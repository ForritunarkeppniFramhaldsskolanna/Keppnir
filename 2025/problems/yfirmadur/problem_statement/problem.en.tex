\problemname{Supervisor}
\illustration{0.4}{hierarchy}{Image taken from \href{https://www.coursesidekick.com/management/study-guides/wmopen-organizationalbehavior/organizational-structures-and-their-history}{coursesidekick.com}}

At large companies it is often quite a lot of work to keep
track of who is in charge of what. To try to make things
easier, it is common to assign everyone to a supervisor, except
the CEO who is their own supervisor. This works well, but then
people start quitting their job and others get new jobs
at the company, so all this planning has to be updated
accordingly.

To save the day, a program that maintains all of this info
is needed, something that can tell any given person who
their supervisor is.

If an employee quits, their supervisor inherits all of their
subordinates. This means if $x$ is the supervisor of $y$, $y$
is the supervisor of $z$ and $y$ quits, $x$ will be the supervisor
of $z$ afterwards. When an employee starts at the company
they are assigned a supervisor. The CEO of the company will never
quit or be replaced, because without them the company
would go bankrupt overnight (according to them anyway).

\section*{Input}
The first line of input contains two integers $n, q$,
the number of employees at the start and the number of
queries that will follow. It will always hold that
$1 \leq n, q \leq 200\,000$. The employees are numbered
$1, 2, \dots, n$ where $1$ is the CEO\@.
Next there is a line with $n$ integers $y_1, y_2, \dots, y_n$.
Here $y_i$ gives the number of the employee that is the supervisor
of employee number $i$, so $1 \leq y_i \leq n$.

Next there are $q$ lines, each with one query.
Each line begins with a \texttt{+}, \texttt{-} or \texttt{?}.
In all cases an integer $x$ will follow on the same line.

If the line begins with a \texttt{+} this means a new
employee is starting whose supervisor will be $x$.
The number of the new employee is the lowest available number.
At the start this number is $n + 1$, but if some employee
quits that number becomes available. So if $2, 8$ and $11$
quit the lowest available number is $2$. But if no one
quits and $n + 1$ was just taken, the next available number
is $n + 2$ and so on.

If the line starts with a \texttt{-} it means that
employee number $x$ just quit.

Finally if the line starts with a \texttt{?} you should
output the number of the supervisor of employee $x$, where
$x$ will always be the number of an employee who is currently
working at the company.
There will never be anyone who is their own supervisor, directly
or indirectly, except the CEO\@.

\section*{Output}
For each query starting with a \texttt{?} print the number
of the supervisor in question on its own line.

\section*{Scoring}
\begin{tabular}{|l|l|l|}
\hline
Group & Points & Constraints \\ \hline
1     & 10   & $n, q \leq 10$. \\ \hline
2     & 15   & $n, q \leq 1\,000$. \\ \hline
3     & 20   & No \texttt{+} or \texttt{-} queries. \\ \hline
4     & 20   & No \texttt{-} queries. \\ \hline
5     & 35   & No further constraints. \\ \hline
\end{tabular}

