\problemname{Veðurspá}
\illustration{0.4}{vidvorun}{Mynd fengin frá \href{vedur.is}{Veðurstofu Íslands}.}

Eitt mikilvægasta verkefni sem finnst á Íslandi er að spá um veðrið.
Veðrið á Íslandi getur reynst mörgum erfitt, til dæmis er Suðurskautslandið eini staður heimsins sem er vindasamari en Ísland.
Veðurstofa Íslands hefur nú sent þér gögnin sín og er þér falið verkefnið að spá um veðrið.

\section*{Inntak}
Fyrsta línan inniheldur tvær heiltölur $n$, fjölda gagnapunkta í safninu, og $m$ fjölda gagnapunkta sem skal spá um í framtíðina.
Önnur línan inniheldur aukaupplýsingar um gagnasafnið, eða breiddargráðu, lengdargráðu, hæð yfir sjávarmáli og upphafsdagsetningu og upphafstíma safnsins.

Næst fylgja $n$ línur þar sem hver lína táknar mælingu.
Hver lína inniheldur hitastig, meðalvindstefnu, meðalvindhraða og rakastig, aðskilin með bilum.
Mælingar eru teknar á klukkustundar fresti.

Hitastig eru rauntölur á bilinu $-50$ til $50$ gefnar með nákvæmlega einum aukastaf eftir punkt, mælt í $^{\circ}C$.
Meðalvindstefnur eru heiltölur á bilinu $0$ upp í $360$, sem táknar átt vindsins í gráðum.
Meðalvindhraðar eru rauntölur á bilinu $0$ upp í $80$ gefnar með nákvæmlega einum aukastaf eftir punkt, mælt í $m/s$.
Rakastig eru heiltölur á bilinu $0$ upp í $110$, sem er mælt í prósentum.

Breiddargráða, lengdargráða og hæð yfir sjávarmáli eru rauntölur gefnar með allt að $6$ aukastöfum eftir punkti.
Upphafsdagsetning er á ISO-8601 formi.
Upphafstími er á ISO-8601 formi, nema án \texttt{Z} aftast.

Gögnin eru alvöru mælingar frá mælitækjum Veðurstofu Íslands frá mismunandi svæðum landsins.
Það getur komið upp að ekki takist að mæla einhver gildi og kemur þá eitt bandstrik \texttt{-} í staðin fyrir tölu þar.

\section*{Úttak}
Skrifaðu út $m$ línur, hver þeirra á sama sniði og gagnapunktarnir í inntakinu, sem tákna spánna þína.
Tölurnar þurfa einnig að uppfylla skorðurnar í inntakinu.
Þú mátt skrifa \texttt{-} til að sleppa að spá um gildi.

\section*{Stigagjöf}
Stig eru gefin út frá réttleika veðurspá, námunduð að næstu heiltölu.
Samtals eru $50$ prufutilvik notuð og summan á niðurstöðum þeirra ákvarðar heildarstig.
Því er veitt $2$ stig fyrir hvert prufutilvik, þá fullt hús fyrir spá sem nær $99.5\%$ eða betri réttleika.
Réttleiki er ekki mældur línulega.

Ef úttakið er ekki á réttu sniði er gefið $0$ stig.
Ef samtals fjöldi stiga yfir öll prufutilvik er $0$ þá er lausnin dæmd röng.
