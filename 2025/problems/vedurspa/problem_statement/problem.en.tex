\problemname{Weather Forecast}
\illustration{0.4}{vidvorun}{Image from \href{vedur.is}{Icelandic Met Office}.}

One of the most important things you can do in Iceland is predict the weather.
The weather in Iceland is difficult for many to endure, as an example Antarctica is
the only place in the world windier than Iceland.
The Icelandic Met Office has now sent you their data and your task is to predict
the weather.

\section*{Input}
The first line of the input contains two integers $n$, the number of data points,
and $m$ the number of data points to predict.
The second line contains extra info about the data set, the latitude, longitude, height
above sea level, the initial day and initial time of the data set.

Next there are $n$ lines, where each line denotes a measurement.
Each such line contains temperature, average wind direction, average wind speed and
humidity, separated by spaces. The measurements are made one hour apart.

The temperatures are real numbers in the range $-50$ to $50$, given with exactly one
digit after the decimal point, measured in $^{\circ}C$.
Average wind directions are integers in the range $0$ to $360$, which denote the direction
of the wind in degrees.
Average wind speeds are real numbers in the range $0$ to $80$ given with exactly one digit
after the decimal point, measured in $m/s$.
Humidities are integers in the range $0$ to $110$, measured in percentages.

Latitude, longitude and height above sea level are real numbers given with up to $6$ digits
after the decimal point.
The initial date is in ISO-8601 format.
The initial time is in ISO-8601 format, but without the trailing \texttt{Z}.

The data is real measurement data from the Icelandic Met Office, taken in different locations
around the country. It may happen that some measurements failed and then the corresponding
value is replaced with \texttt{-}, a single dash.

\section*{Output}
Print $m$ lines, each of them in the same format as the data in the input, which give your
prediction. The numbers have to satisfy the same constraints as in the input.
You may also write \texttt{-} to omit values and make no prediction for them.

\section*{Scoring}
The points are given in accordance with how accurate your prediction is, rounded
to the nearest integer. There are $50$ test cases and the sum of your score on each
of those test cases gives your score. Thus each test case can give at most $2$ points,
with full points given for $99.5\%$ or higher accuracy in the predicted values. Correctness is
not measured linearly.

If the output is not of the correct format that case gets $0$ points.
If the total number of points across all test cases is $0$ the solution is deemed incorrect.
