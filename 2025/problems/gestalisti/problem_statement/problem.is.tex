\problemname{Gestalisti}
\illustration{0.4}{guestlist}{Mynd tekin af \href{https://commons.wikimedia.org/wiki/File:Guest\_list\_of\_people\_attending\_Ulysses\_S.\_Grant\%27s\_funeral,\_August\_8,\_1885\_-\_DPLA\_-\_5076ade4d3accf58b0810e7c8483b4d1\_(page\_1).jpg}{commons.wikimedia.org}}

Nýtt hótel hefur innleitt strangar reglur um aðgang, þar sem aðeins þeir sem eru á sérstökum gestalista fá aðgang.
Vegna mikillar eftirspurnar og tæknilegra vandamála með að halda utan um listann á pappír, þarf að þróa skipanalínuforrit sem sér um að skrá, eyða og leita að nöfnum á listanum.

\section*{Inntak}
\begin{itemize}
    \item Fyrsta línan inniheldur jákvæðu heiltöluna $N$, sem táknar fjölda skipana sem fylgja.
    \item Næstu $N$ línur innihalda skipanir af eftirfarandi gerðum:
    \begin{itemize}
        \item \texttt{+ nafn} – Bætir við nafni á gestalistann.
        \item \texttt{- nafn} – Fjarlægir nafn af gestalistanum.
        \item \texttt{? nafn} – Athugar hvort nafn sé á gestalistanum.
    \end{itemize}
\end{itemize}

Nöfn innihalda aðeins ensku lágstafina \texttt{a} til \texttt{z} og hámarkslengd nafns eru 8 stafir.
Aldrei verður beðið um að fjarlægja nafn sem er ekki á gestalista, eða bæta við nafni sem þegar er á gestalista.

\section*{Úttak}
Fyrir hverja \texttt{? nafn} skipun skal forritið prenta \texttt{Jebb} ef nafnið er á listanum, annars \texttt{Neibb}.

\section*{Stigagjöf}
\begin{tabular}{|l|l|l|}
\hline
Hópur & Stig & Takmarkanir \\ \hline
1     & 20   & Eingöngu \texttt{? nafn} skipanir og $N \leq 20$. \\ \hline
2     & 50   & $N \leq 1\,000$. \\ \hline
3     & 30   & $N \leq 200\,000$. \\ \hline
\end{tabular}
