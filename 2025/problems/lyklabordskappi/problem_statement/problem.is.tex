\problemname{Lyklaborðskappi}
\illustration{0.4}{speedrunner}{Mynd fengin af \href{https://i.kym-cdn.com/entries/icons/original/000/036/070/cover5.jpg}{knowyourmeme.com/}}

Eins og allir vita er Deild Goðsagnanna leikur með mjög vingjarnlegt samfélag.
Þrátt fyrir það detta stundum inn spilarar sem kvarta aðeins meira en aðrir, sérstaklega skógarspilarar.
Meginástæðan fyrir því er að skógarspilarar eyða töluverðum tíma að labba á milli skrímsla og hafa því meiri tíma en aðrir til að senda nokkur skilaboð.
Hann Ingvar er einn þeirra skógarspilara, hann á það til að kvarta örlítið, og ef hann nær því ekki þá gefst hann einfaldlega upp. 

Þar sem Ingvar er vanur spilari þá veit hann nákvæmlega hversu margar sekúndur það tekur hann að labba á milli skrímsla og er það þá tími sem hann getur notað til að skrifa skilaboð.
Einnig hefur hann æft sig vel að skrifa og heldur hann vel utan um hversu marga takka hann getur ýtt á hverri sekúndu.
Gott er að muna að Ingvar telur \texttt{bil} og \texttt{enter} takkana með þegar hann telur á hversu marga takka hann getur ýtt á hverri sekúndu.
Ekki gleyma að það þarf að ýta á \texttt{enter} takkann til að senda hverja línu.

\section*{Inntak}
Inntak samanstendur af tveimur línum.
Fyrsta línan inniheldur þrjár heiltölur $n$, fjöldi orða sem Ingvar ætlar sér að skrifa, $l$, hversu marga takka Ingvar getur ýtt á hverri sekúndu og $t$, fjöldi sekúnda sem Ingvar hefur til að skrifa hverju sinni milli skrímsla.
Önnur línan inniheldur $n$ orð, $w_1, w_2, \dots, w_n$ aðskilin með bilum.

\section*{Úttak}
Ef Ingvar getur ekki skrifað öll orðin sem hann vill skrifa í réttri röð skal prenta út eina línu sem inniheldur \texttt{/ff}.

Annars skal skrifa út orðin í eins fáum línum og mögulegt er.
Hver lína skal tákna skilaboð frá honum sem hann sendir meðan hann ferðast milli skrímsla og þurfa þau að uppfylla tímamörkin.
Sérhvert par af aðliggjandi orðum í skilaboði skal vera aðskilið með að minnsta kosti einu bili.

\section*{Stigagjöf}
\begin{tabular}{|l|l|l|}
\hline
Hópur & Stig & Takmarkanir \\ \hline
1     & 40   & $1 \leq n \leq 1\,000, 4 \leq l \leq 5, t = 1$, lengd hvers orðs er $4$. \\ \hline
2     & 60   & $1 \leq n \leq 1\,000, 4 \leq l \leq 10, 1 \leq t \leq 5$, lengd orða er frá $1$ til $20$. \\ \hline
\end{tabular}

