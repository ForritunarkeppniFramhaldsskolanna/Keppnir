\problemname{ÍÆÍÆÓ}
\illustration{0.4}{fjarrekstur}{Mynd eftur Jónu Þórunn Ragnarsdóttur, fengin af \href{https://commons.wikimedia.org/wiki/File:Fjarrekstur\_vid\_Gaukshofda.JPG}{commons.wikimedia.org}}

Jörmunrekur er orðinn þreyttur á að vera í öllum þessum
forritunardæmum. Til að forðast tölvuheiminn ákvað hann að
flytja út í sveit og gerast bóndi. 

Auðvitað fóru svo hlutir úrskeiðis þegar hann
var að draga í dilka í fyrsta sinn. Hann var að reyna telja
hvað það voru margar kindur komnar í hólf, en vegna þess að
aðrir viðstaddir voru í ullarpeysum taldi hann þá
með líka. Svo þegar hann lauk talningu rann hann til og lenti flatur
á jörðinni. Þetta var ekki sem verst þó hugsaði hann, því hann
fékk þá hugmynd að telja lappirnar sem hann gat séð. Með
þessar tvær tölur ætti hann að geta fundið svarið.

\section*{Inntak}
Inntak samanstendur af tveimur línum.
Fyrsta línan inniheldur eina heiltölu $n$, fjölda kinda og
manneskja samtals, að Jörmunreki frátöldum.
Önnur línan inniheldur heiltöluna $m$ sem táknar fjölda lappa
sem hann taldi.
Hann telur ekki eigin lappir. Engin vera viðstödd þessar
réttir hefur verið limlest. Ef Jörmunrekur sér kind eða
manneskju sér hann allar lappir þeirra.

\section*{Úttak}
Skrifaðu út hvað það eru margar kindur sem Jörmunrekur taldi.
Ef það getur ekki verið að Jörmunrekur taldi rétt, prentið í staðinn
``Rong talning''.

\section*{Stigagjöf}
\begin{tabular}{|l|l|l|}
\hline
Hópur & Stig & Takmarkanir \\ \hline
1     & 100   & $0 \leq n, m \leq 1\,000$ \\ \hline
\end{tabular}

