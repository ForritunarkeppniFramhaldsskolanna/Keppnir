\problemname{Calculator}
\illustration{0.4}{casio}{Image by Christian Horvat, taken from \href{https://commons.wikimedia.org/wiki/File:Casio\_fx-85WA\_20050529.jpg}{commons.wikimedia.org}}

Everything was going swimmingly at Reiknibíllinn until all of a sudden the batteries in
the calculator died! A very important number had been entered into that calculator that must
not be forgotten, so this has to be fixed immediately. Luckily you remember the number, the
main issue is entering that number into the calculator again. The calculator is a bit 
primitive and only supports a few operations.

Can you figure out entering the number given the operations it supports? It only fits
$8$ digits, so if the result of an operation does not fit in $8$ digits the calculator
crashes and everything starts over! Same happens if the result is negative, it does not
support negative numbers either. When the calculator first turns on it shows the number
$0$.

The calculator is rather low on charge, so we want to do this as efficiently as possible.
How much of the charge do you have to expend to enter the number if you do so as efficiently
as possible?

\section*{Input}
The first line of the input contains two integers $A, X$. $A$ is the number of operations
the calculator supports and $X$ is the important number you want to enter into the calculator.
It will always hold that $1 \leq A \leq 5$ and $0 \leq X < 10^8$.
Next there are $A$ lines, each describing one operation.
Each of those lines contains \texttt{op}, $y$, $c$ where $y, c$ are integers and
\texttt{op} is one of the symbols \texttt{+}, \texttt{-}, \texttt{*} or \texttt{/}.
This means this operation applies \texttt{op} to the current value in the calculator
and $y$, with the current value in the calculator as the first number. \texttt{+} and
$y = 5$ thus adds $5$ to the current value in the calculator. $c$ gives the amount of charge
it takes to perform this operation. It will always hold that
$0 \leq y \leq 9$ and $0 \leq c \leq 3$. If the symbol is \texttt{/} then $y \neq 0$.
The division is integers division, so the result is rounded towards $0$.

\section*{Output}
Print the minimum charge expenditure to reach the number $X$.
If there is no way to make the calculator show the number $X$,
instead print \texttt{Engin leid!}.

\section*{Scoring}
\begin{tabular}{|l|l|l|}
\hline
Group & Points & Constraints \\ \hline
1     & 15   & The only operations are \texttt{+} and \texttt{*}, $X \leq 1\,000$. \\ \hline
2     & 15   & The only operations are \texttt{+} and \texttt{*}. \\ \hline
3     & 40   & $c \leq 1$. \\ \hline
4     & 30   & No further constraints. \\ \hline
\end{tabular}

