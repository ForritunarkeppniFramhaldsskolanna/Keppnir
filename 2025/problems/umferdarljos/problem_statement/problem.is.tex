\problemname{Umferðarljós}
\illustration{0.4}{sample1}{Möguleg lausn á sýnidæmi 1.}

Þú hefur fengið starf hjá Vegagerðinni við að endurstilla
umferðarljós höfuðborgarsvæðisins. Eins og skylda er í
öllum ríkisstörfum verður að gera það með sem óskilvirkasta
hætti.

Því mega engin tvö aðlæg umferðarljós vera á sama lit, því
þá væri mögulega hægt að keyra beint í gegnum 2 ljós, og það
væri náttúrulega hættulega skilvirkt!

Einnig þarf að tryggja að engir árekstrar eigi sér stað.
Því þurfa hver gatnamót að hafa grænt ljós á aðra götuna og rautt ljós á hina
eða gult ljós á báðar götur.
Þetta gerir auðvitað ráð fyrir að fólk fylgi umferðarlögunum.

Gatnakerfi höfuðborgarsvæðisins er gefið sem $n$ línustrik
þar sem engin þrjú línustrik skerast í einum punkti. Það er
að segja gatnamót eru ávallt þar sem tvær götur mætast, og
það mætast aldrei þrjár eða fleiri götur á sömu gatnamótunum.
Einnig skerast götur mest í einum punkti.

Gefðu litaskipan á gatnamótunum sem uppfyllir skilyrðin að
ofan. Tvö gatnamót teljast aðlæg ef þau liggja á sama vegi
og það eru engin önnur gatnamót á milli á þeim vegi.

\section*{Inntak}
Inntak byrjar á línu með einni jákvæðri heiltölu $n$, fjöldi
vega. Gefið er að $n \leq 100\,000$.
Næst koma $n$ línur þar sem $i$-ta þeirra gefur
$i$-ta veginn. $i$-ti vegurinn er gefinn sem fjórar heiltölur
$x_1, y_1, x_2, y_2$ þar sem vegurinn liggur frá $(x_1, y_1)$
til $(x_2, y_2)$. Gefið er að $(x_1, y_1) != (x_2, y_2)$.
Öll hnit eru minnst $-1\,000\,000$ og mest $1\,000\,000$.

\section*{Úttak}
Prentið fyrst fjölda gatnamóta á sinni eigin línu.
Svo fyrir hver gatnamót prentið eina línu.
Prentið \texttt{L1 L2 C1 C2} á hverja línu þar sem $L_1$ og $L_2$
eru númer veganna sem skerast. Táknar þá $C_1$ litinn á ljósunum
á þeim mótum fyrir götuna $L_1$ og $C_2$ litinn á ljósunum
fyrir götuna $L_2$ á þeim mótum.
Hér eru $C_1$ og $C_2$ hvort tveggja eitt af stöfunum
\texttt{r}, \texttt{y} eða \texttt{g}. Hér táknar \texttt{r}
rautt, \texttt{y} gult og \texttt{g} grænt.
Það verða aldrei fleiri en $200\,000$ gatnamót í úttakinu.

Ef til eru mörg svör máttu skrifa út hvaða rétta svar sem er.

\section*{Stigagjöf}
\begin{tabular}{|l|l|l|}
\hline
Hópur & Stig & Takmarkanir \\ \hline
1     & 10   & $n \leq 2$. \\ \hline
2     & 20   & $n \leq 10$. \\ \hline
3     & 20   & $n \leq 1\,000$. \\ \hline
4     & 20   & $n \leq 100\,000$, línustrik skarast einungis á endapunktum. \\ \hline
5     & 20   & $n \leq 100\,000$, öll línustrik eru samsíða hnitaásum. \\ \hline
6     & 10   & $n \leq 100\,000$. \\ \hline
\end{tabular}

