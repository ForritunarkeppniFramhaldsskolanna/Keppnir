\problemname{Mögnuð mylla}
\illustration{0.4}{berries}{Mynd eftir Beatrice Murch, fengin af \href{https://commons.wikimedia.org/wiki/File:Tic-tac-toe\_berries.jpg}{commons.wikimedia.org}}

Eitthvað þurfa dómarar að gera meðan á keppni stendur, og
venjuleg mylla er allt of leiðinleg. Fullkomin mylla, sem
hefur komið fram á fyrri keppni, er líka orðin það gömul
að allir dómararnir eru löngu búnir að finna út úr því
hvernig er best að leika. 

Sem betur fer fékk Atli hugljómun og gat bætt ástandið með
nýrri tegund af myllu. Í þessarri myllu eru báðir leikmenn
með stóra og litla stafi, svo annar leikmaður er með
\texttt{x} og \texttt{X} en hinn er með \texttt{o} og
\texttt{O}. Hver leikmaður byrjar með 4 litla stafi og
2 stóra stafi.

Ef leikmaður nær þremur stöfum í röð, stórum eða litlum,
þá vinnur sá leikmaður. Röðin má liggja á ská, svo það eru
átta ólíkar raðir í boði. En ef leikmaður hefur enga leiki
þegar að honum kemur þá tapar hann einnig. Því er aldrei
jafntefli í þessum leik, sem er mun betra en venjuleg mylla
sem endar alltaf í jafntefli milli reyndra leikmanna.

Þegar leikmaður á að gera er þrennt í boði. Í fyrsta lagi
getur hann leikið stórum eða litlum staf á auðan reit, og
á þá einum færri af þeim staf eftir. Ekki er hægt að leika
staf sem leikmaður á engin eintök eftir af. Í öðru lagi getur
leikmaður leikið stóran staf ofan á lítinn staf, sem
fjarlægir litla stafinn (leikmaður fær þann litla staf
ekki til baka). Loks er þriðji valkosturinn að færa stóran
staf sem er þegar í borði ofan á lítinn staf sem er þegar
í borði (leikmaður fær þann litla staf ekki til baka).
Ekki má færa stóran staf sem tilheyrir andstæðingnum.

Til að hrella dómarana tókst þér að hakka þér inn í
tölvuna sem þeir eru að nota til að spila hvorn við annann.
Einnig tókst þér að stilla tækið þannig að þú fengir
alltaf að leika fyrst, svo nú er bara að útbúa forrit
sem getur skúrað gólfið með öllum dómurunum.
Fyrst þú leikur fyrst þá leikur þú \texttt{x} og
\texttt{X}.

\section*{Gagnvirkni}
Þetta er gagnvirkt verkefni. Lausnin þín verður keyrð á móti gagnvirkum dómara
sem les úttakið frá lausninni þinni og skrifar í inntakið á lausninni þinni.
Þessi gagnvirkni fylgir ákveðnum reglum:

Þitt forrit og dómaraforritið skiptast á að prenta út
núverandi ástand leikborðs. Leikborðið eru $3 \times 3$
reitir, gefið sem 3 bókstafir hver á 3 línum, án nokkurra
bilstafa utan nýlínustafanna. Tómur reitur er táknaður með
stafnum \texttt{.}.

Forrit þitt byrjar á að prenta leikborðið eins og það
er eftir fyrsta leik. Svo les forrit þitt inn stöðu borðsins
eftir leik dómara. Þetta endurtekur sig svo. Ef leikur er
búinn, það er að segja ef annar leikmaður á engan leik
eftir eða er með þrjá í röð, prentar dómaraforritið í
staðinn \texttt{Tap!} eða \texttt{Sigur!} eftir því hvort
þú tapaðir eða vannst. Eftir þetta á forrit þitt að ljúka
keyrslu. Dómaraforrit prentar aldrei borð og streng
sem er \texttt{Tap!} eða \texttt{Sigur!}, aðeins annað hvort.

Vertu viss um að gera \texttt{flush} eftir hvern leik, t.d., með
\begin{itemize}
    \item \verb!print(..., flush=True)! í Python,
    \item \verb!cout << ... << endl;! í C++,
    \item \verb!System.out.flush();! í Java.
\end{itemize}

Með verkefninu fylgir tól sem viðhengi til þess að hjálpa við að prófa lausnina þína.

\section*{Stigagjöf}
Lausnin þín verður keyrð á móti mörgum andstæðingum.
Margir leikir verða keyrðir á móti hverjum andstæðing.
Þetta verða samtals $50$ leikir, og fyrir hvern sigur
fást $2$ stig, fyrir tap fást $0$ stig. Ef forrit þitt
prentar ógildan leik eða tekst ekki að ljúka
keyrslu rétt af öðrum ástæðum fást $0$ stig fyrir þann leik, og er
það merkt \texttt{Wrong Answer} frekar en \texttt{Accepted}.
Lokadómur er hins vegar \texttt{Accepted} svo lengi sem 
einhver leikur er \texttt{Accepted}.


