\problemname{Magnificent Noughts \& Crosses}
%% plainproblemname: Magnificent Noughts & Crosses
\illustration{0.4}{berries}{Image by Beatrice Murch, taken from \href{https://commons.wikimedia.org/wiki/File:Tic-tac-toe\_berries.jpg}{commons.wikimedia.org}}

The judges have to pass their time in one way or another
while a contest is ongoing, and the usual distraction of
tic-tac-toe has gotten rather stale. Perfect tic-tac-toe,
as covered in an earlier contest, has also become stale
because all the judges have long figured out how to play
it optimally.

Luckily Atli suddenly had a burst of inspiration and
managed to improve the situation by inventing a new type
of tic-tac-toe. In this version both players have both
lower and upper case characters, so one player uses
\texttt{x} and \texttt{X} while the other uses
\texttt{o} and \texttt{O}. Each player starts with 4 lower
case letters and 2 upper case letters.

If a player manages to place three of their letter in a
row, lower or upper case, that player wins. The row
can be diagonal, so there are $8$ possible rows to win on.
If a player has no legal moves on their turn, they lose.
Thus there is never a tie in this game, a great improvement
over regular tic-tac-toe, which always ends in a tie between
skilled players.

When a player is to move there are three options. The first
one is to play a lower or upper case letter on an empty
space, and then has one less of that letter left.
A player can not play a letter that they have run out of.
Secondly the player can play one of their upper case letters
on top of a lower case letter that's already on the board,
removing the lower case letter (the corresponding player does
not get that lower case letter back). Finally the third
option is to move an upper case letter they have on the board
on top of a lower case character that is also already on the
board (the corresponding player does not get that lower case
letter back either in this case). A player may only move their
own upper case letters.

To get one over on the judges you have now managed to hack
into the computer they are using to play against one another.
You even managed to rig it so that you would always get to
move first. Thus the only thing left to do is to make a 
program that can demolish the morale of the judges. Since
you get the first move you have \texttt{x} and \texttt{X}.

\section*{Interactivity}
This is an interactive problem. Your solution will be tested against an interactive
judge which reads the output of your solution and prints the input your solution
receives. This interaction follows certain rules:

Your program and the judge program will take turns printing
the current board. The board is $3 \times 3$ squares, given
as 3 letters each on 3 lines, with no whitespace aside
from the 3 newline characters. An empty square is denoted
by \texttt{.}.

Your program starts by printing the board after your first
move. Then your program should read the board as it is after
the judge makes its move. This then repeats.
If the game is over, because someone has three in a
row or someone has no moves left, the judge program will
print \texttt{Tap!} (Icelandic for loss) or \texttt{Sigur!}
(Icelandic for victory) depending on whether you won or lost.
After this your program should exit. The judge program never
prints a board and one of the strings \texttt{Tap!} or
\texttt{Sigur!}.

Make sure to \texttt{flush} after each guess, for example using
\begin{itemize}
    \item \verb!print(..., flush=True)! in Python,
    \item \verb!cout << ... << endl;! in C++,
    \item \verb!System.out.flush();! in Java.
\end{itemize}

The task has a testing tool attached to help test your solution.

\section*{Scoring}
Your program will be played against several opponents.
It will play several games against each opponent.
This will be a total of $50$ games, and for each
victory you get $2$ points, getting $0$ points for a loss.
If your program prints an illegal move or otherwise fails
to run correctly you will get $0$ points for that game 
and it will be marked with \texttt{Wrong Answer} rather than
\texttt{Accepted}. The final verdict is however
\texttt{Accepted} as long as some game gets the verdict
\texttt{Accepted}.


