\problemname{Gufustrókur}
\illustration{0.4}{strokur}{Mynd eftir Patriciu Fehrentz}

Oft rýkur gufa upp úr jörðinni kringum Kröflu.
Eftir því hvernig vindáttin liggur getur gufustrókurinn
þá legið í ólíkar áttir og ýmsar nytsamlegar upplýsingar
geta fengist út frá því að rannsaka slíka gufustróka.

Rannsóknarteymi er búið að taka tvær myndir þar sem
gufustrókurinn liggur í tvær ólíkar áttir.
Hvað hefur vindáttin snúist mikið í minnsta lagi?

Stefna gufustróksins verður gefin sem gráðufjöldi
frá norðri, svo norður er $0^{\circ}$, vestur er
$90^{\circ}$, suður er $180^{\circ}$ og austur er
$270^{\circ}$. Ein gráða austan við norður er þá
$359^{\circ}$.

Til dæmis ef myndirnar eru $170^{\circ}$ og $100^{\circ}$
er svarið $70^{\circ}$, því gufustrókurinn getur snúið
í báðar áttir. Eins ef myndirnar eru $355^{\circ}$ og
$7^{\circ}$ er svarið $12^{\circ}$ því $359^{\circ}$ og
$0^{\circ}$ eru aðlæg.

\section*{Inntak}
Inntak samanstendur af tveimur línum.
Fyrri línan gefur stefnu gufustróksins á fyrri mynd,
gefin sem heiltala frá $0$ til $359$.
Seinni línan gefur stefnu gufustróksins á seinni mynd,
gefin sem heiltala frá $0$ til $359$.

\section*{Úttak}
Prentaðu minnsta mögulega fjöldi gráða sem gufustrókurinn hefur
snúist um milli mynda.

\section*{Stigagjöf}
\begin{tabular}{|l|l|l|}
\hline
Hópur & Stig & Takmarkanir \\ \hline
1     & 100   & Engar frekari takmarkanir. \\ \hline
\end{tabular}

