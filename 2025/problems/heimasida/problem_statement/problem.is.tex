\problemname{Heimasíða}
\illustration{0.4}{flow}{Mynd fengin af \href{https://commons.wikimedia.org/wiki/File:Data\_Flow\_of\_the\_Internet\_Protocol\_Suite.PNG}{commons.wikimedia.org}}

Oft skapast vandræði þegar íslenskum fyrirtækjum eða félögum
vantar að stofna heimasíðu. Þessa dagana geta tenglarnir
innihaldið íslenska sérstafi, en vaninn er enn að sleppa
íslenskum sérstöfum. Einnig þarf að sleppa öllum bilstöfum
og öðrum sérstökum táknum.

Mörg félaganna og fyrirtækjanna sem lenda í þessu eru
ekkert sérstaklega tæknivædd, svo þú sérð það í hendi þér
að geta boðið fram þessa þjónustu fyrir himinhátt verð.

Fyrsta skref er þá að henda út öllu í inntakinu sem eru
ekki bókstafir eða tölustafir. Næst þarf að breyta öllum
stórum stöfum í litla stafi. Loks þarf að breyta íslensku
stöfunum. Broddstafir missa einfaldlega brodd sinn.
Svo verður \texttt{ö} að \texttt{o}, \texttt{æ} að
\texttt{ae}, \texttt{ð} að \texttt{d} og loks
\texttt{þ} að \texttt{th}.

\section*{Inntak}
Inntak samanstendur af einni línu.
Þessi lína getur innihaldið alla prentanlega ASCII stafi
ásamt stöfunum
\texttt{Á},
\texttt{á},
\texttt{Ð},
\texttt{ð},
\texttt{É},
\texttt{é},
\texttt{Í},
\texttt{í},
\texttt{Ó},
\texttt{ó},
\texttt{Ú},
\texttt{ú},
\texttt{Ý},
\texttt{ý},
\texttt{Þ},
\texttt{þ},
\texttt{Æ},
\texttt{æ},
\texttt{Ö},
\texttt{ö}.
Í þessum lista er hástafur ávallt á undan samsvarandi
lágstaf.
Hins vegar verða einu bilstafirnir í inntakinu bil
og svo nýlínustafur í lokin.
Línan verður mest $100$ stafir, ásamt nýlínustaf.

\section*{Úttak}
Skrifaðu út hvað heimasíða stofnunarinnar í inntakinu ætti
að vera, út frá reglunum að ofan. Einnig skal bæta við
\texttt{.is} fyrir aftan, og setja allt á eina línu.

\section*{Stigagjöf}
\begin{tabular}{|l|l|l|}
\hline
Hópur & Stig & Takmarkanir \\ \hline
1     & 25   & Einungis ASCII bókstafir í inntaki. \\ \hline
2     & 25   & Einungis ASCII stafir í inntaki. \\ \hline
3     & 50   & Engar frekari takmarkanir. \\ \hline
\end{tabular}

