\problemname{Radicalists}
\illustration{0.4}{judea}{Screen capture from the movie Life of Brian}

Disagreements sometimes arise in the Icelandic Competitive
Programming Society when discussing problems for upcoming
programming contests. Recently these disagreements have
gotten somewhat out of hand and resulted in the group 
splitting up when writing problems to avoid confrontation.

As an example recently the group split in two when a portion
of the problem writers got sick and tired of all the League
of Legends problems and said that the rest of the society
was hopelessly addicted to this boring video game.

This worked for some time until disagreements arose again,
that time some people from both groups complained that the
addiction to Runescape among some members was no better, so
both groups split once more and then there were four.

This was starting to get hard to track. Thus we now need a
program that can maintain the state of things as groups
keep splintering due to groups of radicalists splitting off
from any group they are in.

\section*{Input}
The first line of input contains two positive integers
$n, q$. Here $n$ denotes the number of members and $q$
the number of queries to follow.
Next there are $q$ lines, each containing one query.
Each line begins with one of the letters \texttt{r}, 
\texttt{s} or \texttt{m}. This letter denotes the type
of the query.

If the line starts with a \texttt{r} there will follow a
positive integer $t$.
Then there are $t$ positive integers $x_i$, all on the
same line. This denotes that members number $x_1, \dots,
x_t$ have become radicalists and split away from their 
current group. If $x_i$ and $x_j$ were in the same group
before splitting off they will be in the same group after.
The numbers satisfy $1 \leq x_i \leq n$.

If the line starts with \texttt{s} the program should print
the number of groups the society is currently split into,
printing that number on its own line. Empty groups do not
count.

Finally if the line starts with \texttt{m} there will follow
a single positive integer $u$ on the same line, satisfying
$1 \leq u \leq n$. Then the program should print the members
which are in the same group as $u$, all on one line with
numbers separated by spaces. The members may be printed in
any internal order.
There are at most $2 \cdot 10^6$ numbers in the input.

\section*{Output}
For each query starting with \texttt{s} or
\texttt{m} the program should print one line, as described
above. There will be at most $2 \cdot 10^6$ integers in
the output.

\section*{Scoring}
\begin{tabular}{|l|l|l|}
\hline
Group & Points & Constraints \\ \hline
1     & 10   & $n \leq 10, q = 1$ \\ \hline
2     & 20   & $n, q \leq 1\,000$ \\ \hline
3     & 30   & $n \leq 100\,000$ \\ \hline
4     & 20   & $n \leq 10^7, m = 1$ \\ \hline
5     & 20   & $n \leq 10^7$ \\ \hline
\end{tabular}

