\problemname{Radikalistar}
\illustration{0.4}{judea}{Skjáskot úr kvikmyndinni Life of Brian}

Ósætti koma stundum upp í keppnisforritunarfélaginu þegar
verið er að ræða möguleg dæmi fyrir keppnina. Nýlega hafa
þessi ósætti orðið svo mikil að dæmahöfundar hafa átt það
til að skipta sér niður í ólíka hópa til að vinna ekki með
þeim sem þeir eru ósammála.

Til dæmis splundraðist nýlega hópurinn þegar hluti dæmahöfunda
fékk upp í kok af öllum Deild Goðsagnanna dæmunum og héldu
því fram að restin af félaginu væri vonlaust háð þessum
leiðinlega tölvuleik.

Þetta gekk í einhvern tíma þar til aftur kom upp ósætti. Þá
kvörtuðu bæði stuðningsmenn og andmælendur Deild Goðsagnanna
yfir því að öll dæmin væru um Rúnaheim sem væri ekki 
ásættanlegt heldur, svo báðir helmingar félagsins skiptust
aftur upp og voru þá fjórir hópar talsins.

Þetta var farið að vera erfitt að halda utan um. Því þarf
nú að útbúa forrit sem heldur utan um hvernig hópar skiptast
og skiptast eftir því sem nýjir hópar radikalista myndast
og segja sig frá sínum fyrri hópum.

\section*{Inntak}
Fyrsta lína inntaksins inniheldur tvær jákvæðar heiltölur
$n, q$. Hér er $n$ fjöldi meðlima félagins og $q$ er fjöldi
fyrirspurna sem mun fylgja.
Næst koma $q$ línur, hver með einni fyrirspurn.
Hver línanna byrjar á einum af stöfunum \texttt{r}, 
\texttt{s} eða \texttt{m}. Þessi stafur táknar tegund 
fyrirspurnarinnar.

Ef línan byrjar á \texttt{r} kemur næst jákvæð heiltala $t$.
Loks koma svo $t$ jákvæðar heiltölur $x_i$ á sömu línu.
Þetta táknar að félagsmeðlimir númer $x_1, \dots, x_t$ 
eru radikalistar og segja sig frá núverandi hópum. 
Ef $x_i$ og $x_j$ voru í sama hóp verða þeir áfram saman 
eftirá. Tölurnar uppfylla $1 \leq x_i \leq n$.

Ef línan byrjar á \texttt{s} skal prenta fjölda hópa sem
félagið samanstendur af að svo stöddu á sinni eigin línu.
Tómir hópar teljast ekki með.

Loks ef línan byrjar á \texttt{m} fylgir ein jákvæð heiltala
$u$ á sömu línu, hún uppfyllir $1 \leq u \leq n$. Þá á að
prenta meðlimi hópsins sem meðlimur númer $u$ er í, allt
á einni línu með bilum á milli. Prenta má meðlimina í
hvaða innbyrðis röð sem er.
Það verða mest $2 \cdot 10^6$ tölur í inntakinu.

\section*{Úttak}
Fyrir hverja fyrirspurn sem byrjar á \texttt{s} eða 
\texttt{m} skal prenta eina línu, eins og lýst er að ofan.
Það verða mest $2 \cdot 10^6$ tölur í úttakinu.

\section*{Stigagjöf}
\begin{tabular}{|l|l|l|}
\hline
Hópur & Stig & Takmarkanir \\ \hline
1     & 10   & $n \leq 10, q = 1$ \\ \hline
2     & 20   & $n, q \leq 1\,000$ \\ \hline
3     & 30   & $n \leq 100\,000$ \\ \hline
4     & 20   & $n \leq 10^7, m = 1$ \\ \hline
5     & 20   & $n \leq 10^7$ \\ \hline
\end{tabular}

