\problemname{Inputmaxxing}
\illustration{0.4}{max}{Image of the MAX gene by A2{-}33, taken from \href{https://en.wikipedia.org/wiki/MAX\_(gene)\#/media/File:1an2\_max\_dimer2.png}{commons.wikimedia.org}}

Bráhildur was passing the time by scrolling through the
social media app Klukknatif. She kept running into different
words with the suffix ``maxxing'' and started
wondering what it was all about. Thanks to her experience
with genealogy she assumed it might be related to the gene
MAX, but upon further inspection this turned out not to be
the case. She kept looking into it though, looking into the
operating system MAX, the hamburger chain MAX, the singer
Max Schneider, the athlete Maxx Crosby, the clothing store
T J Maxx and much more. But none of these leads went
anywhere.

She figured it must just be literal then, you can put
absolutely anything in front of the word ``maxxing'' to
participate in this trend. Then you just have to maximise
whatever you put in front of the word. ``Inputmaxxing''
must then be a great way to take part in the trend.

\section*{Input}
The first line of input contains a single integer $n$.
Then follow $n$ lines, each containing a single
integer $x_i$ satisfying $10 \leq x_i \leq 10^9$.

\section*{Output}
Print the ``inputmaxx'', that is, the largest value
in the input.

\section*{Scoring}
\begin{tabular}{|l|l|l|}
\hline
Group & Points & Constraints \\ \hline
1     & 50   & $n = 2$ \\ \hline
2     & 50   & $1 \leq n \leq 1\,000$ \\ \hline
\end{tabular}

