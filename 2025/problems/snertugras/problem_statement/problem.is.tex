\problemname{Snertu gras}
\illustration{0.4}{ginyu}{Dómarar að snerta gras.}

Keppendurnir sem eru mættir í Forritunarkeppni
Framhaldsskólanna eru margir reyndir og hæfir forritarar.
En það er eitt sem þessir keppendur gera að meðaltali
mun minna af en meðaleinstaklingur og mættu endilega læra
að gera oftar.
Það er auðvitað, eins og dæmatitill gefur til kynna, að
snerta gras.

Til að aðstoða keppendur við að venjast því að snerta gras
brúum við hér bilið með því að leyfa ykkur að útbúa forrit
sem snýst um að snerta gras. Þá eruð þið vonandi aðeins meira
undirbúin undir það að snerta alvöru gras á eftir.

Þið fáið kort af staðsetningu ykkar, veggjum og staðsetningum
á grasi. Út frá þessu þurfið þið að finna stystu leið frá
núverandi staðsetningu ykkar að næsta grasbletti. Kortið
verður rúðustrikað, svo eitt skref er hreyfing frá núverandi
reit að reit fyrir norðan, sunnan, vestan eða austan þann
reit. Ekki er hægt að fara út af kortinu, það táknar jaðar
svæðisins sem er óhætt að fara á. Ekki væri gott að fara
snerta of mikið gras og fara venjast þess að vera úti, þá
mynduð þið kannski hætta að forrita, sem væri afar slæmt.

\section*{Inntak}
Inntak byrjar á tveimur heiltölum $h, w$, hæð og breidd
kortsins. Gefið er að $h, w \geq 1$ og $h \cdot w \leq 1\,000\,000$.
Næst fylgja $h$ línur, hver með $w$ stöfum (ásamt
nýlínustaf). Þessir stafir verða allir \texttt{S}, \texttt{.},
\texttt{\#} eða \texttt{G}.
\texttt{S} táknar núverandi staðsetningu þína og mun sá
stafur koma fyrir nákvæmlega einu sinni í inntaki.
Reitir með grasi eru táknaðir með \texttt{G}, reitir með
vegg eru táknaðir með \texttt{\#} og auðir reitir með
\texttt{.}. Hægt er að færa sig yfir á alla reiti nema
þá sem eru með vegg.

\section*{Úttak}
Ef hægt er að komast á reit með grasi, prentið fjölda
skrefa sem það tekur í minnsta lagi.
Prentið annars \texttt{thralatlega nettengdur}.

\section*{Stigagjöf}
\begin{tabular}{|l|l|l|}
\hline
Hópur & Stig & Takmarkanir \\ \hline
1     & 20   & $h, w \leq 2$. \\ \hline
2     & 40   & $h \cdot w \leq 100$. \\ \hline
3     & 15   & Það eru engir \texttt{\#} reitir. \\ \hline
4     & 25   & Engar frekari takmarkanir. \\ \hline
\end{tabular}

\section*{Aukaverðlaun}
Í hádegishlé þessarar keppni hvetjum við ykkur til að snerta
gras. Okkur er svo alvara með þetta að í ár verða aukaverðlaun
fyrir það lið sem sendir okkur bestu myndina af sér í
keppnisbolunum sínum að snerta gras. Hægt er að senda okkur
þá mynd (með liðsnafni) á \texttt{\#jarm} á Discord rás okkar,
eða á \texttt{keppnisforritun@gmail.com}.
