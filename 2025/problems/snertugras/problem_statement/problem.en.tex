\problemname{Touch Grass}
\illustration{0.4}{ginyu}{Judges touching grass.}

The contestants at Forritunarkeppni Framhaldsskólanna are
by and large quite good programmers. But there is something
that these contestants are certainly below average at
achieving, and do much less than the average person does.
They could even do with some training and experience in this
field. And that is, as the problem title suggests, touching
grass.

To assist contestants in getting used to the thought of
touching grass we try to bridge the gap here by making you
make a program that is about touching grass. You will
hopefully feel a bit more mentally ready to touch real
grass afterwards.

You will receive a map of your location, the surrounding
walls and where there is grass. From this you need to find
the shortest path to the nearest patch of grass. The map
is composed of square cells and each step moves you to
an adjacent cell to the north, south, west or east of your
current cell. You can not leave the boundary of the map,
that denotes the boundary of the area it is safe for you
to visit. Going too far and touching too much grass might
get you to consider doing something other than programming,
which would be very bad.

\section*{Input}
The input starts with two integers $h, w$, the height and
width of the map. You may assume that $h, w \geq 1$ and
$h \cdot w \leq 1\,000\,000$.
Next there are $h$ lines, each with $w$ characters (along
with a newline character). These characters will all be
\texttt{S}, \texttt{.}, \texttt{\#} or \texttt{G}.
\texttt{S} denotes your current location and this letter
will appear exactly once in the input. Cells with grass
are denoted with \texttt{G}, cells with walls are denoted
with \texttt{\#} and empty cells with \texttt{.}. 
You can move to any cell except those with walls.

\section*{Output}
If you can reach a cell with grass, print the minimum
number of steps needed to reach a cell with grass.
Otherwise print \texttt{thralatlega nettengdur}
(Icelandic for terminally online).

\section*{Scoring}
\begin{tabular}{|l|l|l|}
\hline
Group & Points & Constraints \\ \hline
1     & 20   & $h, w \leq 2$. \\ \hline
2     & 40   & $h \cdot w \leq 100$. \\ \hline
3     & 15   & There are no \texttt{\#} cells. \\ \hline
4     & 25   & No further constraints. \\ \hline
\end{tabular}

\section*{Extra prize}
At the lunch break of this contest we implore you to go
touch grass. We are so serious about this that this year
there will be a bonus prize for the team that sends us the
best picture of them touching grass while wearing their
contest shirts. You can send this picture (and your team
name) to the \texttt{\#jarm} channel on our Discord,
or to \texttt{keppnisforritun@gmail.com}.
