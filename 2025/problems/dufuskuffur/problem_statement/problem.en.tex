\problemname{Pigeon-holes}
\illustration{0.4}{pigeons}{Image from \href{https://commons.wikimedia.org/wiki/File:TooManyPigeons.jpg}{commons.wikimedia.org}}

In the world of mathematics there is a famous principle
called Dirichlet's box principle. It is also known as the
Pigeonhole principle. The principle says that if you have
more doves than boxes for the doves you have to put more
than one dove in some box to fit them all in a box. We will
look at this phenomenon in this problem, and print the answer
``Dufur passa'' if the doves all fit in the boxes without
cramming more than one dove in a single box. This is actually
related to another mathematical principle called the Principle
of Inclusion-Exclusion, as sometimes doves will be excluded
if cramming more than a single dove in each box is not 
allowed. In this case the answer should be ``Dufur passa
ekki''. This rule is often shortened to be written as PIE,
or even denoted by the Greek letter $\pi$. This actually
relates to a well known movie about a man trying to find
connections and meaning in numbers. This is exactly what we
are trying to do here, like connecting it all to the movie
The Number 23. You can search for meaning in numbers for
as long as you have the time to, trying to check if things
fit together perfectly or whether things are imbalanced.
This is actually very important in this problem. If everything
fits perfectly, with no doves or boxes left over, the answer
should be ``Dufur passa fullkomlega''. Many others have also
considered the meaning of numbers. A well known mason 
received a question about this exact thing, which brings us
to our next topic, that of cods. Iceland won the cod wars,
proving that it is ``stórasta land í heimi''. This should of
course be kept in mind when you solve the problem, as it
is of utmost importance.

\section*{Input}
The input consists of two lines.
The first line contains an integer $n$, the number of doves.
The second line contains an integer $m$, the number of boxes
for the doves.

\section*{Output}
Print whether the doves fit, as described above.

\section*{Scoring}
\begin{tabular}{|l|l|l|}
\hline
Group & Score & Constraints \\ \hline
1     & 100   & $0 \leq n, m \leq 1\,000$ \\ \hline
\end{tabular}

