\problemname{Dúfuskúffur}
\illustration{0.4}{pigeons}{Mynd fengin af \href{https://commons.wikimedia.org/wiki/File:TooManyPigeons.jpg}{commons.wikimedia.org}}

Í stærðfræðiheiminum er til fræg regla sem er oft kölluð
skúffuregla Dirichlet. Á ensku er hún almennt kölluð
\emph{The Pigeonhole Principle}. Reglan segir að ef maður er með fleiri
dúfur en hólf fyrir dúfurnar og maður vill koma þeim öllum
fyrir, þá verður maður að setja fleiri en eina dúfu í
einhvert hólfið. Við ætlum að skoða þetta fyrirbæri í þessu
dæmi, og þá á einmitt að prenta ``Dufur passa'' ef 
dúfurnar komast fyrir án þess að troða fleiri en eina dúfu
í eitthvert hólfið. En þetta tengist einmitt inn á aðra 
stærðfræðireglu sem kallast \emph{Principle of Inclusion-Exclusion}
þar sem stundum verða dúfur útundan ef maður neitar að troða.
Í þessu tilfelli á einmitt að prenta ``Dufur passa ekki''.
Þessi regla er oft stytt sem PIE, eða jafnvel táknuð með
gríska stafnum $\pi$. Þetta tengist einmitt inn á fræga
kvikmynd sem fjallar um mann sem missir sig í að reyna finna
tengingar í tölum. Það er einmitt það sem við erum að gera
hér, eins og að tengja þetta inn á myndina \emph{The Number 23}.
Það má einmitt mikið leita að merkingu í tölum, hvort hlutir
passi vel saman eða hvort hlutir séu í ójafnvægi. Það er
einmitt mjög mikilvægt í þessu dæmi. Því ef allt passar
fullkomlega í dæminu, með engar dúfur eða skúffur afgangs,
á einmitt að prenta ``Dufur passa fullkomlega''. Margir aðrir
hafa líka velt fyrir sér merkingu talna. Vel þekktur múrari
fékk einmitt þá spurningu, sem tengir okkur inn á þorska. 
Ísland vann einmitt þorskastríðið, og er því einmitt stórasta
land í heimi. Hafa skal þetta allt í huga þegar þið leysið
eftirfarandi dæmi.

\section*{Inntak}
Inntak samanstendur af tveimur línum.
Fyrsta línan inniheldur eina heiltölu $n$, fjölda dúfa.
Önnur línan inniheldur heiltöluna $m$ sem táknar fjölda hólfa
fyrir dúfurnar.

\section*{Úttak}
Prentaðu út hvort dúfurnar passi, eins og lýst er að ofan.

\section*{Stigagjöf}
\begin{tabular}{|l|l|l|}
\hline
Hópur & Stig & Takmarkanir \\ \hline
1     & 100   & $0 \leq n, m \leq 1\,000$ \\ \hline
\end{tabular}

