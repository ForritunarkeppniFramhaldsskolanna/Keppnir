\problemname{Nafnauki}
\illustration{0.4}{files}{Mynd eftir Yashar, fengin af \href{https://commons.wikimedia.org/wiki/File:File-extensions-list.png}{commons.wikimedia.org}}

Geir er ekkert sérstaklega vel að sér í tölvumálum, svo þegar
það kemur að því að opna skrár getur hann oft lent í
vandræðum. Þetta truflar oft kennslu hans, þar sem hann þarf
stundum að opna skrár í tíma til að sýna nemendum sínum.
Til þess að flýta fyrir hlutum útbjó vinkona hans lista af
nafnaukum á skrám (file extensions á ensku, einnig þekkt sem
skrárending) og hvað hann ætti að
gera fyrir hverja þeirra. Til dæmis stóð að fyrir MP4 gæti
hann tvísmellt til að opna skrána, svo tvísmellt til að þekja
allan skjáinn, smella til að hefja myndband og loks færa
músina burt af skjánum. 

Þetta var allt gott og blessað, en þegar það kom að því að
nýta þennan lista fattaði Geir að hann hefði ekki hugmynd
hvernig hann ætti að vita hvaða nafnauki væri á skránni.
Getur þú hjálpað honum?

Nafnauki á skrám er ávallt einn til fimm stafir. Finna má
nafnaukann með því að taka allt sem kemur eftir síðasta
punktinum í nafninu á skránni.

\section*{Inntak}
Inntak er ein lína, skrárheiti skrárnar sem Geir er að
reyna opna. Skrárheitið inniheldur aðeins ensku stafina 
\texttt{a} til \texttt{z}, tölustafina $0$ til $9$ ásamt punktum. 
Stafirnir geta verið há- og
lágstafir. Skrárheitið mun alltaf hafa gildan
nafnauka, eins og lýst er að ofan. Skrárheitið er mest
$32$ stafir samtals.

\section*{Úttak}
Skrifaðu út nafnauka skrárnar, með punkti.

\section*{Stigagjöf}
\begin{tabular}{|l|l|l|}
\hline
Hópur & Stig & Takmarkanir \\ \hline
1     & 30   & Nafnaukinn er nákvæmlega 3 stafir. \\ \hline
2     & 40   & Það er nákvæmlega einn punktur í skrárheitinu. \\ \hline
3     & 30   & Engar frekari takmarkanir. \\ \hline
\end{tabular}

