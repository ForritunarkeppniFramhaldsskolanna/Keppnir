\problemname{File Extension}
\illustration{0.4}{files}{Image by Yashar, taken from \href{https://commons.wikimedia.org/wiki/File:File-extensions-list.png}{commons.wikimedia.org}}

Geir is not particularly good with computers, so when it comes
to opening files he often has trouble. This often delays his
teaching as he sometimes has to open files in class to show
his students something. To get things running more smoothly
his friend prepared a list of file extensions (nafnauki in
Icelandic) and what he should do for each of them. For example
it said that for MP4 files he could double click on it to 
open a video player, double click to put it into full screen,
click once more to start the playback and then finally
move his cursor off the player.

This was all well and good, but when it came to using this
list Geir realised he had no idea how to tell what the
extension of a given file is. Can you help him?

File extensions will always be one to five letters long.
They are defined as whatever follows the last dot in the name
of the file.

\section*{Input}
The input is a single line, the name of the file Geir is
trying to open. The file name will contain only the ASCII
letters a to z, lower or upper case, along with dots and
the digits $0$ to $9$.
The file name will always have a valid file extension as
noted above. The file name is at most $32$ letters.

\section*{Output}
Print the file extension of the file, with the dot.

\section*{Scoring}
\begin{tabular}{|l|l|l|}
\hline
Group & Points & Constraints \\ \hline
1     & 30   & The file extension is exactly $3$ letters. \\ \hline
2     & 40   & There is exactly one dot in the file name. \\ \hline
3     & 30   & No further constraints. \\ \hline
\end{tabular}

