\problemname{RedTok ShortReels}
\illustration{0.4}{plus}{Image taken from \href{https://commons.wikimedia.org/wiki/File:Google\_Plus\_logo\_(2011-2015).png}{commons.wikimedia.org}}

Jarmína uses the app Rauðtifa Stuttkefli rather quite a lot.
It's starting to take up so much of her time and she is
so hooked on the dopamine that she gets from it that she
has contacted The Competitive Programming Society of Iceland
to find a way for her to maximise the dopamine she gets
from the app.

The program shows her videos in a fixed order, each of
a particular length $L_i$, given in milliseconds, and some
dopamine amount $D_i$ that she gets for watching it.
It takes $k$ milliseconds to skip a video. She gets no
dopamine for watching part of a video.

She now needs a program that finds her the best way to
watch the videos so she can maximise the dopamine she gets.

Can you help her?

For each video she either needs to watch it in its entirety
or use $k$ milliseconds to skip it, unless she runs out of
time.

\section*{Input}
The first line of input contains two integers $n, k$,
the number of videos and the number of milliseconds it
takes to skip a video.
It always holds that $1 \leq n \leq 1\,000$ and $0 \leq k
\leq 10^9$. 

Next there are $n$ lines, the $i$-th of which
describes the $i$-th video the app shows.
The $i$-th line contains two integers $L_i, D_i$, the length
of the $i$-th video in milliseconds and the amount of
dopamine she gets for watching the entirety of that video.
It always holds that $0 \leq L_i \leq 100\,000$ and $0 \leq D_i \leq 10^9$.

Finally there is a line with an integer $T$, the total
amount of time Jarmína has in milliseconds.
It always holds that $0 \leq T \leq 10^9$. Let $S$
denote the total length of all videos in milliseconds.
It always holds that $0 \leq S \leq 100\,000$.
The videos are given in the order that they appear on
Jarmína's phone.

\section*{Output}
Print the maximum amount of dopamine Jarmína can get
in $T$ milliseconds.

\section*{Scoring}
\begin{tabular}{|l|l|l|}
\hline
Group & Points & Constraints \\ \hline
1     & 10   & $n = 1$. \\ \hline
2     & 20   & $n \leq 20$. \\ \hline
3     & 35   & $k = 0$. \\ \hline
4     & 35   & No further constraints. \\ \hline
\end{tabular}

