\problemname{RauðTifa StuttKefli}
\illustration{0.4}{plus}{Mynd fengin af \href{https://commons.wikimedia.org/wiki/File:Google\_Plus\_logo\_(2011-2015).png}{commons.wikimedia.org}}

Jarmína notar appið Rauðtifa Stuttkefli ansi mikið.
Þetta er farið að taka svo mikinn tíma og hún er svo háð
dópamíninu sem því fylgir að hún er nú búin að leita
til Keppnisforritunarfélagi Íslands til að hámarka dópamín
sitt á appinu.

Forritið sýnir henni myndbönd í fastri röð, hvert þeirra
hefur einhverja lengd $L_i$, gefið í millisekúndum, og
eitthvað ánægjustig $D_i$, gefið í dópamíneiningum.
Það tekur $k$ millisekúndur að 
sleppa myndbandi og fara á næsta myndband.
Ekkert dópamín fæst nema horft sé á heilt myndband.

Henni vantar nú að fá forrit sem segir henni bestu leiðina
til að horfa á myndböndin svo hún nái sem allra mesta 
dópamíninu.

Getur þú hjálpað henni?

Fyrir hvert myndband þarf annað hvort að horfa á það
eða nota $k$ millisekúndur til að sleppa því, nema tíminn
renni út í miðjum klíðum.

\section*{Inntak}
Fyrsta lína inntaksins inniheldur tvær heiltölur
$n, k$, fjöldi myndbanda og fjöldi millisekúndna sem það
tekur að sleppa myndbandi.
Ávallt gildir að $1 \leq n \leq 1\,000$ og $0 \leq k \leq 10^9$.
Næst fylgja $n$ línur, $i$-ta þeirra lýsir $i$-ta myndbandinu
í röðinni.

Á $i$-tu línu eru tvær heiltölur $L_i, D_i$, lengd
myndbandsins í millisekúndum og fjöldi eininga af
dópamíni sem fæst fyrir að horfa á það allt.
Ávallt gildir að $0 \leq L_i \leq 100\,000$ og $0 \leq D_i \leq 10^9$.

Loks fylgir ein lína með heiltölu $T$, heildarfjöldi
millisekúndna sem Jarmína hefur. Ávallt gildir að
$0 \leq T \leq 10^9$. Látum $S$ tákna heildarlengd
allra myndbanda. Ávallt gildir að $0 \leq S \leq 100\,000$.
Myndböndin eru gefin í þeirri röð
sem þau birtast á síma Jarmínar.

\section*{Úttak}
Skrifaðu hámarksfjölda eininga af dópamíni sem Jarmína
getur náð á $T$ millisekúndum.

\section*{Stigagjöf}
\begin{tabular}{|l|l|l|}
\hline
Hópur & Stig & Takmarkanir \\ \hline
1     & 10   & $n = 1$. \\ \hline
2     & 20   & $n \leq 20$. \\ \hline
3     & 35   & $k = 0$. \\ \hline
4     & 35   & Engar frekari takmarkanir. \\ \hline
\end{tabular}

