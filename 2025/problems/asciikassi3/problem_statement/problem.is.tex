\problemname{ASCII kassi 3}
\illustration{0.4}{chair}{Mynd fengin af \href{https://commons.wikimedia.org/wiki/File:Chair-ascii.png}{wikimedia.commons.org}}

Önnur forritunarkeppni, annað dæmi um ASCII kassa! 
Hingað til hefur öllum ASCII kössunum vantað eitthvað.
Listaspírurnar í keppninni kvörtuðu yfir að það vantaði dýpt í þetta, svo í ár
þarf að bæta við þriðju víddina í myndirnar.

Til að teikna kassann skal nota táknin \texttt{+}, \texttt{-}, \texttt{|},
\texttt{/} og \texttt{x}.
Lóðréttar brúnir eru teiknaðar með \texttt{|} og láréttar brúnir með \texttt{-}.
Hornpunktar kassans eru teiknaðir með \texttt{+} og \texttt{/} er fyrir brúnir sem
liggja burt frá sjónarmiði okkar og loks er \texttt{x} notað ef tvær
ekki samsíða brúnir kassans skarast á myndinni. Ef brún og hornpunktur skarast skal áfram
tákna það með \texttt{+}.

Til að kassinn birtist rétt þarf að passa að setja réttan fjölda bila á undan og 
milli stafanna í hverri línu.
Þar að auki má ekki prenta nein auka bil á eftir kassanum í hverri línu, heldur á að 
koma nýlínustafur beint á eftir seinasta tákni kassans í hverri línu.

Kassinn hefur einhverja tiltekna hæð $h \geq 1$, breidd $b \geq 1$ og dýpt $d \geq 1$.
Lóðrétta brúnin eru þá $2$ stykki \texttt{+} og $(h - 2)$ stykki \texttt{|}, nema ef
$h = 1$ þá er brúnin aðeins eitt \texttt{+}.
Eins er lárétta brúnin $2$ stykki \texttt{+} og $(b - 2)$ stykki \texttt{-} og eins ef
$b = 1$ er brúnin í staðinn aðeins eitt \texttt{+}.

Þetta myndar þá fremri hlið kassans, sem er $h \times b$ rétthyrningur. Ef $d > 1$ þarf
næst að teikna $(d - 2)$ stykki $\texttt{/}$ sem liggja upp og til hægri frá öllum
hornpunktum fremri hliðarinnar. Svo þarf að teikna aftari hlið kassans með sama hætti
og fremri hlið við endann á $\texttt{/}$ rununni.

\section*{Inntak}
Fyrsta og eina lína inntaksins inniheldur þrjár heiltölur $h, b, d$, hliðarlengdir
kassans eins og lýst er að ofan.

\section*{Úttak}
Prentið kassa með gefnu hliðarlengdunum, eins og lýst er að ofan.
Hafið í huga að úttakið þarf að vera nákvæmlega rétt, meira að segja bilstafirnir.

\section*{Stigagjöf}
\begin{tabular}{|l|l|l|}
\hline
Hópur & Stig & Takmarkanir \\ \hline
1     & 20   & $1 \leq h, b, d \leq 3$. \\ \hline
2     & 30   & $2 \leq h, b, d \leq 8$. \\ \hline
3     & 20   & $d = 1, 1 \leq h, b \leq 100$. \\ \hline
4     & 30   & $1 \leq h, b, d \leq 100$. \\ \hline
\end{tabular}
