\problemname{Heimanámsálag}
\illustration{0.4}{homework}{Mynd fengin af \href{https://www.publicdomainpictures.net/en/view-image.php?image=132819\&picture=science-homework}{publicdomainpictures.net}}

Fyrir einhverjum árum síðan lét Rektor Menntaskólans á Akureyri alræmd orð falla. Hann hélt því
fram að ef nemendur væri einfaldlega niður í skóla klukkutíma eftir kennslu á hverjum degi þá
væri ekkert heimanám eftir þegar þau færu heim og að kvartanir um heimanámsálag væru því ekki
neitt til að hlusta á.

Nemendur sem hafa verið í, eða eru í, Menntaskólanum á Akureyri (eða flest öðrum framhaldsskólum
landsins) vita að þetta er tóm þvæla. Allir sem hafa þurft láta sig hafa Íslandsáfanga
Menntaskólans á Akureyri þekkja það sérstaklega vel. Til að sýna hversu mikil þvæla þetta er þá
er búið að safna gögnum um allt heimanám sem er sett fyrir og ætlum við að rannsaka hversu mikið
þarf að vinna á dag til að klára allt saman.

Íslendingasögur, atómskáld, læra nöfnin á fossum og hvalategundum Íslands, og eins gott að þú
leggir á minnið uppáhaldssundlaug kennarans til að ná bónusspurningunni! Hrein vitleysa, ég gæti
haldið áfram í allan dag! Það ætti að leggja þetta niður með einu og öllu-

Ha? Nei ég er að skrifa dæmalýsinguna ennþá. Hei! Nei- ég er ekki búinn! Láttu lyklaborðið í frið-

Afsakið þetta.

Hvert verkefni er sett fyrir á einhverjum degi og þarf að ljúka í síðasta lagi á einhverjum degi.
Þar að auki tekur hvert verkefni einhvern tiltekinn fjölda tímaeininga að klára. Ef þú ætlar alltaf að
vinna $x$ tímaeiningar á dag (nema þú sért búin/n með allt heimanám), hvað þarf $x$ að vera
stórt í minnsta lagi til að klára öll verkefni á ásettum tíma?

\section*{Inntak}
Fyrsta lína inntaksins inniheldur eina heiltölu $n$, fjölda verkefna.
Ávallt gildir $0 \leq n \leq 100\,000$.
Næst fylgja $n$ línur þar sem $i$-ta þeirra lýsir $i$-ta verkefninu.

Hver slík lína inniheldur þrjár heiltölur $a_i, b_i, t_i$ þar sem $a_i$
er á hvaða degi það er sett fyrir, $b_i$ er á hvaða degi á að skila, og $t_i$ er hvað verkefnið
tekur margar tímaeiningar.
Athugið að hægt er að vinna í verkefni og skila því í lok dags.
Ávallt gildir að $0 \leq a_i, b_i, t_i \leq 10^9$ og $a_i \leq b_i$.

Athugið að tölurnar í dæminu eru stórar og útreikningar passa mögulega ekki í $32$ bita.

\section*{Úttak}
Prentaðu hvað þú þarft að vinna margar tímaeiningar á dag til að klára öll verkefni
á ásettum tíma.

\section*{Stigagjöf}
\begin{tabular}{|l|l|l|}
\hline
Hópur & Stig & Takmarkanir \\ \hline
1     & 15   & $n = 1$. \\ \hline
2     & 20   & $a_i = b_i$ fyrir öll $i$. \\ \hline
3     & 25   & $n \leq 100$ og $t_i \leq 100$ fyrir öll $i$. \\ \hline
4     & 40   & Engar frekari takmarkanir. \\ \hline
\end{tabular}

