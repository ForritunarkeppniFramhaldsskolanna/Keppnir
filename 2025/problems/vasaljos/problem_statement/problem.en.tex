\problemname{Electric Torch}
\illustration{0.5}{flashlights}{Image from \href{https://xkcd.com/1603/}{xkcd.com}}

Hrolleifs' power generators from last year are on the fritz, so he is trying to find his
electric torch so he can get things fixed. He opens a kitchen drawer and finds his
collection of batteries. He is not very principled when it comes to sorting things, so
some of the batteries are dead. In fact he knows that exactly half of the batteries are
dead and the other half is good. The electric torch needs two good batteries to work, and he
wants to spend as little time as possible trying pairs of batteries. What's the best
way to go about this?

\section*{Interactivity}
This is an interactive problem. Your solution will be tested against an interactive
judge which reads the output of your solution and prints the input your solution
receives. This interaction follows certain rules:

First your program should read a positive integer $n$ on its own line, where $n$ is
the number of batteries in the drawer.
You may assume that $4 \leq n \leq 50$ and $n$ is an even number.

If your program wants to try battery number $i$ and battery number $j$, it should
print the numbers $i$ and $j$ on their own line, separated by a space. The same battery
can not be inserted into two different slots at the same time, so $i \neq j$ must hold.

Your program should then read a string from the input, given on its own line.
That string will be \texttt{Myrkur!} (Icelandic for dark) if one or both of the batteries
are dead, or \texttt{Ljos!} (Icelandic for light) if the electric torch turns on.
If the string was \texttt{Ljos!} your program should exit.

Make sure to \texttt{flush} after each guess, for example using
\begin{itemize}
    \item \verb!print(..., flush=True)! in Python,
    \item \verb!cout << ... << endl;! in C++,
    \item \verb!System.out.flush();! in Java.
\end{itemize}

The sample shows an example with $n = 6$.

The task has a testing tool attached to help test your solution.

Note that the judge program \textbf{will} produce the worst possible case for your solution.

\section*{Scoring}
Your program will be run on many test cases and the worst result over all 
test cases will be used as the final score.
Your program is scored by the number of guesses it makes.
If your program guesses at most $4n^2$ times, it will receive a score.
Fewer guesses result in a higher score and the maximum score is $100$.
If your program guesses more than $4n^2$ times, it will receive a Wrong Answer verdict.
