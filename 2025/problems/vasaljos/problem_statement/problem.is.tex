\problemname{Vasaljós}
\illustration{0.5}{flashlights}{Mynd fengin af \href{https://xkcd.com/1603/}{xkcd.com}}

Aflgjafarnir hans Hrolleifs frá því í fyrra eru bilaðir, svo hann er að reyna finna sér
rafhlöðuvasaljós til að geta bjargað hlutunum. Hann opnar eldhússkúffu og finnur þar safn
sitt af rafhlöðum. Hann er hins vegar ekki mjög duglegur að flokka frá notaðar rafhlöður, svo sumar rafhlöðurnar eru
dauðar. Nánar tiltekið veit hann að helmingur rafhlaðanna er án hleðslu, en hinn helmingurinn er
enn góður. Vasaljósið þarf tvær góðar rafhlöður til að virka, og hann langar að eyða sem minnstum
tíma í að prófa rafhlöður. Hvernig er best að fara að þessu?

\section*{Gagnvirkni}
Þetta er gagnvirkt verkefni. Lausnin þín verður keyrð á móti gagnvirkum dómara
sem les úttakið frá lausninni þinni og skrifar í inntakið á lausninni þinni.
Þessi gagnvirkni fylgir ákveðnum reglum:

Fyrst les forritið þitt jákvæða heiltölu $n$ á einni línu, þar sem $n$ er fjöldi rafhlaða í skúffunni.
Gefið er að $4 \leq n \leq 50$ og að $n$ sé slétt tala.

Svo ef forritið þitt vill prófa rafhlöðu númer $i$ og rafhlöðu númer $j$ þarf hún einfaldlega
að prenta $i$ og $j$ á einni línu, aðskilin með bili. Ekki er hægt að setja sömu rafhlöðu í
tvö hólf, svo það þarf að hafa $i \neq j$.

Svo les forrit þitt streng úr inntakinu á sinni eigin línu. Sá strengur er
\texttt{Myrkur!} ef annað hvort eða bæði
rafhlaðanna er dautt eða \texttt{Ljos!} ef það kviknar á vasaljósinu.
Ef strengurinn var \texttt{Ljos!} á forrit þitt að hætta keyrslu.

Vertu viss um að gera \texttt{flush} eftir hvert gisk, t.d., með
\begin{itemize}
    \item \verb!print(..., flush=True)! í Python,
    \item \verb!cout << ... << endl;! í C++,
    \item \verb!System.out.flush();! í Java.
\end{itemize}

Sýniinntakið sýnir dæmi með $n = 6$.

Með verkefninu fylgir tól sem viðhengi til þess að hjálpa við að prófa lausnina þína.

Athugaðu að yfirferðarforritið \textbf{mun} athuga hvort lausnin þín standist versta mögulega tilfelli fyrir þína lausn.

\section*{Stigagjöf}
Lausnin þín verður keyrð á mörgum prufutilvikum og versta niðurstaða yfir
öll prufutilvik mun gilda til stigagjafar.
Lausnin þín fær stig út frá fjölda giska.
Ef lausnin giskar mest $4*n^2$ sinnum fær hún stig.
Færri gisk gefa fleiri stig og mest er hægt að fá $100$ stig.
Ef lausnin þín giskar oftar en $4*n^2$ sinnum verður hún dæmd röng.
