\problemname{Andmál}
\illustration{0.4}{argument}{Mynd eftir Randall Munroe, fengin af \href{https://xkcd.com/1432/}{xkcd.com}}

Það að semja dæmi fyrir Forritunarkeppni Framhaldsskólanna tekur langan tíma.
Hluti af ástæðunni er einmitt hvað dómararnir eru duglegir að rífast og vera
ósammála hverjum öðrum á fundum. Verstur í þessum málum er Atli sem virðist
alltaf vera ósammála öllu, einfaldlega til þess að vera ósammála. Það er
alveg sama hvað er sagt, hann segir ávallt eitthvað annað.

Til að spara tíma gætir þú kannski búið til gervigreind sem sér um þetta sjálfskipaða
hlutverk Atla. Eina sem hún þarf að gera er að segja eitthvað annað en
inntakið!

\section*{Inntak}
Inntak er á einni línu. 
Inntakið inniheldur allt að 10 stafi ásamt nýlínustaf á endanum.
Þessir stafir geta verið allir bókstafir og tölustafir í ASCII, ásamt bilum.
Þetta eru þá allir enskir stafir \texttt{a} til \texttt{z}, allir enskir
stafir \texttt{A} til \texttt{Z}, tölustafirnir \texttt{0} til \texttt{9}
og bil.

Athugaðu að inntakið getur verið tóm lína!

\section*{Úttak}
Prentaðu eitthvað annað en inntakið. Úttakið má vera mest $10$ stafir
og þarf að uppfylla sömu skorður og er lýst hér að ofan fyrir inntakið.
Einnig þarf úttakið að enda á nýlínustaf.

\section*{Stigagjöf}
\begin{tabular}{|l|l|l|}
\hline
Hópur & Stig & Takmarkanir \\ \hline
1     & 100   & Engar frekari takmarkanir. \\ \hline
\end{tabular}

