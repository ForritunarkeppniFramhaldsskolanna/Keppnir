\problemname{Sovét Union-Find}
\illustration{0.4}{space}{Mynd eftir \href{https://www.vezanmatics.com}{Evan Streb, Vezanmatics}. Notuð með leyfi.}

Ingfríður sat og var að leiðast í sögutíma. Kennarinn var
að tala um sovétríkin og kalda stríðið. Hún var hins vegar
of upptekin að hugsa um forritunarkeppnir og dæmi sem
hún vildi leysa. Hún einbeitti sér því ekkert að því sem
var í gangi.

Hins vegar í lok tímans sagði kennarinn að vinna ætti frjálst
skapandi verkefni tengt þessu tímabili. Hún hugsaði vel og
lengi hvað væri hægt að gera tengt þessu, en var ennþá
föst að hugsa um keppnisforritun. Að lokum small það hjá
henni, Soviet Union, það hlýtur að tengjast \emph{Union-Find} gagnagrindinni.
Soviet Union er náttúrulega enska heitið yfir
Sambandsríki sósíalískra sovétlýðvelda, sem kennarinn var að tala um.
Því getur hún gert verkefni því tengt. Hins vegar er nú
komið alveg fram að skilafrest og hún gleymdi að vinna í
þessu. Getur þú útfært \emph{Sovét Union-Find} til að bjarga
henni?

\emph{Sovét Union-Find} þarf að geta stutt nokkrar aðgerðir.
Í byrjun skiptum við landsvæði heimsins niður í $n$ búta og
númerum þá búta $1, 2, \dots, n$. Í byrjun er hver bútur
sitt eigið sjálfstæða ríki og er sá bútur stjórnandi þess ríkis.
En svo geta komið nokkrar aðgerðir í inntakinu, sú fyrsta
er \texttt{a x y} sem þýðir að ríkið sem inniheldur bút
$x$ tekur yfir ríkið sem inniheldur bút $y$. Stjórnandi
nýja sameinaða ríkisins er þá fyrrum stjórnandi $x$.
Næsta aðgerð er \texttt{b x} sem þýðir að ríkið sem inniheldur
bút $x$ \textit{balkaniserast}. Það þýðir að allir bútar þess
skiptast upp og verða sitt eigið ríki aftur, eins og í
byrjun. Loks er aðgerðin \texttt{c x} sem spyr hver ræður
yfir bút númer $x$ að svo stöddu.


\section*{Inntak}
Fyrsta lína inntaksins inniheldur tvær jákvæðar heiltölur
$n, q$ þar sem $n$ er fjöldi búta sem landsvæði heimsins er
skipt í og $q$ er fjöldi aðgerða. 
Næst fylgja $q$ línur, hver með einni aðgerð.
Fyrsti stafur línunnar er þá ávallt \texttt{a}, \texttt{b}
eða \texttt{c} eins og lýst er að ofan og gefur sá stafur
tegund aðgerðarinnar.

Ef stafurinn var \texttt{a} koma næst tvær jákvæðar heiltölur
$x$ og $y$ sem uppfylla $1 \leq x, y \leq n$. Ef $x$ og $y$
eru þegar hluti af sama ríki gerir aðgerðin ekkert.
Framkvæma á þá sameiningaraðgerðina á $x$ og $y$ eins
og lýst er að ofan.

Ef stafurinn var \texttt{b} kemur næst ein jákvæð heiltala
$x$ sem uppfyllir $1 \leq x \leq n$. Framkvæma á þá
balkaniseringsaðgerðina á $x$ eins og lýst er að ofan.

Loks ef stafurinn var \texttt{c} kemur næst ein jákvæð
heiltala $x$ sem uppfyllir $1 \leq x \leq n$. 
Ávallt gildir að $n, q \leq 100\,000$.

\section*{Úttak}
Fyrir hverja \texttt{c} aðgerð í inntakinu skal prenta
númer bútsins sem ræður yfir bútnum sem er gefinn í
aðgerðinni, eins og lýst er að ofan.
Prenta skal hverja tölu á eigin línu og í sömu röð og 
aðgerðirnar eru gefnar í inntakinu.

\section*{Stigagjöf}
\begin{tabular}{|l|l|l|}
\hline
Hópur & Stig & Takmarkanir \\ \hline
1     & 25   & $n = 2, q \leq 100$ \\ \hline
2     & 25   & $n, q \leq 1\,000$ \\ \hline
3     & 25   & Það eru engar \texttt{b} aðgerðir í inntaki. \\ \hline
4     & 25   & Engar frekari takmarkanir. \\ \hline
\end{tabular}

