\problemname{Soviet Union-Find}
\illustration{0.4}{space}{Image by \href{https://www.vezanmatics.com}{Evan Streb, Vezanmatics}. Used with permission.}

Ingfríður was sitting and staring into open air, bored out
of her mind in history class. The teacher was droning on
about the Soviet Union and the cold war. She was however
too busy thinking about programming contests and problems
she wanted to solve. She was thus not at all focusing on
what was going on in class

And the end of class the teacher said the students should
work on a creative assignment related to this piece of 
history. She thought long and hard what she could do that
is related to this, but was still stuck thinking about 
competitive programming. Finally it clicked in her head 
though. Soviet Union? This must be related to the \emph{Union-Find} data structure!
Thus she will do an assignment about this. Unfortunately
she forgot about this idea for quite a while and now the
deadline is upon her! Can you save her by finishing
the assignment in time?

\emph{Soviet Union-Find} has to be a program that supports a few
queries. At the start we divide the land area of the
world into $n$ pieces, numbering them $1, 2, \dots, n$.
At the start each piece is independent and rules their
own area. But then the queries can change this, the first
type of query being \texttt{a x y}. This means that the
ruler of area $x$ annexes everything the ruler of area
$y$ rules over. The ruler of the new area is the ruler of
area $x$. The second type of query is \texttt{b x} which
means that everything the ruler of area $x$ rules over
gets \textit{balkanised}. This means that all the areas
split up and become independent again, ruling only their
own area once more, like at the beginning. Finally there
is the query \texttt{c x} which asks who currently rules
over area $x$.

\section*{Input}
The first line of input contains two positive integers
$n, q$ where $n$ is the number of areas the world is split
into at the start and $q$ is the number of queries to follow.
Next there are $q$ lines, each with one query.
The first character of each such line is then always one
of \texttt{a}, \texttt{b} or \texttt{c} as described above.
This letter then gives the type of the query.

If the first letter is \texttt{a} there will be two positive
integers $x$ and $y$ on the line that satisfy $1 \leq x, y
\leq n$. If $x$ and $y$ are already have the same ruler this
query does nothing. This means the annexation operation
described above should be performed on areas $x$ and $y$.

If the first letter is \texttt{b} then there will be one
positive integer $x$ on the line satisfying $1 \leq x \leq n$.
Then the balkanisation operation should be performed on the
area the ruler of area $x$ rules over, as described above.k

Finally if the first letter was \texttt{c} there will follow
one positive integer $x$ satisfying $1 \leq x \leq n$.
It will always hold that $n, q \leq 100\,000$.

\section*{Output}
For each \texttt{c} query in the input, print the number
of the ruler of the area given in that query, as described
above. Print each number on its own line and in the same
order as the queries are given in the input.

\section*{Scoring}
\begin{tabular}{|l|l|l|}
\hline
Group & Points & Constraints \\ \hline
1     & 25   & $n = 2, q \leq 100$ \\ \hline
2     & 25   & $n, q \leq 1\,000$ \\ \hline
3     & 25   & There are no \texttt{b} queries in the input. \\ \hline
4     & 25   & No further constraints. \\ \hline
\end{tabular}

