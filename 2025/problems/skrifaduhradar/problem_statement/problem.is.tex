\problemname{Skrifaðu hraðar}
\illustration{0.4}{top20}{Mynd fengin af \href{https://2024.nwerc.eu/main/scoreboard/}{2024.nwerc.eu}}

Í forritunarkeppnum getur oft hjálpað að skrifa hratt,
sérstaklega þegar tímarefsing vegur þungt. Þú veltir nú fyrir
þér hvað það hefði þurft til að sigra fyrri forritunarkeppni.

Þú ert mjög bjartsýn/n svo þú gerir ráð fyrir að þú munir
leysa öll dæmi í fyrstu tilraun, svo þú færð bara tímarefsingu
fyrir fyrstu skil. Besta lið keppninnar leysti öll dæmin,
svo til að vinna þarftu minni tímarefsingu en það lið.
Gerum einnig ráð fyrir að þú hafir
eins mikinn tíma og þú vilt til að leysa keppnina, utan
við það að þurfa ná lægri tímarefsingu. Það dugar ekki
að fá sömu tímarefsingu.

Ef þú skilar lausn á verkefni á mínútu $x$ bætist $x$
við tímarefsingu þína. Það þýðir að lausn sem er skilað eftir
$1$ mínútu og $59$ sekúndur fær tímarefsingu upp á $1$, en
skil eftir $2$ mínútur fær tímarefsingu upp á $2$.

Því ef þú getur skrifað $20$ orð á mínútu og lausn tekur
$100$ orð og þú gerir hana fyrst mun hún gefa tímarefsingu
upp á $5$, en ef hún væri $99$ orð fengirðu tímarefsingu upp
á $4$.

Þú ert búin/n að sjá út lausn á öllum verkefnunum, og veist
hvað lausn þín yrði löng, en veist ekki hvort þú þyrftir
kannski að þjálfa skrifhraða fyrst til að geta unnið.
Því er spurningin, hvað þarftu að geta skrifað mörg orð
á mínútu til að geta unnið?

\section*{Inntak}
Fyrsta lína inntaksins gefur tvær heiltölur $n, T$. $n$
er fjöldi dæma í keppninni sem þú þarft að leysa og $T$
er tímarefsingin sem þú þarft að ná undir. Ávallt gildir
að $1 \leq n \leq 100\,000$ og $1 \leq T \leq 10^{18}$.
Önnur og síðasta lína inntaksins inniheldur $n$ heiltölur
$w_1, w_2, \dots, w_n$. $w_i$ er fjöldi orða sem lausn
$i$-ta dæmisins samanstendur af. Ávallt gildir að 
$0 \leq w_i \leq 10^9$ fyrir öll $i$.
Summa allra $w_i$ verður ekki $0$.

\section*{Úttak}
Skrifaðu út minnsta fjölda orða á mínútu sem þú þarft að
geta skrifað til að vinna.

\section*{Stigagjöf}
\begin{tabular}{|l|l|l|}
\hline
Hópur & Stig & Takmarkanir \\ \hline
1     & 10   & $n = 1$. \\ \hline
2     & 15   & $T = 1$. \\ \hline
3     & 25   & $n \leq 100, \sum_{i=0}^n w_i \leq 1\,000$ \\ \hline
4     & 50   & Engar frekari takmarkanir. \\ \hline
\end{tabular}

