\problemname{Marathon}
\illustration{0.4}{toulouse}{Image taken from \href{https://commons.wikimedia.org/wiki/File:Marathon\_de\_Toulouse\_2014\_-\_3148\_-\_Water\_stop.jpg}{commons.wikimedia.org}}

Konni is just about to run a marathon for the first time. He's well prepared for it and has all 
the equipment he needs to achieve a good time. Among the things he has prepared is a water pouch
that will make sure he can drink constantly throughout the run. He wants to achieve the best time
possible, and thus he needs some help planning on which refilling station he should stop to fill
his water pouch.

It takes $Y$ seconds to fill his water pouch no matter the amount that is already in
the pouch, and the pouch can hold at most $X$ ml of liquid. Konni always remembers to fill
his bag before he starts a marathon and for every metre that he runs, he will drink $1$ml of
liquid. If Konni has no liquid left, he will need to jog until he gets more water.

Can you help Konni in finding the best time he can run the marathon if he chooses to stop
at the refilling stations in the most optimal way?

As everyone knows a marathon is $42\,195$ metres.

\section*{Input}
The first line of input contains three integers, $0 \leq N \leq 1\,000\,000$ the amount of
refilling stations, $0 \leq X \leq 10\,000$ the size of the water pouch in ml,
$0 \leq Y \leq 100$ how long it takes to fill the water pouch.

The next line contains two integers $1 \leq H \leq 10$, the speed Konni runs at in metres
per second and $1 \leq S \leq H$, the speed Konni jogs at in metres per second.

Next will follow $N$ lines, each containing one integer, where line $i$ contains the
position of refilling station number $i$ given in metres from the start.

\section*{Output}
Print the shortest time possible for Konni given that he chooses the refilling stations
in the most optimal way in the format of \href{https://en.wikipedia.org/wiki/ISO\_8601}{$hh:mm:ss$}.

\section*{Scoring}
\begin{tabular}{|l|l|l|}
\hline
Groups & Points & Constraints \\ \hline
1 & 20 & $H=S$ \\ \hline
2 & 30 & $N \leq 20$ \\ \hline
3 & 30 & $N \leq 2\,500, X \leq 1\,000$ \\ \hline
3 & 50 & No further constraints. \\ \hline
\end{tabular}

