\problemname{Spilaröðun}
\illustration{0.4}{impact}{Mynd fengin af \href{https://commons.wikimedia.org/wiki/File:Japan\_Impact\_2023\_-\_Yu-Gi-Oh\_(52765156950).jpg}{commons.wikimedia.org}}

Atli á alveg hrottalegt magn af spilum og lendir stundum
í vandræðum með að halda þeim skipulögðum. Því hugsaði
hann að það gæti verið gott að búa til rafrænan
gagnagrunn fyrir spilin sín.

Gagnagrunnurinn þarf að bjóða upp á að raða á nokkra ólíka
vegu. Til þess að tala um þær leiðir skulum við fyrst fara
yfir hvaða upplýsingar eru á einu spili.

Hvert spil hefur nafn, til dæmis ,,Bláeygður hvítur dreki`` % chktex 26 chktex 33
eða ,,Dimmur Seiðkarl``. Nöfn innihalda ávallt bara enska % chktex 26
stafi og bil, og eru mest $32$ stafir. Nafn mun hvorki
byrja né enda á bili og eru ekki tóm.
Einnig hefur hvert spil átta stafa
ID tölu, til dæmis 55144522. Engin tvö ólík spil hafa sama ID\@.
Einnig hefur hvert spil flokk og mögulega undirflokk.
Við gefum flokkana og undirflokkana í þeirri röð sem á að
raða þeim. Þeir eru

\begin{itemize}
    \item \texttt{Skrimsli}
    \begin{itemize}
        \item \texttt{Venjulegt}
        \item \texttt{Ahrifa}
        \item \texttt{Bodunar}
        \item \texttt{Samruna}
        \item \texttt{Samstillt}
        \item \texttt{Thaeo}
        \item \texttt{Penduls}
        \item \texttt{Tengis}
    \end{itemize}
    \item \texttt{Galdur}
    \begin{itemize}
        \item \texttt{Venjulegur}
        \item \texttt{Bunadar}
        \item \texttt{Svida}
        \item \texttt{Samfelldur}
        \item \texttt{Bodunar}
        \item \texttt{Hradur}
    \end{itemize}
    \item \texttt{Gildra}
    \begin{itemize}
        \item \texttt{Venjuleg}
        \item \texttt{Samfelld}
        \item \texttt{Mot}
    \end{itemize}
    \item \texttt{Annad}
\end{itemize}

Þetta þýðir að \texttt{Skrimsli} hefur undirflokkinn \texttt{Thaeo}, sem kemur á undan undirflokknum \texttt{Tengis} og svo framvegis. Einnig er \texttt{Gildra} á eftir \texttt{Galdur} sama hvaða undirflokk er að ræða. \texttt{Annad} hefur enga undirflokka.

Loks hefur hvert spil útgáfudagsetningu, gefið á forminu
\texttt{yyyy-mm-dd} sem gefur ár, mánuð og dag á
\texttt{ISO-8601} sniði.

\section*{Inntak}
Inntak byrjar á línu með heiltölu $1 \leq n \leq 1\,000$.
Svo fylgja $n$ línur, hver með einu spili. Á þeirri línu
verður nafn, ID, flokkur og útgáfudagsetning spilsins gefin,
aðskilin með kommum. Ef flokkurinn hefur undirflokk er það
gefið á forminu \texttt{flokkur {-} undirflokkur}.

Loks kemur lína með orðunum \texttt{nafn}, \texttt{id},
\texttt{flokkur} og \texttt{dagsetning} aðskilin með bilum
í einhverri röð. Raða á spilunum eftir því sem kemur
fyrst, leysa jafntefli með því sem kemur næst og koll af
kolli.

Nöfn eru röðuð í stafrófsröð, ID í stærðarröð með minnstu
fremst, flokkar í röðinni að ofan og dagsetningar í tímaröð
með elsta fremst. Stafrófsröðin er eftir ASCII gildi sem
þýðir að bil kemur fremst, svo stórir stafir, svo litlir.
Flest forritunarmál raða strengjum svona.

\section*{Úttak}
Eftir að búið er að raða spilunum í rétta röð, prentið
nafnið á hverju spili í þeirri röð. Prentið eitt spil
á hverja línu.

\section*{Stigagjöf}
\begin{tabular}{|l|l|l|}
\hline
Hópur & Stig & Takmarkanir \\ \hline
1     & 5   & $n = 1$. \\ \hline
2     & 10   & Raða á eftir nafni fyrst. \\ \hline
3     & 15   & Raða á eftir ID fyrst. \\ \hline
4     & 20   & Raða á eftir flokki fyrst og svo ID\@. \\ \hline
5     & 20   & Raða á eftir dagsetningu fyrst, svo flokki og svo nafni. \\ \hline
6     & 30   & Engar frekari takmarkanir. \\ \hline
\end{tabular}

