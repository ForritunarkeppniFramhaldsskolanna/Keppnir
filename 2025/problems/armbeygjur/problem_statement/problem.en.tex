\problemname{Press-ups}
\illustration{0.4}{pushup}{Image taken from \href{https://www.publicdomainpictures.net/pictures/380000/velka/push-up-excercise-1610372643pLR.jpg}{publicdomainpictures.net}}

Hrefna started a strict exercise regiment some time ago to try to improve how many
push-ups she can do. Every day she does one more push-up than the day before, no matter
what it takes.

You see that this has netted good results for her and are considering whether it would
be a good idea to try this. But you realise you do not know how many push-ups she did the
first day. When asked she says she does not remember how many push-ups she started with
or for how many days she has been doing the regiment. But she has been meticulously 
counting the push-ups she does, so she can tell you she has done $n$ push-ups in total.

\section*{Input}
The input contains a single positive integer $n$, the total number of push-ups Hrefna
has done.

\section*{Output}
For each possible number of push-ups Hrefna could have started with on the first day,
print one line. That line should contain two integers $f$ and $d$ separated by a space.
This means that Hrefna would have finished $n$ push-ups if she did $f$ push-ups on the
first day and has been doing the regiment for $d$ days in total. $f, d \geq 1$ should
always hold.

Print the lines in ascending order by $f$.

Hrefna never makes mistakes when it comes to something as important as counting
push-ups, so you may assume that there is at least one line in the output.

\section*{Scoring}
\begin{tabular}{|l|l|l|}
\hline
Group & Score & Constraints \\ \hline
1     & 20   & $n \leq 20$ \\ \hline
2     & 20   & $n \leq 1\,000$ \\ \hline
3     & 20   & $n \leq 1\,000\,000$ \\ \hline
4     & 20   & $n \leq 10^{12}$ \\ \hline
5     & 20   & $n \leq 10^{18}$ \\ \hline
\end{tabular}

