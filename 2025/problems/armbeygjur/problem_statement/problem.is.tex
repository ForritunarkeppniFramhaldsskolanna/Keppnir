\problemname{Armbeygjur}
\illustration{0.4}{pushup}{Mynd fengin af \href{https://www.publicdomainpictures.net/pictures/380000/velka/push-up-excercise-1610372643pLR.jpg}{publicdomainpictures.net}}

Hrefna byrjaði fyrir nokkru stranga æfingaráætlun til að reyna að bæta hversu margar armbeygjur hún getur gert.
Á hverjum degi gerir hún eina fleiri armbeygju en daginn áður, sama hvað það tekur mikið á. 

Þú sérð að þetta hefur skilað góðum árangri hjá henni og ert að íhuga hvort það
væri gáfulegt að prófa þetta. En þú veist ekki hvað hún byrjaði á að gera margar
armbeygjur fyrsta daginn. Aðspurð segist hún því miður ekki muna hvenær hún byrjaði nákvæmlega
eða hversu margar armbeygjur hún byrjaði á að gera. Hins vegar er hún
búin að telja allar armbeygjur samviskusamlega og getur því sagt þér að hún sé
búin að gera $n$ armbeygjur samtals.

\section*{Inntak}
Inntak inniheldur eina jákvæða heiltölu $n$, heildarfjöldi armbeygja sem Hrefna
er búin að klára.

\section*{Úttak}
Fyrir hvern fjölda armbeygja sem Hrefna hefði getað byrjað á skal prenta eina línu.
Á þá línu eiga að koma fram tvær tölur $f$ og $d$ aðskilin með bili. Þetta merkir að
Hrefna gæti verið búin með $n$ armbeygjur ef hún gerið $f$ armbeygjur fyrsta daginn
og sé búin að vera að í $d$ daga samtals. Ávallt gildir að $f, d \geq 1$.

Prenta skal línur svo gildin $f$ séu í vaxandi röð.

Hrefna gerir aldrei mistök þegar það kemur að einhverju jafn mikilvægu og að
telja armbeygjur, svo þú mátt gera ráð fyrir að það sé til að minnsta kosti eitt
mögulegt úttak.

\section*{Stigagjöf}
\begin{tabular}{|l|l|l|}
\hline
Hópur & Stig & Takmarkanir \\ \hline
1     & 20   & $n \leq 20$ \\ \hline
2     & 20   & $n \leq 1\,000$ \\ \hline
3     & 20   & $n \leq 1\,000\,000$ \\ \hline
4     & 20   & $n \leq 10^{12}$ \\ \hline
5     & 20   & $n \leq 10^{18}$ \\ \hline
\end{tabular}

