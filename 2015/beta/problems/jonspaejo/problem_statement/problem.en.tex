\problemname{Jón Spæjó}

\illustration{.4}{homer}{}

Helgina fyrir forritunarkeppni framhaldsskólanna var Jón niðri í Háskólanum í
Reykjavík. Hann hafði heyrt að verkefnin sem verða notuð í keppninni séu geymd
í læstum peningaskáp í skólanum. Og viti menn, þarna sér Jón einhvern labba með
bunka af forritunarverkefnum að peningaskáp, slá inn kóða, og setja verkefnin
inn í hann. Því miður náði Jón ekki að sjá verkefnin, en hann náði aftur á móti
að sjá hluta af kóðanum sem sleginn var inn á peningaskápinn.

Núna ætlar Jón að sjá hvort hann geti ekki opnað peningaskápinn með þeim
upplýsingum sem hann hefur. Hann lætur þig fá þann hluta af kóðanum sem hann
sá, og setur spurningamerki á þá staði sem hann sá ekki hvaða stafur var
sleginn inn. Hann veit þó að hver stafur er tala á milli $0$ og $9$. Hann biður
þig um að skrifa út alla kóða sem gætu hugsanlega verið rétti kóðinn, miðað við
það sem Jón sá, í stafrófsröð.

\section*{Inntak}
Ein lína með kóðanum sem Jón sá. Kóðinn inniheldur tölustafi ásamt
spurningamerkjum, þar sem spurningamerki táknar að Jón veit ekki hvaða
tölustafur er á þeim stað. Kóðinn er aldrei lengri en $100$ stafir.

\section*{Úttak}
Forritið á að skrifa út alla mögulega kóða sem passa. Ef það eru
\textbf{fleiri} en $10^4$ möguleikar, skrifið þá út \texttt{Fjoldi: $x$} í
staðin, þar sem $x$ er fjöldi möguleika.

