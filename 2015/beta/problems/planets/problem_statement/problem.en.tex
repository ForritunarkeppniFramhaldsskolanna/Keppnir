\problemname{Plánetur}

\illustration{.4}{planet}{}

Bræðurnur Pí, Kósínus og Sínus hafa nýverið keypt hlut í SpaceX eftir fund með
Elon Musk. Þeim langar öllum að fara í geimferðalag en þeir eru ósammála um
áfangastað. Kósínus vill fara til þeirrar plánetu sem er lengst í burtu af
öllum plánetunum sem þeir geta farið til en Sínus vill bara fara í stutt
ferðalag og kíkja til næstu plánetu. Hins vegar vill Pí láta draum þeirra
Kósínusar og Sínusar beggja rætast og er byrjaður að gera kostnaðaráætlun fyrir
þessi tvö ferðalög.

Vandamálið er að hann veit ekki hvaða pláneta er lengst í burtu og hversu langt
er í hana og ekki heldur hvaða pláneta er næst og fjarlægðina í hana. Hjálpaðu
Pí að finna þessar plánetur og fjarlægðirnar í þær út frá lista af plánetum og
staðsetningu jarðar.

\section*{Inntak}
Fyrsta lína inntaksins inniheldur eina heiltölu $N$, $1 \leq N \leq
10\,000$, sem táknar fjölda mögulegra áfangastaða frá jörðu. Á annarri línu
inntaks er núverandi staðsetning jarðar gefin með þremur rauntölum $x_e$,
$y_e$, $z_e$ í þrívíðu rúmi. Þar á eftir fylgja $N$ línur, hver þeirra
inniheldur þrjár tölur $x_i$, $y_i$ og $z_i$, $-10\,000 \leq x_i,y_i,z_y \leq
10\,000$ sem tákna staðsetningu $i$-ta áfangastaðarins á sama hátt og jörðin.

\section*{Úttak}
Úttakið samanstendur af tveimur línum. Á fyrstu línunni er númerið (talið frá
$1$) á áfangastaðnum samkvæmt röð inntaksins, sem er næst jörðinni og
fjarlægðin í hann. Á næstu línu er númerið á áfangastaðnum sem er fjærst
jörðinni og fjarlægðina í hann. Fjarlægðirnar skal skrifa námundað að $4$
aukastöfum. Áfangastaðirnir munu allir hafa mismunandi fjarlægð.
