\problemname{Deilanleg með 11}

\illustration{.4}{alice}{}

Lewis Carroll er þekktastur fyrir að hafa skrifað bókina um Lísu í undralandi.
Færri vita að hann var líka stærðfræðingur, og hafði mjög gaman af allskonar
stærðfræðiþrautum.

Til dæmis fann hann upp á eftirfarandi reikniriti til að athuga hvort að
heiltala $n$ sé deilanleg með $11$:

\begin{enumerate}
    \item Á meðan $n \geq 10$:
        \begin{enumerate}
            \item Látum $d$ vera aftasta tölustafinn í $n$
            \item Fjarlægjum aftasta tölustafinn úr $n$
            \item Drögum $d$ frá $n$
        \end{enumerate}
    \item Upphaflega talan var deilanleg með 11 ef núverandi $n$ er $0$
\end{enumerate}

Skrifið forrit sem les inn heiltölu $n$, og framkvæmir
þetta reiknirit til að athuga hvort talan sé deilanleg með $11$. Forritið á að
skrifa út gildið á $n$ í hverri ítrun af reikniritinu.

\section*{Inntak}
Inntak inniheldur eina heiltölu $1 \leq n\leq 10^{100}$.

\section*{Úttak}
Úttak á að innihalda gildið á $n$ í hverri ítrun af reikniritinu. Að lokum á að
skrifa út eina línu sem er \texttt{Talan $n$ er deilanleg med 11.} eða
\texttt{Talan $n$ er ekki deilanleg med 11.} eftir því sem við á.

