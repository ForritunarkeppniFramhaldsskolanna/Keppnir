\problemname{Demantar að Eilífu}

\illustration{.5}{diamond}{}

Jón bóndi er nýbyrjaður að læra forritun. Í kennslustund dagsins var kennarinn
að útskýra ``for'' lykkjur. Í miðjum fyrirlestrinum sofnaði Jón og
fór að dreyma um kindurnar í sveitinni. Í lok kennslustundarinnar setti
kennarinn fyrir eftirfarandi verkefni:

\textit{``Þú færð upp gefið stærð demants og bókstaf til að teikna hann með.
Stærð demants er skilgreind sem hæð fyrir ofan míðlínu og hæð fyrir neðan
miðlínu, þar sem miðlínan er með talin.''}

Núna er Jón bóndi í vanda. Þegar hann reynir að rifja upp hvernig ``for''
lykkjur virka fær hann bara myndir af kindum upp í kollinn. Jón er ekki vanur
að svindla, en hann grípur nú til örþrifaráða og biður þig um að gera verkefnið
fyrir sig.

\section*{Inntak}
Ein lína með stærð demantsins og bókstafinn til að teikna hann með, aðskilið
með bili. Stærð demantsins er jákvæð heiltala.

\section*{Úttak}
$2N - 1$ línur með teikningunni af demantinum, þar sem $N$ er stærð demantsins.


\section*{Útskýring á dæmi}
Athuga skal að lína má ekki enda á bili. Ef við breytum bilum í fyrra úttakinu
í `.', þá lítur teikningin svona út:

\begin{verbatim}
....a
...a.a
..a...a
.a.....a
a.......a
.a.....a
..a...a
...a.a
....a
\end{verbatim}

