\problemname{Sjálflýsandi Runur}

Við segjum að runa $A = (a_0, a_1, \ldots, a_{n-1})$ af $n$ heiltölum á
bilinu $0$ upp í $n-1$ sé sjálflýsandi, ef fyrir hverja heiltölu $i$ á þessu
bili þá eru $a_i$ eintök af $i$ í rununni.


\illustration{1}{self}{}

Til dæmis er runan $A = (1,2,1,0)$ sjálflýsandi, því
\begin{itemize}
    \item heiltalan $0$ kemur einu sinni fyrir, og $a_0 = 1$,
    \item heiltalan $1$ kemur tvisvar sinnum fyrir, og $a_1 = 2$,
    \item heiltalan $2$ kemur einu sinni fyrir, og $a_2 = 1$,
    \item heiltalan $3$ kemur aldrei fyrir, og $a_3 = 0$.
\end{itemize}

Skrifið forrit sem les inn heiltölu $n$, og skrifar út sjálflýsandi runu af
lengd $n$. Ef margar svoleiðis runur eru til, þá skiptir ekki máli hver þeirra
er skrifuð út. Ef það eru engar svoleiðis runur til, þá á forritið að skrifa út
\texttt{Engin}.

\section*{Inntak}
Ein lína með heiltölu $n$, þar sem $1 \leq n \leq 100$.

\section*{Úttak}
Ein lína sem inniheldur annaðhvort $n$ heiltölur á milli $0$ og $n-1$ sem
tákna sjálflýsandi runu, eða strenginn \texttt{Engin} ef engin svoleiðis runa
er til.

