\problemname{Primes go both ways}

\illustration{.5}{emirp}{}

Björn, Orri og Sæmundur eru að leika saman út á rólóvelli. Móðir hans Orra
hefur áhyggjur af honum því að hann fór ekki í ullarsokka áður en hann fór út
að leika. Faðir hans Sæmundar ef bifvélavirki þrátt fyirir að vera bóndasonur,
en eftir að faðir hans lést á 8. áratugnum þá ákvað hann að taka ekki við
býlinu, heldur var það bróðir hans sem tók við. Vegna fjárhagsörðugleika varð
hann þó að selja landið og flytja í bæinn en það hafði í för með sér að hann
kynntist Kristínu, en þau eignuðust síðan Björn og systur hans Jónu.

Nú hafa Björn, Orri og Sæmundur áhuga á öllum prímtölum sem eru líka prímtölur
þegar tölunni hefur verið snúið við. Til dæmis er $17$ prímtala en talan $71$ er líka
prímtala.

\section*{Inntak}
Inntakið samanstendur af einni línu sem inniheldur eina tölu $N$, $1 \leq N \leq 1000000$.

\section*{Úttak}
Skrifið út allar prímtölur $p$, $1 \leq p \leq N$, þar sem $p$ skrifuð öfugt er einnig prímtala.
