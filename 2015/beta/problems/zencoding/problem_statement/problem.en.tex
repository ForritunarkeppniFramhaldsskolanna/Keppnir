\problemname{Zen Coding}

\illustration{.4}{coding}{}

Forritarar þurfa stundum að skrifa HTML kóða, en það verk vill oft verða mjög
einhæft með mikið af endurtekningum (eins og að skrifa \texttt{<tag>...</tag>}
milljón sinnum). Þar sem forritarar eru almennt latir og vilja ekki endurtaka
vinnuna sína hefur einhver fundið lausn á þessu.

Zen Coding er lítið mál sem lýsir strúktúr á HTML elementum sem stækka síðan út
í venjulegt HTML. Sem dæmi þá myndi strengurinn
\begin{verbatim}
html>head+body>div+div+p>ul>li*3>a
\end{verbatim}
stækka út í
\begin{verbatim}
<html>
    <head>
    </head>
    <body>
        <div></div>
        <div></div>
        <p>
            <ul>
                <li><a></a></li>
                <li><a></a></li>
                <li><a></a></li>
            </ul>
        </p>
    </body>
</html>
\end{verbatim}

Þið eruð beðin um að útfæra lítinn hluta af þessu máli sem samanstendur af eftirfarandi táknum:
\begin{itemize}
    \item \textbf{>}: HTML tréð sem kemur á eftir tákninu er innan í (e. child) HTML elementinu sem kom á undan.
    \item \textbf{+}: HTML tréð sem kemur á eftir tákninu er samhliða (e. sibling) HTML elementinu sem kom á undan.
    \item \textbf{*n}: HTML elementið sem kom á undan tákninu og allt það HTML tré endurtekið $n$ sinnum.
\end{itemize}

\section*{Inntak}
Inntakið samanstendur af einni línu sem inniheldur streng á Zen Coding forminu.

\section*{Úttak}
Skrifið út HTML kóðann sem útvíkkaður er úr Zen Coding forminu á einni línu án
nokkurra biltákna (whitespace).
