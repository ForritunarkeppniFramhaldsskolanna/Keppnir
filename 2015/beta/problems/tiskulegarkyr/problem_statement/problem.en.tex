\problemname{Tískulegar Kýr}

\illustration{.5}{cow}{}

Þegar Jón bóndi frétti að kýr með tvo bletti væru í tísku keypti hann heila
hjörð af tveggja-bletta kúm. En eins og gengur og gerist er tískan fljót að
breytast, og eru nú kýr með einn blett í tísku.

Jón bóndi vill gera hjörðina sína tískulegri og ákveður að mála hverja kú
þannig að blettirnir tveir sameinist og verði að einum. Skinn kúar er táknað
með $N$ sinnum $M$ ($1 \leq N, M \leq 50$) fylki af
bókstöfum eins og:

\begin{verbatim}
................
..XXXX....XXX...
...XXXX....XX...
.XXXX......XXX..
........XXXXX...
.........XXX....
\end{verbatim}

Hér táknar `X' hluta af bletti. Tvö `X' eru í sama
bletti ef þau eru hlið við hlið lóðrétt eða lárétt (á ská er ekki tekið með),
þannig að skýringarmyndin að ofan inniheldur tvo bletti. Allar kýr í hjörð Jóns
bónda eru með nákvæmlega tvo bletti.

Jón bóndi vill nota eins litla málningu og mögulegt er til að sameina tvo
bletti í einn. í dæminu að ofan getur hann gert það með því að mála aðeins þrjú
`X' í viðbót (nýju bókstafirnir eru táknaðir með `*' að
neðan til að gera þá einfaldari að sjá).

\begin{verbatim}
................
..XXXX....XXX...
...XXXX*...XX...
.XXXX..**..XXX..
........XXXXX...
.........XXX....
\end{verbatim}

Vinsamlegast hjálpaðu Jóni bónda að finna minnsta fjölda nýrra `X'a
sem hann þarf að mála til að sameina blettina tvo í einn stóran blett.

\section*{Inntak}
Fyrsta lína inniheldur tvær heiltölur, $N$ og $M$, aðskildar með
bili. Næstu $N$ línur skilgreina skinn kúarinnar, þar sem hver lína
inniheldur streng af lengdinni $M$ með bókstöfunum `X' og
`.'.

\section*{Úttak}
Ein lína sem inniheldur minnsta fjölda `X'a sem þarf að bæta við
skinnið í inntakinu til að fá einn stóran blett.

\section*{Útskýring á dæmi}
Inntakið sýnir skinn kúar með tvo mismunandi bletti, merktir með 1 og 2 að
neðan:

\begin{verbatim}
................
..1111....222...
...1111....22...
.1111......222..
........22222...
.........222....
\end{verbatim}

Þrjú `X' nægja til að sameina blettina tvo í einn:

\begin{verbatim}
................
..1111....222...
...1111X...22...
.1111..XX..222..
........22222...
.........222....
\end{verbatim}

