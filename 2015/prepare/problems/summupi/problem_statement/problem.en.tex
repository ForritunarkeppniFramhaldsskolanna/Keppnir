\problemname{Leggja saman í $\pi$}

\illustration{.5}{pi}{}

Á morgun er 14. mars 2015, sem líka er hægt að skrifa sem 3/14/15. Þið ættuð nú
öll að kannast við þessa talnarunu, en þetta eru einmitt fyrstu fimm
tölustafirnir í stærðfræðifastanum $\pi \approx 3.1415\cdots$. Það kemur því
ekki á óvart að dagurinn á morgun er tileinkaður $\pi$, og er kallaður $\pi$
dagurinn.

Í tilefni þess erum við búin að undirbúa mörg dæmi tengd $\pi$. Og við
eiginlega gleymdum okkur aðeins í dæmaskrifunum, og bjuggum til of mörg dæmi um
$\pi$! Við ákváðum því að flytja þetta dæmi, sem átti að vera í keppninni á
morgun, í þessa undirbúningskeppni, því það var einfaldlega of auðvelt fyrir
keppnina á morgun. Dæmið hljóðar svona:

Þú færð gefinn lista af $1 \leq n \leq 35$ kommutölum, hver með nákvæmlega einn
tölustaf á undan kommu, og nákvæmlega tíu aukastafi á eftir kommu. Þið eigið að
velja núll eða fleiri af þessum kommutölum, svo að summa þeirra sé sem næst
$\pi$, en þó ekki stærri en $\pi$. Þið eigið svo að skrifa út þessa summu með
nákvæmlega tíu aukastöfum á eftir kommu.

Við vitum að flest ykkar kunna heilan helling af aukastöfum í $\pi$. En til
öryggis, fyrir þá sem eru ekki það kúl, þá fylgir hér $\pi$ námundað upp að
tuttugasta aukastaf:

\begin{verbatim}
3.14159265358979323846
\end{verbatim}

