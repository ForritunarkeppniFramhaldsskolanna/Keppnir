\documentclass{article}
\usepackage[utf8]{inputenc}
\usepackage[T1]{fontenc}
\usepackage[icelandic]{babel}
\usepackage{graphicx}

\title{Myndbandaflokkari}
\author{}
\date{}

\begin{document}
\maketitle

Hver kannast ekki við óreiðuna í myndbandamöppunni á tölvunni sinni. Ekkert
skipulag á kvikmyndum né sjónvarpsþáttum, og maður finnur aldrei það sem mann
langar að horfa á.

Ykkar verkefni í dag er að skrifa forrit sem leysir þennan vanda. Markmiðið er
að forritið gefi notandanum þægilega sýn á myndbandamöppuna, þar sem, til
dæmis, kvikmyndir eru flokkaðar eftir ári og sjónvarpsþættir eru flokkaðir
eftir þáttaröðum. Þá gæti forritið náð í upplýsingar um kvikmyndir og
sjónvarpsþætti af gagnagrunnum á netinu svo sem \texttt{IMDb.com}. Forritið gæti leyft
notandanum að spila kvikmyndir og sjónvarpsþætti, og gæti þá haldið utan um
hvað notandinn er búinn að horfa á.

\vspace{5pt}
\begin{center}
    \includegraphics[scale=1.5]{movie.png}
\end{center}

Hér að neðan eru settar upp grunnkröfur sem forritið verður upp að uppfylla,
og þar að auki eru gefnar hugmyndir að aukavirkni sem mögulegt er að útfæra.
Grunnkröfurnar gera ekki ráð fyrir neinni sérstakri útfærslu á viðmóti, en ykkur
er frjálst að útfæra viðmót í "`console"', með gluggaforriti eða vefforriti.
Eftir að forritið hefur uppfyllt grunnkröfur getið þið svo lagt áherslu á
viðmót, nýtingu gagnagrunna eða aukið virkni forritsins.

Á vefsvæði
Kósínus-deildarinnar\footnote{http://mooshak2.ru.is/fk\_{}2015\_{}kosinus} má
finna \texttt{zip} skrá sem inniheldur dæmi um myndbandamöppu. Þar má finna
allskonar kvikmyndir, sjónvarpsþætti, og annað ótengt efni. Athugið að allar
skrárnar eru tómar.

Gangi ykkur vel!

\section*{Grunnkröfur}
Eftirfarandi listi lýsir kröfum sem forritið verður að uppfylla.
\begin{itemize}
    \item Forritið á að geyma slóð á möppuna sem innihalda myndböndin. Það þarf
        ekki endilega að vera stillanlegt af notanda.
    \item Þegar forritið ræsist, eða þegar notandi gefur forritinu skipun, fer
        það í gegnum allar skrár í myndbandamöppunni (eða myndbandamöppunum) og
        flokkar myndböndin í (að minnsta kosti) tvo flokka: “sjónvarpsþætti” og
        “annað”
        \begin{itemize}
            \item Þið ráðið hvernig forritið finnur út hvort skrá inniheldur
                sjónvarpsþætti eða ekki.
            \item Ein hugmynd er að láta notandann skrá þá sjónvarpsþætti sem
                hann er með í möppunni. Forritið getur þá leitað að öllum
                þáttanöfnum í sérhverju skráarnafni og þannig ákvarðað hvort
                skráin inniheldur sjónvarpsþátt eða ekki.
            \item Önnur hugmynd er að leita að mynstrum sem oft koma upp í
                nöfnum á skrám sem innihalda sjónvarpsþætti eins og t.d.\ 
                `\texttt{the.show.S02E4}' eða `\texttt{show-403}'. Forritið getur þá reynt að
                giska á nafn sjónvarpsþáttarins út frá nafni skráarinnar.
            \item Allt sem flokkast ekki sem sjónvarpsþáttur, flokkast sem "`annað"'.
        \end{itemize}
    \item Forritið á að setja myndböndin fram á snyrtilegan hátt þannig að
        notandinn hafi góða yfirsýn yfir hvaða myndbönd hann á. Þá ættu
        sjónvarsþættir að vera flokkaðir niður í þáttaraðir, og á notandinn að
        geta vafrað um þær.
    \item Notanda á að vera kleift að leita að myndböndum eftir nafni.
\end{itemize}

\section*{Aukavirkni}
Hér koma hugmyndir að aukinni virkni forritsins, en þið eruð einnig hvött til
þess að útfæra ykkar eigin hugmyndir!
\begin{itemize}
    \item Gera notanda kleift að stilla möppuna sem inniheldur myndbönd.
    \item Láta forritið styðja margar möppur sem innihalda myndbönd.
    \item Láta forritið raða myndböndum snyrtilega í möppur.
        \begin{itemize}
            \item T.d.\ ef forritið finnur skrá með nafninu
                \texttt{the.show.S03E14.avi}, þá getur forritið fært það í
                möppuna \texttt{The Show/Season 3/the.show.S03E14.avi}
            \item Forritið gæti svo hreinsað til í myndbandamöppunni, t.d.\ með
                því að eyða tómum möppum eða möppum sem innihalda eingöngu
                óþarfa skrár (eins og t.d.\ textaskrár með lýsingum á þáttum).
        \end{itemize}
    \item Gera grafískt viðmót fyrir forritið. Þetta getur t.d.\ verið gluggaviðmót eða vefviðmót.
    \item Gera notanda kleift að ræsa myndband úr notendaviðmóti.
    \item Halda utan um hvaða myndbönd notandinn er búinn að spila, og jafnvel
        hversu langt notandinn var kominn ef hann kláraði ekki myndbandið.
    \item Ná í upplýsingar um kvikmyndir og sjónvarpsþætti af gagnagrunnum á
        netinu, t.d.\ \texttt{IMDb.com}, \texttt{themoviedb.org}, eða \texttt{omdbapi.com}.
    \item Flokka myndbönd eftir ári, tegund, þáttaröð, leikstjóra, leikurum, o.s.frv.
    \item Gera notanda kleift að halda utan um uppáhalds myndböndin sín, og/eða
        myndbönd sem hann vill horfa á seinna.
    \item Geyma upplýsingar um myndbönd í gagnagrunni, t.d.\ svo ekki þurfi að
        fara í gegnum alla myndbandamöppuna í hvert skipti sem forritið er
        opnað.
    \item Ná í íslenskan texta fyrir kvikmyndir (eða texta á öðrum tungumálum).
        Til að sækja texta getið þið notað API
        \texttt{opensubtitles.org}\footnote{http://trac.opensubtitles.org/projects/opensubtitles/wiki/DevReadFirst}.
\end{itemize}

\end{document}
