\problemname{Dulmál 4}

\illustration{.4}{encryption}{}

Jón og Gunna eru saman í bekk. Þau eru búin að vera að senda hvoru öðru
skilaboð í kennslustundum. Um daginn sá kennarinn hvað þau voru að gera, tók
skilaboðin, og las upp fyrir framan bekkinn. Þetta fannst Jóni og Gunnu vera
mjög óþægilegt, og til að koma í veg fyrir að þetta gerist aftur hafa þau
ákveðið að byrja að dulkóða skilaboðin sem þau senda hvoru öðru. Þau velja sér
einhverja jákvæða heiltölu $k$, og enginn veit hver sú tala er nema þau. Svo
bæta þau við stöfum af handahófi inn á milli stafa í skilaboðunum, þannig að ef
maður les $k$:ta hvern staf í dulkóðuðu skilaboðunum, þá fær maður upprunalega
skilaboðin.

Tökum dæmi. Segjum að $k=3$ og skilaboðin sem á að senda sé ``hittumstaeftir''. Þá gæti dulkóðuðu skilaboðin litið út svona:
\begin{center}
hpnipktoftdhugfmtxspktqracuefofdqtclihtr
\end{center}
Þá ef maður skoðar $k$:ta hvern staf fær maður upphaflegu skilaboðin:
\begin{center}
\textbf{h}pn\textbf{i}pk\textbf{t}of\textbf{t}dh\textbf{u}gf\textbf{m}tx\textbf{s}pk\textbf{t}qr\textbf{a}cu\textbf{e}fo\textbf{f}dq\textbf{t}cl\textbf{i}ht\textbf{r}
\end{center}


Núna eru Jón og Gunna búin að taka eftir hvað Ásgeir er að gera. Þau sjá að
hann er að reyna að brjóta dulkóðann þeirra með því að prufa að afkóða dulkóðuð
skilaboð sem hann kemst yfir með öllum gildum á $k$ frá $1$ upp í lengdina á
dulkóðuðu skilaboðunum.

En Jón og Gunna halda að þau geti villt fyrir Ásgeiri, því hugsanlega geta
fleiri en eitt gildi á $k$ gefið skilaboð sem eru skiljanleg. Til dæmis ef
dulkóðuðu skilaboðin eru ``\texttt{mifaidxzn}'', þá myndi Ásgeir prufa að
afkóða með öllum mögulegum $k$-um, og fá:

\begin{verbatim}
1: mifaidxzn
2: mfixn
3: max
4: min
5: md
6: mx
7: mz
8: mn
9: m
\end{verbatim}

En þarna gæti bæði ``\texttt{min}'' og ``\texttt{max}'' komið til greina sem
skilaboð. Jón og Gunna ætla að nýta sér þetta. Þau velja sér $n$ orð, þar sem
$n$ er í mesta lagi $5$ og hvert orð er í mesta lagi $15$ stafir að lengd, og
ætla svo að búa til dulkóðaðan streng, þannig hvert af þessum orðum komi að
minnsta kosti einu sinni fyrir í listanum sem Ásgeir fær.

Skrifið forrit sem les inn heiltöluna $n$, og svo $n$ orð. Forritið á að skrifa
út \textbf{stystu} dulkóðuðu skilaboðin sem hafa þann eiginleika að öll af
gefnu orðunum koma fyrir þegar hann er afkóðaður með mismunandi gildum af $k$.
Þið megið gera ráð fyrir að svona strengur sé til, og að hann sé aldrei lengri
en $100$ stafir. Ef það eru mörg dulkóðuð skilaboð sem koma til grein, þá
skiptir ekki máli hvert þeirra er skrifað út.

