\problemname{Summa}

Carl Friedrich Gauss er einn af áhrifamestu stærðfræðingum sögunnar. Hann var
undrabarn, og eru til nokkrar sögur af snilli hans. Ein þeirra segir að
grunnskólakennari hans hafi beðið krakkana í bekknum hans að leggja saman allar
heiltölur frá $1$ upp í $100$. Kennarinn áætlaði að þetta myndi taka krakkana
dágóðann tíma. En örfáum mínútum seinna lætur Gauss kennarann vita að hann sé
búinn. Kennarinn trúði honum auðvitað ekki, og fór yfir svarið hans. En hún
varð alveg furðu lostin þegar hún áttaði sig á því að svarið hans var rétt.

Ólíkt Gauss erum við nútímakrakkarnir heppnir að hafa tölvu til að framkvæma
reikninga sem þessa fyrir okkur. Skrifið forrit sem les inn tvær heiltölur $a$
og $b$. Þið megið gera ráð fyrir að $a$ sé minni en $b$, og þær eru báðar á
bilinu $1$ upp í $10\,000$. Forritið á að leggja saman allar heiltölur frá $a$
upp í $b$, og á svo að skrifa út svarið.

