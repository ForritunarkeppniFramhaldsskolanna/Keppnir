\problemname{Aukastafir $\pi$}

Uppáhalds stærðfræði fasti okkar allra er talan $\pi$, en hún er skilgreind sem
hlutfall milli ummáls og þvermáls hrings. Hún er um það bil $3.14159265$, en í
raun og veru er hún með óendanlega marga aukastafi á eftir kommu. Margir
tölvunarfræðingar hafa skemmt sér við að láta tölvur reikna út fleiri og fleiri
aukastafi í $\pi$, og í dag eru búið að finna rúmlega trilljón fyrstu
aukastafina í $\pi$.

Í dag 14. mars 2015, eða 3.14.15. Það er því ekki skrýtið að dagurinn í dag er
tileinkaður $\pi$. Við skulum halda upp á það með því að skrifa forrit sem
skrifar út fyrstu $100$ aukastafi $\pi$.

