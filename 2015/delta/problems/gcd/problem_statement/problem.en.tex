\problemname{Greatest Common Decilitre}

Bjössi bakari er að reyna að baka kransaköku fyrir ferminguna hans Hjalta en
honum vantar ýmis desilítramál til að mæla hráefnin. Hann er hins vegar með
nokkur mál $m_1,m_2,\ldots,m_n$ en hann vill vita hvað er minnsta magnið $g$
sem hann getur mælt með málunum. Hann er með eina stóra skál og í hana getur
hann helt nákvæmlega $m_i$ desilítrum með einhverju máli $m_i$ eða hellt úr
skálinni $m_j$ desilítrum með einhverju máli $m_j$. Hjálpið Bjössa að leysa
þetta vandamál.

\emph{Dæmi}: Ef Bjössi hefur mál af stærð $4$, $6$ og $9$ þá er minnsta
einingin sem hann getur mælt nákvæmlega $1$. Hann byrjar á því að hella $9$
desilítrum í skálina og síðan getur hann tvisvar sinnum hellt úr skálinni $4$
desilítrum með fjagra desilítramálinu.

\section*{Inntak}
Fyrsta lína inntaksins inniheldur eina heiltölu $1 \geq n \geq 10$. Á næstu línu eru $n$
heiltölur $m_1,m_2,\ldots,m_n$ þar sem $1 \geq m_i \geq 1000$ er $i$-ta desilítramálið hans
Bjössa.

\section*{Úttak}
Prentið út á einni línu töluna $g$, minnsta fjölda desílítra sem Bjössi getur
mælt með málunum sínum.
