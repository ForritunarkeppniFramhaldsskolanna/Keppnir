\begin{problem}{Frumtölur}{Inn}{Út}{~}{~}

	Frumtala er heiltala, stærri en einn, sem hefur hefur enga deila aðra en 1 og sjálfa sig. Til dæmis er heiltalan 8 ekki frumtala þar sem hún hefur deilana 1, 2, 4 og 8, en heiltalan 7 er frumtala þar sem hún hefur aðeins deilana 1 og 7 (þ.e.\ 1 og sjálfa sig). Frumtölur koma oft upp í strjálli stærðfræði, og eru lykilatriði í nútíma dulmálsfræði.

	Ef við teljum fyrstu fimm frumtölurnar upp, þá er fyrsta frumtalan $2$, önnur frumtalan $3$, þriðja frumtalan $5$, fjórða frumtalan $7$, og fimmta frumtalan $11$. Þetta dæmi snýst um að finna frumtölu númer $n$. Til dæmis ef $n = 5$, þá viljum við finna fimmtu frumtöluna, sem er auðvitað frumtalan $11$.

	\Input

		Á fyrstu línu er heiltalan $1 \leq T \leq 100$, sem táknar fjölda prófunartilvika sem fylgja. Hvert prófunartilvik samanstendur af einni línu með heiltölunni $1 \leq n \leq 1000$.
	\Output

		Fyrir hvert prófunartilvik á að skrifa út eina línu sem inniheldur frumtölu númer $n$.

	\Examples

		\begin{example}
			\exmp{
4
1
5
15
100
			}{
2
11
47
541
}%
		\end{example}

\end{problem}