\begin{problem}{Bitmask}{Inn}{Út}{~}{~}

	\begin{wrapfigure}{r}{0.36\textwidth}
		\vspace{-25pt}
		\begin{center}
			\includegraphics[scale=0.22]{../Bitmask/binary.jpg}
		\end{center}
		\vspace{-30pt}
	\end{wrapfigure}

	Bitagrímur (eða \textit{bitmasks}) koma oft upp þegar unnið er með bita og bitareikning. Með þessum grímum er hægt að fá fram allskonar bitamynstur. Til dæmis er hægt að núllstilla fyrstu fjóra bitana í 32-bita tölu með því að nota bitagrímuna \texttt{0xFFFFFFF0} og nota svo \texttt{AND} bitaaðgerðina. Í þessu dæmi átt þú að finna svona bitagrímu.

	Þú færð gefnar þrjár 32-bita jákvæðar heiltölu $N$, $L$ og $U$. Þú átt að finna bitagrímu $M$ þannig að $L \leq M \leq U$ og $N \texttt{ OR } M$ er í hámarki. Til dæmis ef $N$ er 100, og $L = 50$, $U = 60$, þá mun $M$ vera $59$ og $N \texttt{ OR } M$ mun vera $127$, sem er í hámarki. Ef fleiri en eitt gildi fyrir $M$ uppfylla þessi skilyrði, þá skaltu nota minnsta gildið.

	\Input

		Á fyrstu línu er heiltalan $1 \leq T \leq 1000$, sem táknar fjölda prófunartilvika sem fylgja. Hvert prófunartilvik samanstendur af einni línu með þremur 32-bita jákvæðum heiltölum $N$, $L$ og $U$, þar sem $L \leq U$.

	\Output

		Fyrir hvert prófunartilvik á að skrifa út minnsta gildi fyrir $M$ þannig að $N \texttt{ OR } M$ er í hámarki.

	\Examples

		\begin{example}
			\exmp{
5
100 50 60
100 50 50
100 0 100
1 0 100
15 1 15
			}{
59
50
27
100
1
}%
		\end{example}

	\Note

		Látum $x$ og $y$ vera 32-bita pósitífar heiltölur. Í forritunarmálunum \texttt{C++}, \texttt{Python} og \texttt{Java} er bitaaðgerðin $x \texttt{ AND } y$ táknuð $x \texttt{ \&{} } y$, og bitaaðgerðin $x \texttt{ OR } y$ táknuð $x \texttt{ | } y$.

		% TODO: Hugsanlega bæta við meiri upplýsingum um AND/OR aðgerðirnar.

\end{problem}