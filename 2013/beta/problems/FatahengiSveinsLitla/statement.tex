\begin{problem}{Fatahengi Sveins Litla}{Inn}{Út}{~}{~}

	\begin{wrapfigure}{r}{0.27\textwidth}
		\vspace{-25pt}
		\begin{center}
			\includegraphics[scale=0.5]{../FatahengiSveinsLitla/coat_monkey.jpg}
		\end{center}
		\vspace{-30pt}
	\end{wrapfigure}

	Sveinn litli hefur ákveðið að stofna leikhús með litlu vinum sínum, Jóni og Palla. Sveinn litli er því miður ekki mikill leikari, en í staðinn fékk hann það spennandi starf að hafa umsjón með fatahenginu. Hans hlutverk er að taka við yfirhöfnum gesta og geyma þær á meðan á sýningu stendur. Eftir sýninguna koma gestirnir og sækja yfirhafnirnar sínar áður en þeir fara.

	Í kvöld er nú ansi stórt kvöld fyrir þá félaga, Svein, Jón og Palla, en það munu koma $N$ ($1 \leq N \leq 9$) gestir að sjá sýninguna þeirra!

	Núna eru gestirnir byrjaðir að streyma inn, og hver á eftir öðrum skilur yfirhöfnina sína eftir í öruggum höndum Jóns litla, yfirumsjónarmanns fatahengisins. Þegar allir N gestirnir hafa skilið eftir yfirhöfnina sína og farnir að horfa á sýninguna, þá tekur Jón litli eftir að hann gleymdi að merkja yfirhafnir gestanna. Núna er hann í vanda, því hann veit ekki hvaða gestur á hvaða yfirhöfn. Hann ákveður að láta sem allt sé í lagi, og þegar gestirnir koma að ná í yfirhafnirnar sínar lætur hann hvern gest fá yfirhöfn af handahófi.

	Núna eru gestirnir farnir. Jón litli fer nú að hugsa með sér hversu slæmt þetta gæti hafa verið. Hann ákveður að telja fjölda möguleika á því að enginn gestur hafi fengið sína yfirhöfn. Eftir að hafa reynt að telja í dálítinn tíma gefst Jón litli upp. Hann er ekki búinn að læra að telja svona hátt, og biður hann þig um að hjálpa sér.

	Skrifið forrit sem telur fjölda möguleika á því að enginn af $N$ gestunum hafi fengið sína yfirhöfn.

	\Input

		Á fyrstu línu er heiltalan $1 \leq T \leq 10$, sem táknar fjölda prófunartilvika sem fylgja. Hvert prófunartilvik samanstendur af einni línu með heiltölunni $1 \leq N \leq 9$.

	\Output

		Fyrir hvert prófunartilvik á að skrifa út eina línu með fjölda möguleika á því að enginn af $N$ gestunum hafi fengið sína yfirhöfn.

	\Examples

		\begin{example}
			\exmp{
3
1
3
5
			}{
0
2
44
}%
		\end{example}

	\Explanation

		Ef það er bara einn gestur, þá mun hann auðvitað fá sína yfirhöfn þegar hann fer. Þá eru því 0 möguleikar á að enginn fái sína yfirhöfn.

Ef það eru þrír gestir, þá eru tveir möguleikar á að enginn fái sína yfirhöfn:
	
	\begin{enumerate}
		\item Gestur nr.\ 1 fær yfirhöfn gests nr.\ 3, gestur nr.\ 2 fær yfirhöfn gests nr.\ 1, og gestur nr.\ 3 fær yfirhöfn gests nr.\ 2.
		\item Gestur nr.\ 1 fær yfirhöfn gests nr.\ 2, gestur nr.\ 2 fær yfirhöfn gests nr.\ 3, og gestur nr.\ 3 fær yfirhöfn gests nr.\ 1.
	\end{enumerate}

\end{problem}