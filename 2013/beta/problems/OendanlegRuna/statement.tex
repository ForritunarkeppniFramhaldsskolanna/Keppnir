\begin{problem}{Óendanleg Runa}{Inn}{Út}{~}{~}

	Jón litli á von á vini sínum í heimsókn, en þeir eru búnir að skipuleggja brjálað LAN kvöld og ætla að spila alla uppáhalds tölvuleikina sína. Jón er búinn að bíða í allan dag og er ekkert smá spenntur. En þá hringir vinur hans og segist því miður ekki geta komist, en hann sé orðinn fárveikur.

	Það er ekkert gaman að spila þessa tölvuleiki einn, þannig Jóni fer að leiðast. Þá fær hann allt í einu snilldar hugmynd. Hann ætlar að búa til óendanlega stóra tölu með því að taka eitthverja litla tölu og bæta henni aftan við sig óendanlega oft.

	Til dæmis, hann velur töluna $6883$, og bætir henni óendanlega oft aftan við sig, og fær þá óendanlega stóru töluna
	\[ 68\underline{83688}36883688368836883688368836883688368836883688368836883\ldots \]
	Hann byrjar að skrifa töluna niður á blað, en eftir að hafa skrifað í hálftíma og fyllt tvær heilar blaðsíður ákveður hann að stoppa. Talan er svo stór! Hún er óendanlega stór! Honum langar að skoða töluna betur, og þá sérstaklega hvernig smá partur af henni lítur út á ákveðnum stöðum. Hann ákveður því að búa til forrit sem les inn upphaflegu töluna sem notuð var til að búa til óendanlega stóru töluna, og svo tvær heiltölur $i$ og $j$. Forritið á svo að skrifa út alla tölustafina frá tölustafi númer $i$ í tölunni til tölustafs númer $j$ í tölunni.

	En þá allt í einu fattar Jón litli að hann kann ekki að forrita! Hann biður þig því um aðstoð.

	Ef við tökum sem dæmi óendanlega stóra töluna sem búin er til með tölunni $6883$, og við viljum fá bútinn af tölunni frá $i = 3$ og upp að $j = 7$, þá mun forritið skila $83688$. Búturinn er undirstrikaður í tölunni að ofan.

	\Input

		Á fyrstu línu er heiltalan $1 \leq T \leq 100$, sem táknar fjölda prófunartilvika sem fylgja. Hvert prófunartilvik samanstendur af tveimur línum.

		Á fyrri línunni eru heiltölurnar $i$ og $j$, aðskildar með bili, þar sem $1 \leq i \leq j \leq 10^9$ og $j - i < 10^4$.

		Á seinni línunni er talan sem notuð var til að búa til óendanlega stóru töluna. Athuga skal að þessi tala getur verið allt að 100 tölustafir að lengd, en hefur að minnsta kosti einn tölustaf.

	\Output

		Skrifa skal út eina línu með bútinum af óendanlega stóru tölunni frá tölustafi númer $i$ til tölustafs númer $j$. Athuga skal að fyrsti tölustafurinn er númer $1$, annar tölustafurinn númer $2$, og svo framvegis.

	\Examples

		\begin{example}
			\exmp{
4
3 7
6883
10 15
123
1 1
9124232342825
5000000 5000004
3665333693
			}{
83688
123123
9
33665
}%
		\end{example}

\end{problem}