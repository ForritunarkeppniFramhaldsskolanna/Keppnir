\begin{problem}{Sierpinski Triangle}{Inn}{Út}{~}{~}

	\definecolor{dark-gray}{gray}{0.15}
	\newcommand{\oddSq}[1]{\colorbox{dark-gray}{\color{white}{\textbf{#1}}}}
	\newcommand{\evenSq}[1]{\textbf{#1}}

	Pascal-þríhyrningurinn er óendanlega stór þríhyrningur af heiltölum, skilgreindur þannig að efst í þríhyrningnum er talan 1, og er talan 1 svo í fyrsta og síðasta dálki hverrar raðar. Tölur annarstaðar í þríhyrningnum eru fengnar með því að leggja saman tölurnar tvær fyrir ofan hana. Fyrstu sjö raðirnar í þríhyrningnum líta þá svona út:

	\begin{center}
		\begin{tabular}{ccccccccccccc}
			&    &    &    &    &    &  \oddSq1\\\noalign{\smallskip\smallskip}
			&    &    &    &    &  \oddSq1 &    &  \oddSq1\\\noalign{\smallskip\smallskip}
			&    &    &    &  \oddSq1 &    &  \evenSq2 &    &  \oddSq1\\\noalign{\smallskip\smallskip}
			&    &    &  \oddSq1 &    &  \oddSq3 &    &  \oddSq3 &    &  \oddSq1\\\noalign{\smallskip\smallskip}
			&    &  \oddSq1 &    &  \evenSq4 &    &  \evenSq6 &    &  \evenSq4 &    &  \oddSq1\\\noalign{\smallskip\smallskip}
		    &  \oddSq1 &    &  \oddSq5 &    & \evenSq{10} &    & \evenSq{10} &    &  \oddSq5 &    &  \oddSq1\\\noalign{\smallskip\smallskip}
		   \oddSq1 &    &  \evenSq6 &    & \oddSq{15} &    & \evenSq{20} &    & \oddSq{15} &    &  \evenSq6 &    &  \oddSq1\\\noalign{\smallskip\smallskip}
		\end{tabular}

	\end{center}

	Ef við litum reitina sem innihalda oddatölur svarta, og reitina sem innihalda sléttar tölur hvíta, þá fáum við mynstur. Þetta mynstur heitir Sierpinski þríhyrningurinn, og er svokallaður "`fractal"'.

	\Input

		Inntakið inniheldur eina heiltölu $1 \leq n \leq 100$.

	\Output

		Skrifa á út fyrstu $n$ raðirnar í Sierpinski þríhyrningnum, þar sem svartur dálkur er táknaður með '\texttt{\#}`, og hvítur dálkur er táknaður með '.`. Athuga skal að þríhyrningurinn á að halla til vinstri.

	\Examples

		\begin{example}
			\exmp{
16
			}{
\#
\#\#
\#.\#
\#\#\#\#
\#...\#
\#\#..\#\#
\#.\#.\#.\#
\#\#\#\#\#\#\#\#
\#.......\#
\#\#......\#\#
\#.\#.....\#.\#
\#\#\#\#....\#\#\#\#
\#...\#...\#...\#
\#\#..\#\#..\#\#..\#\#
\#.\#.\#.\#.\#.\#.\#.\#
\#\#\#\#\#\#\#\#\#\#\#\#\#\#\#\#
}%
		\end{example}

\end{problem}