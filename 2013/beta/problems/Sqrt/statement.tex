\begin{problem}{Sqrt}{Inn}{Út}{~}{~}

	Gömul japönsk aðferð til að reikna $\sqrt{n}$ (kvaðratrót af $n$) er eftirfarandi:

	\begin{quote}
		Búðu til breyturnar $a$ og $b$.\newline
		Breytan $a$ tekur gildið $5 \times n$ og breytan $b$ gildið $5$.\newline
		\newline
		Eftirfarandi skref er hægt að gera oft til að fá meiri nákvæmni (þetta á að gera $i$ sinnum, sjá lýsingu að neðan):

		\begin{quote}
			ef $a \geq b$:
			\begin{quote}
				$a$ tekur gildið $a - b$\newline
				$b$ tekur gildið $b + 10$
			\end{quote}
			annars:
			\begin{quote}
				bæta tveimur núllum fyrir aftan $a$\newline
				bæta einu núlli á milli næstsíðasta og síðasta tölustafsins í $b$
			\end{quote}
		\end{quote}

		Breytan $b$ inniheldur svo svarið.

	\end{quote}

	Þú átt að lesa inn tölurnar $n$ og $i$. Notaðu svo aðferðina til að reikna $\sqrt{n}$. Framkvæmdu milliskrefið $i$ sinnum og skrifaðu svo út lokagildi $b$.

	\Input

		Á fyrstu línu er heiltalan $1 \leq T \leq 100$, sem táknar fjölda prófunartilvika sem fylgja. Hvert prófunartilvik samanstendur af einni línu með tveimur heiltölum $1 < n \leq 10000$ og $1 \leq i \leq 1000$, aðskildum með bili.

	\Output

		Fyrir hvert prófunartilvik á að skrifa út eina línu sem inniheldur lokagildi $b$.

	\Examples

	\begin{example}
\exmp{%
4
2 20
25 8
123 15
54847 360
}{%
14142115
50005
11025
234194363723809544630456655
}%
	\end{example}

\end{problem}