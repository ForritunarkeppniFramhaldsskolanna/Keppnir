\documentclass[11pt,a4paper,oneside]{article}

\usepackage[icelandic]{olymp}
\usepackage[icelandic]{babel}
\usepackage{amsmath}
\usepackage{listings}
\usepackage[utf8]{inputenc}
\usepackage[T1]{fontenc}
\usepackage{color}

\newcommand{\problemstatement}[1]{ \input ../problems/#1/statement.tex }

\title{Forritunarkeppni Framhaldsskólanna 2013}
\date{16. mars 2013}
\author{Háskólinn í Reykjavík}

\begin{document}

	\createsection{\Note}{Athugið}
	\createsection{\Explanation}{Útskýring á dæmi}
	\createsection{\Input}{Inntak}
	\createsection{\Output}{Úttak}

	\maketitle
	\thispagestyle{empty}
	\pagebreak

	\contest
	{Forritunarkeppni Framhaldsskólanna}%
	{Háskólinn í Reykjavík}%
	{16. mars 2013}%

	\problemstatement{TengdurListi}
	\problemstatement{OendanlegRuna}
	\problemstatement{FermatsLastTheorem}
	\problemstatement{SierpinskiTriangle}
	\problemstatement{KonunglegurMatur}
	\problemstatement{Frumtolur}
	\problemstatement{FatahengiSveinsLitla}
	\problemstatement{SkemmtilegarSetningar}
	\problemstatement{TicTacToe}
	\problemstatement{Bitmask}
	
	\problemstatement{Sort}
	\problemstatement{Thyngdarflokkur}
	\problemstatement{Breytunafn}
	\problemstatement{BefungeLoop}
	\problemstatement{Margfoldun}
	\problemstatement{Sqrt}
	\problemstatement{FostudagurinnThrettandi}
	\problemstatement{SamliggjandiTolur}
	\problemstatement{Maurar}
	\problemstatement{Pyramidi}

\end{document}
