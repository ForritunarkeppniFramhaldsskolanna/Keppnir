\begin{problem}
	Við erum með tvö \underline{jafn löng} orð. Við viljum breyta fyrra orðinu í seinna orðið, og eina aðgerðin sem við megum framkvæma er að breyta stafi í fyrra orðinu í einhvern annan staf. Til dæmis gætum við breytt orðinu "`\texttt{dagur}"' í orðið "`\texttt{sigur}"' með því að breyta fyrst \texttt{a}-inu í \texttt{i} (og fá þá orðið \texttt{digur}), og svo \texttt{d}-inu í \texttt{s} (og þar með enda með \texttt{sigur}). Þá er fjöldi aðgerða sem við framkvæmdum $2$.

	Skrifið forrit sem les inn tvö jafn löng orð, og skrifar út minnsta fjölda aðgerða sem þarf til að breyta fyrra orðinu í seinna orðið. Athuga skal að munur er á lágstöfum og hástöfum.

\begin{example}
\exmp{%
\underline{dagur}
\underline{sigur}
2
}%
\end{example}
\begin{example}
\exmp{\underline{Api}
\underline{afi}
2}%
\end{example}
\begin{example}
\exmp{\underline{Amma}
\underline{Amma}
0}%
\end{example}
\begin{example}
\exmp{\underline{Banani}
\underline{Forrit}
6}%
\end{example}
\end{problem}