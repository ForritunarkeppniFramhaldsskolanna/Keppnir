\begin{problem}
	
	% TODO Skólan eða skólann???
	Á hverjum degi hjólar Gunnar í skólann. Hann læsir hjólinu sínu fyrir utan með talnalás. Á talnalásnum eru fjórar talnaskífur, þar sem hver skífa hefur tölurnar frá 0 til 9. Tölurnar á hverri skífu eru raðaðar í hring, þ.e. fyrst kemur 0, svo 1, og svo alveg þar til 9 kemur, en svo kemur 0 aftur, og svo framvegis. Hægt er að snúa hverri skífu fyrir sig. Með einum snúningi er t.d. hægt að fara frá tölunnni 5 yfir í töluna 4 eða töluna 6, og frá tölunni 9 er hægt að fara yfir í töluna 8 eða töluna 0. Talnalásinn opnast þegar efsta talan á hverri talnaskífu myndar ákveðna runu (lykilorðið).

	Þar sem Gunnar þarf að aflæsa lásinn á hverjum einasta degi, þá vill hann reyna að vera eins fljótur og hann getur að opna lásinn, og hann vill því nota sem fæsta snúninga til þess.

	Skrifið forrit sem tekur inn upphafsstöðu talnalássins, og lykilorðið sem notað er til að aflæsa hann. Bæði upphafsstaðan og lykilorðið eru gefin á forminu "`\texttt{$abcd$}"', þar sem $a$ táknar efstu töluna á fyrstu skífunni, $b$ táknar efstu töluna á annarri skífunni, $c$ táknar efstu töluna á þriðju skífunni, og $d$ táknar efstu töluna á fjórðu skífunni. Forritið skrifar svo út minnsta fjölda snúninga sem þarf til að breyta upphafsstöðu talnalássins í lykilorðið, og þar með aflæsa lásinn.

\begin{example}
\exmpexpl{%
Upphafsstaða: \underline{0000}
Lykilorð: \underline{0300}
Minnsti fjöldi snúninga: 3
}{%
Nóg er að snúa annarri skífunni þrisvar sinnum til að fá lykilorðið.
}%
\end{example}
\begin{example}
\exmpexpl{%
Upphafsstaða: \underline{1578}
Lykilorð: \underline{3408}
Minnsti fjöldi snúninga: 6
}{%
% 2+1+3+0
Fyrstu skífunni er snúið tvisvar. Annarri skífunni er snúið einu sinni. Þriðju skífunni er snúið þrisvar sinnum (takið eftir að 7 er snúið að 8, svo 9, svo 0). Fjórða skífan er þegar á sínum stað. Þetta gera alls $2+1+3+0 = 6$ snúninga.
}%
\end{example}
\begin{example}
\exmp{%
Upphafsstaða: \underline{9999}
Lykilorð: \underline{0000}
Minnsti fjöldi snúninga: 4
}%
\end{example}
\end{problem}