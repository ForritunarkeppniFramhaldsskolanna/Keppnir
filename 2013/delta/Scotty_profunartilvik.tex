\documentclass[11pt,a4paper,oneside]{article}
\usepackage{amsmath}
\usepackage{listings}
\usepackage[utf8]{inputenc}
\usepackage[T1]{fontenc}
\usepackage{color}
\usepackage[icelandic]{babel}
\usepackage{enumerate}
\usepackage{graphicx}
\usepackage{wrapfig}
\usepackage{fullpage}
\usepackage{titling}
\newcommand{\subtitle}[1]{%
  \posttitle{%
    \par\end{center}
    \begin{center}\large#1\end{center}
    \vskip0.5em}%
}

\title{Forritunarkeppni Framhaldsskólanna 2013}
\subtitle{Scotty deild - prófunartilvik}
\date{16. mars 2013}
\author{Háskólinn í Reykjavík}

\newcounter{problemCounter}

\newenvironment{example}[1][]{
    \obeylines\obeyspaces\frenchspacing
    \newcommand{\exmp}[1]{
        \begin{minipage}[t]{400px}
        	\vspace{0px}
        	\texttt{##1}
        	\vspace{5px}
        \end{minipage} \\
        \hline
    }

    \newcommand{\exmpexpl}[2]{
        \begin{minipage}[t]{400px}
        	\vspace{0px}
        	\texttt{##1}
        	\vspace{10px}
        \end{minipage} \\
        \hline
        \multicolumn{1}{|c|}{\textbf{Útskýring á dæmi}} \\
        \hline
        \begin{minipage}[t]{400px}
        	\vspace{0px}
        	##2
        	\vspace{5px}
        \end{minipage} \\
        \hline
    }

    \begin{center}
    \begin{tabular}[h]{|l|}
        %\hline
        %\multicolumn{1}{|c|}{\textbf{Dæmi um virkni}} \\
        \hline
}{
    \end{tabular}
    \end{center}
}

\newenvironment{problem}[1][]{
	\stepcounter{problemCounter}
	\noindent
	\textbf{\Large Dæmi \arabic{problemCounter}}
	\nopagebreak
	\vspace{3mm}
	\nopagebreak

}{
	%\pagebreak
	\vspace{2em}

}

\begin{document}

	\maketitle
	\thispagestyle{empty}
	\pagebreak

    \section*{Fyrir hádegi}

	%\problemstatement{hallo_heimur}
    \begin{problem}
        Forritið á bara að skrifa út "`\texttt{Hello World}"'.
    \end{problem}

	%\problemstatement{namundun}
    \begin{problem}
        \begin{example}
\exmp{%
Kommutala: \underline{53.46}
Námunduð: 53
}%
        \end{example}
        \begin{example}
\exmp{%
Kommutala: \underline{6.9}
Námunduð: 7
}%
        \end{example}
        \begin{example}
\exmp{%
Kommutala: \underline{3280.5}
Námunduð: 3281
}%
        \end{example}
    \end{problem}

	%\problemstatement{nidurteljari}
    \begin{problem}
        \begin{example}
\exmp{%
Byrja í: \underline{1}
1
BÚMM!
}%
        \end{example}
        \begin{example}
\exmp{%
Byrja í: \underline{10}
10
9
8
7
6
5
4
3
2
1
BÚMM!
}%
        \end{example}
        \begin{example}
\exmp{%
Byrja í: \underline{22}
22
21
20
19
18
17
16
15
14
13
12
11
10
9
8
7
6
5
4
3
2
1
BÚMM!
}%
        \end{example}
    \end{problem}

	%\problemstatement{jolagjafir}
    \begin{problem}
        \begin{example}
\exmp{%
\underline{2}
2
}%
        \end{example}
        \begin{example}
\exmp{%
\underline{56}
3080
}%
        \end{example}
        \begin{example}
\exmp{%
\underline{42}
1722
}%
        \end{example}
    \end{problem}

	%\problemstatement{fjoldi_stafa}
    \begin{problem}
        \begin{example}
\exmp{%
\underline{Hello World}
10
}%
        \end{example}
        \begin{example}
\exmp{%
\underline{abcdX!X}
6
}%
        \end{example}
        \begin{example}
\exmp{%
\underline{jon@gmail.com}
11
}%
        \end{example}
    \end{problem}

	%\problemstatement{tveirstrengir}
    \begin{problem}
        \begin{example}
\exmp{%
\underline{Forritun}
\underline{Upphitun}
4
}%
        \end{example}
        \begin{example}
\exmp{%
\underline{undercover}
\underline{coffeecake}
10
}%
        \end{example}
        \begin{example}
\exmp{%
\underline{Testing}
\underline{testing}
1
}%
        \end{example}
    \end{problem}

	%\problemstatement{happy_birthday}
    \begin{problem}
        \begin{example}
            \exmp{%
Nafn: \underline{Andri}
Aldur: \underline{98}
Kyn: \underline{kk}

Hann á afmæli í dag,
hann á afmæli í dag.
Hann á afmæli hann Andri,
hann á afmæli í dag.\\

Hann er 98 ára í dag,
hann er 98 ára í dag.
Hann er 98 ára hann Andri,
hann er 98 ára í dag.
            }%
        \end{example}

        \begin{example}
            \exmp{%
Nafn: \underline{Bjarni}
Aldur: \underline{4}
Kyn: \underline{kk}

Hann á afmæli í dag,
hann á afmæli í dag.
Hann á afmæli hann Bjarni,
hann á afmæli í dag.\\

Hann er 4 ára í dag,
hann er 4 ára í dag.
Hann er 4 ára hann Bjarni,
hann er 4 ára í dag.
            }%
        \end{example}

        \begin{example}
            \exmp{%
Nafn: \underline{Elsa}
Aldur: \underline{42}
Kyn: \underline{kk}

Hún á afmæli í dag,
hún á afmæli í dag.
Hún á afmæli hún Elsa,
hún á afmæli í dag.\\

Hún er 98 ára í dag,
hún er 98 ára í dag.
Hún er 98 ára hún Elsa,
hún er 98 ára í dag.
            }%
        \end{example}
    \end{problem}

	%\problemstatement{deiling}
    \begin{problem}
        \begin{example}
            \exmp{%
Fyrri heiltala: \underline{64}
Seinni heiltala: \underline{4}
4 gengur 16 sinnum upp í 64
Enginn afgangur
            }%
        \end{example}

        \begin{example}
            \exmp{%
Fyrri heiltala: \underline{3719}
Seinni heiltala: \underline{3720}
3720 gengur 0 sinnum upp í 3719
Afgangur er 3719
            }%
        \end{example}

        \begin{example}
            \exmp{%
Fyrri heiltala: \underline{8434}
Seinni heiltala: \underline{233}
233 gengur 36 sinnum upp í 8434
Afgangur er 46
            }%
        \end{example}
    \end{problem}

	%\problemstatement{deiling_while}
    \begin{problem}
        \begin{example}
            \exmp{%
Fyrri heiltala: \underline{64}
Seinni heiltala: \underline{4}
4 gengur 16 sinnum upp í 64
Enginn afgangur
Viltu endurtaka vinnslu (j eða n)? \underline{j}
Fyrri heiltala: \underline{3719}
Seinni heiltala: \underline{3720}
3720 gengur 0 sinnum upp í 3719
Afgangur er 3719
Viltu endurtaka vinnslu (j eða n)? \underline{j}
Fyrri heiltala: \underline{8434}
Seinni heiltala: \underline{233}
233 gengur 36 sinnum upp í 8434
Afgangur er 46
Viltu endurtaka vinnslu (j eða n)? \underline{n}
            }%
        \end{example}
    \end{problem}

	%\problemstatement{compression}
    \begin{problem}
        \begin{example}
            \exmp{%
Óþjappað: \underline{10}
Þjappað: 10
            }%
        \end{example}

        \begin{example}
            \exmp{%
Óþjappað: \underline{15 16 17 20 21 22 23}
Þjappað: 15-17 20-23
            }%
        \end{example}

        \begin{example}
            \exmp{%
Óþjappað: \underline{5 8 9 16 24 25 26 32 33}
Þjappað: 5 8-9 16 24-26 32-33
            }%
        \end{example}
    \end{problem}

	%\problemstatement{decompression}
    \begin{problem}
        \begin{example}
            \exmp{%
Þjappað: \underline{10}
Óþjappað: 10
            }%
        \end{example}

        \begin{example}
            \exmp{%
Þjappað: \underline{15-17 20-23}
Óþjappað: 15 16 17 20 21 22 23
            }%
        \end{example}

        \begin{example}
            \exmp{%
Þjappað: \underline{5 8-9 16 24-26 32-33}
Óþjappað: 5 8 9 16 24 25 26 32 33
            }%
        \end{example}
    \end{problem}

	%\problemstatement{gotumalning}
    \begin{problem}
        \begin{example}
            \exmp{%
Fjöldi raða: \underline{2}
Fjöldi dálka: \underline{4}
\#\#\#\#\#\#\#\#\#\#\#\#\#\#\#\#\#\#\#\#\#
\#    \#    \#    \#    \#
\#    \#    \#    \#    \#
\#    \#    \#    \#    \#
\#    \#    \#    \#    \#
\#    \#    \#    \#    \#
\#    \#    \#    \#    \#\\
\vspace{8mm}
\#    \#    \#    \#    \#
\#    \#    \#    \#    \#
\#    \#    \#    \#    \#
\#    \#    \#    \#    \#
\#    \#    \#    \#    \#
\#    \#    \#    \#    \#
\#\#\#\#\#\#\#\#\#\#\#\#\#\#\#\#\#\#\#\#\#
            }%
        \end{example}

        \begin{example}
            \exmp{%
Fjöldi raða: \underline{3}
Fjöldi dálka: \underline{6}
\#\#\#\#\#\#\#\#\#\#\#\#\#\#\#\#\#\#\#\#\#\#\#\#\#\#\#\#\#\#\#
\#    \#    \#    \#    \#    \#    \#
\#    \#    \#    \#    \#    \#    \#
\#    \#    \#    \#    \#    \#    \#
\#    \#    \#    \#    \#    \#    \#
\#    \#    \#    \#    \#    \#    \#
\#    \#    \#    \#    \#    \#    \#\\
\vspace{8mm}
\#    \#    \#    \#    \#    \#    \#
\#    \#    \#    \#    \#    \#    \#
\#    \#    \#    \#    \#    \#    \#
\#    \#    \#    \#    \#    \#    \#
\#    \#    \#    \#    \#    \#    \#
\#    \#    \#    \#    \#    \#    \#
\#\#\#\#\#\#\#\#\#\#\#\#\#\#\#\#\#\#\#\#\#\#\#\#\#\#\#\#\#\#\#
\#    \#    \#    \#    \#    \#    \#
\#    \#    \#    \#    \#    \#    \#
\#    \#    \#    \#    \#    \#    \#
\#    \#    \#    \#    \#    \#    \#
\#    \#    \#    \#    \#    \#    \#
\#    \#    \#    \#    \#    \#    \#
            }%
        \end{example}

        \begin{example}
            \exmp{%
Fjöldi raða: \underline{1}
Fjöldi dálka: \underline{10}
\#\#\#\#\#\#\#\#\#\#\#\#\#\#\#\#\#\#\#\#\#\#\#\#\#\#\#\#\#\#\#\#\#\#\#\#\#\#\#\#\#\#\#\#\#\#\#\#\#\#\#
\#    \#    \#    \#    \#    \#    \#    \#    \#    \#    \#
\#    \#    \#    \#    \#    \#    \#    \#    \#    \#    \#
\#    \#    \#    \#    \#    \#    \#    \#    \#    \#    \#
\#    \#    \#    \#    \#    \#    \#    \#    \#    \#    \#
\#    \#    \#    \#    \#    \#    \#    \#    \#    \#    \#
\#    \#    \#    \#    \#    \#    \#    \#    \#    \#    \#
            }%
        \end{example}
    \end{problem}

    %\problemstatement{talnaspirall}
    \begin{problem}
        \begin{example}
            \exmp{%
Stærð: \underline{13}\\
\  144  143  142  141  140  139  138  137  136  135  134  133  132
\  145  100   99   98   97   96   95   94   93   92   91   90  131
\  146  101   64   63   62   61   60   59   58   57   56   89  130
\  147  102   65   36   35   34   33   32   31   30   55   88  129
\  148  103   66   37   16   15   14   13   12   29   54   87  128
\  149  104   67   38   17    4    3    2   11   28   53   86  127
\  150  105   68   39   18    5    0    1   10   27   52   85  126
\  151  106   69   40   19    6    7    8    9   26   51   84  125
\  152  107   70   41   20   21   22   23   24   25   50   83  124
\  153  108   71   42   43   44   45   46   47   48   49   82  123
\  154  109   72   73   74   75   76   77   78   79   80   81  122
\  155  110  111  112  113  114  115  116  117  118  119  120  121
\  156  157  158  159  160  161  162  163  164  165  166  167  168
            }%
        \end{example}

        \begin{example}
            \exmp{%
Stærð: \underline{119}\\
Ekki lögleg stærð!
            }%
        \end{example}

        \begin{example}
            \exmp{%
Stærð: \underline{9}\\
\   64   63   62   61   60   59   58   57   56
\   65   36   35   34   33   32   31   30   55
\   66   37   16   15   14   13   12   29   54
\   67   38   17    4    3    2   11   28   53
\   68   39   18    5    0    1   10   27   52
\   69   40   19    6    7    8    9   26   51
\   70   41   20   21   22   23   24   25   50
\   71   42   43   44   45   46   47   48   49
\   72   73   74   75   76   77   78   79   80
            }%
        \end{example}
    \end{problem}

	%\problemstatement{ex}
    \begin{problem}
        \begin{example}
            \exmp{%
x = \underline{13.2}
e \^{} 13.2 = 540364.9372
            }%
        \end{example}

        \begin{example}
            \exmp{%
x = \underline{0.5}
e \^{} 0.5 = 1.6487
            }%
        \end{example}

        \begin{example}
            \exmp{%
x = \underline{4.83}
e \^{}  = 125.211
            }%
        \end{example}
    \end{problem}

    %\problemstatement{hamming_distance}
    \begin{problem}
        \begin{example}
            \exmp{%
Fyrri tala: \underline{23}
Seinni tala: \underline{992}
Hamming fjarĺægð er 9
            }%
        \end{example}

        \begin{example}
            \exmp{%
Fyrri tala: \underline{16}
Seinni tala: \underline{17}
Hamming fjarĺægð er 1
            }%
        \end{example}

        \begin{example}
            \exmp{%
Fyrri tala: \underline{99}
Seinni tala: \underline{66}
Hamming fjarĺægð er 2
            }%
        \end{example}
    \end{problem}

	%\problemstatement{stigatafla}
    \begin{problem}
        \begin{example}
            \exmp{%
\underline{888888}

@@@ @@@ @@@ @@@ @@@ @@@
@ @ @ @ @ @ @ @ @ @ @ @
@@@ @@@ @@@ @@@ @@@ @@@
@ @ @ @ @ @ @ @ @ @ @ @
@@@ @@@ @@@ @@@ @@@ @@@
            }%
        \end{example}

        \begin{example}
            \exmp{%
\underline{128901}

\  @ @@@ @@@ @@@ @@@   @
\  @   @ @ @ @ @ @ @   @
\  @ @@@ @@@ @@@ @ @   @
\  @ @   @ @   @ @ @   @
\  @ @@@ @@@   @ @@@   @
            }%
        \end{example}

        \begin{example}
            \exmp{%
\underline{42}

@ @ @@@
@ @   @
@@@ @@@
\  @ @  
\  @ @@@
            }%
        \end{example}
    \end{problem}

	%\problemstatement{logleg_nofn}
    \begin{problem}
        \begin{center}
        \begin{tabular}{|l|}
        \hline
        \multicolumn{1}{|c|}{\textbf{nofn.txt}}\\
        \hline
        \texttt{Jón kk}\\
        \hline
        \end{tabular}
        \end{center}

        \begin{example}
\exmp{%
Skrá: \underline{nofn1.txt}
mannanafn: \underline{Jón}
Löglegt
mannanafn: \underline{Jón Jón}
Löglegt
mannanafn: \underline{Jón Þór}
Ekki löglegt
mannanafn: \underline{Þór}
Ekki löglegt
mannanafn: \underline{Þór Jón}
Ekki löglegt
mannanafn: \underline{bless}
}%
        \end{example}

        \begin{center}
        \begin{tabular}{|l|}
        \hline
        \multicolumn{1}{|c|}{\textbf{nofn2.txt}}\\
        \hline
        \texttt{Elsa kvk}\\
        \texttt{Jón kk}\\
        \texttt{Þór kk}\\
        \texttt{Beta kvk}\\
        \hline
        \end{tabular}
        \end{center}
        
        \begin{example}
\exmp{%
Skrá: \underline{nofn2.txt}
mannanafn: \underline{Beta Þór}
Ekki löglegt
mannanafn: \underline{Beta Elsa}
Löglegt
mannanafn: \underline{Tómatur Elsa}
Ekki löglegt
mannanafn: \underline{Jón Elsa}
Ekki löglegt
mannanafn: \underline{Þór Jón}
Löglegt
mannanafn: \underline{bless}
}%
        \end{example}

        \begin{center}
        \begin{tabular}{|l|}
        \hline
        \multicolumn{1}{|c|}{\textbf{nofn3.txt}}\\
        \hline
        \texttt{Api kk}\\
        \texttt{Banani kvk}\\
        \texttt{Epli kvk}\\
        \hline
        \end{tabular}
        \end{center}
        
        \begin{example}
\exmp{%
Skrá: \underline{nofn3.txt}
mannanafn: \underline{Abi}
Ekki löglegt
mannanafn: \underline{Banani Epli}
Löglegt
mannanafn: \underline{Epli Api}
Ekki löglegt
mannanafn: \underline{Api Api}
Löglegt
mannanafn: \underline{Banani}
Löglegt
mannanafn: \underline{bless}
}%
        \end{example}
    \end{problem}

	%\problemstatement{stysta_leid_skak}
    \begin{problem}
        \begin{example}
            \exmp{%
Upphafsreitur: \underline{a1}
Endareitur: \underline{h7}
\textit{hér á að koma leið með 5 hoppum, til dæmis:}
Stysta leið: a1 c2 e3 g4 f6 h7
            }%
        \end{example}

        \begin{example}
            \exmp{%
Upphafsreitur: \underline{d8}
Endareitur: \underline{e4}
\textit{hér á að koma leið með 3 hoppum, til dæmis:}
Stysta leið: d8 f7 d6 e4
            }%
        \end{example}

        \begin{example}
            \exmp{%
Upphafsreitur: \underline{g1}
Endareitur: \underline{b6}
\textit{hér á að koma leið með 4 hoppum, til dæmis:}
Stysta leið: g1 e2 c3 a4 b6
            }%
        \end{example}
    \end{problem}

    \section*{Eftir hádegi}

    \setcounter{problemCounter}{0}
    
    %\problemstatement{veldi_af_2}
    \begin{problem}
        \begin{example}
            \exmp{%
n = \underline{0}
2\^{}0 = 1
            }%
        \end{example}

        \begin{example}
            \exmp{%
n = \underline{1}
2\^{}0 = 1
2\^{}1 = 2
            }%
        \end{example}

        \begin{example}
            \exmp{%
n = \underline{16}
2\^{}0 = 1
2\^{}1 = 2
2\^{}2 = 4
2\^{}3 = 8
2\^{}4 = 16
2\^{}5 = 32
2\^{}6 = 64
2\^{}7 = 128
2\^{}8 = 256
2\^{}9 = 512
2\^{}10 = 1024
2\^{}11 = 2048
2\^{}12 = 4096
2\^{}13 = 8192
2\^{}14 = 16384
2\^{}15 = 32768
2\^{}16 = 65536
            }%
        \end{example}
    \end{problem}

    %\problemstatement{ofugur_texti}
    \begin{problem}
        \begin{example}
            \exmp{%
\underline{Test}
tseT
            }%
        \end{example}

        \begin{example}
            \exmp{%
\underline{a b c d e f}
a b c d e f
            }%
        \end{example}

        \begin{example}
            \exmp{%
\underline{ab cd ef}
ba dc fe
            }%
        \end{example}
    \end{problem}

    %\problemstatement{kula_i_kassa}
    \begin{problem}
        \begin{example}
\exmp{%
Breidd: \underline{5284.122}
Lengd: \underline{8172.8}
Hæð: \underline{4447.871}
Stærsta r = 2223.9355
}%
        \end{example}

        \begin{example}
\exmp{%
Breidd: \underline{3}
Lengd: \underline{5}
Hæð: \underline{4}
Stærsta r = 1.5
}%
        \end{example}

        \begin{example}
\exmp{%
Breidd: \underline{0}
Lengd: \underline{0}
Hæð: \underline{0}
Stærsta r = 0
}%
        \end{example}
    \end{problem}

    %\problemstatement{lykilord}
    \begin{problem}
        \begin{example}
            \exmp{%
Lykilorð: \underline{hello}
Lykilorðið er veikt
            }%
        \end{example}

        \begin{example}
            \exmp{%
Lykilorð: \underline{!@xYhello}
Lykilorðið er sterkt
            }%
        \end{example}

        \begin{example}
            \exmp{%
Lykilorð: \underline{@}
Lykilorðið er gott
            }%
        \end{example}
    \end{problem}

    %\problemstatement{kelvin_fahrenheit}
    \begin{problem}
        \begin{example}
            \exmp{%
1. Breyta Kelvin í Fahrenheit
2. Breyta Fahrenheit í Kelvin
3. Hætta
Val: \underline{2}
Fahrenheit = \underline{100}
Kelvin = 310.93\\

1. Breyta Kelvin í Fahrenheit
2. Breyta Fahrenheit í Kelvin
3. Hætta
Val: \underline{1}
Kelvin = \underline{310}
Fahrenheit = 98.33\\

1. Breyta Kelvin í Fahrenheit
2. Breyta Fahrenheit í Kelvin
3. Hætta
Val: \underline{3}
            }%
        \end{example}
    \end{problem}

    %\problemstatement{oradad}
    \begin{problem}
        \begin{example}
            \exmp{%
List: \underline{1 2 3}
\textit{hér á að koma umröðun af listanum 1 2 3 sem er ekki röðuð í hækkandi né lækkandi röð, til dæmis:}
1 3 2
            }%
        \end{example}

        \begin{example}
            \exmp{%
List: \underline{5 4 6}
\textit{hér á að koma umröðun af listanum 5 4 6 sem er ekki röðuð í hækkandi né lækkandi röð, til dæmis:}
5 4 6
            }%
        \end{example}

        \begin{example}
            \exmp{%
List: \underline{13 8 6 4 2}
\textit{hér á að koma umröðun af listanum 13 8 6 4 2 sem er ekki röðuð í hækkandi né lækkandi röð, til dæmis:}
2 8 4 6 13
            }%
        \end{example}
    \end{problem}

    %\problemstatement{lodrettur_texti}
    \begin{problem}
        \begin{example}
            \exmp{%
Fjöldi lína: \underline{3}
\underline{123456}
\underline{321}
\underline{123456}\\

131
222
313
4 4
5 5
6 6
            }%
        \end{example}

        \begin{example}
            \exmp{%
Fjöldi lína: \underline{1}
\underline{hello}\\

h
e
l
l
o
            }%
        \end{example}
        
        \begin{example}
            \exmp{%
Fjöldi lína: \underline{2}
\underline{test}
\underline{forritunarkeppnin}\\

tf
eo
sr
tr
\ i
\ t
\ u
\ n
\ a
\ r
\ k
\ e
\ p
\ p
\ n
\ i
\ n
            }%
        \end{example}
    \end{problem}

    %\problemstatement{fizzbuzz}
    \begin{problem}
        \begin{example}
            \exmp{%
n = \underline{1}
1
            }%
        \end{example}

        \begin{example}
            \exmp{%
n = \underline{3}
1
2
Fizz
            }%
        \end{example}

        \begin{example}
            \exmp{%
n = \underline{30}
1
2
Fizz
4
Buzz
Fizz
7
8
Fizz
Buzz
11
Fizz
13
14
FizzBuzz
16
17
Fizz
19
Buzz
Fizz
22
23
Fizz
Buzz
26
Fizz
28
29
FizzBuzz
            }%
        \end{example}
    \end{problem}

    %\problemstatement{inniheldur_tolu}
    \begin{problem}
        \begin{example}
            \exmp{%
Lína: \underline{1337 is elite}
Já
            }%
        \end{example}

        \begin{example}
            \exmp{%
Lína: \underline{42=43}
Nei
            }%
        \end{example}

        \begin{example}
            \exmp{%
Lína: \underline{pi is about 3.1415}
Nei
            }%
        \end{example}
    \end{problem}

    %\problemstatement{dagatal}
    \begin{problem}
        \begin{example}
            \exmp{%
Dagar í mánuðinum: \underline{29}
Fyrsti dagur mánaðarins: \underline{1}\\

Sun Mán Þri Mið Fim Fös Lau
\  1   2   3   4   5   6   7 
\  8   9  10  11  12  13  14 
\ 15  16  17  18  19  20  21 
\ 22  23  24  25  26  27  28 
\ 29 
            }%
        \end{example}

        \begin{example}
            \exmp{%
Dagar í mánuðinum: \underline{31}
Fyrsti dagur mánaðarins: \underline{7}\\

Sun Mán Þri Mið Fim Fös Lau
\                          1 
\  2   3   4   5   6   7   8 
\  9  10  11  12  13  14  15 
\ 16  17  18  19  20  21  22 
\ 23  24  25  26  27  28  29 
\ 30  31 
            }%
        \end{example}

        \begin{example}
            \exmp{%
Dagar í mánuðinum: \underline{31}
Fyrsti dagur mánaðarins: \underline{7}\\

Sun Mán Þri Mið Fim Fös Lau
\                  1   2   3 
\  4   5   6   7   8   9  10 
\ 11  12  13  14  15  16  17 
\ 18  19  20  21  22  23  24 
\ 25  26  27  28  29  30
            }%
        \end{example}
    \end{problem}

    %\problemstatement{slongu_texti}
    \begin{problem}
        \begin{example}
            \exmp{%
Texti: \underline{The answer is 42.}\\

\            s 4
\       w r i   2
\      s e       .
\ h   n
T e a

            }%
        \end{example}

        \begin{example}
            \exmp{%
Texti: \underline{abcdefghijklmno}\\

\              o
\             n
\            m
\           l
\          k
\         j
\        i
\       h
\      g
\     f
\    e
\   d
\  c
\ b
a
            }%
        \end{example}

        \begin{example}
            \exmp{%
Texti: \underline{lolololololololololololol}\\

\ o o o o o o o o o o o o
l l l l l l l l l l l l l
            }%
        \end{example}
    \end{problem}

    %\problemstatement{hjolalas}
    \begin{problem}
        \begin{example}
            \exmp{%
Upphafsstaða: \underline{2584}
Lykilorð: \underline{6849}
Minnsti fjöldi snúninga: 16
            }%
        \end{example}

        \begin{example}
            \exmp{%
Upphafsstaða: \underline{3899}
Lykilorð: \underline{9124}
Minnsti fjöldi snúninga: 15
            }%
        \end{example}

        \begin{example}
            \exmp{%
Upphafsstaða: \underline{9714}
Lykilorð: \underline{9532}
Minnsti fjöldi snúninga: 6
            }%
        \end{example}
    \end{problem}

    %\problemstatement{freq_analysis}
    \begin{problem}
        \begin{example}
            \exmp{%
\underline{Forritunarkeppnin er awesome.}
e 15.3846\%{}
r 15.3846\%{}
n 11.5385\%{}
a 7.69231\%{}
i 7.69231\%{}
o 7.69231\%{}
p 7.69231\%{}
f 3.84615\%{}
k 3.84615\%{}
m 3.84615\%{}
s 3.84615\%{}
t 3.84615\%{}
u 3.84615\%{}
w 3.84615\%{}
            }%
        \end{example}

        \begin{example}
            \exmp{%
\underline{test}
t 50\%{}
e 25\%{}
s 25\%{}
            }%
        \end{example}

        \begin{example}
            \exmp{%
\underline{lollipop}
l 37.5\%{}
o 25\%{}
p 25\%{}
i 12.5\%{}
            }%
        \end{example}
    \end{problem}

    %\problemstatement{fullkomintala}
    \begin{problem}
        \begin{example}
            \exmp{%
n = \underline{120}
Ekki fullkomin
            }%
        \end{example}

        \begin{example}
            \exmp{%
n = \underline{8128}
Fullkomin
            }%
        \end{example}

        \begin{example}
            \exmp{%
n = \underline{2096128}
Ekki fullkomin
            }%
        \end{example}
    \end{problem}

    %\problemstatement{translate}
    \begin{problem}
        \begin{example}
            \exmp{%
\underline{leg=ben}
\underline{carrot=gulerod}
\underline{sulten=hungry}
\underline{er=am}
\underline{jeg=i}
\underline{text:}
jeg er sulten
i am hungry
            }%
        \end{example}

        \begin{example}
            \exmp{%
\underline{my=mein}
\underline{programming=programmierung}
\underline{where=wo}
\underline{car=auto}
\underline{is=ist}
\underline{text:}
where is my car
wo ist mein auto
            }%
        \end{example}

        \begin{example}
            \exmp{%
\underline{x=y}
\underline{y=z}
\underline{z=x}
\underline{text:}
\underline{x x y z y x z}
y y z x z y x
            }%
        \end{example}
    \end{problem}

    %\problemstatement{bmi}
    \begin{problem}
        \begin{example}
            \exmp{%
Fjöldi: \underline{3}\\

Nafn 1: \underline{John}
Hæð 1: \underline{1.5}
Þyngd 1: \underline{60}\\

Nafn 2: \underline{Jane}
Hæð 2: \underline{1.4}
Þyngd 2: \underline{55}\\

Nafn 3: \underline{Bob}
Hæð 3: \underline{2.0}
Þyngd 3: \underline{95}\\

Bob 23.8
John 26.7
Jane 28.1
            }%
        \end{example}

        \begin{example}
            \exmp{%
Fjöldi: \underline{2}\\

Nafn 1: \underline{Alice}
Hæð 1: \underline{1.85}
Þyngd 1: \underline{75}\\

Nafn 2: \underline{Bob}
Hæð 2: \underline{1.96}
Þyngd 2: \underline{94}\\

Alice 21.9
Bob 24.5
            }%
        \end{example}

        \begin{example}
            \exmp{%
Fjöldi: \underline{1}\\

Nafn 1: \underline{Abc}
Hæð 1: \underline{2.4}
Þyngd 1: \underline{135}\\

Abc 23.4
            }%
        \end{example}
    \end{problem}

    %\problemstatement{stae_pusl}
    \begin{problem}
        \begin{example}
            \exmp{%
Þraut:
\underline{\   GREEN}
\underline{+ ORANGE}
\underline{= COLORS}
Lausn:
\   83446
+ 135684
= 219130
            }%
        \end{example}

        \begin{example}
            \exmp{%
Þraut:
\underline{\   HVAR}
\underline{+    ER}
\underline{= HELGI}
Lausn:
Engin lausn
            }%
        \end{example}

        \begin{example}
            \exmp{%
Þraut:
\underline{\   CROSS}
\underline{+  ROADS}
\underline{= DANGER}
Lausn:
\   96233
+  62512
= 158746
            }%
        \end{example}
    \end{problem}

    %\problemstatement{samlagning}
    \begin{problem}
        \begin{example}
            \exmp{%
Tölur: \underline{8 7 9 45}
Samtals: 69
Minnsti kostnaður: 108
            }%
        \end{example}

        \begin{example}
            \exmp{%
Tölur: \underline{9 7 8 37 69 42 85 80}
Samtals: 337
Minnsti kostnaður: 877
            }%
        \end{example}

        \begin{example}
            \exmp{%
Tölur: \underline{5 -8 -2 14 35 66}
Samtals: 110
Minnsti kostnaður: 148
            }%
        \end{example}
    \end{problem}

\end{document} 