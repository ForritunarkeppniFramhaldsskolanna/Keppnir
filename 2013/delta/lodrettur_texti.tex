\begin{problem}
	Skrifið forrit sem les inn allt að 15 línur af texta. Forritið skrifar svo línurnar aftur út, nema lóðréttar.

\begin{example}
\exmp{%
Fjöldi lína: \underline{4}
\underline{Halló heimur!}
\underline{Testing 123}
\underline{4567}
\underline{Lol lol lol}\\

H T 4 L
a e 5 o
l s 6 l
l t 7 
ó i   l
\  n   o
h g   l
e      
i 1   l
m 2   o
u 3   l
r
!
}%
\end{example}
\end{problem}