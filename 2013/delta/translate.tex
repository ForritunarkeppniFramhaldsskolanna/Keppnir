\begin{problem}
	Mikið er til af (misgóðum) forritum til að þýða texta af einu tungumáli yfir á annað. Í þessu dæmi eigið þið að útfæra mjög einfalda útgáfu af svona þýðanda, en hann þýðir textann orð fyrir orð með notkun orðabókar sem hann les inn.

	Skrifið forrit sem les inn orðabók og texta. Orðabókin er listi af línum á forminu "`\texttt{XXX=YYY}"', þar sem \texttt{XXX} er orð á upphaflega tungumálinu, og \texttt{YYY} er samsvarandi orð á tungumálinu sem þýða á yfir í. Gera má ráð fyrir að hvert orð komi aðeins einu sinni fyrir, og að hvert orð í textanum komi fyrir í orðabókinni. Síðasta lína orðabókarinnar inniheldur aðeins "`\texttt{text:}"'. Eftir að orðabókin hefur verið lesin inn á að lesa eina línu sem inniheldur textann sem á að þýða. Forritið skrifar svo út þýdda textann. Gera má ráð fyrir að allir stafir séu lágstafir, bæði í orðabókinni og í textanum, og að engin tákn önnur en bókstafir og bil komi fyrir.

\begin{example}
\exmp{%
\underline{halló=hello}
\underline{heimur=world}
\underline{text:}
\underline{halló heimur}
hello world
}%
\end{example}
\begin{example}
\exmp{%
\underline{hello=halló}
\underline{world=heimur}
\underline{text:}
\underline{hello world}
halló heimur
}%
\end{example}

\begin{example}
\exmp{%
\underline{funny=fyndið}
\underline{this=þetta}
\underline{is=er}
\underline{lol=haha}
\underline{text:}
\underline{lol this is funny lol lol}
haha þetta er fyndið haha haha
}%
\end{example}
\end{problem}