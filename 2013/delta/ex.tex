\begin{problem}
	Stærðfræðilega talan $e$ er um það bil $2.718281828\ldots$\\
	Þessa tölu er hægt að reikna með óendanlegu summunni $$e = \frac{1}{0!} + \frac{1}{1!} + \frac{1}{2!} + \dotsm + \frac{1}{n!} + \dotsm$$
	Þessi formúla er sértilvik af formúlunni til að reikna $e^x$ þegar $x = 1$, en sú formúla er óendanlega summan $$e^x = \frac{x^0}{0!} + \frac{x^1}{1!} + \frac{x^2}{2!} + \dotsm + \frac{x^n}{n!} + \dotsm$$

	Skrifið forrit sem spyr um kommutöluna $x$ ($x > 0$) og reiknar nálgun á $e^x$ með því að nota 20 fyrstu liðina í formúlunni að ofan. Forritið skrifar svo útkomuna á skjáinn, námundað að fjórum aukastöfum.

	Athuga skal að $n!$, eða $n$ hrópmerkt (e. factorial), er margfeldið af fyrstu $n$ náttúrulegu tölunum, svo $n! = 1\cdot 2\cdot \dotsm \cdot (n-1) \cdot n$. Þegar $n=0$, þá er $n!$ skilgreint sem $1$.

\begin{example}
\exmp{%
x = \underline{3.5}
e \^{} 3.5 = 33.1155
}%
\end{example}
\begin{example}
\exmp{%
x = \underline{1}
e \^{} 1 = 2.7183
}%
\end{example}
\begin{example}
\exmp{%
x = \underline{0.23}
e \^{} 0.23 = 1.2586
}%
\end{example}
\end{problem}