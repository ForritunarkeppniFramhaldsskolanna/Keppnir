\begin{problem}
	Þegar maður velur sér lykilorð er mikilvægt að það hafi ákveðna eiginleika svo það sé öruggt. Við skulum segja að "`sterkt"' lykilorð hafi alla þrjá af eftirtöldum eiginleikum:

	\begin{itemize}
		\item Sé að minnsta kosti átta stafir að lengd (dæmi: "`\texttt{lykilord}"')
		\item Innihaldi bæði há- og lágstafi (dæmi: "`\texttt{lykiLORD}"')
		\item Innihaldi bókstafi og að minnsta kosti einn tölustaf eða tákn (dæmi: "`\texttt{lyk1L0RD}"' eða "`\texttt{lykiL\#{}RD}"')
	\end{itemize}

	Við skulum kalla lykilorð "`gott"' ef það hefur tvo af eiginleikunum, en "`ásættanlegt"' ef það hefur aðeins einn af eiginleikunum. Lykilorð sem hefur enga af eiginleikunum skulum við kalla "`veikt"'.

	Skrifið forrit sem les inn lykilorð og tilgreinir hvort það sé "`sterkt"', "`gott"', "`ásættanlegt"', eða "`veikt"'.

\begin{example}
\exmp{%
Lykilorð: \underline{lykill}
Lykilorðið er veikt
}%
\end{example}
\begin{example}
\exmp{%
Lykilorð: \underline{lykilord}
Lykilorðið er ásættanlegt
}%
\end{example}
\begin{example}
\exmp{%
Lykilorð: \underline{Lykilord}
Lykilorðið er gott
}%
\end{example}
\begin{example}
\exmp{%
Lykilorð: \underline{Lyki10rd}
Lykilorðið er sterkt
}%
\end{example}
\end{problem}