\begin{problem}
	Dæmi í forritunarkeppnum segja oft keppendum nákvæmlega hvað þeir eiga að gera. Þetta finnst okkur ekki alveg nógu sniðugt. Í þessu dæmi ætlum við gera hið gagnstæða. Við ætlum að segja ykkur hvað þið megið ekki gera:

	\begin{center}
		\textit{Þið megið ekki raða listanum af heiltölunum sem þið lesið inn.}
	\end{center}

	Skrifið forrit sem les inn eina línu sem inniheldur lista af ólíkum heiltölum, aðskildum með bili. Gera má ráð fyrir að listinn innihaldi að minnsta kosti þrjár tölur.

	Forritið á að skrifa listann aftur út, á sama formi, með stökunum í hvaða röð sem er. Eina skilyrðið er að listinn sé ekki raðaður, hvorki í hækkandi né lækkandi röð.

\begin{example}
\exmp{%
Listi: \underline{1 2 3 4 5}
3 5 1 4 2
}%
\end{example}
\begin{example}
\exmp{%
Listi: \underline{8 100 15 28 17 3 2 88}
8 100 15 28 17 3 2 88
}%
\end{example}
\end{problem}