\begin{problem}
	Ein leið til að afkóða dulkóðaðan texta er að nota tíðnigreiningu á dulkóðaða textann. Þá er oft hægt að giska á hvað hver dulkóðaður stafur var í upprunalega textanum. Til dæmis er líklegt að stafurinn sem kemur oftast fyrir í dulkóðaða textanum sé, í upprunalega textanum, stafurinn sem kemur oftast fyrir í tungumálinu (t.d. stafurinn A í íslensku eða E í ensku).

	Skrifið forrit sem les inn eina línu af dulkóðuðum texta, og skrifar út þá bókstafi sem koma fyrir í textanum og tíðnina á þeim bókstafi í prósentum (tíðni bókstafs er hversu oft sá bókstafur kemur fyrir deilt með fjölda bókstafa í textanum). Ekki er gerður greinarmunur á há- og lágstöfum. Raða á bókstöfunum eftir tíðni í minnkandi röð, og svo eftir bókstöfum í stafrófsröð.

\begin{example}
\exmp{%
\underline{Hello World}
l 30\%
o 20\%
d 10\%
e 10\%
h 10\%
r 10\%
w 10\%
}%
\end{example}
\begin{example}
\exmp{%
\underline{Ryppe gsrkyi hmeq qexxmw svgm vyxvyq fperhmx.}
e 10.5263\%
m 10.5263\%
x 10.5263\%
y 10.5263\%
p 7.89474\%
q 7.89474\%
r 7.89474\%
v 7.89474\%
g 5.26316\%
h 5.26316\%
s 5.26316\%
f 2.63158\%
i 2.63158\%
k 2.63158\%
w 2.63158\%
}%
\end{example}
\end{problem}