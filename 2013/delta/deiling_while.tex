\begin{problem}
	Búið til forrit sem spyr um tvær heiltölur. Forritið á svo að skrifa út hversu oft seinni talan gengur upp í fyrri töluna, og svo hve mikill afgangur er eftir við deilinguna. Ef enginn afgangur er eftir, þá á forritið að birta það.

	Forritið býður svo notanda að endurtaka vinnsluna þar til notandi kýs að hætta.

\begin{example}
\exmp{%
Fyrri heiltala: \underline{60}
Seinni heiltala: \underline{7}
7 gengur 8 sinnum upp í 60
Afgangur er 4
Viltu endurtaka vinnslu (j eða n)? \underline{j}
Fyrri heiltala: \underline{60}
Seinni heiltala: \underline{15}
15 gengur 4 sinnum upp í 60
Enginn afgangur
Viltu endurtaka vinnslu (j eða n)? \underline{j}
Fyrri heiltala: \underline{82742576}
Seinni heiltala: \underline{5582}
5582 gengur 14823 sinnum upp í 82742576
Afgangur er 590
Viltu endurtaka vinnslu (j eða n)? \underline{n}
}%
\end{example}
\end{problem}