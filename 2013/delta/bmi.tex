\begin{problem}
	Skrifið forrit sem les inn heiltöluna $n$. Forritið á svo að lesa inn nafn, hæð (í metrum), og þyngd (í kílóum) á $n$ manneskjum. Forritið á svo, fyrir hverja manneskju, að skrifa út nafn hennar og BMI stig. Þessi listi á að vera raðaður í hækkandi röð eftir BMI stigi. Ef manneskja hefur hæð $h$ og þyngd $w$, þá er BMI stig hennar reiknað með formúlunni
	\[
		\mathrm{BMI} = \frac{w}{h^2}
	\]

\begin{example}
\exmp{%
Fjöldi: \underline{5}\\

Nafn 1: \underline{Jónas}
Hæð 1: \underline{1.7}
Þyngd 1: \underline{90}\\

Nafn 2: \underline{Magnús}
Hæð 2: \underline{1.8}
Þyngd 2: \underline{80}\\

Nafn 3: \underline{Bjarki}
Hæð 3: \underline{1.78}
Þyngd 3: \underline{60}\\

Nafn 4: \underline{Þórir}
Hæð 4: \underline{1.5}
Þyngd 4: \underline{50}\\

Nafn 5: \underline{Gunnar}
Hæð 5: \underline{2.1}
Þyngd 5: \underline{112}\\

Bjarki 18.9
Þórir 22.2
Magnús 24.7
Gunnar 25.4
Jónas 31.1
}%
\end{example}
\end{problem}