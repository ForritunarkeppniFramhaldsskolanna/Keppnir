\begin{problem}
	Skrifið forrit sem les inn eina línu af texta. Forritið skrifar svo textann út á eftirfarandi hátt. Fyrsta stafinn skal skrifa út í fyrsta dálknum í línunni sem við skulum kalla miðlínuna. Staðsetning fylgjandi stafa fer eftir stafnum sem kemur á undan. Ef stafurinn sem verið er að skrifa út er minni en stafurinn á undan, þá á að skrifa hann út í næsta dálk í næstu línu að ofan. Ef stafurinn er sá sami og stafurinn á undan, þá á að skrifa hann út í næsta dálk í sömu línu. Ef stafurinn er stærri en stafurinn á undan, þá á að skrifa hann út í næsta dálk í næstu línu að neðan.

	Til að bera saman tvo stafi á að bera saman ASCII gildi þeirra. Flest forritunarmál gera það sjálfkrafa þegar verið er að bera saman tvo stafi, svo líklega þarf ekki að huga sérstaklega að því.

\begin{example}
\exmp{%
Texti: \underline{helloworld}\\

\     w r
\    o o l
h ll     d
\ e
}%
\end{example}

\begin{example}
\exmp{%
Texti: \underline{3.1415926535897932384626433832795028841971693}\\

\             9 9   8 6 6
\      9 6   8 7 3 3 4 2 4  8   9   88
\   4 5 2 5 5     2       33 3 7 5 2  4 9   9
3 1 1     3                  2   0    1 7 6 3
\ .                                       1
}%
\end{example}
\end{problem}