\begin{problem}
	Tala er fullkomin ef að summa deila tölunnar (talan sjálf ekki tekin með) er jöfn tölunni.
	Tökum sem dæmi töluna $6$. Deilar hennar eru 1, 2 og 3 (við sleppum deilinum $6$). Summa deilanna er $1 + 2 + 3 = 6$ og þar af leiðandi er talan $6$ fullkomin.
	Gríski stærðfræðingurinn Evklíður sem var uppi 300 f.kr.\ fann formúlu fyrir fullkomnar \textit{sléttar} tölur. Ekki er enn vitað hvort til séu fullkomnar oddatölur.
	Formúlan var $2^{p - 1} \times (2^p - 1)$ og skilar hún fullkomnum tölum þegar bæði $p$ er frumtala og $2^p - 1$ er frumtala. Frumtala er heiltala, stærri en $0$, sem hefur enga deila nema $1$ og sjálfa sig.

	Skrifið forrit sem les inn eina heiltölu $n$ ($1 < n < 10^8$), og skrifar út "`\texttt{Fullkomin}"', ef talan er fullkomin, en "`\texttt{Ekki fullkomin}"', ef talan er ekki fullkomin.

\begin{example}
\exmp{%
n = \underline{6}
Fullkomin
}%
\end{example}
\begin{example}
\exmp{%
n = \underline{31}
Ekki fullkomin
}%
\end{example}
\begin{example}
\exmp{%
n = \underline{32}
Ekki fullkomin
}%
\end{example}
\begin{example}
\exmp{%
n = \underline{33550336}
Fullkomin
}%
\end{example}
\begin{example}
\exmp{%
n = \underline{33550338}
Ekki fullkomin
}%
\end{example}
\end{problem}