\begin{problem}
	Ef við erum með tvær tvíundatölur, þá er Hamming fjarlægð á milli þeirra skilgreind sem fjöldi bita sem eru ólíkir á milli talnanna. Til dæmis ef við höfum tvíundatölurnar 1011 (sem er talan 11 í tugakerfinu) og 110 (sem er talan 6 í tugakerfinu), þá er Hamming fjarlægð á milli þeirra þrír, þar sem aftasti bitinn (þ.e. bitinn sem er lengst til hægri) er ekki eins í báðum tölum, og sama á við um þriðju- og fjórðu öftustu bitana. Það eru þrír ólíkir bitar, og því er Hamming fjarlægðin þrír.

	Skrifið forrit sem les inn tvær jákvæðar 32-bita heiltölur sem gefnar eru á tugaformi (þ.e. í tugakerfinu), og skrifar út Hamming fjarlægðina á milli þeirra.

\begin{example}
\exmp{%
Fyrri tala: \underline{11}
Seinni tala: \underline{6}
Hamming fjarlægð er 3
}%
\end{example}
\begin{example}
\exmp{%
Fyrri tala: \underline{6}
Seinni tala: \underline{11}
Hamming fjarlægð er 3
}%
\end{example}
\begin{example}
\exmp{%
Fyrri tala: \underline{1234}
Seinni tala: \underline{1234}
Hamming fjarlægð er 0
}%
\end{example}
\begin{example}
\exmp{%
Fyrri tala: \underline{538214}
Seinni tala: \underline{12428}
Hamming fjarlægð er 13
}%
\end{example}
\end{problem}