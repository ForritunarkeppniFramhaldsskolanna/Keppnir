\problemname{Mylla}
\illustration{0.3}{tic_tac_toe}{Mynd fengin af \href{https://commons.wikimedia.org/wiki/File:Tic-tac-toe_mural_in_Gy\%C3\%B6r.jpg}{wikimedia.org}}%
Uppáhaldsspilið hans Hjalta er mylla. Leikurinn virkar þannig að tveir leikmenn
skiptast á að setja niður tákn á $3\times 3$ töflu, en þeir mega ekki setja
tákn á reit þar sem tákn hefur þegar verið sett. Leikmaðurinn sem byrjar notar
táknið \texttt{X}, en hinn leikmaðurinn táknið \texttt{O}. Leikmaður vinnur ef
hann nær að setja þrjú eintök af tákninu sínu í beina röð, hvort sem það er
lárétt, lóðrétt eða á ská. Leikurinn endar þegar leikmaður vinnur, eða í
jafntefli ef engir tómir reitir eru eftir.

Hjalti er mikil keppnismanneskja og elskar ekkert meira en að vinna Guðjón í myllu. Hjalti segir að hann sé miklu betri í myllu en Guðjón og þess vegna leyfir hann Guðjóni alltaf að byrja.
Þar sem það er mikill rígur á milli þeirra þá treysta þeir ekki hvorum öðrum að fara yfir niðurstöðuna og því biður Hjalti þig, hlutlausan þriðja aðila, um að fara yfir niðurstöðuna.
Hjalta er alveg sama hvort Guðjón vann eða ekki, það eina sem skiptir hann máli er hvort hann sjálfur vann.

\section*{Inntak}
Inntakið er á þremur línum, þar sem hver lína er ein röð í leikborðinu.
Hver röð getur innihaldið táknin \texttt{X}, \texttt{O} eða \texttt{\_}, þar sem \texttt{X} eru reitirnir hans Guðjóns, \texttt{O} reitirnir hans Hjalta og \texttt{\_} tómu reitirnir.
Inntakið er alltaf gild lokastaða.

\section*{Úttak}
Skrifið \texttt{Jebb} ef Hjalti vann annars \texttt{Neibb}.

\section*{Stigagjöf}
\begin{tabular}{|l|l|l|}
\hline
Hópur & Stig & Takmarkanir \\ \hline
1     & 100   & Engar frekari takmarkanir \\ \hline
\end{tabular}

