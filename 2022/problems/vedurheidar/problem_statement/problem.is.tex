\problemname{Veður - Lokaðar heiðar}
\illustration{0.3}{vedur_heidar_visir}{Mynd fengin af \href{https://www.visir.is/g/20222229194d/buid-ad-opna-vegina-um-hellisheidi-og-threngsli}{visir.is}. Ljósmyndari: Vilhelm Gunnarsson}

Núna er enn ein lægðin búin að leggjast yfir landið. Vegagerðin heyrir í þér til að hjálpa við að ákvarða hvaða heiðum þarf að loka.

Til að einfalda er sami vindhraði allstaðar á landinu. Þú færð gefinn vindhraðann á landinu, lista af heiðum, og við hvaða vindhraða er öruggt að ferðast um hverja heiði, í mesta lagi.


\section*{Inntak}
Fyrsta línan í inntakinu inniheldur eina heiltölu $v$ ($0 \leq v \leq 200$), vindhraðann á Íslandi.
Önnur línan í inntakinu inniheldur eina heiltölu $n$ ($1 \leq n \leq
    100$), fjölda heiða.

Næstu $n$ línur munu hver og ein samanstanda af streng $s_i$, nafni á heiði, og heiltölu $k_i$ ($0 \leq k_i \leq 200$), sem táknar hámarks vindhraðann sem heiðin þolir, aðskilin með bili.

\section*{Úttak}

Skrifið út $n$ línur, eina línu fyrir hverja heiði, sem segir hvort það sé öruggt að ferðast um hverja heiði í vindhraða $v$.
Hver lína er annaðhvort ``$s_i$ opin'', ef það er öruggt að ferðast um heiði $s_i$, eða ``$s_i$ lokud'', ef það er ekki öruggt að ferðast um heiði $s_i$.

Athugaðu að röðin á heiðunum í úttakinu skal vera í sömu röð og var gefin í inntakinu.

\section*{Stigagjöf}
\begin{tabular}{|l|l|l|}
    \hline
    Hópur & Stig & Takmarkanir               \\ \hline
    1     & 100  & Engar frekari takmarkanir \\ \hline
\end{tabular}

