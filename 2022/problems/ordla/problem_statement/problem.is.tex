\problemname{Orðla}
\illustration{0.3}{letters}{Mynd fengin af \href{https://unsplash.com/photos/gL7oJLJOb_I}{Unsplash}}%
Orðaleikurinn Orðla hefur farið um eins og eldur í sinu á netinu undanfarið.
Leikurinn virkar þannig að þú ert með orðabók af $n$ fimm-stafa enskum orðum,
en tölvan hefur valið eitt af þessum orðum í leyni. Leikmaðurinn má giska eins
oft og hann vill á orð úr orðabókinni til að finna leynda orðið, en því færri
gisk sem hann þarf, því betra.

Eftir hvert gisk aðstoðar tölvan leikmanninn með því að lita orðið hans, og
tákna þannig hversu nálægt leikmaðurinn er leynda orðinu. Tölvan litar orðið á
eftirfarandi hátt:

\begin{enumerate}
    \item Stafir sem eru þegar á réttum stað eru litaðir grænir, táknað með \texttt{O}.
    \item Stafir sem eru í leynda orðinu en eru ekki á réttum stað eru litaðir
        gulir, táknað með \texttt{/}. Ef stafur kemur oftar fyrir í
        ágiskuninni en leynda orðinu, þá litar tölvan bara vinstrustu eintökin
        af þessum staf, eða eins mörg eintök og eru í leynda orðinu.
    \item Restin af stöfunum eru litaðir gráir, táknað með \texttt{X}.
\end{enumerate}

Sem dæmi, ef leikmaður giskar á \texttt{error} en leynda orðið er
\texttt{racer}, þá mun tölvan lita ágiskunina með \texttt{//XXO}.

\section*{Gagnvirkni}
Þetta er gagnvirkt vandamál. Lausnin þín verður keyrð á móti gagnvirkum dómara
sem les úttakið frá lausninni þinni og skrifar í inntakið á lausninni þinni.
Þessi gagnvirkni fylgir ákveðnum reglum:

Dómarinn skrifar fyrst út eina línu með heiltölunni $n$ ($1 \leq n \leq 500$),
fjöldi orða í orðabókinni. Næst skrifar dómarinn $n$ línur með orðunum úr
orðabókinni, sem hvert samanstendur af fimm enskum lágstöfum. Öll orðin eru
mismunandi. Dómarinn velur svo eitt af þessum orðum alveg af handahófi.

Næst giskar lausnin þín á orð úr orðabókinni með því að skrifa út línu með því
orði. Dómarinn svarar með því að skrifa út eina línu sem inniheldur lituðu
útgáfuna af ágiskuninni eins og útskýrt er að ofan. Ef litunin er
\texttt{OOOOO} þá hefur lausnin þín giskað á rétt orð og á að hætta þegar í
stað. Annars á lausnin þín að giska aftur, eins og lýst er í byrjunninni á
þessari málsgrein.

Vertu viss um að gera \texttt{flush} eftir hvert gisk, t.d., með
\begin{itemize}
    \item \texttt{print(..., flush=True)} í Python,
    \item \texttt{cout << ... << endl;} í C++,
    \item \texttt{System.out.flush();} í Java.
\end{itemize}

Með dæminu fylgir tól til þess að hjálpa við að prófa lausnina þína.

\section*{Stigagjöf}
Lausnin þín verður keyrð á 100 mismunandi prófunartilfelli. Ef lausnin þín þarf
$g$ gisk til að leysa prófunartilfelli sem inniheldur $n$ orð, þá fær lausnin
% $\mathrm{max}(0, 1 - x / (1 + exp(-x)))$ stig fyrir það prófunartilfelli, þar
% sem $x = (g-1) / n$.
$\mathrm{max}(0, 1 - (g-1)^2/n)$ stig fyrir það prófunartilfelli.
Lokastigafjöldi er svo summa stiga úr öllum prófunartilfellum.

Við ábyrgjumst að $30$ prófunartilfelli hafa $n\leq 10$, $50$ prófunartilfelli
hafa $10 < n \leq 100$, og $20$ prófunartilfelli hafa $n = 500$.

