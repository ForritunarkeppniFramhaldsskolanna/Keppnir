\problemname{Veður - Vegakerfi}
\illustration{0.3}{vegakerfi}{Mynd fengin af \href{https://www.vegagerdin.is/ferdaupplysingar/faerd-og-vedur/allt-landid-faerd-kort/}{vegagerdin.is}}

Vegagerðin hefur samband við þig aftur varðandi annað verkefni.
Nú eru margir Íslendingar að ferðast um landið, þrátt fyrir slæma veðrið.
Því koma oft inn fyrirspurnir til Vegagerðarinnar, til dæmis, hvort hægt
sé að komast frá Kópavogi til Egilsstaða við núverandi aðstæður.

Vegagerðin gefur þér lýsingu á vegakerfi Íslands þar sem
öllum vegum landsins er lýst. Sérhverjum vegi er lýst með
tveimur endapunktum og vindstigsþröskuldi.
Ef vindhraðinn er minni en eða jafn vindstigsþröskuldi vegarins þá
er vegurinn opinn og þá má ferðast um veginn, annars er vegurinn
lokaður.

Nú þarft þú að skrifa forrit sem getur svarað þessum fyrispurnum.
Fyrirspurnir innihalda upplýsingar um upphafsstaðsetningu, endastaðsetningu
og vindhraðann á landinu. Svarið skal vera \texttt{Jebb} ef til er leið
frá upphafsstaðsetningu til endastaðsetningar sem notar einungis opna vegi.
Annars skal svarið vera \texttt{Neibb}.

\section*{Inntak}
Fyrsta línan í inntakinu inniheldur þrjár heiltölur $n$ ($1 \leq n \leq
10^5$), fjöldi gatnamóta, $m$ ($0 \leq m \leq 10^5)$, fjöldi vega,
og $q$ ($1 \leq q \leq 10^5)$, fjöldi fyrirspurna.

Næst fylgja $m$ línur, þar sem $i$-ta línan lýsir vegi númer $i$.
Hver lína inniheldur þrjár heiltölur $u_i$ ($1 \leq u_i \leq n$) og $v_i$
($1 \leq v_i \leq n$), endapunkta vegarins, og $t_i$ ($0 \leq t_i \leq 10^9$),
vindstigsþröskuld vegarins.

Að lokum koma $q$ línur, þar sem $j$-ta línan lýsir fyrirspurn númer $j$.
Til að tryggja að fyrirspurnum sé svarað einni í einu þá eru þær dulkóðaðar.
Hver lína inniheldur því þrjár dulkóðaðar heiltölur $a'_j$, $b'_j$ og $h'_j$
Táknum fjölda fyrirspurna sem hafa verið svarað með \texttt{Jebb} upp að þessu
með $x$.
Til að fá rétt gildi á $a_j, b_j$ og $h_j$ skal beita XOR aðgerðinni, táknuð með
$\hat{}$ virkjanum í flestum forritunarmálum, og því er
\begin{itemize}
    \item $a_j = a'_j \hat{} x$,
    \item $b_j = b'_j \hat{} x$,
    \item $h_j = h'_j \hat{} x$.
\end{itemize}
Hver fyrirspurn inniheldur því þrjár heiltölur $a_j$ ($1 \leq a_j \leq n$), upphafspunkt,
$b_j$ ($1 \leq b_j \leq n$, endapunkt og $h_j$ ($0 \leq h_j \leq 10^9$), vindhraðann.

\section*{Úttak}
Fyrir hverja fyrirspurn skal svara annaðhvort \texttt{Jebb}, ef til er leið
frá gefna byrjunarpunktinum til gefna endapunktsins sem hefur engan lokaðan veg,
eða \texttt{Neibb} ef ekki er til leið.

\section*{Stigagjöf}
\begin{tabular}{|l|l|l|}
\hline
Hópur & Stig & Takmarkanir \\ \hline
1     & 10   & \begin{tabular}{@{}c@{}}$1 \leq n, q \leq 20$, fyrirspurnir koma í lækkandi röð eftir vindhraða, vegakerfið myndar eina rás sem\\
                                       táknar hringveginn og gatnamót eru tengd í hækkandi röð ($1$ við $2$, $2$ við $3$, ...) og svo $n$ við $1$.\end{tabular}  \\ \hline
2     & 30   & $1 \leq n, q \leq 100$, fyrirspurnir koma í lækkandi röð eftir vindhraða. \\ \hline
3     & 20   & $1 \leq n \leq 250$, fyrirspurnir koma í lækkandi röð eftir vindhraða.  \\ \hline
4     & 30   & Fyrirspurnir koma í lækkandi röð eftir vindhraða. \\ \hline
5     & 10   & Engar frekari takmarkanir.  \\ \hline
\end{tabular}

