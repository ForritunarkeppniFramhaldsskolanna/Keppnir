\problemname{Kolkrabbaleikarnir}
\illustration{0.3}{glassbridge}{Mynd fengin af \href{https://upload.wikimedia.org/wikipedia/commons/9/96/Hongya_Valley_glass_bridge_(cropped).jpg}{wikimedia.org}}

Kolkrabbaleikarnir 2022 eru hafnir. Arnar, einnig þekktur sem Kolkrabbi,
bauð mörgum keppendum að taka þátt í þessarri æsispennandi keppni þar sem
keppt er um risastór peningaverðlaun.
Keppnin samanstendur af mörgum mismunandi leikjum.
Keppendur voru $1\,234\,567$ í upphafi, en eftir að hafa keppt í síðustu fjórum 
leikjum hefur mörgum keppendum verið útrýmt úr keppninni.
Því eru aðeins $m$ keppendur eftir.

Nú er komið að fimmta leiknum og er keppendum raðað í eina röð. Það er enginn annar
en Hlini sem er fyrstur í röðinni. Hann stígur áfram í herbergi næsta leiks
þar sem hann tekur eftir að hann stendur á palli. Hann sér annan pall hinum megin í
herberginu og stóra brú á milli. En þetta er engin eðlileg brú. Brúin samanstendur
af $n$ röðum, þar sem hver röð er gerð úr tveimur gler reitum. Keppendur eiga því að
komast yfir glerbrúnna með því að velja annanhvorn gler reitinn til að stíga á í
hverri röð.

Gler reitirnir líta eins út en það er stórhættulegt að gera ráð fyrir að þeir séu
eins. Annar reiturinn í hverri röð er úr tempruðu gleri og þolir þyngd einnar 
manneskju. Hinn reiturinn er ekki úr tempruðu gleri og myndi því brotna við þyngd
manneskjunnar sem stígur á reitinn. Manneskjan myndi þá falla niður og vera úr leik.

Eftir að Arnar hefur útskýrt leikreglurnar þá segir Hlini: ``Þetta er bara fifty-fifty,
annaðhvort kemst ég yfir eða ekki.'' Arnar útskýrir fyrir honum að svo sé ekki, þar sem
það eru helmingslíkur að hann giski rétt í hvert skipti sem hann stígur áfram.
Allir keppendur, fyrir utan Hlina, eru með fullkomið minni. Allir keppendur eru með
fullkomið jafnvægi.

Margir áhorfendur eru til staðar og er algengur leikur fyrir þá að giska hversu margir
keppendur verða eftir í lok hvers leiks. Hvað má búast við að margir keppendur verði
eftir í lok leiksins?

\section*{Inntak}
Inntak er ein lína með tveimur heiltölum $n$, fjöldi raða í brúnni, og $m$, fjöldi keppenda
í leiknum, þar sem $0 \leq n \leq 10^6$ og $0 \leq m \leq 10^6$.

\section*{Úttak}
Úttak skal vera ein lína með einni rauntölu, fjölda fólks sem má búast við að séu eftirstandandi
í lok leiksins.
Úttakið er talið rétt ef talan er annaðhvort nákvæmlega eða hlutfallslega 
ekki lengra frá réttu svari en $10^{-6}$.
Þetta þýðir að það skiptir ekki máli með hversu margra aukastafa nákvæmni 
tölurnar eru skrifaðar út, svo lengi sem þær er nógu nákvæmar.

%\section*{Útskýring á sýnidæmi}
%Sýnidæmi $1$ tilheyrir hópi 3 og í því eru tvær raðir í brúnni og tveir leikmenn.
%Það eru fjórar mismunandi útkomur á leiknum í þessu tilviki.
%\begin{itemize}
%    \item Fyrri leikmaður giskar rétt fyrir báðar raðir. 
%          Tveir leikmenn komast yfir.
%    \item Fyrri leikmaður giskar rétt fyrir fyrri röðina, en rangt fyrir seinni röðina.
%          Einn leikmaður kemst yfir.
%    \item Fyrri leikmaður giskar rangt fyrir fyrri röðina og seinni leikmaður giskar rétt fyrir seinni röðina.
%          Einn leikmaður kemst yfir.
%    \item Fyrri leikmaður giskar rangt fyrir fyrri röðina og seinni leikmaður giskar rangt fyrir seinni röðina.
%          Enginn leikmaður kemst yfir.
%\end{itemize}
%Meðalfjöldinn sem kemst yfir er því $1$.

\section*{Stigagjöf}
\begin{tabular}{|l|l|l|}
\hline
Hópur & Stig & Takmarkanir \\ \hline
1     & 10   & Brúin er aðeins ein röð.\\ \hline
2     & 20   & Hlini er eini keppandinn eftir.\\ \hline
3     & 30   & $0 \leq n, m \leq 10$ \\ \hline
4     & 30   & $0 \leq n, m, \leq 1000$ \\ \hline
5     & 10   & Engar frekari takmarkanir. \\ \hline
\end{tabular}

