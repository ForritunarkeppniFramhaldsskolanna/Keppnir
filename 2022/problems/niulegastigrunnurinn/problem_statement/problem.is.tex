\problemname{Níulegasti grunnurinn}
\illustration{0.3}{niu}{Mynd fengin af \href{https://www.flickr.com/photos/jblndl/283365812/in/photolist-r3jMs-i82Vr-cyqA9-4cZetF-4d4e3U-2i59gMC-m41sj8-2geq8Xe-6Xn9Qg-RK8uaV-21TzxS3-2dpyE2S-MCPZZC-2g98EQ2-2gi2bFP-4eLfTe-2g98Cib-6xvN67-Q5go5o-4UcV8j-4bXE58-TkuBsJ-2kdkw2L-vwbQc-9hQSaS-9Roa2W-JF3oiq-2i8UY8o-2i3UYtb-2hSbu57-2m4B4go-2hC9BuA-2irDGUN-2hC8FFh-2irEPHm-pRN6Bw-2iEdvop-2i3UYsK-7jf4GV-7G5bAg-2ibATiK-6dov8g-2hTwBAs-29snpkB-2i3FXpF-2hNh8Vh-2i9rv5i-2i9u2kN-2i59gMc-2hd2N43}{flickr.com}}

Jörmunrekur var að leika sér að prófa að skrifa út nokkrar tölur í ólíkum grunnum. Hann prófaði að byrja með $203433$ sem dæmi. Uppáhalds
talan hans Jörmunreks er níu, svo honum finnst þetta ekki merkileg tala. En ef hann skrifar hana út með grunntölu $16$ fæst $31AA9$ sem
er strax betra því þar er ein nía. En ef hann skrifar hana út með grunntölu $12$ fæst $99889$. Þetta er frábært, þrjár níur, varla hægt
að óska eftir einhverju betra. Eða hvað?

\section*{Inntak}

% O(cube root(n) * log(n)) intended

Inntakið er ein lína sem inniheldur tvær heiltölur $1 \leq n, d \leq 10^{18}$. Heiltalan $n$ er talan sem á að skrifa út í einhverjum grunni og
talan $d$ er hún sem Jörmunrekur vill að komi oftast fyrir. Aðeins á að íhuga grunna með grunntölu að minnsta kosti $2$.

\section*{Úttak}

Prentið út hversu oft $d$ getur í mesta lagi komið fyrir ef rétt grunntala er valin.

\section*{Stigagjöf}
\begin{tabular}{|l|l|l|}
    \hline
    Hópur & Stig & Takmarkanir                                 \\ \hline
    1     &  40  & $1 \leq n, d \leq 10^6$ \\ \hline
    2     &  30  & $1 \leq n, d \leq 10^{12}$ \\ \hline
    3     &  30  & Engar frekari takmarkanir \\ \hline
\end{tabular}
