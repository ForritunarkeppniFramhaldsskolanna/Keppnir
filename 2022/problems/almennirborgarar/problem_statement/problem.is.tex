\problemname{Almennir Borgarar}
\illustration{0.3}{queue}{Mynd fengin af \href{https://unsplash.com/photos/ysambitxV8M}{Unsplash}}%
Í mörg ár hafa keppendur Forritunarkeppni Framhaldsskólanna fengið gómsæta
hamborgara frá Hamborgarabúllu Tómasar í hádegismat. Keppendur mynda eina langa
röð á meðan þeir bíða eftir borgurunum, en kokkar Búllunnar vinna hörðum höndum
við að undirbúa borgarana. Kokkar Búllunnar eru mjög færir, og taka enga stund
að undirbúa borgara. Eina undantekningin er að þeir þurfa að bíða eftir að
borgararnir eldist á grillinu. Búllan er með $n$ lítil grill (númeruð frá $1$ til
$n$), en hvert grill getur bara eldað einn borgara í einu. Grillin eru líka
misheit og taka því mislangan tíma að elda borgara. Kokkarnir mældu þennan tíma
og komust að því að grill númer $i$ tekur $t_i$ sekúndur að elda einn borgara.

Nú bíður Benni spenntur í röðinni eftir að fá borgara, en það eru $m$ keppendur
fyrir framan hann í röðinni. Ef kokkarnir nota grillin á sem bestan hátt, hvað
er langt í að Benni fái borgarann sinn?

\section*{Inntak}
Fyrsta línan í inntakinu inniheldur tvær heiltölur $n$ ($1 \leq n \leq
2\cdot10^5$), fjöldi grilla, og $m$ ($0 \leq m \leq 10^9$), fjöldi keppenda
fyrir framan Benna í röðinni.

Síðan kemur lína með $n$ heiltölum $t_1, t_2, \ldots, t_n$, þar sem $t_i$ ($1
\leq t_i \leq 10^9$) er tíminn sem það tekur að elda borgara á grilli númer
$i$.

\section*{Úttak}
Skrifið út í hversu margar sekúndur Benni þarf að bíða áður en hann fær borgarann
sinn, ef kokkarnir nota grillin á sem bestan máta.

\section*{Stigagjöf}
\begin{tabular}{|l|l|l|}
\hline
Hópur & Stig & Takmarkanir \\ \hline
1     & 20   & $n, m, t_i \leq 100$ \\ \hline
2     & 20   & $n, m \leq 10^3$ \\ \hline
3     & 20   & $m, t_i \leq 10^3$ \\ \hline
4     & 10   & $m \leq 10^5$ \\ \hline
5     & 30   & Engar frekari takmarkanir\\ \hline
\end{tabular}

