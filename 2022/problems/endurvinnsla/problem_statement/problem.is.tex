\problemname{Endurvinnsla}
\illustration{0.3}{endurvinnsla_pic}{Mynd fengin af \href{https://upload.wikimedia.org/wikipedia/commons/thumb/5/59/Recycle_Plastic_Bird.jpg/1200px-Recycle_Plastic_Bird.jpg?20160522183713}{commons.wikimedia.org}}

Í landinu Ekkitilistan, þá virkar flokkun á plasti þannig að ef ákveðin prósenta af hlutunum er ekki úr plasti, til dæmis óhrein ílát, einhver setti pappa í plast tunnu, og svo framvegis. þá þarf að farga öllum pokanum, þar sem það væri of mikil vinna að aðskilja plastið frá því sem er ekki plast.

Í mismunandi borgum Ekkitilistan, þá gilda mismunandi reglur, t.d. í Varaldreitilburg þá má aðeins 5\% af hlutunum vera ekki-plast, en í Ekkitilburg er prósentan 7\%.

Þú hefur ákveðið að veita endurvinnslunni hjá Ekkitilistan hjálp, með því að skrifa forrit sem reiknar út hvort að ákveðinn pokinn af endurvinnslu sé í raun endurvinnanlegur.

\section*{Inntak}
Fyrsta línan er nafnið á borginni.
Önnur línan er hlutfallið $0 \leq p \leq 1$ sem poki af endurvinnslu má vera af efni sem er ekki plast, án þess að pokanum sé fargað. $p$ er gefið með nákvæmlega tveimur aukastöfum.
Þriðja línan er $n$, fjöldi hluta sem eru í endurvinnslu pokanum.
Næstu $n$ línur munu lýsa hverjum hlut í pokanum, hver þannig lína mun annaðhvort vera ``plast'', sem merkir að hluturinn sé úr plasti, eða ``ekki plast'', sem merkir að hluturinn sé ekki úr plasti.


\section*{Úttak}
Skrifið út \texttt{Jebb} ef það er hægt að endurvinna innihald pokans, en \texttt{Neibb} ef það þarf að farga pokanum.

\section*{Stigagjöf}
\begin{tabular}{|l|l|l|}
    \hline
    Hópur & Stig & Takmarkanir          \\ \hline
    1     & 100  & $1 \leq n \leq 10^5$ \\ \hline
\end{tabular}

