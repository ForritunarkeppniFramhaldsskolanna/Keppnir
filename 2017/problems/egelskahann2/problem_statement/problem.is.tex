\problemname{Ég elska hann}
\illustration{0.3}{flower-0}{Fyrsta sýnidæmið}
Gunna litla var mjög hrifin af Jóni litla. Hún sveiflaði sér í rólunni, horfði á
Jón leika sér með hinum börnunum og skoðaði blómið sitt.
Jón talar aldrei við hana. Hún heldur fastar um blómið sitt.

Hún ætlar að leyfa blóminu að ákveða hvort hún fari að tala við hann eða ekki.
Hún byrjar á að númera laufblöðin á blóminu frá $1$ upp í $N$ rangsælis, þar
sem $N$ er fjöldi laufblaða.

Hún byrjar hjá laufblaði númer $1$ og segir ``Hann elskar mig''. Hún lætur
þetta laufblað vera, og heldur áfram á blað númer $2$. Þá segir hún ``Hann
elskar mig ekki'' og rífur blaðið af. Svo fer hún á blað númer $3$, segir
``Hann elskar mig''. Hún heldur svona áfram, segir ``Hann elskar mig'' og
``Hann elskar mig ekki'' til skiptis, og rífur blaðið sem hún er á af þegar
hún segir ``Hann elskar mig ekki''. Þetta gerir hún, hring eftir hring, þar til
aðeins eitt laufblað er eftir.

Getur þú talið með Gunnu, og sagt henni númerið á síðasta laufblaðinu?

\section*{Inntak}
Heiltalan $1 \leq N$, fjöldi laufblaða á blóminu til að byrja með.

\section*{Úttak}
Ein lína með númerinu á laufblaðinu sem eftir stendur.

\section*{Útskýring á sýnidæmum}
Gunna gerir eftirfarandi í fyrsta sýnidæminu:
\begin{itemize}
    \item Hún byrjar á laufblaði $1$, og segir ``Hann elskar mig''.
    \item Hún fer á laufblað $2$, segir ``Hann elskar mig ekki'', og rífur laufblaðið af.
    \item Hún fer á laufblað $3$, og segir ``Hann elskar mig''.
    \item Hún fer á laufblað $4$, segir ``Hann elskar mig ekki'', og rífur laufblaðið af.
    \item Hún fer á laufblað $5$, og segir ``Hann elskar mig''.
    \item Hún fer á laufblað $1$, segir ``Hann elskar mig ekki'', og rífur laufblaðið af.
    \item Hún fer á laufblað $3$, og það er síðasta laufblaðið.
\end{itemize}

\section*{Stigagjöf}
Lausnin mun verða prófuð á miserfiðum inntaksgögnum, og er gögnunum skipt í
hópa eins og sýnt er í töflunni að neðan. Lausnin mun svo fá stig eftir því
hvaða hópar eru leystir.

\begin{tabular}{|l|l|l|l|}
\hline
Hópur & Stig & Inntaksstærð \\ \hline
1     & 20   & $N \le 10$ \\ \hline
2     & 20   & $N \le 1000$ \\ \hline
3     & 30   & $N \le 10^6$ \\ \hline
4     & 30   & $N \le 10^{18}$ \\ \hline
\end{tabular}

