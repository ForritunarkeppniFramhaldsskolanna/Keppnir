\problemname{Fullkomin mylla}

\illustration{0.45}{mylla}{Fullkomin mylla}

Arnar og Hannes eru mjög keppnissamir vinir og elska að keppa á móti hvor öðrum. Þeir
eru alltaf að leita að nýjum leikjum til að keppa í og ekki verra ef það eru leikir
sem krefjast mikillar hugsunar.

Hannes fann nýja gerð af myllu sem kallast \emph{Fullkomin mylla}
og taldi að hann ætti góðan möguleika á að sigra Arnar í þessum leik. Hann fór því
til Arnars og spurði hvort hann væri til í að keppa. Arnar sagðist vera til í að keppa
en aðeins ef að sá sem tapar myndi borga fyrir matinn næst þegar þeir færu á 
American Style.

Til að vera vissir um að betri leikmaðurinn vinnur, og að það sé ekki heppni
sem ráði úrslitunum, þá ákveða þeir að spila nokkrar lotur, og sá sem er fyrr
að vinna $n$ lotur vinnur veðmálið. Hver lota mun samanstanda af $5$ leikjum af
Fullkomnri myllu, og sá sem vinnur fleiri af þessum $5$ leikjum vinnur lotuna.

Að auki ætla þeir ekki að eyða tíma í óþarfa leiki. Sér í lagi munu þeir stoppa
lotuna um leið og það er kominn sigurvegari fyrir lotuna. Til dæmis, ef Arnar
vinnur fyrstu þrjá leikina í lotu, þá munu þeir ekki spila fleiri leiki í
þessari lotu, af því Arnar mun vinna lotuna sama hver úrslitin í síðustu
tveimur leikjunum verða.

\section*{Reglur}
Fullkomin mylla er spiluð þannig að búin er til ein stór mylla og síðan í hverjum
reit á stóru myllunni er lítil mylla. Spilarar skiptast á að gera og velja reiti í litlu
myllunum til að spila í þangað til að kominn er upp mylla. Ef að spilara tekst að ná
myllu í lítilli myllu þá er sá reitur í stóru myllunni orðinn að tákni þess spilara.
Ef spilara tekst að ná myllu í stóru myllunni vinnur hann leikinn.

Spilari $1$ byrjar og getur valið hvaða reit sem er í stóru myllunni til að spila
í, og velur síðan reit í litlu myllunni sem er þar. Eftir það þarf Spilari $2$
að setja í reit í stóru myllunni sem samsvarar reitnum í litlu myllunni
sem Spilari $1$ valdi. Ef það er komin mylla í þeim reit þá má Spilari $2$
velja hvaða reit sem er til að spila í. Þetta endurtekur sig síðan þangað
til annar hvor einstaklingurinn nær stórri myllu eða enginn getur unnið
og þar af leiðandi er jafntefli.

\section*{Inntak}
Fyrsta línan inniheldur heiltölu $N$ sem er fjöldi lota sem þarf að vinna til 
þess að vinna veðmálið. Næsta lína inniheldur streng $S$ sem lýsir því hver
sigraði hvern leik, A ef Arnar vann og H ef Hannes vann. Gera má ráð fyrir að
engin jafntefli hafi komið upp, og að strengurinn lýsi nákvæmlega þeim leikjum
sem voru spilaðir, í þeirri röð sem þeir voru spilaðir.


\section*{Úttak}
Prentið út hver tapaði veðmálinu.

\section*{Útskýring á sýnidæmum}
Í fyrra sýnidæminu þá vinnur sá sem er fyrr að vinna $2$ lotur. Svona fara leikirnir fram:
\begin{enumerate}
    \item Fyrsta lota byrjar. Arnar vinnur fyrsta leikinn í þessari lotu.
    \item Hannes vinnur annan leikinn.
    \item Arnar vinnur þriðja leikinn. Staðan er núna 2-1 fyrir Arnari, en Hannes á enn séns á að vinna, svo þeir halda áfram.
    \item Hannes vinnur fjórða leikinn.
    \item Arnar vinnur fimmta leikinn, og vinnur Arnar því fyrstu lotuna 3-2.
    \item Önnur lota byrjar. Arnar vinnur fyrsta leikinn.
    \item Arnar vinnur líka annan leikinn.
    \item Hannes kemur nú sterkur inn, og vinnur þriðja leikinn.
    \item Hannes vinnur fjórða leikinn.
    \item Hannes er á dúndur ferð, vinnur fimmta leikinn, og vinnur því aðra lotuna 3-2. Nú hafa báðir unnið eina lotu.
    \item Þriðja lota byrjar. Hannes vinnur fyrsta leikinn.
    \item Arnar vinnur annan leikinn.
    \item Hannes vinnur þriðja leikinn.
    \item Arnar vinnur fjórða leikinn.
    \item Arnar vinnur fimmta leikinn, og vinnur því lotuna 3-2. Nú hefur Arnar unnið tvær lotur, og er því sigurvegari!
\end{enumerate}

\noindent
Í seinna sýnidæminu þá vinnur sá sem er fyrr að vinna $2$ lotur. Svona fara leikirnir fram:
\begin{enumerate}
    \item Fyrsta lota byrjar. Hannes vinnur fyrsta leikinn.
    \item Hannes vinnur annan leikinn.
    \item Hannes er á dúndur ferð, og vinnur líka þriðja leikinn. Nú er Hannes kominn með þrjú stig, en það eru bara tveir leikir eftir í þessari lotu. Arnar hefur því engan kost á að vinna þessa lotu, og því vinnur Hannes þessa lotu 3-0.
    \item Önnur lota byrjar. Arnar vinnur fyrsta leikinn, og vinnur þá líka smá virðingu til baka eftir rústið hjá Hannesi í fyrstu lotu.
    \item Hannes lætur þetta ekki á sig hafa, og vinnur annan leikinn.
    \item Hannes er kominn sterkur aftur inn, og vinnur þriðja leikinn.
    \item Hannes kastar, og hann SKORAR! Hann vinnur fjórða leikinn. Hann er
        því kominn með þrjú stig í þessari lotu, en Arnar aðeins eitt. Það er
        bara einn leikur eftir, og sama hvernig hann fer mun Hannes vinna þessa
        lotu. Þeir stoppa því þessa lotu og Hannes vinnur hana 3-1. Hannes hefur nú unnið tvær lotur, og vinnur því veðmálið.
\end{enumerate}

\section*{Stigagjöf}
Lausnin mun verða prófuð á miserfiðum inntaksgögnum, og er gögnunum skipt í
hópa eins og sýnt er í töflunni að neðan. Lausnin mun svo fá stig eftir því
hvaða hópar eru leystir.

\begin{tabular}{|l|l|l|l|}
\hline
Hópur & Stig & Inntaksstærð & Önnur skilyrði  \\ \hline
1     & 50         & $ 1 \le N \le 1000$ & Engin lota klárast snemma (allar lotur tóku nákvæmlega 5 leiki)\\ \hline
2     & 50         & $ 1 \le N \le 1000$ & \\ \hline
\end{tabular}

