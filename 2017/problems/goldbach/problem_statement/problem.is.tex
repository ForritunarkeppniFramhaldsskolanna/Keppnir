\problemname{Önnur tilgáta Goldbachs}

Heiltalan $P$ er kölluð frumtala ef einu heiltölurnar sem ganga upp í hana eru
$1$ og $p$ sjálf. Til dæmis er $20$ ekki frumtala, því heiltalan $5$ gengur upp
í hana. Aftur á móti er $11$ frumtala, því aðeins $1$ og $11$ ganga upp í $11$.

Fræg tilgáta um frumtölur er Tilgáta Goldbachs, en hún segir:
\begin{quote}
    Allar sléttar heiltölur stærri en $2$ er hægt að tákna sem summu tveggja frumtalna.
\end{quote}

Þessi tilgáta er frá árinu 1742. Enn þann dag í dag hefur engum tekist að sanna tilgátuna,
né koma með mótdæmi gegn henni. Okkur datt í hug að láta ykkur sanna hana hér í dag,
en það væri of auðvelt.

Við kynnum heldur erfiðari tilgátu þekkt sem Önnur tilgáta Goldbachs:
\begin{quote}
    Allar oddatölur stærri en $5$ er hægt að tákna sem summu þriggja frumtalna.
\end{quote}

Í þessu verkefni gefum við oddatöluna $N$ sem er stærri en $5$. Við biðjum þig
að finna þrjár frumtölur $P_1$, $P_2$, $P_3$ þannig að $P_1 + P_2 + P_3 = N$,
eða tilkynna okkur að $N$ brjóti Aðra kenningu Goldbachs.

\section*{Inntak}
Inntakið inniheldur eina oddatölu $N > 5$. 

\section*{Úttak}
Skrifið út þrjár frumtölur aðskildar með einu bili þar sem summa þeirra er $N$.
Ef það eru margir möguleikar megið þið skrifa hvern þeirra sem er. Ef engar
slíkar tölur eru til, skrifið þá út \texttt{Neibb}.

\section*{Útskýring á sýnidæmum}
Í fyrsta sýnidæminu er $N = 65$. Ef þú skoðar úttakið sérðu að allar tölurnar
eru frumtölur og summa þeirra $65$. Þessar tölur geta komið út í hvaða röð sem
er. Aðrar mögulegar lausnir eru \texttt{11 37 17} og \texttt{11 11 43}.

\section*{Stigagjöf}
Lausnin mun verða prófuð á miserfiðum inntaksgögnum, og er gögnunum skipt í
hópa eins og sýnt er í töflunni að neðan. Lausnin mun svo fá stig eftir því
hvaða hópar eru leystir.

\begin{tabular}{|l|l|l|l|}
\hline
Hópur & Stig & Inntaksstærð \\ \hline
1     & 20   & $N \le 31$ \\ \hline
2     & 25   & $N \le 500$ \\ \hline
3     & 25   & $N \le 10^4$ \\ \hline
4     & 30   & $N \le 10^8$ \\ \hline
\end{tabular}
