\problemname{XORsistinn}

Presturinn Gunnar fékk dularfull skilaboð í pósti í gær þar sem stóð að eitt af 
sóknarbörnum hans væri mögulega andsetið. Á bakhliðinni var búið að skrifa
,,Þú ert okkar eina von XORsist!". Gunnar var nefnilega líka særingamaður 
og hefur mikla reynslu af því að bæla í burtu púka og djöfla.

Hann ákvað að skrifa niður nöfnin á öllum í sókninni sinni og endaði með nöfn
númeruð frá $1$ upp í $N$. Seinna fattaði hann að ef hann myndi taka XOR af öllum tölum frá
$1$ upp í $N$ myndi það gefa honum upplýsingar um hver væri andsetinn; ef talan er $0$ þá er enginn
andsetinn, ef hún er á bilinu $1$ og upp í $N$ þá er það manneskjan á listanum hans sem
passar við þá tölu. Ef hún er hinsvegar stærri en $N$ þá er það Gunnar sjálfur sem er
andsetinn og þá erum við öll í vanda! 


\section*{Útskýring á XOR}
XOR, venjulega táknuð með $~\hat{}~$ í forritunarmálum, er aðgerð sem tekur tvær
tölur og skilar nýrri tölu. Aðgerðin er framkvæmd á tölurnar tvær á
tvíundarformi (e.\ binary), bita fyrir bita. Eftirfarandi tafla sýnir hvernig
nýr biti er reiknaður út frá samsvarandi bitum í tölunum tveimur.
\begin{center}
\begin{tabular}{|c|c|c|}
\hline
A & B & A XOR B \\ \hline
1 & 1 & 0 \\ \hline
1 & 0 & 1 \\ \hline
0 & 1 & 1 \\ \hline
0 & 0 & 0 \\ \hline
\end{tabular}
\end{center}

Tökum dæmi. Talan $1337$ í binary er 10100111001 og talan $1993$ í binary er 11111001001.
\begin{center}
\begin{tabular}{|c|ccccccccccc|}
\hline
1337 & 1 & 0 & 1 & 0 & 0 & 1 & 1 & 1 & 0 & 0 & 1\\ \hline
1993 & 1 & 1 & 1 & 1 & 1 & 0 & 0 & 1 & 0 & 0 & 1 \\ \hline
1337 XOR 1993 & 0 & 1 & 0 & 1 & 1 & 1 & 1 & 0 & 0 & 0 & 0 \\ \hline
\end{tabular}
\end{center}
Útkoman er 01011110000 sem er talan $752$.

\section*{Inntak}
Fyrsta og eina línan inniheldur heiltölu $N$, hversu mörg sóknarbörn eru í sókninni hans Gunnars.

\section*{Úttak}
Prentið út númerið á sóknarbarninu sem er andsetið, \texttt{Enginn} ef talan er $0$ og
\texttt{Gunnar} ef talan er stærri en $N$.

\section*{Útskýring á sýnidæmum}
Í fyrsta sýnidæminu er $N = 3$. Ef við reiknum XOR af öllum tölum frá $1$ upp í
$N$, þá fáum við $(1\textrm{ XOR  }2)\textrm{ XOR }3 = 0$. Það er því enginn andsetinn, og svarið er \texttt{Enginn}.

Í öðru sýndæminu er $N = 6$. Ef við reiknum XOR af öllum tölum frá $1$ upp í
$6$, þá er útkoman $7$. Gunnar er því andsetinn, og svarið er \texttt{Gunnar}.

Í þriðja sýndæminu er $N = 5$. Ef við reiknum XOR af öllum tölum frá $1$ upp í
$5$, þá er útkoman $1$. Barn númer $1$ er því andsetið, og svarið er \texttt{1}.

\section*{Stigagjöf}
Lausnin mun verða prófuð á miserfiðum inntaksgögnum, og er gögnunum skipt í
hópa eins og sýnt er í töflunni að neðan. Lausnin mun svo fá stig eftir því
hvaða hópar eru leystir.

\begin{tabular}{|l|l|l|}
\hline
Hópur & Stig & Inntaksstærð  \\ \hline
1     & 50         & $ 1 \le N \le 1000$ \\ \hline
2     & 50         & $ 1 \le N \le 10^{18}$ \\ \hline
\end{tabular}

