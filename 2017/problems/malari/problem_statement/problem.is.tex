\problemname{Veggurinn, seinni hluti}

\noindent
Þegar verktakar Trumps eru búnir að reisa vegginn hans þá hringir
Trump í Magga málara:

\begin{quote}
\textbf{TRUMP:} ``I need you to paint my wall. The builders have no clue how to do it. Sad!''\\
\textbf{MAGGI:} ``Ok, hvernig á hann að vera á litinn?''\\
\textbf{TRUMP:} ``It should be great: Orange like my beautiful face!''\\
\textbf{MAGGI:} ``Ok, ég mæti með appelsínugula málningu.''
\end{quote}

\noindent
Veggurinn er $N$ einingar að breidd, og eru einingarnar númeraðar frá $1$ upp í
$N$.
Bíllinn hans Magga er ekki mjög stór og rúmar ekki alla málninguna sem hann
þarf, og þarf hann því að fara nokkrar ferðir í málningarbúðina. Eftir hverja ferð kemur hann að veggnum 
og byrjar að mála einingu númer $i$. Síðan málar hann næstu einingar við hliðina, $i+1$, $i+2$,
$\ldots$, alveg þar til hann hefur málað $j$-tu eininguna, en þá er hann búinn með málninguna
og þarf að fara aftur í búðina.

Maggi á það til að ruglast, og málar því stundum yfir einingar sem hann
hefur málað áður. Og nú, eftir $M$ bílferðir, er Maggi alveg búinn að gleyma
hve margar einingar hann er búinn að mála! Sem betur fer var Trump að fylgjast
með honum að mála og getur sagt honum hve margar einingar hann er búinn að
mála.

\section*{Inntak}
Fyrsta lína inntaksins inniheldur tvær heiltölur $N$, fjölda eininga í veggnum,
og $M$, fjöldi bílferða. Síðan koma $M$ línur. $k$-ta línan inniheldur tvær
heiltölur $i$ og $j$, þar sem $i \le j$, sem er upphafs- og lokaeiningin sem
Maggi málar eftir bílferð númer $k$.

\section*{Úttak}
Skilið fjölda eininga sem Maggi málaði og prentið ``The Mexicans took our jobs! Sad!'' ef sá fjöldi er stærri en $N / 2$, en annars ``The Mexicans are Lazy! Sad!''.

\section*{Stigagjöf}
Lausnin mun verða prófuð á miserfiðum inntaksgögnum, og er gögnunum skipt í
hópa eins og sýnt er í töflunni að neðan. Lausnin mun svo fá stig eftir því
hvaða hópar eru leystir.

\begin{tabular}{|l|l|l|l|}
\hline
Hópur & Stig & Inntaksstærð \\ \hline
1     & 30         & $ 1 \le N \le 1000$, $1 \le M \le 10^4$ \\ \hline
2     & 30         & $ 1 \le N \le 10^6$, $1 \le M \le 10^5$ \\ \hline
3     & 40         & $ 1 \le N \le 10^{12}$, $1 \le M \le 10^5$ \\ \hline
\end{tabular}
