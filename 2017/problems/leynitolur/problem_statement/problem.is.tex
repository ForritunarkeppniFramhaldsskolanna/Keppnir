\problemname{Leynitölur}

Jón litli er með lista af 100 uppáhalds tölunum sínum, allt heiltölur á bilinu
$0$ upp í $10^{18}$. Honum er mjög annt um þessar tölur, og vill ekki að neinn
komist að því hverjar tölurnar eru. Hann hefur því ákveðið að dulkóða tölurnar
sínar, og gerir það á eftirfarandi hátt:

\begin{enumerate}
    \item Hann tekur tölu $x$ úr listanum sínum.
    \item Hann margfaldar $x$ með tölunni $230\,309\,227$ og leggur svo töluna
        $68\,431\,307$ við útkomuna. Hann deilir svo útkomunni með $2^{64}$, og
        kallar afganginn úr deilingunni $y$.
    \item Hann gleymir nú tölunni $x$, og geymir í staðinn $y$ sem er dulkóðaða talan.
\end{enumerate}

Hann gerði þetta við allar tölurnar í listanum sínum, og er því núna með 100
dulkóðaðar tölur. En hann gleymdi mikilvægasta hlutanum: hann veit ekki hvernig
hann getur afkóðað dulkóðuðu tölurnar sínar. Getur þú hjálpað honum?

\section*{Inntak}
Inntakið inniheldur hundrað heiltölur, sem hver er dulkóðuð tala, ein á hverri
línu.

\section*{Úttak}
Skrifið út eina línu fyrir hverja dulkóðaða heiltölu í inntakinu. Þessi lína á
að innihalda afkóðuðu töluna, sem er heiltala á milli $0$ og $10^{18}$, eða
töluna $0$ ef þið vitið ekki hver afkóðaða talan er. Þið megið gera ráð fyrir
að það sé nákvæmlega ein tala á milli $0$ og $10^{18}$ sem er afkóðuð útgáfa af
samsvarandi tölu í inntakinu. Þ.e. ef hún er dulkóðuð, þá fæst samsvarandi
tala í inntakinu.

\section*{Stigagjöf}
Lausin verður keyrð á lista af 100 dulkóðuðum tölum. Listinn er alltaf sá sami,
og er sá sem er sýndur hér fyrir neðan. Lausnin fær 1 stig fyrir hverja tölu
sem hún nær að afkóða.

Jón gaf ykkur aukalega eftirfarandi upplýsingar um upprunalega listann af
tölunum:

\begin{itemize}
\item 10 tölur eru mjög litlar ($< 10$)
\item 10 tölur eru nokkuð litlar ($< 1000$)
\item 7 tölur eru ``Perfect tölur''
\item 10 tölur eru ``Factorial tölur''
\item 10 tölur eru á forminu $2^n$
\item 10 tölur eru ``Fibonacci tölur''
\item 10 tölur eru ``Catalan tölur''
\item 10 tölur eru ``Motzkin tölur''
\item 10 tölur eru ``Triangular tölur''
\item 13 tölur eru mjög stórar ($< 10^{18}$)
\end{itemize}

\section*{Útskýring á sýnidæmi}
Í sýnidæminu er gefinn listi af 100 dulkóðuðum tölum sem lausnin ykkar verður
prófuð á. Úttakið í sýnidæminu gefur dæmi um hvað lausnin ykkar gæti skrifað
út. Þar eru allar tölurnar, nema þrjár, $0$.

Á línu 5 í úttakinu skilaði lausnin tölunni $6$. Prufum að dulkóða þessa tölu:
\begin{enumerate}
    \item Margföldum $6$ með tölunni $230\,309\,227$ og fáum $1\,381\,855\,362$.
    \item Leggjum svo töluna $68\,431\,307$ við útkomuna, og fáum $1\,450\,286\,669$.
    \item Framkvæmum svo deilinguna $1\,450\,286\,669 / 2^{64}$ og fáum út $0$ með afganginn $1\,450\,286\,669$.
    \item Afgangurinn $1\,450\,286\,669$ er því dulkóðaða útgáfan af tölunni $6$.
\end{enumerate}
Ef við skoðum nú línu 5 í inntakinu, þá er það einmitt talan $1\,450\,286\,669$
sem var dulkóðaða talan sem Jói litli vildi afkóða. Svarið $6$ er því rétt, og
lausnin fær eitt stig fyrir þessa tölu. Það vill líka svo til að talan $6$ er
einmitt bæði ``Perfect tala'' og lítil tala ($< 10$).

Á línu 38 í úttakinu skilaði lausnin tölunni $42$. Prufum að dulkóða þessa tölu:
\begin{enumerate}
    \item Margföldum $42$ með tölunni $230\,309\,227$ og fáum $9\,672\,987\,534$.
    \item Leggjum svo töluna $68\,431\,307$ við útkomuna, og fáum $9\,741\,418\,841$.
    \item Framkvæmum svo deilinguna $9\,741\,418\,841 / 2^{64}$ og fáum út $0$ með afganginn $9\,741\,418\,841$.
    \item Afgangurinn $9\,741\,418\,841$ er því dulkóðaða útgáfan af tölunni $42$.
\end{enumerate}
En ef við skoðum nú línu 38 í inntakinu, þá var það talan $68\,431\,307$ sem
Jói litli vildi afkóða. Svarið $42$ er því \textbf{ekki} rétt, og lausnin fær
ekki stig fyrir þessa tölu.

Á línu 40 í úttakinu skilaði lausnin tölunni $806\,515\,533\,049\,393$. Prufum að dulkóða þessa tölu:
\begin{enumerate}
    \item Margföldum $806\,515\,533\,049\,393$ með tölunni $230\,309\,227$ og fáum $185\,747\,968\,980\,098\,654\,649\,211$.
    \item Leggjum svo töluna $68\,431\,307$ við útkomuna, og fáum $185\,747\,968\,980\,098\,723\,080\,518$.
    \item Framkvæmum svo deilinguna $185\,747\,968\,980\,098\,723\,080\,518 / 2^{64}$ og fáum út $10\,069$ með afganginn $7\,702\,901\,917\,247\,859\,014$.
    \item Afgangurinn $7\,702\,901\,917\,247\,859\,014$ er því dulkóðaða útgáfan af tölunni $806\,515\,533\,049\,393$.
\end{enumerate}
Ef við skoðum nú línu 40 í inntakinu, þá er það einmitt talan $7\,702\,901\,917\,247\,859\,014$
sem var dulkóðaða talan sem Jói litli vildi afkóða. Svarið $806\,515\,533\,049\,393$ er því rétt, og
lausnin fær eitt stig fyrir þessa tölu. Það vill líka svo til að talan $806\,515\,533\,049\,393$ er
einmitt ``Fibonacci tala''.

