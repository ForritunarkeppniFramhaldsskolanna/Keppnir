\problemname{Órökrétt}

Anna er virtur rökfræðingur. Nýlega hefur hún verið að rannsaka röksegðir á
mismunandi formi. Tvö algeng form eru ONF (ogað normlegt form) og ENF (eðað
normlegt form).

Í ONF þá er röksegðin samsett af einni eða fleiri klausum með OG-um á milli,
þar sem hver klausa samanstendur af einni eða fleiri breytu (eða neitun breytu)
með EÐA-um á milli. Dæmu um röksemd á ONF formi er eftirfarandi:

\begin{center}
    \texttt{(a EDA !b) OG (c) OG (!a EDA !c EDA b)}
\end{center}

ENF er svipað, en þá er röksegðin samsett af einni eða fleiri klausum með EÐA-um á milli,
þar sem hver klausa samanstendur af einni eða fleiri breytu (eða neitun breytu)
með OG-um á milli. Dæmi um röksemd á ENF formi er eftirfarandi:

\begin{center}
    \texttt{(c) EDA (b OG !a) EDA (!b OG a OG !c)}
\end{center}

Algengt verkefni fyrir rökfræðing er að athuga hvort það sé hægt að gera gefna
röksegð sanna með því að setja gildin SATT/ÓSATT í breyturnar. Ef það er hægt,
þá segir maður að röksegðin sé fullnægjanleg. Til dæmis er hægt að gera
eftirfarandi röksegð sanna með því að setja \texttt{a = ÓSATT}, \texttt{b =
ÓSATT}, \texttt{c = SATT}, og hún er því fullnægjanleg:

\begin{center}
    \texttt{(a EDA !b) OG (c) OG (!a EDA !c EDA b)}
\end{center}

Aftur á móti er ekki hægt að gera eftirfarandi röksegð sanna, sama hvaða gildi
við látum breyturnar hafa, og hún er því ekki fullnægjanleg:

\begin{center}
    \texttt{(a EDA b) OG (!a EDA !b)}
\end{center}

Anna var að gera snilldarlega uppgötvun. Hún fann upp á skilvirkri leið til að
breyta röksegð á ONF formi yfir í jafngilda röksegð á ENF formi. Núna biður hún
þig um að hjálpa sér að athuga hvort röksegðin sem er á ENF formi sé
fullnægjanleg.

\section*{Inntak}
Ein lína sem inniheldur röksegð á ENF formi. Hver breyta inniheldur bara enska
lágstafi, og hefur lengd á bilinu 1 til 5.

\section*{Úttak}
Ein lína sem inniheldur \texttt{Jebb} ef röksegðin er fullnægjanleg, og \texttt{Neibb} ef hún er ekki fullnægjanleg.

\section*{Stigagjöf}
Lausnin mun verða prófuð á miserfiðum inntaksgögnum, og er gögnunum skipt í
hópa eins og sýnt er í töflunni að neðan. Lausnin mun svo fá stig eftir því
hvaða hópar eru leystir.

\begin{tabular}{|l|l|l|l|}
\hline
Hópur & Stig & Skilyrði \\ \hline
1     & 50   & Það eru bara $10$ breytur, táknaðar með einum bókstaf frá \texttt{a} til \texttt{j}, og það eru í mesta lagi $10$ klausur. \\ \hline
2     & 50   & Það eru allt að $100$ breytur, og það eru í mesta lagi $100$ klausur. \\ \hline
\end{tabular}

