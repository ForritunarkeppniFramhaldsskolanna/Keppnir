\problemname{Siggi sement}

\illustration{0.45}{cement}{Mynd eftir \href{https://www.flickr.com/photos/65924740@N00/6573336709/}{judy\_and\_ed}}

Siggi vinnur sem malbikari fyrir borgina. Einn daginn sér hann óvenjulega stóra holu í úthverfum Reykjavíkur, en íbúar úthverfisins eru mjög óþolinmóðir og vilja láta laga þetta strax.

Siggi mælir gatið og sér að það er nákvæmlega $K$ einingar að stærð. Hann
ákveður að laga gatið sjálfur, og fer því í næstu sementverslun. Þar fást pokar
af sementi af allskyns stærðum og gerðum. Hann er slæmur í bakinu, svo hann
ætlar að kaupa nákvæmlega tvo poka svo að þyngdin dreifist jafnt. Og þar að
auki vill hann ekki halda á meiri þyngd en hann þarf, svo hann vill að pokarnir
tveir rúmi saman nákvæmlega $K$ einingar.

Þú færð gefinn lista af öllum pokum sem eru til, og stærðum þeirra. Það gætu
verið til margir af sömu stærð. Getur þú hjálpað Sigga að velja tvo poka
sem hafa samtals stærð nákvæmlega $K$?

\section*{Inntak}
Fyrsta línan inniheldur tvær heiltölur $N$, fjöldi poka í búðinni, og $K$, stærðin
á holunni. Svo fylgja $N$ línur, ein fyrir hvern sementspoka, sem inniheldur
heiltölu $S_i$, stærðin á sementspokanum í einingum.

\section*{Úttak}
Skrifið út stærðir á tveimur pokum úr versluninni sem uppfylla skilyrðin. Það
má bara velja hvern poka einu sinni (en það má velja tvo mismunandi poka af
sömu stærð).

Ef margar lausnir koma til greina, þá megið þið skrifa einhverja þeirra út.
Ef það eru engar lausnir, skrifið þá út \texttt{Neibb}.

\section*{Útskýring á sýnidæmum}
Í fyrsta sýnidæminu getur Siggi valið poka af stærð $3$ og $17$. Samtals hafa
þeir stærð $3+17=20$, eins og beðið var um. Önnur lausn sem kæmi til greina
væri að velja poka $4$ og $16$.

Í öðru sýnidæminu er bara einn poki til í búðinni, svo hann getur ekki valið
tvo poka.

Í þriðja sýnidæminu getur Siggi valið poka af stærð $5$ og $5$.

\section*{Stigagjöf}
Lausnin mun verða prófuð á miserfiðum inntaksgögnum, og er gögnunum skipt í
hópa eins og sýnt er í töflunni að neðan. Lausnin mun svo fá stig eftir því
hvaða hópar eru leystir.

\begin{tabular}{|l|l|l|l|}
\hline
Hópur & Stig & Inntaksstærð \\ \hline
	1     & 30         & $ 1 \le N \le 10^3$,  $ 1 \le K \le 10^9$\\ \hline
	2     & 10         & $ 1 \le N \le 10^3$,  $ 1 \le K \le 10^{18}$\\ \hline
	3     & 30         & $ 1 \le N \le 2\cdot 10^5$,  $ 1 \le K \le 100$\\ \hline
	4     & 30         & $ 1 \le N \le 2\cdot 10^5$,  $ 1 \le K \le 10^{9}$\\ \hline
% 5     & 20         & $ 1 \le N \le 2\cdot 10^5$,  $ 1 \le K \le 10^{18}$\\ \hline
\end{tabular}
