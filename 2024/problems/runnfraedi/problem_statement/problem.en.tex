\problemname{Rúnnfræði}
\illustration{0.5}{rounding}{Image from \href{https://commons.wikimedia.org/wiki/File:Comparison\_rounding\_graphs\_SMIL.svg}{commons.wikimedia.org}}

Dagur is working on the homework he was assigned in the statistics course he is taking.
Dagur is however not sure whether he should be rounding his answers or truncating them.
If his answer were $123.456$ then he could round to the next hundredth and get $123.46$. 
If he however truncates to the same accuracy he gets $123.45$.
Dagur always rounds up on fives, so $123.45$ rounded to tenths $123.5$.
More precisely, to round we start by choosing a power of $10$.
Next we change our number to the closest integer multiple of this power of $10$,
choosing the larger multiple in case of a tie.
When we truncate we also start by choosing a power of $10$.
But instead of choosing the closest integer multiple we choose the largest integer multiple
of this power of $10$ that is not larger than our number.
To try to figure out whether he should be rounding or truncating he looks at examples from his teacher.
Can you help him try to figure out if his teacher is rounding or truncating?

\section*{Input}
The input contains two lines, each with one real number.
These values will never be negative and will always have at least one digit after the decimal point.
The last digit after a decimal point will not be $0$ if there is more than one digit after the decimal point.
The second value always has strictly fewer digits after the decimal point.
Each value will have at most $5$ digits before the decimal point.
It is guaranteed that the second value can be achieved by rounding or truncating the first value, or possibly both.

\section*{Output}
If the teacher must be rounding and not truncating, print \texttt{Runnun}.
If the teacher must be truncating and not rounding, print \texttt{Styfun}.
Otherwise print \texttt{Veit ekki}.

\section*{Scoring}
\begin{tabular}{|l|l|l|}
\hline
Group & Points & Constraints \\ \hline
1     & 20   & Both values are $< 0.5$, at most $5$ digits after the decimal point. \\ \hline
2     & 30   & At most $5$ digits after the decimal point, generally easier cases. \\ \hline
3     & 30   & At most $5$ digits after the decimal point. \\ \hline
4     & 10   & At most $500$ digits after the decimal point. \\ \hline
5     & 10   & At most $1\,000\,000$ digits after the decimal point. \\ \hline
\end{tabular}
