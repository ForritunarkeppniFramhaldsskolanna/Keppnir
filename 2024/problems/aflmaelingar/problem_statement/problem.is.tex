\problemname{Aflmælingar}
\illustration{0.5}{nixie}{Mynd fengin af \href{https://commons.wikimedia.org/wiki/File:\%D0\%98H-14\_(IN-14)\_Nixie\_Tubes\_Displaying\_\%2225\%22.jpg}{commons.wikimedia.org}}

Hrolleifur er búinn að eignast $300$ aflgjafa og einn aflmæli. Hann getur stillt hvern aflgjafa á $0\%$ til $100\%$ og svo kveikt á kerfinu
til að fá mælingu á samtals afli. Aflgjafarnir bjóða einungis upp á heiltöluprósentur. Aflmælirinn les svo summu aflsins sem allir aflgjafarnir
gefa í heild. Hins vegar er mælirinn gamall og notar Nixie túbur, og þú átt aðeins $K$ perur. Svo þú færð aðeins síðustu $K$ tölustafina í svarinu.

Hver aflgjafi $i$ hefur eitthvað grunnafl $s_i$, svo ef hann er stilltur á $p_i$ prósentur er aflið $p_i \cdot s_i$.
Nánar tiltekið er $s_i$ aflið þegar stillt er á $1$ prósentu.
Hrolleifur hefur bara visst mikinn tíma, svo nú vill hann komast að því hvað grunnafl hvers aflgjafa er í aðeins $q$ mælingum.
Þú veist að grunnaflið er einhver heiltala frá og með $0$ til og með $99$.
Getur þú hjálpað honum?

\section*{Gagnvirkni}
Þetta er gagnvirkt verkefni. Lausnin þín verður keyrð á móti gagnvirkum dómara
sem les úttakið frá lausninni þinni og skrifar í inntakið á lausninni þinni.
Þessi gagnvirkni fylgir ákveðnum reglum:

Fyrst les forritið þitt tvær heiltölur á einni línu $K, q$, þar sem $K$ er fjöldi pera og $q$ er fjöldi mælinga sem þú átt að framkvæma.

Næst skrifar lausnin þín út $n$ heiltölur $p_1, \dots, p_n$, prósentan sem þú stillir hvern aflgjafa á. 
Eftir það les forritið þitt $K$ stafa tölu, talan sem mælirinn sýnir.

Eftir $q$ slíkar mælingar skal forritið skrifa $n$ tölur $s_1, \dots, s_n$, grunnafl hvers mælis.

Vertu viss um að gera \texttt{flush} eftir hvert gisk, t.d., með
\begin{itemize}
    \item \texttt{print(..., flush=True)} í Python,
    \item \texttt{cout << ... << endl;} í C++,
    \item \texttt{System.out.flush();} í Java.
\end{itemize}

Sýniinntakið sýnir dæmi með $n = 4, K = 4, q = 2$.
Lausnin verður keyrð á þessu sýniinntaki, en niðurstaðan mun ekki hafa áhrif á stigagjöf. Það þýðir að forritið þitt þarf ekki að leysa sýnidæmið rétt til að fá stig.

Með verkefninu fylgir tól sem viðhengi til þess að hjálpa við að prófa lausnina þína.

\section*{Stigagjöf}
\begin{tabular}{|l|l|l|}
\hline
Hópur & Stig & Takmarkanir \\ \hline
1     & 30   & $K = 2, q = 300, n = 300$. \\ \hline
2     & 30   & $K = 4, q = 150, n = 300$. \\ \hline
3     & 40   & $K = 3, q = 200, n = 300$. \\ \hline
\end{tabular}
