\problemname{Ríkjafræði 2}
\illustration{0.5}{morphism}{Image from \href{https://commons.wikimedia.org/wiki/File:Commutative\_diagram\_for\_morphism.svg}{commons.wikimedia.org}}

Jörmunrekur thought he had achieved a decent understanding of category theory, but suddenly things got more complicated when he
started looking deeper into things! Category theory often uses diagrams as visual aids for theorems or definitions. The diagrams
usually represent categories consisting of objects and arrows between them. The thing that has gotten more complicated since last
time Jörmunrekur looked at categories is that now if there is an arrow from $x$ to $y$, the same might not be true for $y$ to $x$.
If it is possible to travel both ways, this has to be given as two separate arrows, one going either way. Jörmunrekur continues
undeterred though and thinks he could gain a better understanding of this phenomenon if he draws some diagrams of his own. The
diagrams start out containing only some points to denote the objects, but no arrows.

\section*{Input}
The input begins with one line containing two positive integers $n$ and $q$, where $n$ is the number of objects and $q$ is the
number of queries in the input. The objects are numbered from $1$ to $n$ and $1 \leq n \leq 1\,000$ will always hold. Next there are $q$
lines, each with three integers $o, x, y$, separated by spaces. $o$ is always either $0$ or $1$ and $1 \leq x, y \leq n$. If $o = 0$
the query means that Jörmunrekur draws an arrow from object $x$ to object $y$. If $o = 1$ then Jörmunrekur is wondering whether 
there is some sequence of arrows starting at $x$ and going to $y$.

\section*{Output}
Print a single line for each query starting with a $1$. If it's possible to trace some sequence of arrows from $x$ to $y$ print
\texttt{Jebb}. Otherwise print \texttt{Neibb}.

\section*{Note}
Inputs and outputs are fairly large. Make sure to use sufficiently fast I/O methods.

\section*{Scoring}
\begin{tabular}{|l|l|l|}
\hline
Group & Points & Constraints \\ \hline
1     & 40   & $1 \leq q \leq 1\,000$. \\ \hline
2     & 30   & $1 \leq q \leq 500\,000$, the arrows never form a cycle. \\ \hline
3     & 30   & $1 \leq q \leq 500\,000$. \\ \hline
\end{tabular}

