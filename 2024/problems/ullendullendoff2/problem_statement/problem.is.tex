\problemname{Úllen Dúllen Doff 2}
\illustration{0.4}{point}{Mynd fengin af \href{https://flickr.com/photos/pipstar/75570251}{flickr.com}}

Lárus er millistjórnandi hjá glæsilegu fyrirtæki.
Undir honum eru $n$ starfsmenn og allir nema einn þeirra eru frændfólk hans.

Þegar nýtt verkefni kemur upp þá sér Lárus um að úthluta því á starfsmann.
Lárus vill setja sem minnsta vinnu á frændur sínar.
Hann getur samt ekki bara alltaf valið sama aðilann.

Til að tryggja að fólk gruni Lárus ekki um frændhygli þá notar hann fágaða aðferð til að velja starfsmann af handahófi.
Hann raðar fólkinu upp í hring og notar þulu til þess að velja af handahófi hver fær verkefnið.
Hann velur fyrsta starfsmann til að benda á og þylur fyrsta orðið.
Svo fer hann í gegnum þuluna og bendir á næsta starfsmann til hægri í hringnum fyrir hvert orð sem hann þylur.

Þulan hljómar svo:
\begin{quote}
    Úllen dúllen doff kikke lane koff koffe lane bikke bane úllen dúllen doff.
\end{quote}

Hvernig getur Lárus raðað starfsmönnunum þannig að frændfólk hans fái ekki verkefnið?

\section*{Inntak}
Fyrsta línan inniheldur eina heiltölu $n$, fjölda starfsmanna.
Næst fylgja $n$ línur, þar sem hver lína inniheldur eitt nafn.
Fyrsta nafnið er starfsmaðurinn sem er ekki hluti frændfólksins.

Þú mátt gera ráð fyrir að sérhvert nafn sé einstakt og samanstendur af $1$ til $10$ enskum lágstöfum.

\section*{Úttak}
Skrifaðu út $n$ línur, þar sem hver lína inniheldur eitt nafn á starfsmanni og skal ekkert nafn vera endurtekið.
Lárus mun nota röðina sem þú gefur og þylja þuluna til að velja starfsmanninn sem tekur við nýja verkefninu.
Ef röðin sem þú gefur verður til þess að frændi eða frænka Lárusar verði fyrir valinu þá verður lausnin þín dæmd röng.

\section*{Stigagjöf}
Fjöldi starfsmanna, $n$, getur verið frá $1$ upp í $20$.
Til er stigahópur fyrir hvert mögulegt gildi á $n$ og er hver hópur virði $5$ stiga.
Leysa þarf öll prufutilvikin í hóp til að öðlast stiginn fyrir þann hóp.

\section*{Útskýring á sýnidæmum}
Í fyrra sýnidæminu má til dæmis nota upprunalegu röðina í inntakinu því þá er farið í gegnum hana á eftirfarandi máta:

\begin{itemize}
    \item Úllen: Arnar
    \item dúllen: Atli
    \item doff: Bjarni
    \item kikke: Bjarki
    \item lane: Hannes
    \item koff: Unnar
    \item koffe: Arnar
    \item lane: Atli
    \item bikke: Bjarni
    \item bane: Bjarki
    \item úllen: Hannes
    \item dúllen: Unnar
    \item doff: Arnar
\end{itemize}

Þar sem Arnar verður fyrir valinu að lokum er úttakið talið rétt.

Í seinna sýnidæminu er eitt mögulegt svar að fara í gegnum röðina á eftirfarandi máta:
\begin{itemize}
    \item Úllen: v
    \item dúllen: x
    \item doff: y
    \item kikke: a
    \item lane: b
    \item koff: c
    \item koffe: p
    \item lane: q
    \item bikke: r
    \item bane: s
    \item úllen: t
    \item dúllen: u
    \item doff: z
\end{itemize}

Að lokum verður z fyrir valinu og því er úttakið talið rétt.

Athugið að mörg önnur rétt úttök koma til greina og að í seinna sýnidæminu komumst við ekki að síðustu tveimur gildunum í röðinni.
