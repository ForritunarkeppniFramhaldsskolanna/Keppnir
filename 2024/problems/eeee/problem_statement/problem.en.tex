\problemname{Enn Eitt EEEE Erfiði}
\illustration{0.5}{ddi}{Image taken from \href{https://en.wikipedia.org/wiki/DDI}{wikipedia.com}}

Arnar is going mad from all these abbreviations and shorthands at his job. Everything from DDI, i13n, o11y to LASIK. He can't read
half a page without having to stop to look up all kinds of acronyms. Eventually he decides to do something about this and is going
to collect some info on the complexity of his reading material to show how bad the problem has gotten. He is of course busy doing his
job, so he gets you to calculate the complexity of all the acronyms in his data set. The complexity of an acronym is the number of acronyms
you need to define to understand the meaning of the initial acronym. For example, the complexity of LASER (Light Amplified by Stimulated
Emission of Radiation) has a complexity of $1$, since it does not refer to any further acronyms. The complexity of DDI is $5$ however as it
refers to DNS, DHCP and IPAM, which in turn refers to IP.
Note that acronyms can refer to themselves and to each other in a cyclic manner.

\section*{Input}
The first line of input contains an integer $1 \leq n \leq 30 \, 000$, the number of acronyms defined in the input.
The next $n$ lines each contain one definition of an acronym. The line starts with the acronym being defined, and then there is a
number $0 \leq k \leq n$, the number of words it refers to. The $k$ words then follow. Each of the words contains only upper case characters
or only lower chase characters. The ones with only upper case characters are other acronyms, the others not. 
All strings in the input are of length at most $20$ and contain only English characters. The total length of all strings in
the input is at most $2 \, 000 \, 000$. All acronyms referred to in the input are defined somewhere in the input.

\section*{Output}
For each acronym that appears in the input, print its complexity on its own line, in the same order as the acronyms are defined in
the input.

\section*{Scoring}
\begin{tabular}{|l|l|l|}
\hline
Group & Points & Constraints \\ \hline
1     & 10     & Acronyms do not refer to other acronyms, $1 \leq n \leq 1 \, 000$. \\ \hline
2     & 15     & All acronyms are defined before being referred to and no two different acronyms refer to the same acronym, $1 \leq n \leq 1 \, 000$. \\ \hline
3     & 10     & All acronyms are defined before being referred to and no two different acronyms refer to the same acronym. \\ \hline
4     & 40     & $1 \leq n \leq 1 \, 000$.  \\ \hline
5     & 25     & No further constraints.  \\ \hline
\end{tabular}

\section*{Explanation of samples}

In the first sample LASER, IP, DNS and DHCP do not refer to any other acronyms, so they only need to define themselves.
Thus their complexities are all $1$.
LASIK however refers to LASER, so you need to define both, so it has complexity $2$. 
The same applies to IPAM referring to IP so its complexity is $2$.
To understand DDI you need to define DNS, DHCP and IPAM.
This does however not make the complexity $4$ but $5$ instead as you also need to define IP to understand IPAM.

In the second sample YARA, DB and UNIX all have complexity $1$.
Now some acronyms refer to themselves however.
This does not affect their complexity though.
To define PHP you only need to define PHP, as once its defined that solves the self-reference, so the complexity is $1$.
Similarly for XAMPP you just need to define DB and PHP additionally, so the complexity is $3$.
For the same reason GNU has complexity $3$.
Next we see that HIRD and HURD refer to each other cyclically, so to understand one we need to understand the other.
Together they refer to UNIX as well, so to understand one of them we need to define HIRD, HURD and UNIX.
Thus they both have complexity $3$.
Finally we see that GNULINUX refers to UNIX twice, but that doesn't affect how many definitions need to be made total.
To define GNULINUX one needs to define GNULINUX, UNIX and GNU which gives a complexity of $3$.
