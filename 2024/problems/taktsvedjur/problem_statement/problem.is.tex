\problemname{Taktsveðjur}
\illustration{0.5}{OIG2}{}

Atli stundar það alveg vandræðalega mikið að spila tölvuleiki þegar hann ætti að vera að gera eitthvað gáfulegra.
Oft er fólk að reyna ná á honum en fær ekkert svar því hann er staddur í sínum sýndarveruleika og heyrir ekkert.
Nánast undantekningalaust er hann að spila uppáhalds leikinn sinn Taktsveðjur þegar hann er í sýndarveruleikanum.
Taktsveðjur snýst um að sveifla sverðum og reyna að hitta allar nóturnar sem birtast,
oft í takt við einhverja góða tónlist.
Mikil orka fer í það að reyna fá sem hæstu stig fyrir gefið lag og alltaf að reyna slá metið sitt.
Þar sem Atli er of upptekinn við að sveifla höndunum og reyna fá sem flest stig, 
þá þarf einhver annar að sjá um það að telja saman stigin!

Í Taktsveðjum fást stig fyrir hverja nótu sem maður hittir. 
Hver nóta getur gefið $1$ til $115$ stig, ef manni mistekst að slá nótuna rétt fást $0$ stig.
Ofan á þetta bætist svo stigamargfaldari sem er háður hversu mörgum nótum í röð maður nær.
Í byrjun er margfaldarinn $1$, svo nótan gefur manni jafn mörg stig og hún er virði.
Ef maður nær að hitta tvær nótur í röð fer margfaldarinn upp í $2$ og þá fær maður tvöföld stig nótnanna.
Ef margfaldarinn er $2$ og maður nær $4$ í röð fer margfaldarinn upp í $4$.
Eins ef hann er $4$ og maður nær $8$ í röð fer margfaldarinn upp í $8$, sem er hæsta mögulega gildið.
Ef nóta færir mann upp í nýjan margfaldara telst nýi margfaldarinn fyrir þá nótu.
Til dæmis ef maður hittir fyrstu tvær nóturnar í lagi fær maður strax tvöföld stig fyrir þá seinni.
Ef maður missir af nótu lækkar margfaldarinn um leið um eitt þrep, 
svo frá $8$ í $4$, $4$ í $2$ eða $2$ í $1$.
Ef margfaldarinn er $1$ helst hann þar ef maður missir af nótu.

\section*{Inntak}
Fyrsta línan inniheldur eina heiltölu $n$, fjölda nótna í laginu sem Atli reyndi við.
Næst koma $n$ línur, hver með stigin fyrir eina nótu í laginu. 
Hver lína inniheldur sem sagt eina heiltölu $x$ sem uppfyllir $0 \leq x \leq 115$.

\section*{Úttak}
Prentið samtals stigafjölda Atla fyrir lagið.

\section*{Stigagjöf}
\begin{tabular}{|l|l|l|}
\hline
Hópur & Stig & Takmarkanir \\ \hline
1     & 20   & $1 \leq n \leq 100$, Atli hittir engar nótur. \\ \hline
2     & 20   & $1 \leq n \leq 100$, Atli hittir aldrei tvær eða fleiri nótur í röð. \\ \hline
3     & 20   & $1 \leq n \leq 100$, Atli missir ekki af neinum nótum. \\ \hline
4     & 20   & $1 \leq n \leq 100$. \\ \hline
5     & 20   & $1 \leq n \leq 100\,000$. \\ \hline
\end{tabular}
