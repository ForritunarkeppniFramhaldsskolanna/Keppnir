\problemname{Hnappasetningaskipti}
\illustration{0.5}{typewriter}{Mynd fengin af \href{https://commons.wikimedia.org/wiki/File:Typewriter-12_hg.jpg}{commons.wikimedia.org}}

Á sínum tíma kepptu Arnar, Bjarki og Unnar saman í keppnisforritun. Á háskólastigi er einnig keppt í þriggja manna
liðum eins og á framhaldsskólastigi, en hins vegar fær hvert lið aðeins eina tölvu. Því skiptir miklu máli að 
nota tímann við tölvuna vel meðan hinir tveir liðsfélagarnir reyna að leysa önnur dæmi og undirbúa sig undir það
að forrita hratt og rétt þegar að þeim kemur.

Til þess að nýta tímann sem best ákvað Bjarki að læra á \emph{dvorak} hnappasetninguna, því hann telur að hún
geri honum kleift að skrifa hraðar. Arnari fannst þetta góð hugmynd, og bjó því til sína eigin hnappasetningu
sem hann kallaði \emph{bjarki}, og notar hana. Unnar heldur sig hins vegar enn við \emph{qwerty}, honum finnst þetta ekki
fyrirhafnarinnar virði. En þetta veldur nú vandræðum. Þegar þeir skiptast á að skrifa við tölvuna eru þeir
stundum búnir að skrifa heilan helling inn í tölvuna áður en þeir fatta að hún er stillt á vitlausa
hnappasetningu! Því þarf nú að búa til forrit sem getur tekið það sem búið er að skrifa og breytt því í
það sem þeir ætluðu að skrifa.

Hér fylgir tafla sem gefur hvað hver hnappur samsvarar í ólíkum hnappasetningum. Bil eru eins í öllum
hnappasetningum. Töfluna má einnig nálgast á TSV formi sem viðhengi.

\begin{minipage}{0.45\textwidth}
\begin{tabular}{|l|l|l|} \hline
\emph{qwerty} & \emph{dvorak} & \emph{bjarki} \\ \hline
\verb!~! & \verb!~! & \verb!0! \\
\verb!1! & \verb!1! & \verb!2! \\
\verb!2! & \verb!2! & \verb!4! \\
\verb!3! & \verb!3! & \verb!8! \\
\verb!4! & \verb!4! & \verb!6! \\
\verb!5! & \verb!5! & \verb!1! \\
\verb!6! & \verb!6! & \verb!3! \\
\verb!7! & \verb!7! & \verb!5! \\
\verb!8! & \verb!8! & \verb!7! \\
\verb!9! & \verb!9! & \verb!9! \\
\verb!0! & \verb!0! & \verb!=! \\
\verb!-! & \verb![! & \verb!-! \\
\verb!=! & \verb!]! & \verb!/! \\
\verb!q! & \verb!'! & \verb!b! \\
\verb!w! & \verb!,! & \verb!j! \\
\verb!e! & \verb!.! & \verb!a! \\
\verb!r! & \verb!p! & \verb!r! \\
\verb!t! & \verb!y! & \verb!k! \\
\verb!y! & \verb!f! & \verb!i! \\
\verb!u! & \verb!g! & \verb!g! \\
\verb!i! & \verb!c! & \verb!u! \\
\verb!o! & \verb!r! & \verb!s! \\
\verb!p! & \verb!l! & \verb!t! \\ \hline
\end{tabular}
\end{minipage}

\begin{minipage}{0.45\textwidth}
\begin{tabular}{|l|l|l|} \hline
\emph{qwerty} & \emph{dvorak} & \emph{bjarki} \\ \hline
\verb![! & \verb!/! & \verb!.! \\
\verb!]! & \verb!=! & \verb!,! \\
\verb!a! & \verb!a! & \verb!l! \\
\verb!s! & \verb!o! & \verb!o! \\
\verb!d! & \verb!e! & \verb!e! \\
\verb!f! & \verb!u! & \verb!m! \\
\verb!g! & \verb!i! & \verb!p! \\
\verb!h! & \verb!d! & \verb!d! \\
\verb!j! & \verb!h! & \verb!c! \\
\verb!k! & \verb!t! & \verb!n! \\
\verb!l! & \verb!n! & \verb!v! \\
\verb!;! & \verb!s! & \verb!q! \\
\verb!'! & \verb!-! & \verb!;! \\
\verb!z! & \verb!;! & \verb![! \\
\verb!x! & \verb!q! & \verb!]! \\
\verb!c! & \verb!j! & \verb!y! \\
\verb!v! & \verb!k! & \verb!z! \\
\verb!b! & \verb!x! & \verb!h! \\
\verb!n! & \verb!b! & \verb!w! \\
\verb!m! & \verb!m! & \verb!f! \\
\verb!,! & \verb!w! & \verb!x! \\
\verb!.! & \verb!v! & \verb!'! \\
\verb!/! & \verb!z! & \verb!~! \\ \hline
\end{tabular}
\end{minipage}

\section*{Inntak}
Fyrsta línan er á forminu \texttt{type1 on type2} þar sem \texttt{type1} og \texttt{type2} eru eitt af qwerty, 
dvorak eða bjarki. \texttt{type2} er hnappasetning lyklaborðsins og \texttt{type1} er hnappasetningin sem sá
sem situr við tölvuna er vanur.
Næst kemur ein lína, það sem sá sem situr við tölvuna skrifaði. Þessi texti er ávallt ein lína en getur
innihaldið alla stafina sem koma fram í töflunni að ofan. Þessi lína inniheldur mest $1\,000$ stafi og að minnsta
kosti einn staf sem er ekki bilstafur.

\section*{Úttak}
Prentið það sem sá sem situr við tölvuna ætlaði að skrifa. Úttak verður tekið gilt þó svo að bil séu ekki nákvæmlega eins og
í inntaki, svo lengi sem orðin séu eins.
Til dæmis eru tvö samhliða bil og eitt bil talið það sama.

\section*{Stigagjöf}
\begin{tabular}{|l|l|l|}
\hline
Hópur & Stig & Takmarkanir \\ \hline
1     & 1    & Inntak er sýniinntak. \\ \hline
2     & 11   & Inntak byrjar á \texttt{qwerty on qwerty}. \\ \hline
3     & 11   & Inntak byrjar á \texttt{dvorak on dvorak}. \\ \hline
4     & 11   & Inntak byrjar á \texttt{bjarki on bjarki}. \\ \hline
5     & 11   & Inntak byrjar á \texttt{dvorak on qwerty}. \\ \hline
6     & 11   & Inntak byrjar á \texttt{bjarki on qwerty}. \\ \hline
7     & 11   & Inntak byrjar á \texttt{qwerty on dvorak}. \\ \hline
8     & 11   & Inntak byrjar á \texttt{bjarki on dvorak}. \\ \hline
9     & 11   & Inntak byrjar á \texttt{qwerty on bjarki}. \\ \hline
10    & 11   & Inntak byrjar á \texttt{dvorak on bjarki}. \\ \hline
\end{tabular}
