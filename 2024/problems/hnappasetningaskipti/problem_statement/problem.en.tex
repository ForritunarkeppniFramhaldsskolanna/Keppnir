\problemname{Hnappasetningaskipti}
\illustration{0.5}{typewriter}{Image from \href{https://commons.wikimedia.org/wiki/File:Typewriter-12_hg.jpg}{commons.wikimedia.org}}

A while back Arnar, Bjarki and Unnar competed together in competitive programming. At the university level,
contestants also compete in three person teams, but each team only gets one computer. Therefore it is very
important to use the time at the computer well, while the other two contestants try to solve other problems,
and prepare to program their solutions quickly and correctly when it is their turn.

To make the most of his time, Bjarki decided to learn the \emph{dvorak} keyboard layout, because he believes it
allows him to type faster. Arnar thought this was a good idea and thus made his own keyboard layout which
he called \emph{bjarki} and moved onto using it. Unnar has decided to just stick to \emph{qwerty} however, he does not 
think this is worth the effort. But this is now causing problems! When the three swap turns being at the
computer, they sometimes start typing a whole bunch of things without realising the computer has the wrong
keyboard layout active! Thus, they need a program that can take what has already been written and change it
to what they intended to write.

A table follows that shows which keys correspond to one another in different keyboard layouts. Spaces
are the same in all keyboard layouts. The table can also be found in the TSV format in the attachments.

\begin{minipage}{0.45\textwidth}
\begin{tabular}{|l|l|l|} \hline
\emph{qwerty} & \emph{dvorak} & \emph{bjarki} \\ \hline
\verb!~! & \verb!~! & \verb!0! \\
\verb!1! & \verb!1! & \verb!2! \\
\verb!2! & \verb!2! & \verb!4! \\
\verb!3! & \verb!3! & \verb!8! \\
\verb!4! & \verb!4! & \verb!6! \\
\verb!5! & \verb!5! & \verb!1! \\
\verb!6! & \verb!6! & \verb!3! \\
\verb!7! & \verb!7! & \verb!5! \\
\verb!8! & \verb!8! & \verb!7! \\
\verb!9! & \verb!9! & \verb!9! \\
\verb!0! & \verb!0! & \verb!=! \\
\verb!-! & \verb![! & \verb!-! \\
\verb!=! & \verb!]! & \verb!/! \\
\verb!q! & \verb!'! & \verb!b! \\
\verb!w! & \verb!,! & \verb!j! \\
\verb!e! & \verb!.! & \verb!a! \\
\verb!r! & \verb!p! & \verb!r! \\
\verb!t! & \verb!y! & \verb!k! \\
\verb!y! & \verb!f! & \verb!i! \\
\verb!u! & \verb!g! & \verb!g! \\
\verb!i! & \verb!c! & \verb!u! \\
\verb!o! & \verb!r! & \verb!s! \\
\verb!p! & \verb!l! & \verb!t! \\ \hline
\end{tabular}
\end{minipage}

\begin{minipage}{0.45\textwidth}
\begin{tabular}{|l|l|l|} \hline
\emph{qwerty} & \emph{dvorak} & \emph{bjarki} \\ \hline
\verb![! & \verb!/! & \verb!.! \\
\verb!]! & \verb!=! & \verb!,! \\
\verb!a! & \verb!a! & \verb!l! \\
\verb!s! & \verb!o! & \verb!o! \\
\verb!d! & \verb!e! & \verb!e! \\
\verb!f! & \verb!u! & \verb!m! \\
\verb!g! & \verb!i! & \verb!p! \\
\verb!h! & \verb!d! & \verb!d! \\
\verb!j! & \verb!h! & \verb!c! \\
\verb!k! & \verb!t! & \verb!n! \\
\verb!l! & \verb!n! & \verb!v! \\
\verb!;! & \verb!s! & \verb!q! \\
\verb!'! & \verb!-! & \verb!;! \\
\verb!z! & \verb!;! & \verb![! \\
\verb!x! & \verb!q! & \verb!]! \\
\verb!c! & \verb!j! & \verb!y! \\
\verb!v! & \verb!k! & \verb!z! \\
\verb!b! & \verb!x! & \verb!h! \\
\verb!n! & \verb!b! & \verb!w! \\
\verb!m! & \verb!m! & \verb!f! \\
\verb!,! & \verb!w! & \verb!x! \\
\verb!.! & \verb!v! & \verb!'! \\
\verb!/! & \verb!z! & \verb!~! \\ \hline
\end{tabular}
\end{minipage}

\section*{Input}
The first line is of the form \texttt{type1 on type2} where \texttt{type1} and \texttt{type2} are qwerty,
dvorak or bjarki. \texttt{type2} is the keyboard layout currently active and \texttt{type1} is the layout
that the person at the computer is used to. Next there is a single line, the text that the person at the
computer wrote. This text is always a single line but can contain all of the characters in the table above.
This line will contain at most $1\,000$ letters and at least one letter that is not whitespace.

\section*{Output}
Print what the person at the computer intended to write. The output will be considered correct even if the
whitespace is not an exact match, as long as the words match. For example a single space and two consecutive spaces
will be considered equivalent.

\section*{Scoring}
\begin{tabular}{|l|l|l|}
\hline
Group & Points & Constraints \\ \hline
1     & 1    & The input is the sample input. \\ \hline
2     & 11   & The input starts with \texttt{qwerty on qwerty}. \\ \hline
3     & 11   & The input starts with \texttt{dvorak on dvorak}. \\ \hline
4     & 11   & The input starts with \texttt{bjarki on bjarki}. \\ \hline
5     & 11   & The input starts with \texttt{dvorak on qwerty}. \\ \hline
6     & 11   & The input starts with \texttt{bjarki on qwerty}. \\ \hline
7     & 11   & The input starts with \texttt{qwerty on dvorak}. \\ \hline
8     & 11   & The input starts with \texttt{bjarki on dvorak}. \\ \hline
9     & 11   & The input starts with \texttt{qwerty on bjarki}. \\ \hline
10    & 11   & The input starts with \texttt{dvorak on bjarki}. \\ \hline
\end{tabular}
