\problemname{Sannvirði}
\illustration{0.4}{game_show}{}

Dröfn Karen hefur rosalega gaman af nýja leikþættinum Sannvirði í sjónvarpsdagskránni.
Það sem henni finnst skemmtilegast af öllu er að sjá hvað sigurvegarinn er ánægður þegar hann hlýtur verðlaunin.
Hún horfir æsispennt á hvern einasta þátt.

Hver einasti þáttur hefst á því að dómarar sýna einhvern hlut sem má finna til sölu, til dæmis ísskáp, bíl, gulrót, ilmvatn, og svo framvegis.
Allir keppendur fá að sjá hlutinn jafn lengi.
Keppendur eiga svo að giska á hvert sannvirði hlutarins er.
Sá keppandi sem giskar næst rétta svarinu, en ekki yfir, er sigurvegarinn.
Keppendurnir senda inn ágiskanirnar sínar samtímis og fá ekki að sjá ágiskanir annarra keppenda,
en áhorfendur eins og Dröfn fá allar þessar upplýsingar í hendurnar.
Klippt er yfir í mjög stutt viðtöl við hvern keppanda þar sem keppandinn fær að útskýra ágiskun sína.

Það væri nú ekki sjónvarpsefni án auglýsinga.
Áður en sigurvegarinn er tilkynntur er klippt yfir í auglýsingahlé.
Dröfn iðar alveg af spenningi.
Hún hefur sínar eigin hugmyndir um sannvirðið og hugsar með sér hvaða keppandi myndi sigra ef einhver hugmynd hennar væri rétta svarið.

Geturðu fundið sigurvegarann fyrir sérhverja hugmynd Drafnar?


\section*{Inntak}
Fyrsta línan inniheldur eina heiltölu $n$, fjölda keppenda.

Næst fylgja $n$ línur, hver þeirra lýsir keppanda.
Lýsing á keppanda samanstendur af nafni og ágiskun, aðskilin með bili.
Sérhvert nafn samanstendur af minnsta lagi $1$ og mesta lagi $10$ enskum lágstöfum.
Sérhver ágiskun er heiltala á bilinu $0$ upp í $10^9$, báðar þar með taldar.
Þú mátt gera ráð fyrir að engir tveir keppendur giskuðu á sama virði.

Næst kemur ein lína sem inniheldur eina heiltölu $q$, fjölda hugmynda.

Að lokum koma $q$ línur, hver þeirra lýsir einni af hugmyndum Drafnar.
Sérhver hugmynd er heiltala á bilinu $0$ upp í $10^9$, báðar þar með taldar.

\section*{Úttak}
Fyrir hverja hugmynd, í sömu röð og þær koma í inntaki, skaltu skrifa út nafn keppandans sem sigrar skyldi sú hugmynd Drafnar um sannvirðið reynast rétt.
Ef enginn keppandi sigrar fyrir þá hugmynd skaltu skrifa út \texttt{:(}, því Dröfn verður leið ef enginn hlýtur verðlaunin.

\section*{Stigagjöf}
\begin{tabular}{|l|l|l|}
\hline
Hópur & Stig & Takmarkanir \\ \hline
1     & 10   & $n = 1$, $q = 1$ \\ \hline
2     & 15   & $n = 1$, $1 \leq q \leq 1\,000$ \\ \hline
3     & 20   & $1 \leq n \leq 1\,000$, $q = 1$ \\ \hline
4     & 25   & $1 \leq n \leq 1\,000$, $1 \leq q \leq 1\,000$ \\ \hline
5     & 30   & $1 \leq n \leq 200\,000$, $1 \leq q \leq 200\,000$ \\ \hline
\end{tabular}
