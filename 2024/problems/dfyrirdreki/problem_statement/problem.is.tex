\problemname{D Fyrir Dreki}
\illustration{0.5}{equations}{Mynd fengin af \href{https://commons.wikimedia.org/wiki/File:Quadratic_eq_discriminant.svg}{commons.wikimedia.org}}

Daði Dreki er alltaf að æfa sig að verða meira hugaður og það næsta hjá honum er skoða hversu hugaður hann er í heimi stærðfræðinnar.
Eins og allir vita þá eru það bara þau djörfu og hugrökku sem geta fundið rætur á annars stigs margliðum og vill því Daði verða einn þeirra.
Það er eitt sem flækist samt alltaf fyrir Daða og það er að það geta verið ein, tvær eða engar rauntölurætur á hverri annars stigs margliðu.
Daði biður þig því um hjálp við að finna hversu margar rauntölurætur eru á hverri annars stigs margliðu.
Getur þú hjálpað Daða að verða meira hugaður?

Annars stigs raunmargliða er stæða á forminu $ax^2 + bx + c$ þar sem $a$, $b$ og $c$ eru gefnar rauntölur með $a \neq 0$ og $x$ er breyta.\\
Dæmi um þetta eru: 
\begin{itemize}
	\item $-2x^2 + 7x - 1$ þar sem $a = -2$, $b = 7$ og $c = -1$
	\item $x^2 - 9$ þar sem $a = 1$, $b = 0$ og $c = -9$ 
	\item $10x^2 - x$ þar sem $a = 10$, $b = -1$ og $c = 0$
\end{itemize}


\section*{Inntak}
Inntakið samanstendur af þremur línum, hver með einni heiltölu.
Fyrsta línan inniheldur gildið á $a$, önnur línan inniheldur gildið á $b$ og þriðja línan inniheldur á $c$. 
Heiltölurnar $a, b$ og $c$ eru stuðlarnir fyrir annars stigs margliðuna $ax^2 + bx + c$, þar sem $-100 \leq a, b, c \leq 100$ og $a \neq 0$. 

\section*{Úttak}
Skrifaðu út hversu margar rauntölurætur annars stigs margliðan hans Daða hefur.

\section*{Stigagjöf}
\begin{tabular}{|l|l|l|}
\hline
Hópur & Stig & Takmarkanir \\ \hline
1     & 30   & Alltaf tvær rætur \\ \hline
2     & 30   & Annaðhvort engar eða tvær rætur \\ \hline
3     & 40   & Engar frekari takmarkanir \\ \hline
\end{tabular}

