\problemname{Tarot Póker}
\illustration{0.5}{card}{Mynd fengin af \href{https://commons.wikimedia.org/wiki/File:Tarot\_Card\_-\_The\_High\_Priest.jpg}{commons.wikimedia.org}}

Konráði var boðið að spila póker með KFFÍ (KeppnisForritunar Félagi Íslands), en Atli var löngu búinn að sníkjast til að búa til
algrím sem spilar póker fullkomlega. Til að reyna koma í veg fyrir svindl þá dró Konráð fram sataníska tarot spilastokk sinn og
samdi nýja útgáfu af póker á staðnum, til þess að Atli gæti ekki svindlað með gamla forritinu sínu. Þá vaknar spurningin, er
virkilega siðferðislega rétt að hjálpa Atla að svindla áfram? Því miður þýðir lítið að velta sér upp úr því, Atli var svo
slunginn að bara það inn í keppnisforritunarverkefni, svo það er ekki annað í boði en að svindla fyrir hans hönd ef menn vilja
fá stig í keppninni.

Nýju reglurnar eru sem fylgir. Tarot stokkurinn inniheldur 22 háspil, þau eru númeruð frá núll upp í tuttugu-og-einn og táknuð
með rómverskum stöfum (nema $0$ sem er \texttt{O}). Utan við þetta eru 14 spil til viðbótar í fjórum ólíkum sortum, sem gera
56 spil samtals og kallast lágspil. Sortirnar eru stafir, bikarar, sverð og mynt. Munum tákna þessar sortir með stöfunum \texttt{S}, \texttt{B},
\texttt{V} og \texttt{M}. Spilin 14 í hverri sort eru númeruð frá $1$ til $10$ og svo taka við Gosi, Riddari, Drottning og Kóngur. Við munum merkja
síðustu fjögur með stöfunum \texttt{G}, \texttt{R}, \texttt{D} og \texttt{K}. Sortin verður tekin fram fyrst, svo dæmi um
nokkur spil eru \texttt{V10}, \texttt{M1}, \texttt{BD} og \texttt{SR}. Loks bara til að gera lífið erfiðara bætið Konráð
jókerum í stokkinn. Jókerar verða táknaðir með \texttt{J}. Jókerar eru algildisspil, sem merkir að láta má jókerspilið vera
hvað sem er þegar maður setur fram hendi sína, jafnvel þó það sé jafnt öðru spili í borði. Sérhverjum jóker verður að úthluta
einu spili öðrum en jóker þegar hendi er spilað.

Þegar verið er að bera saman stök spil þá telst \texttt{XXI} hæst, svo fer það niður í \texttt{O} í þeirri röð. Öll háspilin
eru sem sagt hærri en öll lágspilin. Meðal lágspilanna telst \texttt{K} hæst, svo \texttt{D}, svo \texttt{R}, svo \texttt{G} og
loks tölurnar frá $10$ til $1$ í þeirri röð. Í sortum teljast myntir hæst, svo sverð, svo bikarar og svo stafir. Lítum svo á
að spilin $1$ til $10$ og svo \texttt{G}, \texttt{R}, \texttt{D}, \texttt{K} séu í röð. Eins eru \texttt{O} upp í \texttt{XXI}
í röð, en ekki má mynda röð með t.d. \texttt{MD}, \texttt{MK}, \texttt{O}, \texttt{I}, \texttt{II}. Raðir mega heldur ekki
innihalda endurtekin gildi.

Háspilin eru ekki með neina sort, en eru samt með gildi. Því til dæmis ef spilari er með eitt háspil á hendi telst það sjálfkrafa
sem hæsta spil í hendi. Háspil og jóker geta einnig myndað par sem er hærra en öll pör af lágspilum. Hins vegar er aldrei hægt
að mynda lit með háspil því háspil teljast ekki vera af neinni sort. En háspil geta myndað pör, fullt hús og allt sem hefur aðeins
með gildi að gera. Enn fremur ef há- og lágspil eru með sömu tölu á, t.d. \texttt{M4} og \texttt{IV} kallast gildin hliðstæð. 
Athugum að gosar, riddarar, drottningar og kóngar eru ekki hliðstæð neinum háspilum því þau hafa enga tölu á. 
Hins vegar er gildið $15$ hliðstætt \texttt{XV}.

Eftirfarandi samsetningar af spilum teljast góðar, með bestu samsetninguna efst og verstu neðst. Jafntefli eru leyst með því að
bera saman hæsta spil handanna, af þeim sem mynda samsetninguna. Ef þau eru jöfn eru næst hæstu spil hvorrar hendi borin saman af þeim
sem mynda samsetninguna, og svo framvegis. Ef þau eru öll eins þá eru hæstu spil handarinnar utan samsetningarinnar borin saman. Eins
ef þau eru eins er næst hæsta spil utan samsetningarinnar skoðað, og þar fram eftir götunum.

\begin{tabular}{|c|c|c|} \hline
    Sæti & Heiti & Myndun \\ \hline
    1 & Fimm eins & Fimm spil af sama gildi. \\ \hline
    2 & Engin lágspil & Fimm háspil. \\ \hline
    3 & Litaröð & Fimm spil í röð af sömu sort. \\ \hline
    4 & Fjölbreytni & Fjögur lágspil af ólíkum sortum og eitt háspil. \\ \hline
    5 & Stórt summuhús & Fjögur lágspil og háspil hliðstætt summu þeirra. \\ \hline 
    6 & Hliðstætt hús & Fjögur lágspil af sama gildi og hliðstætt háspil. \\ \hline 
    7 & Summuhús & Þrjú lágspil og háspil hliðstætt summu þeirra. \\ \hline 
    8 & Töluleysa & Fimm spil án tölu á, þ.e. \texttt{G}, \texttt{R}, \texttt{D}, \texttt{K}. \\ \hline 
    9 & Ferna & Fjögur spil af sama gildi. \\ \hline
    10 & Fullt hús & Þrjú spil af sama gildi, og tvö önnur spil af sama gildi. \\ \hline 
    11 & Litur & Öll fimm spil af sömu sort. \\ \hline
    12 & Röð & Fimm spil í röð. \\ \hline
    13 & Tvær hátvennur & Tvö pör af hliðstæðu há- og lágspili. \\ \hline
    14 & Tvenna og hátvenna & Par með sama gildi og par af hliðstæðu há- og lágspili. \\ \hline
    15 & Hátvenna & Par af hliðstæðu há- og lágspili. \\ \hline
    16 & Þrenna & Þrjú spil af sama gildi. \\ \hline
    17 & Tvær tvennur & Tvö pör af spilum með sama gildi. \\ \hline
    18 & Tvenna & Par af spilum með sama gildi. \\ \hline
    19 & Engin háspil & Fimm lágspil. \\ \hline
    20 & Hátt spil & Hæsta spil á hendi myndar mynstrið. \\ \hline
\end{tabular}

\section*{Inntak}
Inntakið mun innihalda tvær línur. Hver lína inniheldur fimm spil, aðskilin með bilum. Fyrri línan gefur spilin hans Konráðs
og seinni línan gefur spilin hans Atla. Athugið að sama spil getur komið oftar en einu sinni fyrir í inntaki.

\section*{Úttak}
Prentaðu fyrst tvær tölur, sæti sem hendi Konráðs fær í töflunni að ofan í besta falli og sæti sem hendi Atla fær í töflunni að ofan í besta falli.
Skrifaðu svo út hvor vinnur á næstu línu, þ.e. prentaðu \texttt{Konrad} ef Konráð vinnur, \texttt{Atli} ef Atli vinnur og \texttt{Jafntefli} annars.

\section*{Stigagjöf}
\begin{tabular}{|l|l|l|}
\hline
Hópur & Stig & Takmarkanir \\ \hline
1     & 40   & Engir jókerar, engin háspil. \\ \hline
2     & 40   & Engir jókerar. \\ \hline
3     & 20   & Engar frekari takmarkanir. \\ \hline
\end{tabular}

