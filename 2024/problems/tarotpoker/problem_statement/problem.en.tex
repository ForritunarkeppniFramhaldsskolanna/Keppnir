\problemname{Tarot Póker}
\illustration{0.5}{card}{Image from \href{https://commons.wikimedia.org/wiki/File:Tarot\_Card\_-\_The\_High\_Priest.jpg}{commons.wikimedia.org}}

Konráð was invited to play poker with KFFÍ (Competitive Programming Society of Iceland),
but Atli had of course long ago made an algorithm that plays poker perfectly and had been abusing that fact thoroughly. 
To try to prevent cheating Konráð pulled out his satanic tarot playing cards and made up a new version of poker on the spot,
such that Atli could not use his old program to cheat.
Thus the question arises, is it really ethical to help Atli keep cheating?
Unfortunately for you there's little point in pondering that conundrum,
seeing as Atli was devious enough to make that task a problem on this contest,
which leaves you little choice if you wish to maximise your winning chances.

The new rules are as follows. The tarot deck contains 22 high cards, 
numbered from zero to twenty-one and denoted by roman numerals (except $0$ which is \texttt{O}).
Outside these cards there are fourteen other cards in each of four suits,
which total 56 cards and they are called the low cards. 
The suits are wands, cups, swords and pentacles. 
We will denote these suits by \texttt{S}, \texttt{B}, \texttt{V} and \texttt{M} respectively.
The fourteen cards in each suit are numbered from $1$ to $10$, with a Jack, Knight, Queen and King following in that order.
We will denote the last four by the letters \texttt{G}, \texttt{R}, \texttt{D} and \texttt{K} respectively.
The suit will always be written first,
so as examples \texttt{V10}, \texttt{M1}, \texttt{BD} and \texttt{SR} are all in the deck.
Finally to make life harder Konráð also adds jokers to the deck.
The jokers will be denoted by \texttt{J}. 
Jokers are wildcards, meaning they can be treated as whichever card the player prefers when the hand is played,
even if that card is already in play. 
Each joker must be given the value of one non-joker card when played.

When individual cards are compared, \texttt{XXI} is considered the highest card, with the order going down to \texttt{O},
in that order. That is to say, any high card is considered higher than any low card.
Among the low cards \texttt{K} is considered highest, followed by \texttt{D}, then \texttt{R},
then \texttt{G} and finally the numbers $10$ down to $1$ in decreasing order.
When cards differ only in suit we consider pentacles highest, then swords, then cups and finally wands.
We consider the values $1$ through $10$ and then \texttt{G}, \texttt{R}, \texttt{D}, \texttt{K} to be sequential.
Similarly \texttt{O} through \texttt{XXI} are sequential. A sequence can however not contain high and low cards.
For example \texttt{MD}, \texttt{MK}, \texttt{O}, \texttt{I}, \texttt{II} does not constitute a sequence.
Sequences may not contain repeated values.

The high cards have no suit, but still have a value. 
Thus for example if a player has a single high card in their hand along with a joker, 
they can make a pair higher than any pair of low cards.
A player can however never for a flush with a high card as it has no suit.
High cards can thus form pairs, full houses and all other hands that only depend on value and not suit.
Furthermore, a high and low card with the same number on it, for example \texttt{M4} and \texttt{IV}, we will call comparable.
Note that jacks, knights, queens and kings are not comparable to any high card as they don't have a number on them.
However, we will consider \texttt{XV} to be comparable to the value $15$.

The following combinations of cards are considered to be good, 
with the best combination at the top and the worst at the bottom. 
Ties are broken by comparing the highest card in each hand that is part of their best combination.
If those are equal the second best card in each hand that is part of their best combination is considered.
This continues until both combinations are exhausted. 
After that ties are resolved by comparing the highest card in each hand not part of the combination.
Then the second highest card in each hand not part of the combination is considered, and so on.

\begin{tabular}{|c|c|c|} \hline
    Rank & Name & Condition \\ \hline
    1 & Five of a kind & Five cards with the same value. \\ \hline
    2 & No lows & Five high cards. \\ \hline
    3 & Straight flush & Five sequential cards of the same suit. \\ \hline
    4 & Diversity & Four low cards of different suits and a high card. \\ \hline
    5 & Large sum house & Four low cards and a high card comparable to their sum. \\ \hline 
    6 & Comparable house & Four low cards and a high card comparable to each of them. \\ \hline 
    7 & Sum house & Three low cards and a high card comparable to their sum. \\ \hline 
    8 & Numberless & Five cards without numbers, i.e. \texttt{G}, \texttt{R}, \texttt{D}, \texttt{K}. \\ \hline 
    9 & Four of a kind & Four cards of the same value. \\ \hline
    10 & Full house & Three cards of the same value, and two others of the same value. \\ \hline 
    11 & Flush & Five cards of the same suit. \\ \hline
    12 & Straight & Five sequential cards. \\ \hline
    13 & Two comparable pairs & Two pairs of comparable high and low cards. \\ \hline
    14 & Comparable pair and pair & A pair of comparable high and low cards and a pair of cards with the same value. \\ \hline
    15 & Comparable pair & A pair of comparable high and low cards. \\ \hline
    16 & Three of a kind & Three cards with the same value. \\ \hline
    17 & Two pairs & Two pairs of cards with equal values. \\ \hline
    18 & Pair & A pair of cards with equal values. \\ \hline
    19 & No highs & Five low cards. \\ \hline
    20 & Highest card & The combination consists of the highest card in hand. \\ \hline
\end{tabular}

\section*{Input}
The input will contain two lines. Each line will contain five cards, separated by spaces. 
The first line gives the cards in Konráð's hand and the second gives the cards in Atli's hand.
Note that the same card can appear multiple times in the input.

\section*{Output}
First print two integers, 
the rank of Konráð's hand in the table above in the best case and the rank of Atli's hand in the table above in the best case.
Then on the next line print who wins.
I.e. print \texttt{Konrad} if Konráð wins, \texttt{Atli} if Atli wins and \texttt{Jafntefli} otherwise.

\section*{Scoring}
\begin{tabular}{|l|l|l|}
\hline
Group & Points & Constraints \\ \hline
1     & 40   & No jokers, no high cards. \\ \hline
2     & 40   & No jokers. \\ \hline
3     & 20   & No further constraints. \\ \hline
\end{tabular}

