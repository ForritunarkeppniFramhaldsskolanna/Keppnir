\problemname{Golf}
\illustration{0.5}{golf}{Mynd fengin af \href{https://commons.wikimedia.org/wiki/File:Bruges\_Public\_Library\_Manuscript\_251\_folio\_149r\_detail\_golf.jpg}{commons.wikimedia.org}}

Meðlimir Keppnisforritunarfélags Íslands ákváðu að skrá sig í svokallaða Kóða Golf keppni.
Þeir héldu að þetta væri keppni þar sem ætti að forrita sem stystar lausnir á keppnisforritunarverkefnum,
en því miður var sú ekki raunin!
Þetta var golfmót fyrir forritara í staðinn!
Fyrst þeir voru þegar skráðir ákváðu meðlimir félagsins margir hverjir að taka samt sem áður þátt.
En þeir komu ekki með blað og blýant til að halda utan um stigin,
heldur komu þeir bara með fartölvur til að forrita í.
Lausnin hlýtur því að vera ljós, það þarf að útbúa forrit sem heldur utan um golfstigin!

Allir þátttakendur byrja á pari og á hverri holu þarf þátttakandi einhvern fjölda högga undir eða yfir pari.
Par er fyrirfram ákveðinn fjöldi höggva sem skipuleggjendur meta að hæfur leikmaður þurfi til að ljúka holu.
Stigin eru ávallt miðað við par og eru lögð saman.
Það þýðir að $0$ táknar par og breytir ekki stigum þátttakanda.
Jákvæðar tölur eru niðurstöður yfir pari, svo $2$ táknar tveimur yfir pari og hækkar stig þátttakanda um $2$.
Neikvæðar tölur eru niðurstöður undir pari, svo $-3$ táknar þremur undir pari og lækkar stig þátttakanda um $3$.
Ef einhver er tveimur undir pari, svo einum yfir og loks á pari eru stigin þeirra $-1$.
Markmiðið er að vera með sem fæst högg, svo því lægri stig því betra.

\section*{Inntak}
Fyrsta línan inniheldur tvær jákvæðar heiltölur $n, q$ þar sem $n$ er fjöldi þátttakenda og $q$
er fjöldi fyrirspurna.
Næst fylgir ein lína með $n$ nöfnum, aðskildum með bilum, nöfn þátttakendanna. 
Engir tveir ólíkir þátttakendur hafa sama nafn.
Næst koma $q$ fyrirspurnir, hver á sinni línu.
Hver þeirra byrjar á \texttt{!} eða \texttt{?}. 
Ef línan byrjar á \texttt{?} fylgir næst nafn þátttakanda.
Þá skal prenta tvær tölur á einni línu,
hvaða sæti þeir eru í og hvað þeir eru langt frá pari.
Ef línan byrjar á \texttt{!} er verið að gefa niðurstöður næstu holu.
Á eftir \texttt{!} kemur jákvæð heiltala $k$, fjöldi þátttakenda sem voru ekki á pari.
Loks koma svo $k$ ólík nöfn og stig á línunni,
þeir þátttakendur sem voru ekki á pari og hvað þeir voru langt frá pari.
Til dæmis myndi \texttt{! 2 Atli 3 Arnar -3} tákna að allir nema Arnar og Atli voru á pari,
og að Atli var þremur yfir og Arnar þremur undir.

Öll nöfn innihalda bara enska há- og lágstafi. Hvert nafn er mest $20$ stafir.
Samtals fjöldi stafa í inntaki er mest $1 \, 000 \, 000$.
Öll stig gefin í inntaki eru minnst $-10^9$ og mest $10^9$.

\section*{Úttak}
Fyrir hverja línu í inntakinu sem byrjar á \texttt{?} skal prenta hvaða sæti sá
þátttakandi er í og hvað þeir eru langt frá pari á einni línu.
Ef það eru fleiri en einn þátttakandi með jafn mörg stig teljast þeir allir vera í efsta
sameiginlega sætinu.
Til dæmis ef annað til fimmta sæti er sameiginlegt teljast þeir allir vera í öðru sæti.

\section*{Stigagjöf}
\begin{tabular}{|l|l|l|}
\hline
Hópur & Stig & Takmarkanir \\ \hline
1     & 20   & $1 \leq n, q \leq 100$, allir eru á pari. \\ \hline
2     & 20   & $n = 1$, $1 \leq q \leq 100$. \\ \hline
3     & 20   & $1 \leq n, q \leq 100$, í öllum \texttt{?} fyrirspurnum eru engin jafntefli. \\ \hline
4     & 30   & $1 \leq n, q \leq 100$. \\ \hline
5     & 10   & $1 \leq n, q \leq 100 \, 000$. \\ \hline
\end{tabular}
