\problemname{Golf}
\illustration{0.5}{golf}{Image from \href{https://commons.wikimedia.org/wiki/File:Bruges\_Public\_Library\_Manuscript\_251\_folio\_149r\_detail\_golf.jpg}{commons.wikimedia.org}}

The members of the Competitive Programming Association of Iceland decided to partake in a Code Golf contest.
They thought this would be a contest of who can code the shortest solution to competitive programming problems,
but unfortunately this was not the case!
This was a golfing contest for programmers!
Since they had already registered many of the members decided to go anyway.
They didn't bring paper and pencils to record their scores though,
as they had been expecting a programming contest.
But for the same reason they did bring their laptops with them.
The solution is clear, a program that keeps track of golf scores has to be coded!

All participants start at par, and on each hole each participant is some number of shots over or under par.
Par is a predetermined number of shots that designers estimate it will take a proficient golfer to complete the hole.
The scores are always compared to par and are added together.
This means $0$ denotes par, which does not change a contestant's score.
Positive numbers denote results over par, 
so $2$ denotes two over par and increases a participant's score by $2$.
Negative numbers denote results under par, 
so $-3$ denotes three under par and decreases a participant's score by $3$.
If someone is two under par, then one over, and finally on par their total score is then $-1$.
The goal is to have as few shots as possible, so the lower the score, the better.

\section*{Input}
The first line of input contains two positive integers $n, q$ where $n$ is the number of participants
and $q$ is the number of queries.
Next there is a line with $n$ names, separated by spaces, the names of the participants.
No two different participants will have the same name.
Next there are $q$ queries, each on its own line.
Each query begins with \texttt{!} or \texttt{?}.
If the line begins with \texttt{?} the name of a contestant will follow.
Then you are to print two numbers on a single line,
which place they are in and how far they are from par.
If the line begins with \texttt{!} the query gives the results of the next hole.
After the \texttt{!} there is a positive integer $k$,
the number of contestants who were not on par.
Then there are $k$ different names of participants along with their scores.
For example \texttt{! 2 Atli 3 Arnar -3} would mean that everyone but Arnar and Atli were on par,
and that Atli was three over par and Arnar three under par.

All names only contain English upper and lower case letters.
Each name is at most $20$ letters
The total number of letters in the input is at most $1 \, 000 \, 000$.
All points in the input are at least $-10^9$ and at most $10^9$.

\section*{Output}
For each line in the input that begins with \texttt{?} print the place of that
contestant along with how far from par they are.
If there is more than one contestant with the same score,
they are all considered to be in the highest place among them.
For example if second to fifth place all have the same score,
they are all considered to be in second place.

\section*{Scoring}
\begin{tabular}{|l|l|l|}
\hline
Group & Points & Constraints \\ \hline
1     & 20   & $1 \leq n, q \leq 100$, everyone is on par. \\ \hline
2     & 20   & $n = 1$, $1 \leq q \leq 100$. \\ \hline
3     & 20   & $1 \leq n, q \leq 100$, in all \texttt{?} queries there are no ties. \\ \hline
4     & 30   & $1 \leq n, q \leq 100$. \\ \hline
5     & 10   & $1 \leq n, q \leq 100 \, 000$. \\ \hline
\end{tabular}
