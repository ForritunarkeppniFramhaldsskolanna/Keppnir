\problemname{Flugvallakóðar}
\illustration{0.5}{airport}{Mynd fengin af \href{https://flic.kr/p/4x29UW}{flickr.com}}

Flugvellir eru þekktir með tveimur gerðum af kóðum: IATA og ICAO.
Til dæmis hefur flugvöllurinn í Keflavík IATA kóðann KEF og ICAO kóðann BIKF.

Í Absúrdistan, hins vegar, þá eru engir alþjóðlegir staðlar notaðir, allt þarf að vera innlent og upprunið í Absúrdistan til að það megi nota það í Absúrdistan.
Þannig að þegar fyrsti flugvöllurinn í Absúrdistan var byggður í höfuðborginni, Sillysocks, þá var kóðinn fyrir hann valinn af þingmönnum á þinginu í Absúrdistan sem fylgir eftirfarandi reglum:

\begin{itemize}
    \item Kóðinn er alltaf 3 stafir.
    \item Fyrstu 3 stafirnir í nafni borgarinnar eru notaðir, ef enginn annar flugvöllur er með þann kóða.
    \item Annars eru fremstu þrír stafirnir sem mynda kóða sem er ekki tekinn valdir, þeir þurfa ekki að vera hlið við hlið en þeir þurfa að vera í sömu röð í nafni og í kóða.
    \item Fremstu þrír merkir hér að fyrsti valdi stafurinn sé eins framarlega og hægt er. 
    \item Ef tveir valkostir hafa fyrsta valda staf á sama stað er sá kóði valinn sem hefur annann valinn staf framar.
    \item Loks ef tveir valkostir hafa fyrsta og annann staf á sama stað er sá kóði valinn sem hefur þriðja valinn staf framar.
\end{itemize}

Þessi staðall ber heitið AANS eða Absúrdistan Airport National Standard.
Þú færð aðgang að gagnagrunni yfir flugvelli, og þarft að segja til um hvað AANS kóðinn er fyrir hvern flugvöll.

Háir og láir stafir skipta ekki máli í inntaki, þar sem AANS kóðar eru alltaf skrifaðar með hástöfum.
Hins vegar verður að passa að prenta AANS kóðann í úttakinu með hástöfum.

Til dæmis væri Sillysocks flugvöllurinn í Absúrdistan með AANS kóðann \texttt{SIL}, en ef það væri byggður annar flugvöllur við bæinn Silverstone, þá fengi hann AANS kóðann \texttt{SIV}, þar sem \texttt{SIL} er nú þegar í notkun, og V kemur eftir L í nafninu á Silverstone.

\section*{Inntak}
Fyrsta línan í inntakinu inniheldur eina jákvæða heiltölu $n$, fjöldi flugvalla í Absúrdistan.
Næstu $n$ línur innihalda nöfnin á bæjum flugvallanna, hvert nafn er strengur sem inniheldur aðeins enska há- og lágstafi.
Hvert nafn er minnst $3$ stafir og heildarfjöldi stafa í inntaki er mest $2 \, 000 \, 000$.

Athugið að inntakið kemur í þeirri röð sem flugvellirnir voru byggðir, þannig að því fyrr sem flugvöllurinn er í inntakinu, því hærri forgang hefur hann í vali á kóða.

\section*{Úttak}
Fyrir hvern flugvöll þá á að skrifa út AANS kóðann fyrir hann.
Ef enginn gildur AANS kóði er í boði þýðir það að flugvöllurinn var aldrei byggður og því skal prenta \texttt{":("} í staðinn.

\section*{Stigagjöf}
\begin{tabular}{|l|l|l|}
\hline
Hópur & Stig & Takmarkanir \\ \hline
1     & 10   & Nafn hvers flugvallar er mest $200$ stafir, $n = 1$. \\ \hline
2     & 20   & Nafn hvers flugvallar er mest $200$ stafir, $1 \leq n \leq 100$, ef til er kóði notar hann fyrstu þrjá stafi nafnsins. \\ \hline
3     & 20   & Nafn hvers flugvallar er mest $200$ stafir, $1 \leq n \leq 100$, ef til er kóði notar hann fyrstu tvo stafi nafnsins. \\ \hline
4     & 20   & Nafn hvers flugvallar er mest $200$ stafir, $1 \leq n \leq 100$, ef til er kóði notar hann fyrsta staf nafnsins. \\ \hline
5     & 20   & Nafn hvers flugvallar er mest $200$ stafir, $1 \leq n \leq 100$. \\ \hline
6     & 5    & Nafn hvers flugvallar er mest $200$ stafir, $1 \leq n \leq 10 \, 000$. \\ \hline
7     & 5    & Engar frekari takmarkanir. \\ \hline
\end{tabular}
