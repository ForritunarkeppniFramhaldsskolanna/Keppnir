\problemname{Flugvallakóðar}
\illustration{0.5}{airport}{Image from \href{https://flic.kr/p/4x29UW}{flickr.com}}

Airports are known by two types of codes: IATA and ICAO.
As an example the airport in Keflavík has the IATA code KEF and the ICAO code BIKF.

In Absurdistan however there are no international standards in use, 
everything has to be domestic and be originally made in Absurdistan if it is to be used in Absurdistan.

Thus, when the first airport in Absurdistan was built in the capital city, Sillysocks, 
the system of codes for airports was decided by the parliament of Absurdistan.
It consists of the following rules:

\begin{itemize}
    \item The code is always 3 letters.
    \item The first 3 letters of the name of the city are used, unless some other airport already has that code.
    \item If that code is already taken, the available code appearing first in the city name is chosen. 
        The letters do not have to be adjacent in the name, but they have to be in the same order in the name and in the code.
    \item If there are several options, the code with the first letter appearing earliest in the city name is chosen.
    \item If two such options have the first letter in the same place, the one with the second letter appearing earliest is chosen.
    \item Finally if two such options have the first two letters in the same place, the one with the third letter appearing earliest is chosen.
\end{itemize}

This standard is called AANS, or Absurdistan Airport National Standard.
You are given access to a database of airports and need to determine the AANS code for each airport.

Upper and lower case letters are not distinguished as the AANS codes must always be upper case.
You must make sure to respect this in the output.

As an example the airport at Sillysocks in Absurdistan would have the AANS code \texttt{SIL}. 
But if they were then to build another airport at the town Silverstone, it would get the AANS code \texttt{SIV},
as \texttt{SIL} is already in use and V comes after L in Silverstone.

\section*{Input}
The first line of the input contains a single positive integer $n$, the number of airports in Absurdistan.
The next $n$ lines each contain the names of the airport towns. 
The names contain only English lower and upper case letters.
Each name is at least $3$ letters and the total number of letters in the input is at most $2 \, 000 \, 000$.

Note that the input is given in the order the airports were built,
so an airport appearing earlier in the input has priority over the later one in choosing a code.

\section*{Output}
For each airport, print its AANS code.
If there is no valid AANS code possible, that means the airport was never built, so instead you should print \texttt{":("}.

\section*{Scoring}
\begin{tabular}{|l|l|l|}
\hline
Group & Points & Constraints \\ \hline
1     & 10   & Each airport name is at most $200$ letters, $n = 1$. \\ \hline
2     & 20   & Each airport name is at most $200$ letters, $1 \leq n \leq 100$, if a code exists it uses the first 3 letters of the name. \\ \hline
3     & 20   & Each airport name is at most $200$ letters, $1 \leq n \leq 100$, if a code exists it uses the first 2 letters of the name. \\ \hline
4     & 20   & Each airport name is at most $200$ letters, $1 \leq n \leq 100$, if a code exists it uses the first letter of the name. \\ \hline
5     & 20   & Each airport name is at most $200$ letters, $1 \leq n \leq 100$. \\ \hline
6     & 5    & Each airport name is at most $200$ letters, $1 \leq n \leq 10 \, 000$. \\ \hline
7     & 5    & No further constraints. \\ \hline
\end{tabular}
