\problemname{Púsluspil}
\illustration{0.4}{puzzle}{Mynd fengin af \href{https://flic.kr/p/jagpmg}{flickr.com}}

Davíð hefur mjög gaman af púsluspilum, það vill svo til að hann keypti sér nýlega púsluspil frá Púsluspilabúð Balda.
Hann fer rakleitt heim og byrjar að púsla.
Þegar hann er nánast búinn fattar Davíð að það vantar nokkur púsl í nánast kláruðu púsluspilamyndina, en kassinn er tómur!
Það hefur greinilega vantað nokkur púsl í púsluspilakassann.
Hann fer til Balda og útskýrir hvað gerðist, og Balda líður svo illa með þetta að hann gefur Davíði annað samskonar púsluspil, en leyfir honum að halda gallaða kassanum.
Davíð fer heim til að klára púsluðu myndina, og kemst að því að það vantar líka nokkur púsl í nýja kassann. Það vill hins vegar svo til að þetta voru aðrir púslbitar en vantaði úr fyrsta kassanum.

Núna leitar Davíð til þín, til að ákvarða hvort hann geti yfir höfuð klárað púsluspilið.

Davíð hefur þurft að fara $n$ sinnum til Balda að fá púslkassa, og er þar af leiðandi með $n$ kassa sem hann getur nýtt.
Klárað púsluspil hefur $m$ púslbita og kassi númer $i$ hefur $k_i$ púsl bita.

Púslbitar eru númeraðir með tölunum frá $1$ upp í $m$. Púslbiti $p$ í kassa $a$ er samskonar púslbiti og $p$ í kassa $b$, þar sem $a$ og $b$ eru púslkassar.

Púsl er talið klárað þegar öll púsluspil, $1, 2, \ldots, m$, eru komin saman til að mynda púslmyndina.

Athugaðu að púsluspilin eru ekki endilega í raðaðri röð, nema annað sé tekið fram.

\section*{Inntak}
Fyrsta línan inniheldur tvær heiltölur $n$ og $m$, aðskildar með bili.

Næstu $n$ línur lýsir hver einum púslkassa.
Lýsing á púslkassa hefst á heiltölu $k_i$, fjöldi púsla í kassa $i$, þar sem $0 \leq k_i \leq m$. Næst fylgja $k_i$ heiltölur $p_1, p_2, \ldots, p_{k_i}$, sem tákna púslbitana í kassa $i$, þar sem $1 \leq p_i \leq m$.
Tölurnar eru aðskildar með bilum.

Það mun alltaf gilda að $0 \leq n \cdot m \leq 500\,000$.

\section*{Úttak}
Skrifaðu út \texttt{"Jebb"} ef Davíð getur klárað púslið, eða \texttt{"Neibb"} ef Davíð getur það ekki.

\section*{Stigagjöf}
\begin{tabular}{|l|l|l|}
\hline
Hópur & Stig & Takmarkanir \\ \hline
1     & 10   & $n = 1, 1 \leq m \leq 100$, bitar eru í hækkandi röð, fyrsti bitinn er alltaf $1$ \\ \hline
2     & 10   & $0 \leq n \leq 100, m = 0$ \\ \hline
3     & 10   & $n = 0, 0 \leq m \leq 100$ \\ \hline
4     & 35   & $0 \leq n, m \leq 500$\\ \hline
5     & 35   & $0 \leq n, m \leq 500\,000$ \\ \hline
\end{tabular}

