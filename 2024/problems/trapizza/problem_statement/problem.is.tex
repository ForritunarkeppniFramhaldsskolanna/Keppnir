\problemname{Trapizza}
\illustration{0.5}{Trapizza}{Mynd teiknuð af Evu.}
Tryggvi, eigandi Mahjong Pizza, og Ómar, eigandi Trapizzu, eru erkióvinir. 
Þeir rekast alltaf á hvorn annan niðri í bæ og enda samtöl þeirra alltaf í endalausum rifrildum, oftar en ekki tengt pizzunum þeirra.
Tryggvi segir að trapisulaga pizzur séu siðblinda og að samfélagið ætti ekki að leyfa að selja þær á landinu.
Ómar segir að hringlaga pizzur eru í fortíðinni og að framtíðin er að hafa frumlegri lögun á hlutum.

Síðasta rifrildi þeirra endaði með því að bera saman hvor pizzan væri stærri þar sem verðið á pizzunum er eins.
Þeir kunna ekkert að reikna hvor er með stærri pizzuna, enda eru þeir engir stærðfræðingar, og biðja því um hjálp frá þér.
Tryggvi segir þér þvermálið á pizzunum sem hann selur og Ómar segir þér hliðarlengdir og hæð pizzunnar sem hann selur.
Getur þú hjálpað þeim að leysa málin og stöðva þessi endalausu rifrildi?

\section*{Inntak}
Inntak er fjórar línur. Fyrsta línan inniheldur eina heiltölu $0 \leq d \leq 100$, þvermál Mahjong pizzunnar.
Seinni línan inniheldur eina heiltölu $0 \leq a \leq 100$, hliðarlengd annarar samsíðu hliðar Trapizzu pizzunnar.
Þriðja línan inniheldur eina heiltölu $0 \leq b \leq 100$, hliðarlengd hinnar samsíðu hliðar Trapizzu pizzunnar.
Fjórða línan inniheldur eina heiltölu $0 \leq h \leq 100$, hæð Trapizzu pizzunnar.

\section*{Úttak}
Ef Trapizza hefur stærri pizzuna þá skal skrifa út \textit{Trapizza!}. Ef Mahjong hefur stærri pizzuna þá skal skrifa út \textit{Mahjong!}.
Ef pizzurnar eru jafn stórir þá skal skrifa út \textit{Jafn storar!}.

\section*{Stigagjöf}
\begin{tabular}{|l|l|l|}
\hline
Hópur & Stig & Takmarkanir \\ \hline
1     & 100   & Engar frekari takmarkanir. \\ \hline
\end{tabular}
