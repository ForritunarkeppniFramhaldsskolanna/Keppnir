\problemname{Lykilorð}
\illustration{0.5}{already_used}{You can always find worse password policies\textellipsis}

In digital systems requiring authentication, it is rather common that a user needs to create their own password to pair with their username.
The passwords should not be stored in plaintext, instead they should be processed by a hash function.
Simply put, the hash function transforms the password from plaintext to an integer.
Many passwords can end up as the same integer and the process is difficult to invert which makes storing the integer safe.
Then the password entered during the login process can be hashed and a check can be made whether the correct integer appears.
If the integers are equal, then it is astronomically unlikely that the password used in the login attempt is different from the one originally set by the user.

Many of these systems provide the users with rules which their passwords must satisfy.
A sensible lower bound on length is set to ensure passwords are long enough not to be susceptible to brute force attacks.
The method in question is based on trying all possible passwords which satisfy the password rules.
This is usually done by starting at the lowest allowed length and working up to higher length passwords.
Additionally, there are arguments for an upper bound on a password,
such as some hash function algorithms only working text containing at most $64$ symbols.
Another good rule is forbidding the use of common passwords.
The reason for this rule is that the outputs of the hash functions for those passwords are known,
thus it is easy to reverse the process for those outputs.
For the purpose of this task, we assume the hash functions accept at most $32$ symbols,
setting a base upper bound on the length of all passwords.

Years ago, system developers thought that by imposing more rules on passwords they would improve security.
Many websites required that passwords must include uppercase letters, lowercase letters and digits, one of each type.
Other systems took this a step further and forbid the use of consecutive equal symbols or incrementing digits or adjacent letters in the alphabet.
For a long time this was considered an improvement, but made matters worse.
By imposing all these rules, the set of valid passwords became much smaller, meaning the brute force attack has fewer passwords to try.
The passwords also became harder for users to remember.
Naturally, the users reused passwords between systems as the could, even writing them down on sticky notes stuck to their table.
Luckily, we know better today! Or do we?

You are given rules for password creation in different systems.
Your program should create passwords based on the given rules.
Rules are given in rule sets.
We say that a password satisfies a rule set if it satisfies all the rules in that rule set.
A password is considered valid if it satisfies at least one rule set.
However, if there are no rule sets provided, then all passwords are valid.

It saddens the author profusely that a good portion of the test cases are real rules in real systems which still exist to this day.

\section*{Input}
The first line of input consist of two integers $n$, the number of rule sets, and $m$, the number of passwords to output.
You may assume that $1 \leq m \leq 2\,000\,000$.

Then $n$ descriptions of rule sets follow.
Each description of a rule set starts with one line with a positive integer denoting the number of rules in the rule set.
Then the rules follow, one line per rule.

The systems are international, therefore the rules are given in English.
The following rules are defined with the prefix \texttt{The password must:}
\begin{enumerate}
    \item \texttt{contain at least \{count\} symbols}
    \item \texttt{contain at most \{count\} symbols}
    \item \texttt{contain at least \{count\} symbols from \{symbol set\}}
    \item \texttt{contain at most \{count\} symbols from \{symbol set\}}
    \item \texttt{contain \{count\} consecutive equal symbols}
    \item \texttt{contain \{count\} consecutive incrementing symbols}
    \item \texttt{contain \{count\} consecutive decrementing symbols}
    \item \texttt{contain an english word}
    \item \texttt{include \{substring\} as a substring}
    \item \texttt{start with \{prefix\}}
    \item \texttt{end with \{suffix\}}
    \item \texttt{be an english word}
    \item \texttt{be in top \{count\} most common passwords}
\end{enumerate}

There will be at least $0$ and at most $30$ rules in total over all rule sets.
For rules $5$ to $13$, the prefix \texttt{"not "} may be added to get an opposite effect.
Some of the rules have variables and the full description of the input constraints is complicated, but you may assume they are reasonable.
For example, the \texttt{count} in rules $1$ and $2$ can range from $0$ to $32$.
However, in rules $5$ to $7$, the variable \texttt{count} is at least $2$ and at most $5$.
Note that collectively, the rules may impose restrictions such that no password is valid.

The symbols appearing in the input are those with ASCII values between $33$ and $126$.
This includes English uppercase letters, English lowercase letters, digits, punctuation and some other symbols, such as curly braces.
Note that spaces cannot appear in passwords nor in the variables of rules in the input.

You will receive a list of common words to assist you in creating passwords that satisfy the requirements.
\footnote{The words were found in the list \href{https://github.com/tkoop/popular-english-words/}{here} which collected them from Wikipedia.}
You will also receive a list of passwords ordered by popularity, the top password being the most commonly used one.
\footnote{The passwords were found in the list \href{https://github.com/danielmiessler/SecLists/tree/master/Passwords/Common-Credentials}{here} which is constructed from real world statistical data.}
These lists will be used to verify the correctness of your output.
\footnote{The lists are unchanged, except for a limitation on the number of passwords and one password being removed which had symbols outside the symbol set of this task.
The contents of these lists do not represent the views of anyone associated with the task.
They are simply raw data ordered by frequency.}


\section*{Output}
If $m$ or more passwords satisfy the given rules, you should first output a line containing the text \texttt{"Mogulegt!"}.
Then $m$ distinct passwords should follow, where each password satisfies at least one rule set, except in the case of no rule sets.
In theses cases, you will at most need to output $1\,000\,000$ passwords and at most $6\,291\,456$ bytes, or $6$ MB, are required for a correct output.

However, if the number of distinct passwords satisfying the requirements are lower than $m$,
you should output one line with the text \texttt{"Omogulegt!"}.

\section*{Scoring}
There are $100$ test cases in total and you will receive one point for each correctly solved test case.

\section*{More rules}
A few more rules follow here that are not used in the task but still exist in the real world.
\begin{itemize}
    \item \texttt{not be the same as the username}
    \item \texttt{not be same as previous \{count\} passwords}
    \item \texttt{not be changed within \{count\} hours of last change}
    \item \texttt{not be same as other passwords in system}
    \item \texttt{not contain a reference to a pop culture icon}
    \item \texttt{not include your social security number or any subset of your social security number that is more than a single number}
    \item \texttt{not include words that can be found in any dictionary regardless of language}
    \item \texttt{be your date of birth in format ddmmyyyy}
\end{itemize}
