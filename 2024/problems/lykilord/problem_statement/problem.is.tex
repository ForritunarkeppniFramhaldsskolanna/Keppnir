\problemname{Lykilorð}
\illustration{0.5}{already_used}{Það má alltaf finna verri lykilorðareglur\textellipsis}

Í stafrænum kerfum með auðkenningu er algengt að notendur þurfi að smíða sín eigin lykilorð til að para við notandanafnið sitt.
Lykilorðin skal ekki geyma í sínu hráa formi, heldur eru þau sett í gegnum tætifall.
Til einföldunar má segja að tætifallið breyti lykilorðinu úr hráu textaformi yfir í heiltölu.
Mörg lykilorð geta endað í sömu heiltölu og erfitt er að snúa ferlinu við, sem gerir það öruggt að geyma heiltöluna.
Ofan á það er þá hægt að tæta lykilorð við innskráningu seinna í framtíðinni og athuga hvort sama tala komi út og er geymd í gagnagrunni kerfisins.
Ef heiltölurnar passa að þá eru hverfandi líkur á því að lykilorðið sem notað var við innskráningu sé frábrugðið því sem var upprunalega stillt af notandanum.

Mörg þessarra kerfa bera undir notendur reglur sem lykilorð þeirra þurfa að standast.
Skynsamleg mörk á lágmarkslengd eru sett til að tryggja að lykilorð séu nógu löng til þess að ekki sé hægt að beita ofbeldisaðferðinni.
Sú aðferð felst í því að prófa öll möguleg lykilorð sem uppfylla reglurnar.
Þá er oftast byrjað á stystu leyfilegu lykilorðunum og unnið sig upp í lengd.
Einnig eru til rök fyrir því að takmarka hámarkslengd þar sem sum reiknirit fyrir tætiföll virka bara fyrir texta með mest $64$ tákn.
Önnur góð regla er að ekki leyfa notkun á algengustu lykilorðum sögunnar.
Ástæðan fyrir þeirri reglu er að útkomur tætifalla fyrir þau lykilorð eru þekkt og því auðvelt að snúa ferlinu fyrir þær útkomur.
Í þessu verkefni gerum við ráð fyrir að tætiföllin virki fyrir mest $32$ tákn og því eru það grunn efri mörk á lengd allra lykilorða.

Fyrir mörgum árum datt kerfishönnuðum í hug að með fleiri reglum um lykilorðin myndu þau tryggja að lykilorðin væru öruggari.
Kröfur voru gerðar á mörgum vefsíðum að lykilorð skuli innihalda hástafi, lágstafi og tölustafi, minnst einn af hverri týpu.
Önnur kerfi tóku þetta skrefinu lengra og leyfðu ekki endurtekna stafi í röð né hækkandi runur af tölustöfum eða aðliggjandi bókstafi í stafrófsröð.
Lengi var þetta talið bæta öryggi en gerði illt verra.
Með því að bæta við þessum reglum varð heildarmengi leyfilegra lykilorða smærra, og því færri lykilorð í boði til að prófa með ofbeldisaðferðinni.
Einnig urðu lykilorð erfiðari fyrir notendur að muna.
Því endurnýttu þeir sömu lykilorð milli kerfa eins og þeir gátu og skrifuðu þau jafnvel niður á miða sem þeir límdu við borðin sín.
Sem betur fer vitum við betur í dag! Eða hvað?

Þú færð gefnar reglur fyrir lykilorðasmíði í mismunandi kerfum og á forritið að smíða lykilorð út frá gefnu reglunum.
Reglurnar eru gefnar í söfnum.
Hvert safn getur innihaldið margar reglur.
Við segjum að lykilorð uppfylli kröfur reglusafns ef það uppfyllir allar reglurnar í reglusafninu.
Til að lykilorð teljist gilt þarf það að uppfylla að minnsta kosti eitt reglusafn.
Hins vegar, ef engin reglusöfn eru gefin eru öll lykilorð talin gild.

Það hryggir höfund verkefnisins að segja að stór hluti prufutilvika séu alvöru reglur í alvöru kerfum þann dag sem verkefnið var skrifað.

\section*{Inntak}
Fyrsta línan inniheldur tvær heiltölur $n$, fjölda reglusafna, og $m$, fjölda lykilorða sem skal skrifa út.
Þú mátt gera ráð fyrir að $m \leq 2\,000\,000$.

Næst fylgja lýsingar á $n$ reglusöfnum.
Hver lýsing á reglusafni byrjar á einni línu með einni jákvæðri heiltölu sem táknar fjölda reglna í reglusafninu.
Næst fylgja reglurnar, ein lína fyrir hverja reglu.

Kerfin eru alþjóðleg og því eru reglurnar gefnar upp á ensku.
Eftirfarandi reglur eru skilgreindar með forskeytinu \texttt{The password must:}
\begin{enumerate}
    \item \texttt{contain at least \{count\} symbols}
    \item \texttt{contain at most \{count\} symbols}
    \item \texttt{contain at least \{count\} symbols from \{symbol set\}}
    \item \texttt{contain at most \{count\} symbols from \{symbol set\}}
    \item \texttt{contain \{count\} consecutive equal symbols}
    \item \texttt{contain \{count\} consecutive incrementing symbols}
    \item \texttt{contain \{count\} consecutive decrementing symbols}
    \item \texttt{contain an english word}
    \item \texttt{include \{substring\} as a substring}
    \item \texttt{start with \{prefix\}}
    \item \texttt{end with \{suffix\}}
    \item \texttt{be an english word}
    \item \texttt{be in top \{count\} most common passwords}
\end{enumerate}

Það verða minnst $0$ og mest $30$ reglur samanlagt yfir öll reglusöfn.
Fyrir reglur $5$ til $13$ má rita \texttt{"not "} fyrir framan þær til að fá andstæða virkni.
Sumar reglurnar taka við breytum og eru full lýsing á skorðunum þar flókin en gera má ráð fyrir að þær séu skynsamlegar.
Til dæmis ef það eru margar reglur að þá getur breytan \texttt{count} í reglum $1$ og $2$ verið frá $0$ upp í $32$.
Hins vegar, í reglum $5$ til $7$ er breytan \texttt{count} minnsta lagi $2$ og mesta lagi $5$.
Athugaðu samt sérstaklega að samanlagt geta reglurnar orðið til þess að engin lykilorð séu gild.

Táknin sem geta komið fyrir í inntaki eru þau sem hafa ASCII gildi á bilinu $33$ upp í $126$.
Þetta inniheldur enska hástafi, enska lágstafi, tölustafi, greinarmerki og nokkur önnur tákn, til dæmis slaufusviga.
Athugið sérstaklega að bilstafur getur hvorki komið fyrir í lykilorðum né breytum reglna í inntaki.

Þú færð lista af algengum orðum í ensku til að aðstoða þig við að búa til lykilorð sem uppfylla kröfurnar.
\footnote{Orðin voru fengin úr listanum \href{https://github.com/tkoop/popular-english-words/}{hér} sem safnaði þeim saman frá Wikipedia.}
Þú færð einnig einnig lista af lykilorðum sem eru röðuð eftir vinsæld, efsta lykilorðið er algengasta lykilorðið.
\footnote{Lykilorðin voru fengin úr listanum \href{https://github.com/danielmiessler/SecLists/tree/master/Passwords/Common-Credentials}{hér} sem er byggður á alvöru tölfræði.}
Þessir listar eru notaðir til að staðfesta að svarið þitt sé talið rétt.
\footnote{Listarnir er óbreyttir, fyrir utan takmörkun á stærð og eitt lykilorð utan stafamengis verkefnisins var fjarlægt.
Innihald þeirra táknar ekki skoðanir neins sem kemur að þessu verkefni heldur eru þetta hrá gögn sem hefur verið raðað eftir tíðni.}

\section*{Úttak}
Ef til eru $m$ eða fleiri mismunandi lykilorð sem uppfylla gefnu reglurnar skaltu fyrst skrifa út línu með textanum \texttt{"Mogulegt!"}.
Þar á eftir skaltu skrifa út $m$ mismunandi lykilorð þar sem hvert lykilorð uppfyllir allar reglurnar í að minnsta kosti einu reglusafni í inntaki, nema í tilvikinu þar sem engin reglusöfn eru gefin því þá eru öll lykilorð gild.
Í þessum tilvikum þarf mest að skrifa út $1\,000\,000$ lykilorð og þarf mest $6\,291\,456$ eða $6$ MB til að skrifa út svarið.

Hins vegar, ef fjöldi mismunandi lykilorða sem uppfylla kröfurnar er lægri en $m$, skaltu skrifa út eina línu með textanum \texttt{"Omogulegt!"}.

\section*{Stigagjöf}
Prufutilvikin eru $100$ samtals og þú færð eitt stig fyrir hvert leyst prufutilvik.

\section*{Fleiri reglur}
Hér eru nokkrar fleiri reglur sem komust ekki í verkefnið en eru samt til úti í heiminum.
\begin{itemize}
    \item \texttt{not be the same as the username}
    \item \texttt{not be same as previous \{count\} passwords}
    \item \texttt{not be changed within \{count\} hours of last change}
    \item \texttt{not be same as other passwords in system}
    \item \texttt{not contain a reference to a pop culture icon}
    \item \texttt{not include your social security number or any subset of your social security number that is more than a single number}
    \item \texttt{not include words that can be found in any dictionary regardless of language}
    \item \texttt{be your date of birth in format ddmmyyyy}
\end{itemize}
