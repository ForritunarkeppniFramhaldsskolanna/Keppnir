\problemname{ASCII Kassi 2}
\illustration{0.4}{fish}{Mynd fengin af \href{https://commons.wikimedia.org/wiki/File:Fish-shell-logo-ascii.svg}{wikimedia.commons.org}}

Í fyrra litum við aftur í tímann og létum nemendur teikna ASCII kassa sem dæmi.
Í ár viljum við líta á hlutina frá nýju sjónarhorni!
Því er dæmið í ár að teikna ASCII kassa, en á ská!

Síðast var notað við táknin \texttt{+}, \texttt{-} og \texttt{|} til að teikna kassann.
Þegar þessu hefur verið snúið um $45^\circ$ fást þá táknin \texttt{x}, \texttt{/} og \texttt{\textbackslash}.
Hornin á kassanum verða því táknuð með \texttt{x} en hin tvö táknin notuð til að teikna hliðarnar.

Til að kassinn birtist rétt þarf að passa að setja réttan fjölda bila á undan og milli stafanna í hverri línu.
Þar að auki má ekki prenta nein auka bil á eftir kassanum í hverri línu, heldur á að koma nýlínustafur beint á eftir
seinasta tákni kassans í hverri línu.
Efsta lína kassans er þá ávallt einhver fjöldi bila og svo eitt \texttt{x}.
Þar næst kemur lína með \texttt{/} vinstra megin fyrir neðan \texttt{x} og
Þetta heldur svo áfram þar til hliðarnar eru af réttri lengd.
\texttt{\textbackslash} hægra megin fyrir neðan \texttt{x}, nema hliðarlengd kassans sé $0$.
Loks kemur svo \texttt{x} vinstra megin fyrir neðan \texttt{/} og hægra megin fyrir neðan \texttt{\textbackslash}.
Þetta er svo endurtekið með spegluðum hætti til að klára kassann.

\section*{Inntak}
Fyrsta og eina lína inntaksins inniheldur eina heiltölu $n$, hliðarlengd kassans.

\section*{Úttak}
Prentið kassa með hliðarlengd $n$ eins og lýst er að ofan.
Hafðu í huga að úttakið þarf að vera nákvæmlega rétt, meira að segja bilstafirnir.

\section*{Stigagjöf}
\begin{tabular}{|l|l|l|}
\hline
Hópur & Stig & Takmarkanir \\ \hline
1     & 30   & $0 \leq n \leq 3$. \\ \hline
2     & 30   & $0 \leq n \leq 10$. \\ \hline
3     & 20   & $0 \leq n \leq 100$. \\ \hline
4     & 20   & $0 \leq n \leq 1\,000$. \\ \hline
\end{tabular}
