\problemname{ASCII Kassi 2}
\illustration{0.4}{fish}{Image from \href{https://commons.wikimedia.org/wiki/File:Fish-shell-logo-ascii.svg}{wikimedia.commons.org}}

Last year we looked back in time and made contestants draw ASCII boxes as a challenge.
This year we want to look at things from a new perspective!
Therefore the challenge this year is to draw an ASCII box, but diagonally!

Last time the symbols \texttt{+}, \texttt{-} and \texttt{|} were used to draw the box.
When this has been rotated by $45^\circ$ you get the symbols \texttt{x}, \texttt{/} and \texttt{\textbackslash}.
The corners of the box are therefore drawn with a \texttt{x} and the other symbols are used for the sides.

To make sure the box is printed correctly you need to put the right number of spaces before and between the
symbols on each line. The last non-whitespace character should be immediately followed by a newline character.
Additionally you must not print any extra spaces after the box on any line.
The first line of the box is therefore always some number of spaces followed by a single \texttt{x}.
Next there is a line that has a \texttt{/} below and to the left of the \texttt{x} and
a \texttt{\textbackslash} below and to the right of the \texttt{x}, unless the side length of the box is $0$.
This continues until the sides have the correct length.
Finally there is a line with a \texttt{x} below and to the left of the last \texttt{/} and another
\texttt{x} below and to the right of the \texttt{\textbackslash}.
This is then repeated in a mirrored way to make the bottom half of the box.

\section*{Input}
The first and only line of the input contains a single integer $n$, the side length of the box.

\section*{Output}
Print a box with side length $n$ as described above.
Be aware that the output must match exactly, even the whitespace characters.

\section*{Scoring}
\begin{tabular}{|l|l|l|}
\hline
Group & Points & Constraints \\ \hline
1     & 30   & $0 \leq n \leq 3$. \\ \hline
2     & 30   & $0 \leq n \leq 10$. \\ \hline
3     & 20   & $0 \leq n \leq 100$. \\ \hline
4     & 20   & $0 \leq n \leq 1\,000$. \\ \hline
\end{tabular}
