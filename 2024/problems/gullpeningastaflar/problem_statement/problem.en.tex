\problemname{Gullpeningastaflar}
\illustration{0.4}{stacks}{Photo from \href{https://flickr.com/photos/filterforge/8735533688/}{flickr.com}}

In front of you there are $n$ stacks, each containing $n$ gold coins.
Only one of these stacks contains real gold coins, the other stacks contain counterfeit coins.
The counterfeit coins are very well made and are nearly indistinguishable from the real ones.
The only difference is in their weight and even that one is minute.
The counterfeit coins weigh $n$ units each and the real gold coins weigh $n+1$ units each.

Your goal is to find the stack of real gold coins,
so that you may take the coins and leave the worthless counterfeit coins behind.
Since it is difficult to discern between the coins based on their weights, you have brought a scale.
You can take as many coins from each stack as you wish and place them on the scale.
The scale will then show you the total weight of the coins.
The coins then go to their respective stacks after each use of the scale.

Can you find which stack contains real gold in as few weighings as possible?

\section*{Interactivity}
This is an interactive problem.
Your program will be tested against an interactive judge which reads the output of your program and prints the input your program receives.
This interaction follows certain rules:

Your program should first read one line with one integer $n$.
In the secret test cases the value of $n$ will always be $1\,024$, but in the samples $n$ is smaller.

Then your program can either make a final guess or ask questions to obtain information.

In order to ask a question, your program must write a line starting with the symbol \texttt{?} and
then $n$ space separated integers, denoting how many coins from each stack you place on the scale.
Your program should then read one line with one integer, the total weight of the coins on the scale.

Once your program has determined the answer, it should write out the symbol \texttt{!} followed by a space and an integer denoting the number of the stack.
The stacks are numbered from $1$ to $n$.
If the answer is incorrect, the program will receive a Wrong Answer verdict. Otherwise, it will be scored as described below.

The task has a testing tool attached to help test your solution.

\section*{Scoring}
Your program will be run on many test cases and the worst result over all test cases will be used as the final score.
Your program is scored by the number of questions it asks.
If your program asks fewer than $2n$ questions, it will receive a score.
Fewer questions result in a higher score and the maximum score is $100$.
If your program asks $2n$ or more questions, it will receive a Wrong Answer verdict.
