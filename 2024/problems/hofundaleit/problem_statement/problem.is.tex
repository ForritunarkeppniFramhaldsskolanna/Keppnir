\problemname{Höfundaleit}
\illustration{0.5}{dewey}{Mynd fengin af \href{https://commons.wikimedia.org/wiki/File:HK\_Wan\_Chai\_Library\_Inside\_Bookcase\_a.jpg}{commons.wikimedia.org}}

Hallgerður Stuttbrók er stödd á bókasafninu á Reyðarfirði. 
Hún er að leita sér að einhverju skemmtilegu til að lesa, en á erfitt með að finna bókina.
Þetta er vegna þess að hún man ekki hver höfundur bókarinnar er, en á bókasafninu er bókum raðað eftir höfundi.
Getur þú hjálpað henni að finna númer bókanna sem hún vill lesa í röðinni?

\section*{Inntak}
Fyrsta línan inniheldur tvær heiltölur $n$, fjöldi bóka á bókasafninu, og $q$, fjöldi bóka sem Hallgerði langar að lesa.
Næstu $n$ línur munu hver innihalda lýsingu á einni bók, fyrst titil bókarinnar og svo höfund, aðskilin með kommu.
Loks koma $q$ línur sem hver innihalda einn bókatitil, hver þeirra lýsir bók sem Hallgerður vill lesa.
Titlar og höfundanöfn munu bara innihalda enska há- og lágstafi ásamt undirstrikum.
Engar tvær ólíkar bækur hafa sama titil.
Sérhver titill og sérhvert höfundarnafn verður mest $25$ stafir að lengd.
Samtals lengd allra strengja í inntaki verður mest $10^6$ stafir samtals.

\section*{Úttak}
Fyrir hverja bók sem Hallgerður vill lesa, prentið númer hvað hún er í röðinni ef öllum bókum er raðað eftir höfundanafni.
Hér lítum við svo á að fyrsta bókin sé númer $1$, næsta númer $2$ og svo framvegis.
Ef bók er ekki til skal prenta $-1$ í staðinn.
Ef höfundur er með fleiri en eina bók er bókunum innbyrðis raðað eftir titli.
Röðin er venjulega stafrófsröð strengja út frá ASCII-gildi.
Athugið að þetta er sama röð og innbyggða röðunarfall flestra forritunarmála skilar.
Til dæmis \texttt{sorted} í Python eða \texttt{std::sort} í C++.

\section*{Stigagjöf}
\begin{tabular}{|l|l|l|}
\hline
Hópur & Stig & Takmarkanir \\ \hline
1     & 10   & $0 \leq n, q \leq 100$, engin af bókunum sem beðið er um eru til á safninu. \\ \hline
2     & 10   & $0 \leq n, q \leq 100$, enginn höfundur er með fleiri en eina bók, bókasafnsbókum er raðað eftir höfundi, allar bækur til á safninu. \\ \hline
3     & 20   & $0 \leq n, q \leq 100$, enginn höfundur er með fleiri en eina bók, allar bækur til á safninu. \\ \hline
4     & 20   & $0 \leq n, q \leq 100$, enginn höfundur er með fleiri en eina bók. \\ \hline
5     & 20   & $0 \leq n, q \leq 100$. \\ \hline
6     & 20   & $0 \leq n, q \leq 100\,000$. \\ \hline
\end{tabular}
