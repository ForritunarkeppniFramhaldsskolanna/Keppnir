\problemname{Undratré}
Unnur hefur mikinn áhuga á plöntum. Hún vökvar allar plönturnar sínar einu sinni á dag.
Flestar plönturnar eru frekar eðlilegar en eitt tré er sérstaklega einkennilegt og einstakt. Það er undratréið hennar Unnar.
Upprunalega er undratréið bara ein grein sem stendur uppúr jörðinni, sú grein er kölluð rót undratrésins. Undratréið er sérstakt á þann máta að það vex mismunandi eftir því hvar það er vökvað.
Ef grein er vökvuð þá vex önnur grein út frá henni. Ekki má vökva sömu grein of oft, því ef einhver grein er vökvuð oftar en $k$ sinnum þá visnar tréð og deyr. Unnur vökvar nákvæmlega eina grein á dag.
Ef grein hefur aldrei verið vökvuð þá er hún kölluð lauf. Undratréið er því alltaf með allavega eitt lauf.
Breidd undratrésins er skilgreind sem fjöldi laufa. Því er breidd undratrésins upprunalega $1$.
Hæð undratrésins er skilgreind sem fjöldi greina í lengstu leið frá rót að laufi, þar með talin rótin og laufið. Því er hæð undratrésins upprunalega $1$.
Unnur undrar sig á því hversu breitt og hátt undratréið getur orðið eftir $n$ daga, getur þú hjálpað Unni?

\section*{Inntak}
Fyrsta og eina línan inniheldur tvær heiltölur $n$ og $k$, fjöldi daga og hversu oft má vökva hverja grein.

\section*{Úttak}
Tvær línur þar sem fyrri línan innheldur minnstu og mestu mögulega hæð trésins eftir $n$ daga.
Seinni línan skal innihalda minnstu og mestu mögulega breidd trésins eftir $n$ daga.

\section*{Stigagjöf}
\begin{tabular}{|l|l|l|l|}
\hline
Hópur & Stig & Inntaks takmarkanir \\ \hline
    1 &   50 & $0 \leq n \leq 10^{9}$, $k = 2$ \\ \hline
    2 &   50 & $0 \leq n \leq 10^{15}$, $1 \leq k \leq 10^4$ \\ \hline
\end{tabular}
