\problemname{Einkunnir}
Karen þarf að fara yfir krossapróf nokkurra nemenda.
Fyrir hverja spurningu er nákvæmlega eitt svar rétt.
Allar spurningarnar hafa sama vægi og ekki er dregið frá fyrir röng svör.
Kareni finnst ekki gaman að fara yfir þetta próf í höndunum og hefur því búið til lítið forrit þar sem nemendurnir merkja kross við svar í hverri spurningu.
Forritið er hannað þanning að ekki er unnt að skila prófinu nema merkt sé við nákvæmlega einn kross í hverri spurningu.
Hjálpið Kareni að reikna út einkunn hvers nemenda.

\section*{Inntak}
Í fyrstu línu eru tvær heiltölur $S$ og $N$ sem standa fyrir fjölda spurninga á prófinu og fjöldi nemenda.
Í næstu línu er svarlykillinn fyrir prófið, stórir bókstafir á bilinu \texttt{A} til \texttt{Z} aðskilnir með bili, þar sem t.d. \texttt{A D E} þýðir að svörin við spurningum $1$, $2$ og $3$ á þriggja spurninga prófi eru \texttt{A}, \texttt{D} og \texttt{E} hver um sig í þeirri röð sem um var getið.
Þar á eftir fylgja $2N$ línur fyrir svör nemendanna. Fyrst kemur lína með nafni nemandans, lengd nafnsins er mesta lagi $20$ stafir, og síðan lína með svörum nemandans á sama formi og svarlykillinn.

\section*{Úttak}
Ein lína fyrir hvern nemanda með nafni nemandans og einkunn hans.
Einkunnir nemandanna eru á skalanum $0$ til $10$ og skal skrifa þær út í þeirri röð sem nemendurnir eru gefnir í inntakinu. Einkunnir eru gefnar í heilum og hálfum með nákvæmlega einum aukastaf og námundað upp í óvissu, t.d. gæfi $8.25$ einkunina $8.5$ en $7.24$ gæfi einkunina $7$.

\section*{Stigagjöf}
\begin{tabular}{|l|l|l|l|}
\hline
Hópur & Stig & Takmarkanir \\ \hline
1     & 100     & $1 \leq S \leq 5 \cdot 10^3, 1 \leq N \leq 100$ \\ \hline
\end{tabular}
