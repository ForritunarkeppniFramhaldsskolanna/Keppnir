\problemname{Hraðskrif}
\illustration{0.3}{keyboard}{\href{https://pixabay.com/images/id-2308477/}{Keyboard} eftir DomAlberts, Pixabay}%
Við keppnisforritun er hentugt að geta skrifað hratt á lyklaborð.
Bjarki æfir sig oft á TypeRacer til þess að ná fleiri orðum á mínútu.
Stundum þegar Bjarki er byrjaður skrifa orð þá vildi hann helst geta ýtt á TAB takkann sinn til þess að klára orðið sem hann er að hugsa um.
Það myndi spara honum mikinn tíma en væri aðeins mögulegt ef það kæmi bara eitt orð til greina.

Bjarki setur því upp orðabók á tölvuna sína sem inniheldur öll þau orð sem eru í textanum sem hann vill skrifa.
Til dæmis getur textinn sem Bjarki skrifar verið \texttt{ANNA OG AMMA ETA MAT SAMAN} sem krefst þess að Bjarki slær á $26$ lykla.
Ef hann notar orðabókina og slær á lyklana
\texttt{AN! O! AM! E! M! S!}, þar sem að \texttt{!} táknar að Bjarki hafi ýtt á TAB takkann, þá slær hann á $19$ lykla og sparar sér því $7$ lykla.
Athugið að hann getur ekki notað TAB takkann strax á eftir \texttt{A} því það er óvíst á þeim punkti hvort hann vill skrifa \texttt{ANNA} eða \texttt{AMMA}.
Athugið einnig að ef aðeins eitt orð er í orðabókinni þá getur Bjarki slegið á TAB takkann strax, því þó hann hafi ekki slegið inn neinn staf, þá er aðeins eitt orð sem kemur til greina.

Gefin textinn sem Bjarki skrifar, hvað getur hann sparað sér marga áslætti á lyklaborðinu?

\section*{Inntak}
Fyrsta línan inniheldur eina heiltölu $1 \leq n \leq 5 \cdot 10^5$, fjöldi orða sem Bjarki skrifar í textanum.
Næsta lína inniheldur textann sem Bjarki skrifar.
Textinn samanstendur af einu eða fleiri orðum sem eru aðskilin með bilum og er fjöldi tákna í mesta lagi $10^6$.
Orðin innihalda einungis hástafi úr enska stafrófinu.

\section*{Úttak}
Skrifið út eina línu með fjölda áslátta á lyklaborðið sem Bjarki sparar sér.

\section*{Stigagjöf}
\begin{tabular}{|l|l|l|}
\hline
Hópur & Stig & Takmarkanir \\ \hline
1     & 30   & $1 \leq n \leq 10$ og hvert orð er í mesta lagi $10$ tákn \\ \hline
2     & 40   & Hvert orð er í mesta lagi $10$ tákn \\ \hline
3     & 30   & Engar frekari takmarkanir \\ \hline
\end{tabular}
