\problemname{Stikl}
\illustration{0.35}{tiles}{\href{https://pixabay.com/images/id-937881/}{Number tiles} eftir geralt, Pixabay}%
Þú elskar að spila leiki með vini þínum. Um þessar mundir er uppháldið ykkar leikurinn \emph{Stikl}.
Til að byrja með eru $n$ reitir settir í línu, og tala af handahófi skrifuð á hvern þeirra.
Reglur leiksins eru einfaldar. Báðir leikmenn byrja á reit af handahófi. Þeir
skiptast svo á að kasta tening og framkvæma svo fjölda skrefa samkvæmt því.
Hvert skref virkar þannig að leikmaðurinn færir sig á fyrsta reitinn til hægri sem hefur \emph{ekki} lægra númer en reiturinn sem hann er nú þegar á.
Það erfiðasta við þennan leik er að finna út hvar leikmaðurinn endar eftir ákveðinn fjölda af skrefum.

\section*{Inntak}
Fyrsta lína inniheldur tvær heiltölur $n$ og $q$ ($1 \leq n, q \leq 10^5$), fjöldi reita og fjöldi fyrirspurna.
Önnur lína inniheldur $n$ heiltölur $a_1,a_2,\ldots,a_n$ ($1 \leq a_i \leq 10^9$) sem tákna tölurnar á reitunum frá vinstri til hægri.
Svo fylgja $q$ fyrirspurnir, hver á línu sem inniheldur tvær heiltölur $s$ og
$d$ ($1 \leq s \leq n$, $1 \leq d \leq 10^5$), númerið á reitnum sem
leikmaður byrjar á og fjöldi skrefa sem hann framkvæmir.

\section*{Úttak}
Fyrir hverja fyrirspurn skrifið út eina línu með númeri reitsins sem
leikmaðurinn endar á, ef hann byrjar á reit $s$ og framkvæmir $d$ skref. Ef
leikmaðurinn myndi hoppa út fyrir enda reitanna, skrifið þá ``\texttt{leik lokid}''
í þeirri fyrirspurn.

\section*{Stigagjöf}
\begin{tabular}{|l|l|l|}
\hline
Hópur & Stig & Takmarkanir \\ \hline
1     & 42   & $n, q \leq 100$ \\ \hline
2     & 26   & $s$ er alltaf $1$ \\ \hline
3     & 32   & Engar frekari takmarkanir\\ \hline
\end{tabular}

