\problemname{Talnalás}
\illustration{0.3}{lock}{\href{https://pixabay.com/images/id-2580718/}{Bicycle lock} eftir Photorama, Pixabay}
Anna litla hjólaði í skólann í morgun. Hún læsti hjólinu sínu í
hjólageymslunni með talnalás, og ruglaði svo tölunum á lásnum. Talnalásinn
samanstendur af $n$ talnaskífum sem hver hefur tölur á bilinu $0$ til $9$, en
hægt er að snúa skífunum í báðar áttir. Með einum snúning er því hægt að færa
skífu yfir á tölu sem er einni lægri eða einni hærri en talan sem var þar
áður. Skífan myndar hring, svo ef maður hækkar töluna $9$ þá fer maður aftur
á töluna $0$, og ef maður lækkar töluna $0$ þá fer maður aftur á töluna $9$.

Anna er núna að fara heim og þarf því að opna talnalásinn. Þetta gerir hún
með því að snúa skífunum, einn snúning í einu á hvaða skífu sem er, svo að
tölurnar á skífunum myndi á endanum ákveðna talnarunu. En Anna er mjög
hjátrúarfull og í gegnum tíðina hefur hún fundið sér $m$ happatölur. Eftir
hvern snúning sem hún framkvæmir vill hún vera viss um að talnaröðin sem
skífurnar mynda sé örugglega ein af happatölunum hennar, því annað boðar
ólukku.

Gefin upprunalega talnarunan á lásnum, talnarunan sem þarf til að opna
lásinn, og listi af öllum happatölunum hennar Önnu, hjálpaðu henni að finna í
hvaða röð hún á að snúa skífunum til að mynda talnarununa til að opna lásinn,
þannig að tölurnar á skífunum myndi happatölu eftir hvern snúning. Þar sem Anna
er að drífa sig heim þá vill hún gera þetta með sem fæstum mögulegum snúningum.

\section*{Inntak}
Fyrsta lína inniheldur tvær heiltölur $n$ og $m$ ($1 \leq n$, $1 \leq m$),
fjöldi skífa á talnalásnum og fjöldi happatalna. Önnur lína inniheldur
$n$-stafa tölu, þar sem $i$-ti stafurinn táknar töluna á $i$-tu skífunni í
upprunalegu talnarununni á talnalásnum. Þriðja lína inniheldur
$n$-stafa tölu, þar sem $i$-ti stafurinn táknar töluna á $i$-tu skífunni í
talnarununni sem þarf til að opna lásinn. Að lokum fylgja $m$ línur, hver með
$n$-stafa tölu sem tákna happatölurnar hennar Önnu. Að auki eru upprunalega
talnarunan og talnarunan til að opna lásinn happatölur.

Gera má ráð fyrir því að upprunalega talnarunan og talnarunan til að opna
lásinn séu mismunandi.

\section*{Úttak}
Skrifið út eina línu með heiltölunni $k$, minnsta fjölda hreyfinga sem Anna
þarf til að opna lásinn þannig að talnarunan á lásnum myndi happatölu eftir
hverja einustu hreyfingu. Skrifið svo út $k+1$ línur, hver með $n$-stafa tölu,
þar sem sú fyrsta er upprunalega talnarunan á lásnum, og seinni $k$ sýna
talnarununa á lásnum eftir hvern snúning sem Anna þarf að framkvæma. Síðasta
talnarunan þarf því að vera talnarunan sem Anna þarf til að opna lásinn. 

Ef það eru margar mögulegar lausnir, þá skiptir ekki máli hverja þeirra þið
skrifið út. Ef það er ekki til nein lausn, skrifið þá bara út eina línu sem
inniheldur ``\texttt{Neibb}''.

\section*{Stigagjöf}
\begin{tabular}{|l|l|l|}
\hline
Hópur & Stig & Takmarkanir \\ \hline
1     & 9   & $n = 1$, $m \leq 10$ \\ \hline
2     & 14   & $n = 2$, $m \leq 100$ \\ \hline
3     & 17   & $n = 3$, $m \leq 10^3$ \\ \hline
4     & 20   & $n = 5$, $m \leq 10^5$ \\ \hline
5     & 12   & $n = 50$, $m \leq 10$ \\ \hline
6     & 10   & $n = 50$, $m \leq 20$ \\ \hline
7     & 18   & $n = 50$, $n\cdot m \leq 10^5$ \\ \hline
\end{tabular}

