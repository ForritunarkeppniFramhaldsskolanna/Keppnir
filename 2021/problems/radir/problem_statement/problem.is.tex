\problemname{Raðir}
\illustration{0.4}{cards}{\href{https://unsplash.com/photos/tSE54tVzlqM}{Three of Hearts} eftir Harry Shelton, Unsplash}%
Þér finnst rosalega gaman að spila á spil með vinkonu þinni, en hún var nýlega
að kynna þig fyrir nýjum leik. Til að byrja með fær hvor leikmaður $p$ spil til
að hafa á hendi. Þið skiptist svo á að gera, en sá sem á leik dregur eitt spil
á hendi og hendir svo einu spili af hendi (mögulega spilinu sem var dregið).
Markmið leiksins er að fá \emph{röð} á hendi. Röð samanstendur af þremur spilum
af sömu sort sem uppfyllir það að, ef minnsta spilið hefur gildi $a$, þá hafa
hin tvö spilin gildi $a+1$ og $a+2$. Þú getur tilkynnt sigur þegar þinni umferð
er lokið, svo lengi sem þú ert með að minnsta kosti eina röð á hendi.

Þú varst rétt í þessu að tapa fyrir vinkonu þinni. Þú varðst ekki fyrir miklum
vonbrigðum þar sem þú varst bara að kynnast leiknum, en vilt engu að síður læra
af mistökum þínum. Þú lítur á spilin þín og veltir því fyrir þér hversu snemma
þú hefðir getað fengið röð ef þú hefðir spilað á sem bestan máta.

\section*{Inntak}
Fyrsta línan inniheldur tvær heiltölur $n$ og $p$ ($3 \leq p \leq n \leq
10^6$), heildarfjöldi spila og fjöldi spila sem þú getur haft á hendi.
Svo fylgja $n$ línur, þar sem $j$-ta línan inniheldur tvær heiltölur $c_j$ og
$k_j$ ($1 \leq c_j \leq 4$, $1 \leq k_j \leq 13$), sem táknar sort og gildi á
ákveðnu spili. Fyrstu $p$ spilin tákna þau spil sem þú byrjaðir með á hendi, og
seinni $n-p$ spilin eru þau spil sem þú dróst, í þeirri röð sem þau dróst þau.

\section*{Úttak}
Skrifaðu út eina heiltölu sem táknar fæsta fjölda umferða sem það hefði getað
tekið þig að fá röð, ef þú hefðir spilað á sem bestan máta. Ef það er engin
leið til að fá röð þá skaltu skrifa út ``\texttt{Neibb}''.

\section*{Stigagjöf}
\begin{tabular}{|l|l|l|}
\hline
Hópur & Stig & Takmarkanir \\ \hline
1     & 27   & $n \leq 100$ \\ \hline
2     & 23   & $n \leq 5\,000$ \\ \hline
3     & 15   & $n \leq 10^6$, öll spil af sömu sort og hafa gildi $1$, $2$ eða $3$ \\ \hline
4     & 35   & Engar frekari takmarkanir\\ \hline
\end{tabular}
