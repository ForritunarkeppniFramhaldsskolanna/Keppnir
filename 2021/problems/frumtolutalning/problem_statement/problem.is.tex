\problemname{Frumtölutalning}
\illustration{0.3}{primes}{\href{https://pixabay.com/images/id-5319713/}{Prime Numbers} eftir geralt, Pixabay}%
Heiltala $p$ er frumtala ef hún er stærri en $1$ og það er aðeins hægt að deila
henni með $1$ og $p$ án þess að fá afgang. Til dæmis er $10$ ekki frumtala þar
sem það er hægt að deila henni með $1$, $2$, $5$ og $10$ án þess að fá afgang.
Aftur á móti er $11$ frumtala þar sem það er bara hægt að deila henni með $1$ og
$11$ án þess að fá afgang. Á sama hátt sjáum við að $2$, $3$ og $17$ eru
frumtölur, en $4$, $9$ og $15$ eru ekki frumtölur.

Ef maður skoðar tölurnar á bilinu $1$ upp í $100$ þá getur maður séð að $25$
þeirra eru frumtölur. Með aðeins meiri handavinnu er hægt að komast að því að
það eru hvorki meira né minna en $168$ frumtölur á bilinu $1$ upp í $1\,000$.

Gefnar tvær heiltölur $a$ og $b$, reiknaðu hvað það eru margar frumtölur á
bilinu $a$ upp í $b$.

\section*{Inntak}
Inntak er ein lína með tveimur heiltölum $a$ og $b$ ($1 \leq a \leq b$).

\section*{Úttak}
Skrifið út fjölda frumtala á bilinu $a$ upp í $b$.

\section*{Stigagjöf}
\begin{tabular}{|l|l|l|}
\hline
Hópur & Stig & Takmarkanir \\ \hline
1     & 19   & $b \leq 10^3$ \\ \hline
2     & 23   & $b \leq 10^9$, $b-a \leq 10^3$ \\ \hline
3     & 14   & $b \leq 10^{18}$, $b-a \leq 10^5$ \\ \hline
4     & 13   & $b \leq 10^7$ \\ \hline
5     & 16   & $b \leq 10^{15}$, $b-a \leq 10^7$ \\ \hline
6     & 15   & $b \leq 10^{11}$ \\ \hline
\end{tabular}

