\problemname{Teningakast}
\illustration{0.35}{paradice}{\href{https://pixabay.com/images/id-1265633/}{Dice} eftir EsaRiutta, Pixabay}%
Meðlimir KFFÍ eru að plana borðspilakvöld og eru að velja spunaspilakerfi sem hentar til að spila saman gegnum netið.
Spilakerfið Ravendice verður fyrir valinu. Það er ekki vinsælasta spunaspilakerfið að svo stöddu svo ekki er búið að
búa til sjálfvirkt kerfi fyrir teningaköst og annað slíkt. Hver spilari þarf því að kasta teningum sjálfur og tilkynna
niðurstöðuna upphátt. Atli er að halda utan um spilið og er farið að gruna að sumir séu að svindla og vill fá þína aðstoð
til að athuga hvort niðurstöðurnar séu raunhæfar.

Hvaða teningum á að kasta er gefinn sem strengur. $n$d$m$ táknar að kasta eigi $n$ $m$-hliða teningum og leggja saman
niðurstöðuna. Hér eru $m$-hliða teningar jafn líklegir til að gefa niðurstöðurnar $1, 2, \dots$ og upp í $m$. $n$ og $m$
geta verið hvaða heiltölur sem er stærri en $0$. Í þessu dæmi munu $n$ og $m$ hins vegar vera minni en $10^4$. $n$d$m!$
táknar `exploding dice' sem þýðir að ef hæsta niðurstaðan fæst á teningi, þ.e. $m$, þá skuli kasta teningnum aftur og
bæta þeirri niðurstöðu við. Þetta getur gerst oft í röð ef hæsta niðurstaðan fæst oft í röð. `Exploding dice' mun aldrei
hafa $m = 1$. Teningakastsstrengurinn er runa teningakasta eða talna með $+$ eða $-$ á milli, mögulega einnig með $-$ fremst. 
Tölurnar verða einnig minni en $10^4$. Til dæmis táknar \verb|3d6+1d4!-2| að kasta eigi þremur sex-hliða teningum, einum 
`exploding' fjögurra hliða teningi, leggja það allt saman og svo draga tvo frá niðurstöðunni.

\section*{Inntak}
Fyrsta línan inniheldur eina heiltölu $q$ ($1 \leq q \leq 10^5$). Síðan fylgja $q$ fyrirspurnir, hver á $2$ línum.
Fyrri línan inniheldur teningakastsstreng eins og lýst er að ofan. Seinni línan inniheldur eina
heiltölu $r$ ($-10^{18} \leq r \leq 10^{18}$). Heildarlengd allra teningakastsstrengjanna verður mest $10^5$ stafir.

\section*{Úttak}
Skrifið út \texttt{Raunhaeft} ef hægt er að fá $r$ sem niðurstöðu úr teningakastinu, annars \texttt{Svindl} fyrir
hverja fyrirspurn í sömu röð með hvert svar á sinni eigin línu.

\section*{Stigagjöf}
\begin{tabular}{|l|l|l|}
\hline
Hópur & Stig & Takmarkanir \\ \hline
1     & 30   & Ekkert \verb|!| í inntaki \\ \hline
2     & 70   & Engar frekari takmarkanir \\ \hline
\end{tabular}

