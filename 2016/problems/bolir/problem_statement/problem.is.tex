\problemname{Bolir}

\illustration{0.4}{bolur}{Keppendur fá boli}

Eitt það skemmtilegasta við að taka þátt í forritunarkeppninni er auðvitað að fá bol. En í ár gætu skipuleggjendurnir hafa klúðrað málunum\ldots Þegar keppandi skráir sig í keppnina þá gefur hann upp hvaða bolastærð hann vill. En þessar upplýsingar týndust, svo skipuleggjendurnir pöntuðu í staðinn boli í alls konar stærðum í von um að allir keppendurnir gætu fengið stærð sem þeir væru ánægðir með, þó það væri ekki endilega stærðin sem þeir báðu um.

Nú er dagur keppninnar runninn upp, og bæði keppendurnir og bolirnir eru komnir í hús. Það eru jafn margir keppendur og bolir, eða $N$ keppendur og $N$ bolir. Bolastærðirnar eru táknaðar með heiltölum frá $1$ upp í $K$, og því hærri sem talan er því stærri er bolurinn. Keppendurnir eru spurðir um hvaða bolastærðir þeir væru ánægðir með að vera í, og gefur keppandi $i$ upp bil af bolastærðum frá $L_i$ upp í $R_i$, þar sem $L_i$ er minnsta bolastærðin sem hann væri ánægður með að vera í, og $R_i$ er stærsta bolastærðin sem hann væri ánægður með að vera í.

Þú færð líka gefinn lista af stærðunum á bolunum sem skipuleggjendurnir pöntuðu. Getur þú athugað hvort hægt sé að úthluta bolunum þannig að hver keppandi fái bol sem hann væri ánægður með að vera í?

\section*{Inntak}
Á fyrstu línu er heiltalan $N$ sem táknar bæði fjölda keppenda og fjölda bola. Þar eftir fylgja $N$ línur, ein fyrir hvern keppanda. Af þessum línum mun lína númer $i$ innihalda tvær heiltölur $L_i$ og $R_i$, þar sem $1 \leq L_i \leq R_i \leq K$, sem tákna bilið af bolastærðum sem keppandi $i$ væri ánægður með. Að lokum kemur ein lína með $N$ heiltölum á bilinu $1$ upp í $K$, þar sem hver heiltala táknar stærð á einum boli.

\section*{Úttak}
Ein lína sem inniheldur \texttt{Jebb} ef hægt er að úthluta bolunum til keppendanna þannig að hver keppandi fái bol sem hann er ánægður með, eða \texttt{Neibb} annars.

\section*{Útskýring á sýnidæmum}
Í fyrsta sýnidæminu eru $N=4$ keppendur og bolir. Keppendurnir fjórir vilja boli af eftirfarandi stærðum:
\begin{enumerate}
    \item stærð á milli $1$ og $3$,
    \item stærð á milli $1$ og $10$,
    \item stærð á milli $2$ og $2$, og
    \item stærð á milli $2$ og $3$.
\end{enumerate}
Bolirnir fjórir eru af stærðum $1$, $2$, $2$ og $9$. Ef keppandi $1$ fær bol af stærð $1$, keppandi $2$ fær bol af stærð $9$, og keppendur $3$ og $4$ fá boli af stærð $2$, þá hefur öllum keppendum verið úthlutað bolum sem þeir eru ánægðir með, og svarið er því \texttt{Jebb}.

Í öðru sýnidæminu eru þrír sem vilja bara bol af stærð $2$, en það eru bara tveir bolir af stærð $2$. Það er því ekki hægt að úthluta öllum boli sem þeir eru ánægðir með, og svarið er því \texttt{Neibb}.

\section*{Stigagjöf}
Lausnin mun verða prófuð á miserfiðum inntaksgögnum, og er gögnunum skipt í
hópa eins og sýnt er í töflunni að neðan. Lausnin mun svo fá stig eftir því
hvaða hópar eru leystir.

\begin{tabular}{|l|l|l|l|}
\hline
Hópur & Stig & Inntaksstærð & Önnur skilyrði \\ \hline
1 & 15 & $N \leq 10$ & Stærsta bolastærðin $K$ er $1\,000$ \\ \hline
2 & 15 & $N \leq 100$ & Stærsta bolastærðin $K$ er $100$ \\ \hline
3 & 20 & $N \leq 1\,000$ & Stærsta bolastærðin $K$ er $1\,000$ \\ \hline
4 & 20 & $N \leq 10^5$ & Stærsta bolastærðin $K$ er $2$ \\ \hline
5 & 15 & $N \leq 10^5$ & Stærsta bolstærðin $K$ er $10^5$ \\ \hline
6 & 15 & $N \leq 10^5$ & Stærsta bolastærðin $K$ er $10^9$ \\ \hline
\end{tabular}
