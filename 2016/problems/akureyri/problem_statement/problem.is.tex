\problemname{Akureyri}
Árið 2016 er merkilegt ár fyrir Forritunarkeppni Framhaldsskólanna. Í fyrsta
skipti í sögu keppninnar verður keppnin haldin á tveimur stöðum! En nú er árið
3016 og keppnin er haldin um allt land, ekki bara í Reykjavík og Akureyri.
Keppendur fara ekki lengur á keppnina, heldur kemur keppnin til þeirra.  En
skipulagsteymi forritunarkeppninnar er í miklum vandræðum með að skipuleggja
bolina fyrir keppendur þetta árið. Keppendur eru nú í þúsundum, jafnvel
tugþúsundum en það er alltof stutt í keppnina svo að hægt sé að fara yfir
þetta í höndunum. Á árinu 3016 er fólk svo vant því að tölvur forriti fyrir
fólk, fyrir utan stöku snillinga sem eru að keppa í Forritunarkeppni
Framhaldskólanna að skipuleggjendurnir eru búnir að steingleyma hvernig á að
forrita og hafa ekki hugmynd hvernig þetta vandamál væri leysanlegt í tölvu.

Sem betur fer var ákveðið að frysta Hjalta árið 2016 og geyma hann í djúpsvefni
ef ske kynni að skipuleggjendur yrðu í svo miklum vandræðum að geta ekki leyst
úr flækjunni sjálf. Það sem skipuleggjendurnir þurfa hjálp Hjalta við er að fara
yfir keppendalistann og skrá niður hversu marga boli þarf að senda á hvern stað
á landinu. Hjalti er hins vegar búinn að vera sofandi í 1000 ár og er því
svolítið ringlaður. Hjálpaðu Hjalta að redda málunum!

\section*{Inntak}
Inntakið byrjar á einni línu sem inniheldur staka heiltölu $1\leq N\leq
10\,000$ sem er fjöldi keppenda. Næst fylgja $N$ pör af línum, samtals $2N$
línur. Í hverju pari af línum inniheldur fyrri línan nafn keppanda en seinni
línan staðsetningu hans. Báðir strengirnir samanstanda eingöngu af enskum
bókstöfum. Þeir munu innihalda að minnsta kosti einn staf og aldrei fleiri en
$100$ stafi.

\section*{Úttak}
Fyrir hvert bæjarfélag sem birtist sem staðsetning keppanda, prentið út nafn
bæjarfélagsins og síðan fjölda keppanda sem staðsettir eru í tilteknu
bæjarfélagi. Prenta má þennan lista út í hvaða röð sem er.

\section*{Útskýring á sýnidæmum}
Í fyrsta sýnidæmi eru tveir keppendur á Akureyri, þeir Bjarki og Jonas en hins
vegar eru Hjalti, Gunnar og Tomas í Reykjavik og gefur því þetta úttakið sem
sýnt er.

\section*{Stigagjöf}
Lausnin mun verða prófuð á miserfiðum inntaksgögnum, og er gögnunum skipt í
hópa eins og sýnt er í töflunni að neðan. Lausnin mun svo fá stig eftir því
hvaða hópar eru leystir.

\begin{tabular}{|l|l|l|}
\hline
Hópur & Stig & Önnur skilyrði \\ \hline
1     & 35 & Einu staðsetningarnar eru \texttt{Akureyri} og \texttt{Reykjavik}, og hvor staðsetning kemur minnst einu sinni \\ \hline
2     & 15 & Einu staðsetningarnar eru \texttt{Akureyri} og \texttt{Reykjavik} \\ \hline
2     & 50 &  \\ \hline
\end{tabular}
