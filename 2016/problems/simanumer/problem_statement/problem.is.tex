\problemname{Símanúmer}
Kristín hefur risið til metorða innan Rannsóknardeildar Lögreglunnar í
Reykjavík undanfarin ár, einna helst vegna færni sinnar í tölvunarfræði, og er
orðinn einn helsti rannsakandi deildarinnar. Við rannsóknir mála vinnur hún
oftast með vitnum og hefur hún tekið eftir því, í vinnu sinni, að vitni muna
símanúmer mjög illa. Flest vitni muna bara fyrstu stafina í símanúmerum.

Þau símanúmer sem vitnin gefa upp eru oftar en ekki lykillinn að lausn
sakamálanna, en það kerfi sem lögreglan notar við að fara í gegnum símanúmerin
er ekki skilvirkt. Hópur lögregluþjóna fer yfir öll skráð símanúmer og tínir
til þau símanúmer sem byrja á þeim tölustöfum sem vitnin gefa upp.

Kristín, verandi tölvunarfræðingur, veit að hægt er að gera leitina töluvert
skilvirkari, en er mjög upptekin og biður því ykkur um að útfæra leitina fyrir
sig. Forritið á að geta tekið við upphafi símanúmers, sem við köllum
\emph{fyrirspurn}, frá vitni og segir svo til um hversu mörg símanúmer byrja á
þeirri fyrirspurn.

\section*{Inntak}
Fyrsta lína inntaksins inniheldur heiltölu $N$ sem segir til um fjölda
símanúmera í safninu sem lögreglan hefur yfir að ráða. Næstu $N$ línur
innihalda símanúmer safnsins, eitt símanúmer í hverri línu. Engin tvö símanúmer
eru eins, og hvert þeirra samanstendur af $7$ tölustöfum. Næsta lína inniheldur
heiltölu $Q$ sem segir til um fjölda fyrirspurna. Næstu $Q$ línur innihalda
fyrirspurnirnar, ein fyrirspurn í hverri línu. Hver fyrirspurn samanstendur af
$1$ til $7$ tölustöfum.

\section*{Úttak}
Skrifið út eina línu fyrir sérhverja fyrirspurn með fjölda símanúmera í safninu
sem byrja á þeirri fyrirspurn.

\section*{Stigagjöf}
Lausnin mun verða prófuð á miserfiðum inntaksgögnum, og er gögnunum skipt í
hópa eins og sýnt er í töflunni að neðan. Lausnin mun svo fá stig eftir því
hvaða hópar eru leystir.

\begin{tabular}{|l|l|l|l|}
\hline
Hópur & Stig & Inntaksstærð & Önnur skilyrði \\ \hline
1 & 5 & $N=1$, $Q=1$ & Hver fyrirspurn inniheldur nákvæmlega $3$ tölustafi \\ \hline
2 & 5 & $N = 1$, $Q=1$ & \\ \hline
3 & 15 & $N \leq 100$, $Q \leq 100$ & Hver fyrirspurn inniheldur nákvæmlega $3$ tölustafi \\ \hline
4 & 20 & $N \leq 100$, $Q \leq 100$ & \\ \hline
5 & 25 & $N \leq 10^5$, $Q \leq 10^5$ & Hver fyrirspurn inniheldur nákvæmlega $3$ tölustafi \\ \hline
6 & 30 & $N \leq 10^5$, $Q \leq 10^5$ & \\ \hline
\end{tabular}
