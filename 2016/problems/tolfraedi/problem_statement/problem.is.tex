\problemname{Tölfræði}

Murray Christianson var einn afkastamesti skoski stærðfræðingurinn á 19.\ öld en
hann var aðallega þekktur fyrir verk sín í fléttufræði og tölfræði.  Meðal
þess sem hann hefur gert má nefna \emph{Setningu Christanson} sem snýr að
talningu umraðana og umraðanamynstra. Þrátt fyrir það lagði hann þó ávallt mestu
áhersluna á tölfræði og gagnavinnslu. Í hans daga voru engar tölvur sem gátu
haldið utan um gígantísk gagnasöfn og unnið með þau, né var til það sem við
köllum í dag \emph{Big Data}. Hann vann þó oft með gríðarlegt magn af gögnum og
reiknaði ýmsar niðurstöður í höndunum, þar á meðal alla konunglega tölfræði
fyrir Viktoríu drottningu og hirð hennar.

Þegar hann lést árið 1901, sama ár og Viktoría drotting lést og rétt áður
en Játvarður VII tók við krúnunni, var hann að vinna við að reikna ýmsa tölfræði
yfir nýja hirðmenn konungs. Sumir hirðmenn voru að hætta og nýir ráðnir, og í
hvert skipti sem einhverjar breytingar áttu sér stað þurfti hann að reikna
aldur yngsta og elsta hirðmanns konungs og meðalaldur þeirra allra.

Til að sýna yfirburði tölva yfir handútreikninga, þá þarft þú að skrifa forrit
sem vinnur verk Murray Christianson margfalt hraðar. Í þessu dæmi færðu gefið
aldur manneskju $x_i$ og hvort að hún sé að hætta í hirðinni, táknað með
\texttt{R}, eða að byrja í hirðinni, táknað með \texttt{A}. Fyrir hverja
breytingu, prentaðu út lægsta og hæsta aldur manneskju í hirðinni og
meðalaldurinn.

\section*{Inntak}
Inntakið byrjar á einni línu sem inniheldur eina jákvæða heiltölu $Q$ sem
táknar fjölda breytinga sem eiga sér stað. Því næst fylgja $Q$ línur, ein fyrir
hverja breytingu, og á línu $i$ er einn bókstafur, \texttt{A} eða \texttt{R}, og
síðan jákvæð heiltala $x_i$.

Gera má ráð fyrir að aðeins hirðmenn sem eru nú þegar í hirðinni hætta.

\section*{Úttak}
Eftir hverja breytingu á að prenta út línu sem inniheldur lægsta aldur, hæsta
aldur og meðalaldur þeirra sem eru í hirðinni á þeim tímapunkti. Ef enginn er í
hirðinni á þeim tímapunkti skal línan innihalda \texttt{-1~-1~-1}. Úttakið er
talið rétt ef hver tala er annaðhvort nákvæmlega eða hlutfallslega ekki lengra
frá réttu svari en $10^{-4}$. Þetta þýðir að það skiptir ekki máli með hversu
margra aukastafa nákvæmni tölurnar eru skrifaðar út, svo lengi sem þær er nógu
nákvæmar.

\section*{Útskýring á sýnidæmum}
Fyrsta sýnidæmið sýnir manneskju með aldurinn $1$ bætta við hirðina og er því
lægsti aldurinn $1$, hæsti aldurinn $1$ og meðalaldurinn $1$. Næst er bætt við
manneskju með aldurinn $2$ og er þá lægsti aldurinn áfram $1$, hæsti aldurinn $2$
og meðalaldurinn $1.5$. Og þannig gengur þetta koll af kolli.

Í næsta sýnidæmi bætast við manneskjur með aldurinn $2$, $5$ og $2$. Á þeim
tímapunkti er lægsti aldurinn $2$, hæsti aldurinn $5$ og meðalaldurinn $3$,
eins og sést á þriðju línu í úttakinu. Næst er manneskja með aldurinn $2$ rekin
úr hirðinni, en lægsti aldurinn breytist ekki því enn er ein manneskja með
aldurinn $2$ í hirðinni. Í lokin er seinni manneskjan með aldurinn $2$ rekin úr
hirðinni og breytist því lægsti aldurinn og meðalaldurinn í $5$ því það er
aldurinn á einu manneskjunni sem er eftir.

\section*{Stigagjöf}
Lausnin mun verða prófuð á miserfiðum inntaksgögnum, og er gögnunum skipt í
hópa eins og sýnt er í töflunni að neðan. Lausnin mun svo fá stig eftir því
hvaða hópar eru leystir.

\begin{tabular}{|l|l|l|l|}
\hline
Hópur & Stig & Inntaksstærð & Önnur skilyrði  \\ \hline
1     & 20         & $ 1 \le Q \le 1\,000$, $1 \leq x_i \leq 10^6$ & Engar $R$ breytingar \\ \hline
2     & 20         & $ 1 \le Q \le 1\,000$, $1 \leq x_i \leq 10^6$ & \\ \hline
3     & 20         & $ 1 \le Q \le 100\,000$, $1 \leq x_i \leq 100$ & \\ \hline
4     & 20         & $ 1 \le Q \le 100\,000$, $1 \leq x_i \leq 10^9$ & Engar $R$ breytingar \\ \hline
5     & 20         & $ 1 \le Q \le 100\,000$, $1 \leq x_i \leq 10^9$ & \\ \hline
\end{tabular}
