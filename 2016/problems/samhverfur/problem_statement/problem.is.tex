\problemname{Samhverfudulritun}
Sigga litla hefur mikið vera að fylgjast með þróun skammtatölva. Henni finnst stafa ógn af þessum tölvum því að þær geta brotið RSA dulritunarkerfið, sem er byggt á frumtölum, og er notað mjög víða við að dulrita samskipti á internetinu í dag. Ef ekki er búið að skipta um dulritunarkerfi þegar fyrsta alvöru skammtatölvan lítur dagsins ljós, þá mun enginn vera öruggur á netinu!

Sigga litla tekur þetta mjög nærri sér og ákveður að búa til sitt eigið dulritunarkerfi. Í staðinn fyrir að notast við frumtölur ákveður hún að nota samhverfur. Tala er samhverfa ef hún er eins hvort sem maður les hana aftur á bak eða áfram. Til dæmis eru 5, 11, 121 og 9779 samhverfur.

Til að athuga öryggi dulkerfisins þarf hún lista af samhverfum af misstórum stærðum. Hún biður þig um að finna, fyrir hvert $k$ á milli $1$ og $100$, stærstu samhverfuna sem er ekki stærri en $2^k$.

\section*{Inntak}
Það er ekkert inntak.

\section*{Úttak}
Fyrir hvert $k$ á milli $1$ og $100$ má vera ein lína sem inniheldur tvær heiltölur: $k$ og stærsta samhverfan sem er ekki stærri en $2^k$. Línurnar mega koma í hvaða röð sem er, og ekki þarf að skrifa út línu fyrir öll möguleg $k$ (en því fleiri línur, því fleiri stig).

\section*{Stigagjöf}
Lausnin fær eitt stig fyrir hvert $k$ sem kemur fyrir í úttakinu og hefur rétt svar.

\section*{Útskýring á sýnidæmum}
Í sýnidæminu inniheldur úttakið fimm línur, ein fyrir hvert $k$ í $\{1,4,3,10,6\}$. Hér sjáum við að ekki er skrifuð út lína fyrir hvert $k$ á milli $1$ og $100$, heldur bara fyrir fimm tölur. Þegar $k=4$, þá á seinni heiltalan að vera stærsta samhverfan sem er ekki stærri en $2^4 = 16$, en hún er einmitt $11$. Lausn sem skrifar þetta úttak fær því eitt stig fyrir $k=4$. Á sama hátt sjáum við að hún fengi þrjú stig í viðbót fyrir $k$ í $\{1,3,10\}$. Fyrir $k=6$ á seinni talan að vera stærsta samhverfan sem er ekki stærri en $2^6 = 64$. Í þessu úttaki er skrifuð út talan $44$, en svarið er $55$ fyrir þetta $k$. Lausnin fær því ekki stig fyrir þetta $k$. Í heildina myndi lausn sem skrifar þetta úttak fá $4$ stig.
