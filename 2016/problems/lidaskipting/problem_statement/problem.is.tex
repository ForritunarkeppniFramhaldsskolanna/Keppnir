\problemname{Liðaskipting}
Undanfarin ár hafa keppendur fengið að búa til sín eigin lið fyrir Forritunarkeppni Framhaldsskólanna, en eins og þið vitið mega vera þrír saman í liði. Þetta hefur þó ekki alltaf gengið nógu vel því stundum verða einhverjir útundan. Skipuleggjendurnir hafa því ákveðið að sjá um liðaskiptingu sjálfir, og passa að enginn verði útundan. En þeir átta sig á því að það er ekki alltaf hægt að skipta í þriggja manna lið án þessa að neinn sé útundan.

Á næsta ári ætla $N$ manns að keppa í Forritunarkeppni Framhaldsskólanna. Geturðu hjálpað skipleggjendunum með að athuga hvort það sé hægt að skipta $N$ manns í þriggja manna lið þannig að enginn sé útundan?

\section*{Inntak}
Ein lína með jákvæðu heiltölunni $N$ sem táknar fjölda manns sem ætla að keppa.

\section*{Úttak}
Ein lína sem inniheldur \texttt{Jebb} ef hægt er að skipta $N$ manns í þriggja manna lið þannig að enginn sé útundan, eða \texttt{Neibb} ef það er ekki hægt.

\section*{Útskýring á sýnidæmum}
Í fyrsta sýnidæminu ætla $N=3$ manns að keppa. Svarið er \texttt{Jebb} því það er hægt að búa til eitt þriggja manna lið, og þá er enginn útundan.

Í öðru sýnidæminu ætla $N=14$ manns að keppa. Það er ekki nægur fjöldi til að búa til fimm þriggja manna lið, og ef það eru búin til fjögur lið þá eru tveir manns útundan. Það er því ekki hægt að skipta $14$ manns í þriggja manna lið, og svarið er \texttt{Neibb}.

Í síðasta sýnidæminu ætla $N=300\,000\,000\,000$ manns að keppa. Hér er hægt að búa til $100\,000\,000\,000$ þriggja manna lið og þá er enginn útundan. Svarið er því \texttt{Jebb}.

\section*{Stigagjöf}
Lausnin mun verða prófuð á miserfiðum inntaksgögnum, og er gögnunum skipt í
hópa eins og sýnt er í töflunni að neðan. Lausnin mun svo fá stig eftir því
hvaða hópar eru leystir.

\begin{tabular}{|l|l|l|}
\hline
Hópur & Stig & Inntaksstærð \\ \hline
1 & 10 & $N \leq 3$ \\ \hline
2 & 10 & $N \leq 30$ \\ \hline
3 & 20 & $N \leq 30\,000$ \\ \hline
4 & 20 & $N \leq 10^{12}$ \\ \hline
5 & 40 & $N \leq 10^{100}$ \\ \hline
\end{tabular}
