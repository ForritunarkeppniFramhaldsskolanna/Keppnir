\problemname{Frumtölur}
Verið velkomin í Forritunarkeppni Framhaldsskólanna! Ef þið hafið keppt áður þá vitið þið kannski hvað við dæmahöfundarnir erum helteknir af frumtölum. Fyrir ykkur nýliðana þá fylgir hér stutt kynning á frumtölum.

Heiltala $N$ er frumtala ef það er aðeins hægt að deila henni með $1$ og $N$. Til dæmis er $10$ ekki frumtala þar sem það er hægt að deila henni með $1$, $2$, $5$ og $10$. Aftur á móti er $11$ frumtala þar sem það er bara hægt að deila henni með $1$ og $11$. Á sama hátt sjáum við að $2$, $3$ og $17$ eru frumtölur, en $4$, $9$ og $15$ eru ekki frumtölur.

Í þessu fyrsta verkefni ætlum við að láta ykkur skrifa forrit sem finnur fyrstu $100$ frumtölurnar. Forritið má skrifa út frumtölurnar í hvaða röð sem er, og það þarf ekki að skrifa út allar af fyrstu $100$ frumtölunum. Lausnin fær eitt stig fyrir hverja frumtölu sem hún skrifar út, svo því fleiri því betra. En ef forritið skrifar út tölu sem er ekki ein af fyrstu $100$ frumtölunum, þá mun lausnin fá $0$ stig.

\section*{Inntak}
Það er ekkert inntak.

\section*{Úttak}
Fyrstu hundrað frumtölurnar, eða hlutmengi af þeim, í hvaða röð sem er. Hver tala á að vera á sér línu.

\section*{Útskýring á sýnidæmum}
Í sýnidæminu eru fyrstu fimm tölurnar skrifaðar út. Lausn sem skrifar þessar fimm tölur út fær $5$ stig.
