\problemname{Heiltölusumma}

Hefur þú heyrt um stærðfræðinginn Carl Friedrich Gauss? Það er til skemmtileg
saga af honum frá því hann var í grunnskóla. Einn daginn var kennarinn hans
orðinn þreyttur á uslanum í krökkunum, svo hann setti fyrir erfitt
stærðfræðidæmi til að halda krökkunum uppteknum. Verkefnið var að leggja saman
allar heiltölur á milli $1$ og $100$. Kennaranum brá þegar Gauss kom að
kennaraborðinu, aðeins örstuttri stundu seinna, og sagðist vera búinn.
Kennarinn trúði honum auðvitað ekki og bað hann um að segja sér svarið.
Kennarinn var ekki sjálfur búinn að leysa verkefnið, enda bjóst hann ekki við
að neinn myndi klára það svona fljótt, svo að hann þurfti nokkrar mínútur til
að athuga svarið. Og viti menn, Gauss var með rétt svar!

Sem betur fer höfum við tölvur í dag til að gera svona handavinnu fyrir okkur.
Auðvitað fann Gauss aðferð til að leggja tölurnar saman hratt án þessa að
framkvæma mikla handavinnu, en við erum ekki öll eins klár og hann. Í þessu
verkefni ætlum við engu að síður að biðja ykkur um að leysa sama
stærðfræðidæmi: Gefin heiltala $N$, hver er summan af öllum heiltölum á milli
$1$ og $N$?

\section*{Inntak}
Ein lína með heltölunni $N$.

\section*{Úttak}
Ein lína með summunni af öllum heiltölum á milli $1$ og $N$.

\section*{Útskýring á sýnidæmum}
Í fyrsta sýnidæminu er $N = 4$. Heiltölurnar á milli $1$ og $4$ eru $1$, $2$,
$3$ og $4$, og summa þeirra er $1+2+3+4=10$.

Í öðru sýnidæminu er $N = -1$. Heiltölurnar á milli $1$ og $-1$ eru $-1$, $0$,
og $1$, og summa þeirra er $(-1)+0+1 = 0$.

\section*{Stigagjöf}
Lausnin mun verða prófuð á miserfiðum inntaksgögnum, og er gögnunum skipt í
hópa eins og sýnt er í töflunni að neðan. Lausnin mun svo fá stig eftir því
hvaða hópar eru leystir.

\begin{tabular}{|l|l|l|}
\hline
Hópur & Stig & Inntaksstærð \\ \hline
1 & 10  & $1\leq N\leq 100$ \\ \hline
1 & 20  & $1\leq N\leq 10^5$ \\ \hline
1 & 20  & $1\leq N\leq 10^9$ \\ \hline
1 & 20  & $-100\leq N\leq 100$ \\ \hline
1 & 30  & $-10^9\leq N\leq 10^9$ \\ \hline
\end{tabular}
