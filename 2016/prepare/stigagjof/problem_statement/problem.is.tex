\problemname{Stigagjöf}

Keppnin í ár mun vera með aðeins öðru sniði en fyrri keppnir. Ein stærsta
breytingin (fyrir utan krúttlegri hóp af dæmasmiðum en nokkurntíman hefur sést
áður) er stigagjöfin.

Fyrir hvert dæmi er hægt að fá stig á bilinu $0$ til $100$. Því fleiri stig,
því betra. Hvert lið má senda inn eins margar lausnir og það vill. Ef mörgum
lausnum hefur verið skilað í sama dæmi, þá er það lausnin sem fékk flestu
stigin sem gildir, og þessi lausn ákvárðar stigafjöldann fyrir þetta dæmi.

Heildarstigafjöldi liðs er svo summa stiga sem liðið hefur fengið fyrir hvert
dæmi. Liðum er svo raðað í lækkandi röð eftir heildarstigafjölda. Ef tvö lið
hafa sama heildarstigafjölda, þá er liðið sem fékk þennan heildarstigafjölda á
undan sem er ofar á stigatöflunni.

Í þessu dæmi þurfið þið hvorki að spá í heildarstigafjölda né tíma. Þið fáið
gefið nafn á dæmi, svo lista af lausnum sem lið hefur sent inn, og eigið að
finna stigafjöldann sem liðið hefur fyrir uppgefið dæmi.

\section*{Inntak}
Á fyrstu línu er nafnið á dæminu sem spurt er um. Á annarri línu kemur heiltala
$1 \leq N \leq 100$ sem táknar fjölda lausna sem liðið hefur skilað inn. Svo
fylgja $N$ línur. Hver af þessum línum táknar lausn sem liðið hefur skilað inn,
og samanstendur af heiltölunni $1 \leq t\leq 360$, sem segir á hvaða mínútu
lausninni var skilað inn, nafninu á dæminu sem lausninni er ætlað, og loks
heiltölunni $0 \leq s\leq 100$ sem táknar stigafjöldann sem lausnin fékk. Nöfn
dæmanna munu aðeins innihalda enska lágstafi og tölustafi, og lausnirnar
koma inn í hækkandi röð eftir tíma.

\section*{Úttak}
Ein lína með stigafjöldanum sem liðið hefur fyrir dæmið sem spurt er um.

\section*{Útskýring á sýnidæmum}
Í fyrsta sýnidæminu er spurt um dæmið \texttt{stigagjof}, en liðið er búið að
skila þremur lausnum í það dæmi. Af þessum lausnum fékk önnur lausnin hæsta
stigafjöldann, $100$. Þetta er því stigafjöldinn sem liðið hefur fyrir þetta
dæmi, og er því svarið.

Í öðru sýnidæminu er spurt um dæmið \texttt{budarkassi2}. Liðið hefur skilað
tveimur lausnum í það dæmi, og af þeim fékk betri lausnin $60$ stig.

Í þriðja sýnidæminu er spurt um dæmið \texttt{budarkassi1}. Liðið hefur ekki
skilað neinum lausnum í það dæmi, og hefur liðið því engin stig fyrir þetta
dæmi, og svarið er $0$.

\section*{Stigagjöf}
Lausnin mun verða prófuð á miserfiðum inntaksgögnum, og er gögnunum skipt í
hópa eins og sýnt er í töflunni að neðan. Lausnin mun svo fá stig eftir því
hvaða hópar eru leystir.

\begin{tabular}{|l|l|l|l|}
\hline
Hópur & Stig & Önnur skilyrði \\ \hline
1     & 50      & Eina dæmið sem kemur í inntakinu er dæmið sem spurt er um \\ \hline
2     & 50      &  \\ \hline
\end{tabular}
