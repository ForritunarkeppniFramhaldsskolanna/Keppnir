\problemname{Sjálfsalastuð}

Í Háskólanum í Reykjavík eru stundaðar miklar rannsóknir innan tölvunarfræði.
Til að halda rannsakendum virkum og þægum, þá þarf að gefa þeim mikið Kók og
það fellur á deildarforseta tölvunarfræðideildar að fara í sjálfsalann og kaupa
Kók fyrir alla starfsmenn deildarinnar. Þar sem deildarforseti er mjög
upptekinn, þá kaupir hann yfirleitt fleiri en eina flösku í hverri ferð.

Sjálfsalarnir í HR taka aðeins 100 kr, 50 kr og 10 kr peninga, og þeir gefa til
baka í eins fáum myntum og mögulegt er. Sjálfsalarnir afgreiða jafnframt aðeins
eina flösku í einu.

Vegna mikilvægi rannsóknanna sem fram fara í tölvunarfræðideildinni, hefur HR
samið um það að fá Kókflöskuna á 80 kr. Deildarforseta finnst mjög leiðinlegt
að setja peninga í sjálfsala, svo hann biður ykkur að skrifa forrit sem,
að gefnum fjölda flaska sem hann þarf að kaupa og þann fjölda sem hann hefur af
hverri mynt, finnur minnsta mögulega fjölda mynta sem hann þarf að setja í
sjálfsalann til að fá allar flöskurnar. Við megum gera ráð fyrir því að
deildarforseti taki ávallt nægan pening með sér í sjálfsalann til að greiða
fyrir allar flöskurnar.

\section*{Inntak}
Inntakið innheldur fjórar heiltölur, aðskildar
af einu orðabili. Fyrsta talan $1 \leq C \leq 150$ er fjöldi flaska
sem deildarforseti ætlar að kaupa. Næstu þrjár tölurnar
$n_{10}$, $n_{50}$ og $n_{100}$
eru fjöldi 10 kr, 50 kr og 100 kr mynta, þar sem $0 \leq n_{10}
\leq 500$, $0 \leq n_{50} \leq 100$ og $0 \leq n_{100} \leq 50$.

\section*{Úttak}
Skrifið út eina heiltölu sem táknar minnsta fjölda mynta sem deildarforseti
þarf að setja í sjálfsalann.

\section*{Útskýring á sýnidæmum}
Í fyrsta sýnidæminu er deildarforseti með tvo tíkalla, einn fimmtíukall, einn
hundraðkall og ætlar að kaupa tvær Kókflöskur. Hann greiðir fyrir fyrstu
flöskuna með hundraðkallinum og fær tvo tíkalla til baka. Hann hefur þá sett
eina mynt í sjálfsalann. Næst setur hann fimmtíukallinn, ásamt þremur tíköllum
í sjálfsalann, samtals fjórar myntir, og fær þá seinni flöskuna. Í heildina
hefur hann þá sett fimm myntir í sjálfsalann.

Í öðru sýnidæminu er deildarforseti með einn tíkall, fjóra fimmtíukalla, einn
hundraðkall og ætlar aftur að kaupa tvær Kókflöskur. Hann greiðir fyrir
fyrstu flöskuna með hundraðkallinum og fær tvo tíkalla til baka. Næst setur hann
fimmtíukallana tvo í sjálfsalann og fær þá seinni flöskuna. Hann hefur þá sett
þrjár myntir samtals í sjálfsalann.

Í því þriðja er deildarforseti með 200 tíkalla og þrjá fimmtíukalla. Hann
kaupir fyrstu þrjár flöskurnar með einum fimmtíukalli og þremur tíköllum. Þá
hefur hann sett samtals $3 \cdot 4 = 12$ myntir í sjálfsalann. Hann á þá bara
tíkalla eftir til að kaupa 17 síðustu flöskurnar. Hann þarf að setja 8 myntir í
sjálfsalann fyrir hverja flösku, og þarf því að setja samtals $8 \cdot 17 =
136$ myntir í sjálfsalann til að kaupa þessar 17 flöskur. Til að kaupa allar 20
flöskurnar hefur hann þá sett $136 + 12 = 148$ myntir í sjálfsalann.

\section*{Stigagjöf}
Lausnin mun verða prófuð á miserfiðum inntaksgögnum, og er gögnunum skipt í
hópa eins og sýnt er í töflunni að neðan. Lausnin mun svo fá stig eftir því
hvaða hópar eru leystir.

\begin{tabular}{|l|l|l|}
\hline
Hópur & Stig & Önnur skilyrði \\ \hline
1        & 10   & Eingöngu tíkallar \\ \hline
2        & 30   & Eingöngu tíkallar og fimmtíukallar \\ \hline
2        & 60   &  \\ \hline
\end{tabular}
