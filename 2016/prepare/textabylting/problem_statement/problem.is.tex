\problemname{Textabylting}

Nýjasta bylting í tölvubransanum er að nú er hægt að lesa texta á hlið!
Notagildi þessa fídusar er næstum ótakmarkað. Til dæmis getur maður núna lesið
á tölvuna á meðan maður liggur á koddanum uppí rúmi. Geggjað!

Því miður er þetta þó ekki sjálfsagt og þarf að útfæra þennan fídus í hverju
forriti fyrir sig. Þú ert með eitt svoleiðis forrit, og þitt verkefni er að
útfæra þennan byltingarkennda fídus.

\section*{Inntak}
Á fyrstu línu er ein heiltala, $N$, sem táknar fjölda lína í textanum sem á að
bylta. Þar eftir fylgja $N$ línur af texta sem á að bylta. Textinn inniheldur
bara enska stafi, tölur og bil.

\section*{Úttak}
Bylti textinn.

\section*{Stigagjöf}
Lausnin mun verða prófuð á miserfiðum inntaksgögnum, og er gögnunum skipt í
hópa eins og sýnt er í töflunni að neðan. Lausnin mun svo fá stig eftir því
hvaða hópar eru leystir.

\begin{tabular}{|l|l|l|l|}
\hline
Hópur & Stig & Inntaksstærð & Önnur skilyrði \\ \hline
1     & 30   & $N=1$ & \\ \hline
2     & 30   & $1\leq N \leq 100$ & Allar línurnar eru jafn langar \\ \hline
3     & 40   & $1\leq N \leq 100$ & \\ \hline
\end{tabular}
