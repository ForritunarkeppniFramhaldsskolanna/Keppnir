\problemname{Miði}
\illustration{0.3}{note_passing}{Mynd fengin af \href{https://flic.kr/p/xnKuj}{flickr.com}}

Litli Jón er skotinn í litlu Gunnu. Hann ætlar sér að senda henni miða
þegar þau eru í miðri kennslustund í skólanum. Miðinn fer í hendur
margra nemenda til að komast frá Jóni til Gunnu. Einnig er hætta á að
kennarinn sjái miðann og taki hann. Jón er hræddur um að bekkjarfélagar
hans eða jafnvel kennarinn lesi skilaboðin í miðanum.

Því hefur Jón fundið upp nýja dulkóðunaraðferð. Fyrsta sem hann gerir
er að snúa skilaboðunum við því þá er mikið erfiðara að lesa það.
Til þess að veita aðferðinni meira öryggi þá sendir hann bara smá hluta
af skilaboðunum í hvert sinn, þannig hann þarf að senda marga miða.
Þessa miða sendir hann svo í öfugri röð. Jóni finnst hann sjálfur vera rosa
sniðugur.

Hver hefur ekki lent í því að fá ástarbréf án þess að hafa einhverja
vitneskju um dulkóðunaraðferðina sem var notuð til að skrifa bréfið?
Gunna hefur lent í því. Því getur hún ekki lesið upprunalega skilaboðin.
Gunna situr því í kennslustofunni með $n$ miða og þarf að púsla þeim 
saman til að skilja þá. Eina sem hún veit er hvað stendur á hverjum miða
og í hvaða röð þeir komu. Geturðu hjálpað Gunnu að lesa skilaboðin?

\section*{Inntak}
Fyrsta línan í inntakinu inniheldur eina heiltölu $1 \leq n \leq 2 \cdot 10^6$,
fjölda miða.
Næst fylgja $n$ línur, hver með einum streng, textinn á hverjum miða 
fyrir sig í þeirri röð sem þeir bárust Gunnu.
Fjöldi stafa í upprunalegu skilaboðunum er mesta lagi $2 \cdot 10^6$.
Skilaboðin innihalda eingöngu enska lágstafi og engin bil.
Hver einasti miði inniheldur að minnsta kosti einn staf.

\section*{Úttak}
Skrifið út eina línu sem inniheldur upprunalega skilaboðin sem Jón vildi
senda Gunnu.

\section*{Stigagjöf}
\begin{tabular}{|l|l|l|}
\hline
Hópur & Stig & Takmarkanir \\ \hline
1     & 50   & $n \leq 100$, fjöldi stafa í upprunalega skilaboðunum er í mesta lagi $1\,000$ \\ \hline
2     & 50   & Engar frekari takmarkanir\\ \hline
\end{tabular}

