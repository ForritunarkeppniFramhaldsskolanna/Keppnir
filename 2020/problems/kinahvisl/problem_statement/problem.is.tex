\problemname{Kínahvísl}
\illustration{0.3}{circle}{Mynd fengin af \href{https://www.pxfuel.com/en/free-photo-eukij}{pxfuel.com}}

Bartosz og vinir hans eru í hvísluleik í leikskólanum sínum.

Hvísluleikurinn virkar þannig að $N$ manns sitja í hring. Einhver ákveðin manneskja (köllum hana $1$) hugsar um orð, og hvíslar því orði að manneskjunni sem situr vinstra megin við sig, manneskju $2$. Manneskja $2$ hvíslar síðan orðinu að manneskjunni sem situr vinstra megin við sig, manneskju $3$. Þetta gengur svo koll af kolli þangað til manneskjan sem situr hægra megin við manneskju $1$, köllum hana $N$, fær orð hvíslað til sín; á þeim tímapunkti segir manneskja $N$ upphátt hvaða orð hún heyrði, og sömuleiðis segir manneskja $1$ hvaða orð hún hugsaði um í byrjun.

Nema á þessum leikskóla, þá heyra allir krakkarnir orðið sem er hvíslað að þeim rétt, fyrir utan einn staf, sem er öðruvísi.

\section*{Inntak}
Á fyrstu línu er orðið sem manneskja $1$ hugsaði um. Einnig kallað
upphafsorðið. Á annari línu er orðið sem manneskja $N$ heyrði. Einnig kallað
lokaorðið. Upphafsorðið og lokaorðið eru jafn löng. Vert er að taka fram að
aðeins eru notaðir hinir 26 ensku lágstafir.

\section*{Úttak}
Ein línu sem inniheldur minnsta mögulega $N$. Það er, fæsti fjöldi krakka sem
þarf til að upphafsorðið verði að lokaorðinu með því að spila Hvísluleikinn.

\section*{Stigagjöf}
\begin{tabular}{|l|l|l|}
\hline
Hópur & Stig & Takmarkanir \\ \hline
1     & 10   & Upphafsorðið og lokaorðið eru sama orðið \\ \hline
2     & 20   & Lengd upphafsorðsins og lokaorðsins eru bæði $\leq 100$ \\ \hline
3     & 70   & Lengd upphafsorðsins og lokaorðsins eru bæði $\leq 10^6$ \\ \hline
\end{tabular}
