\problemname{RNG Test}
\illustration{0.3}{teningar}{Áður fyrr voru teningar notaðir til að framleiða prófunargögn}

Eins og vanalega fyrir Forritunarkeppni Framhaldsskólanna þarf að semja alveg gífurlegt magn
af prófunargögnum til að prófa allar lausnir á. Þar sem ekki er raunhæft að skrifa öll gögnin
handvirkt beitir KFFÍ (Keppnisforritunarfélag Íslands) slembitalnaframleiðendum (e.\ random number
generators). Þar sem stjórn félagsins er mjög tortryggin er útfærslunum sem fylgja forritunar-
málunum ekki treyst. Því var Atla úthlutað það verkefni að búa til línulegan slembitalnaframleiðanda.
Þar sem Atli svaf hins vegar yfir sig eins og vanalega þarf einhver annar að redda málunum, þ.e.a.s.\ 
þú (eins og vanalega).

Línulegur slembitalnaframleiðandi er skilgreindur útfrá þremur heiltölum $a$,
$b$, $x_0$ og látum þá $f(x) = (ax + b)\ \%\ m$. Prósentumerkið táknar módulus,
og þýðir það að útkoman er reiknuð módulus $m$. Þ.e.a.s.\ við tökum afganginn
af niðurstöðunni þegar henni er deilt með tölunni $m$.

Til að fá $n$-tu slembitöluna er $f$ beitt $n$ sinnum á upphafsstakið $x_0$.
Þetta þýðir að núllta talan er $x_0$, fyrsta er $f(x_0)$, næsta er $f(f(x_0))$
o.s.frv.

\section*{Inntak}
Eina línan í inntakinu inniheldur fimm heiltölur $a$, $b$, $x_0$, $n$ og $m$ eins og lýst er að ofan.
Heiltölurnar eru ekki neikvæðar, og $m$ er alltaf jákvæð.

\section*{Úttak}
Skrifa skal út $n$-tu slembitöluna sem línulegi slembitalnaframleiðandinn framleiðir.

\section*{Stigagjöf}
\begin{tabular}{|l|l|l|}
\hline
Hópur & Stig & Takmarkanir á inntaki sem gilda í þessum ákveðna stigahópi \\ \hline
1 & 40 & $a = 10, b = 3, m = 10^3 + 9, x_0 \leq 10^9, 1 \leq n \leq 10^6$ \\ \hline
2 & 20 & $m = 10^5 + 3, a, b, x_0 \leq 10^9, 1 \leq n \leq 10^{18}$ \\ \hline
3 & 20 & $m = 10^9 + 7, a, b, x_0 \leq 10^9, 1 \leq n \leq 10^{18}$ \\ \hline
4 & 20 & $a, b, x_0, m \leq 10^9 + 7, 1 \leq n \leq 10^{18}$ \\ \hline
\end{tabular}
