\problemname{Þjarki}
\illustration{0.3}{robot}{Mynd fengin af \href{https://flic.kr/p/kTYqmv}{flickr.com}}

Í frítíma sínum hefur Gunnar forritað lítinn þjarka sem fylgir örvum á gólfinu.
Gólfinu hans er skipt upp í reiti, en á hverjum reit hefur hann sett niður ör
sem bendir á einn af fjórum reitum sem eru í kringum reitinn.

Gunnar setur þjarkinn niður á reit á gólfinu. Þjarkurinn skoðar örina á reitnum
sem hann er á, og fer á þann reit sem örin bendir á. Þar endurtekur þjarkurinn
leikinn; skoðar örina á þeim reit, fylgir henni, og svo koll af kolli.

Gunnar er ekki alveg viss hvort þjarkurinn virki eins og hann eigi að virka, og
hefur ekki tíma til að fylgjast með honum allan daginn enda þarf hann að fara í
vinnuna. Gunnar biður þig um aðstoð. Gefinn reiturinn sem þjarkurinn byrjar á,
og hversu mörg skref þjarkurinn á að taka, geturðu hjálpað Gunnari að finna
hvaða reit þjarkurinn á að enda á?

\section*{Inntak}
Fyrsta línan í inntakinu inniheldur tvær heiltölur $n$ og $m$ ($1 \leq n,m \leq
500$), fjöldi raða og fjöldi dálka á gólfinu.

Svo fylgja $n$ línur, hver með $m$ stöfum, sem saman tákna gólfið hans Gunnars.
Stafirnir geta verið
`\texttt{\^{}}' (ör sem bendir upp),
`\texttt{<}' (ör sem bendir til vinstri),
`\texttt{v}' (ör sem bendir niður) eða
`\texttt{>}' (ör sem bendir til hægri).

Svo kemur ein lína með heiltölunni $q$ ($1 \leq q \leq 10^4$), fjöldi
fyrirspurna. Svo fylgja $q$ línur, ein fyrir hverja fyrirspurn, sem inniheldur
þrjár heiltölur $x$, $y$ og $k$ ($1\leq x \leq n$, $1 \leq y \leq m$ og $1 \leq
k \leq 10^9$), þar sem $x$ táknar röð reitsins sem þjarkurinn byrjar á, $y$
táknar dálk reitsins sem þjarkurinn byrjar á, og $k$ táknar fjölda skrefa sem
þjarkurinn á að taka.

\section*{Úttak}
Fyrir hverja fyrirspurn, skrifið út eina línu með tveimur heiltölum $x'$ og
$y'$, þar sem $x'$ er röð reitsins sem þjarkurinn endar á, og $y'$ er dálkur
reitsins sem þjarkurinn endar á.

Það mun aldrei koma upp tilfelli þar sem þjarkurinn mun labba út af gólfinu.

\section*{Stigagjöf}
\begin{tabular}{|l|l|l|}
\hline
Hópur & Stig & Takmarkanir \\ \hline
1     & 15   & $n=1$, $q=1$, $k \leq 1\,000$, $q\leq 100$ og stafirnir geta bara verið `\texttt{<}' og `\texttt{>}' \\ \hline
2     & 15   & $n=1$ , $q\leq 100$ og stafirnir geta bara verið `\texttt{<}' og `\texttt{>}' \\ \hline
3     & 15   & $q\leq 100$ og stafirnir geta bara verið `\texttt{<}', `\texttt{>}' og `\texttt{v}' \\ \hline
4     & 15   & $k \leq 1\,000$, $q\leq 100$ \\ \hline
5     & 15   & $q\leq 10$ \\ \hline
6     & 25   & Engar frekari takmarkanir\\ \hline
\end{tabular}

