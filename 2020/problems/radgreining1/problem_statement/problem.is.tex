\problemname{Raðgreining 1}
\illustration{0.3}{corona}{Mynd fengin af \href{https://en.wikipedia.org/wiki/File:2019-nCoV-CDC-23312_without_background.png}{Wikipedia}, public domain}%
Þú vinnur á rannsóknarstofu þar sem verið er að raðgreina erfðamengi veirunnar
2019-nCoV, betur þekkt sem Kórónaveiran. Með raðgreiningu er verið að finna út
hvernig DNA röð veirunnar lítur út, en DNA röð veirunnar er strengur af lengd
$n$ sem inniheldur stafina \texttt{G}, \texttt{T}, \texttt{A} og \texttt{C}.

Aðferðin sem rannsóknarstofan þín notar til að raðgreina getur aðeins fundið
smá bút af DNA röðinni í einu. Sem dæmi, ef DNA röð veirunnar er af lengd $6$,
þá væri hægt að nota aðferðina til að greina DNA bútinn sem byrjar á staf $1$
og endar á staf $4$ í DNA röð veirunnar, og svo greina DNA bútinn sem byrjar á
staf $3$ og endar á staf $6$ í DNA röð veirunnar. Ef fyrri greiningin skilaði
DNA bútinum \texttt{GCAT} og seinni greiningin DNA bútinum \texttt{ATTC}, þá
væri hægt að leiða það út að DNA röð veirunnar er í raun \texttt{GCATTC}.

Á þennan hátt er búið að raðgreina mismunandi búta af DNA röð veirunnar sem
byrja á mismunandi stöðum, og það eina sem á eftir að gera er að taka bútana
saman og finna hver DNA röð veirunnar er í heild sinni.

Gefnir þeir bútar sem búið er að greina, og hvar hver bútur byrjar í DNA röð
veirunnar, skrifaðu forrit sem setur þá saman og finnur út eins mikið af DNA
röð veirunnar og hægt er.

\section*{Inntak}
Fyrsta línan í inntakinu inniheldur tvær heiltölur $n$ og $m$ ($1 \leq n, m \leq
500$), lengdin á DNA röð veirunnar og fjöldi búta sem búið er að raðgreina.

Svo fylgja $m$ línur, ein fyrir hvern bút sem búið er að raðgreina. Hver af
þessum línum byrjar á heiltölu $s$ ($1 \leq s \leq n$), staðsetningin í DNA röð
veirunnar þar sem þessi bútur byrjar, og svo fylgir búturinn sjálfur, sem er
strengur af lengd $k$ ($1\leq k \leq n-s+1$) sem inniheldur stafina \texttt{G},
\texttt{T}, \texttt{A} og \texttt{C}.

\section*{Úttak}
Skrifið út eina línu sem inniheldur stafina í DNA röð veirunnar. Ef margir
möguleikar koma til greina fyrir ákveðinn staf í DNA röðinni, táknið þá þann
staf sem `\texttt{?}'. Ef eitthvað misræmi kemur upp, eins og að ákveðinn
stafur í DNA röðinni hefur mismunandi gildi í mismunandi bútum, þá á bara að
skrifa út eina línu sem inniheldur \texttt{Villa}.

\section*{Stigagjöf}
\begin{tabular}{|l|l|l|}
\hline
Hópur & Stig & Takmarkanir \\ \hline
1     & 33   & $m=1$ \\ \hline
2     & 33   & Engin misræmi koma upp \\ \hline
3     & 34   & Engar frekari takmarkanir\\ \hline
\end{tabular}

