\problemname{Blaðra}
\illustration{0.3}{balloons}{Mynd fengin af \href{https://flic.kr/p/8K1F5u}{flickr.com}}

Það sameiginilega við afmæli og forritunarkeppnir eru blöðrurnar.
Þú ert mætt(ur) í 20 ára afmæli Forritunarkeppni Framhaldsskólanna.
Þar færðu rosalega flotta blöðru.

Ó NEI!

Þú missir tak á blöðrunni og hún flýgur upp.
Ef þú finnur langa stöng eða stiga þá geturðu kannski teygt þig í
blöðruna og ýtt henni niður.

Blaðran var á hreyfingu þegar þú misstir takið og er hún því með upphafshraðann $v$.
Hröðun blöðrunnar er $a$ og þú áætlar að það taki þig $t$ sekúndur að ná í tækin
og tólin til að bjarga blöðrunni.
Nú þarftu bara að finna vegalengdina $d$ sem blaðran hefur farið.
Sem betur fer lærðirðu í skólanum að $d = vt + \frac{1}{2}at^2$.
Hvert er gildið á $d$?

\section*{Inntak}
Inntakið er ein lína og samanstendur af þremur heiltölum $-1\,000 \leq v \leq 1\,000$, upphafshraða blöðrunnar,
$-1\,000 \leq a \leq 1\,000$, hröðun blöðrunnar og $0 \leq t \leq 1\,000$, tíminn sem blaðran er á hreyfingu.

\section*{Úttak}
Skrifaðu út eina línu með tölunni $d$.
Úttakið er talið rétt ef talan er annaðhvort nákvæmlega eða hlutfallslega ekki 
lengra frá réttu svari en $10^{-5}$. Þetta þýðir að það skiptir ekki máli með hversu 
margra aukastafa nákvæmni talan eru skrifuð út, svo lengi sem hún er nógu nákvæm.

\section*{Stigagjöf}
\begin{tabular}{|l|l|l|}
\hline
Hópur & Stig & Takmarkanir \\ \hline
1     & 100   & Engar frekari takmarkanir\\ \hline
\end{tabular}

