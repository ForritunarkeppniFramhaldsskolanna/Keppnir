\problemname{Fleytitala}
\illustration{0.3}{fleyta}{Mynd fengin af \href{https://flic.kr/p/fLPcfh}{flickr.com}}

Friðrik styttir sér stundir með því að fleyta kerlingar.
Hann er orðinn svo góður í að fleyta kerlingar að hann getur
ákveðið hversu oft steinninn mun skoppa.

Friðrik byrjar á að kasta steininum $d$ metra áfram.
Eftir hvert skopp helmingast vegalengdin sem steinnin fer áfram.
Hversu marga metra áfram fer steinninn samtals?

\section*{Inntak}
Inntak er tvær línur.
Fyrri línan í inntakinu inniheldur eina heiltölu $0 \leq d \leq
10^6$, vegalengdin sem steinnin fer í upprunalega kasti Friðriks.
Seinni línan í inntakinu inniheldur eina heiltölu $0 \leq k \leq 10^{18}$,
fjölda skoppa.

\section*{Úttak}
Skrifið út eina lína með einni rauntölu, samtals vegalengdina sem steinninn ferðast.

Úttakið er talið rétt ef talan er annaðhvort nákvæmlega eða hlutfallslega ekki 
lengra frá réttu svari en $10^{-5}$. Þetta þýðir að það skiptir ekki máli með hversu 
margra aukastafa nákvæmni tölurnar eru skrifaðar út, svo lengi sem þær er nógu nákvæmar.

\section*{Stigagjöf}
\begin{tabular}{|l|l|l|}
\hline
Hópur & Stig & Takmarkanir \\ \hline
1     & 25   & $0 \leq k \leq 10$ \\ \hline
2     & 50   & $0 \leq k \leq 1\,000$ \\ \hline
4     & 25   & Engar frekari takmarkanir\\ \hline
\end{tabular}

