\problemname{Lægð yfir landinu}
\illustration{0.4}{barometer}{Mynd fengin af \href{https://flic.kr/p/67pJnh}{flickr.com}}

Já fínt, já sæll. Er lægð yfir landinu eða!?

Siggi er að verða alveg brjálaður á vonskuveðrinu sem hefur geisað yfir landið
undanfarið. Hann komst að því að þetta var allt þessum árans lægðum að kenna,
en lægðir myndast þar sem loftþrýstingur er minni en loftþrýstingur í kring.

Á morgun langar hann að fara út að leika, en er hræddur um að það verði lægð
yfir landinu. Á vedur.is fann hann Siggi spákort af Íslandi sem var búið að
skipta upp í reiti, en hver reitur sýnir loftþrýstinginn á því svæði á morgun.
Getur þú aðstoðað Sigga að finna hvort það sé einhver lægð á spánni fyrir
morgundaginn?

Reitur á spákortinu inniheldur lægð ef loftþrýstingurinn á reitnum er minni en
loftþrýstingurinn á öllum fjórum reitunum í kring, fyrir neðan, ofan, vinstra
megin og hægra megin (ekki á ská). Athugið að reitir sem eru á jaðarnum
innihalda ekki lægð þar sem þeir hafa ekki fjóra reiti í kring.

\section*{Inntak}
Fyrsta línan í inntakinu inniheldur tvær heiltölur $n$ og $m$ ($3 \leq n,m \leq
50$), fjöldi raða og fjöldi dálka á spákortinu.

Síðan koma $n$ línur, ein fyrir hverja röð, þar sem hver lína inniheldur $m$
heiltölur, eina fyrir hvern dálk. Hver tala táknar loftþrýsting í millibörum á
þeim reit, og eru á bilinu $1$ upp í $10^9$. Spákortið mun aldrei innihalda tvo
reiti hlið við hlið sem hafa nákvæmlega sama loftþrýsting.

\section*{Úttak}
Skrifið út \texttt{Jebb} ef það er lægð á spákortinu, en \texttt{Neibb} ef það
eru engar lægðir.

\section*{Stigagjöf}
\begin{tabular}{|l|l|l|}
\hline
Hópur & Stig & Takmarkanir \\ \hline
1     & 20   & $n = 3$, $m = 3$ \\ \hline
2     & 35   & $n = 3$ \\ \hline
3     & 45   & Engar frekari takmarkanir\\ \hline
\end{tabular}

