\problemname{Manhattanstíflur}
\illustration{0.3}{gridlock}{Mynd fengin af \href{https://commons.wikimedia.org/wiki/File:Gridlock.svg}{commons.wikimedia.org}}

Eins og er vel þekkt samanstendur gatnakerfi Manhattan af götum sem liggja frá norðri til suðurs og frá austri til vesturs. Skulum númera
göturnar sem liggja frá norðri til suðurs $0, 1, \dots, n$ þar sem $0$ er vestasta slíka gatan. Númerum einnig göturnar frá austri til
vesturs með $0, 1, \dots, m$ þar sem $0$ er nyrsta slíka gatan. Þá getum við táknað gatnamót með númerum gatnanna sem mynda mótin, frá
$(0, 0)$ til $(n, m)$. Við höfum áhuga á að geta komist frá norðvesturhorninu $(0, 0)$ í suðausturhornið $(n, m)$ með því að ferðast
meðfram götunum. Hins vegar er alltaf
verið að loka leiðum milli aðlægra gatnamóta vegna framkvæmda eða annarra uppákoma. Getur þú svarað því hvenær slíkar framkvæmdir loka 
á allar leiðir frá $(0, 0)$ til $(n, m)$?

\section*{Inntak}
Fyrsta línan inniheldur eina jákvæða heiltölu $q$. Næst kemur lína með tveimur jákvæðum heiltölum $n, m$, fjölda
gatna í hvora stefnu. Loks fylgja $q$ línur, hver þeirra með upplýsingar um götubút sem er lokað. Hver lína hefur fjórar heiltölur
$x_1, y_1, x_2, y_2$ sem uppfylla $0 \leq x_1, x_2 \leq n$ og $0 \leq y_1, y_2 \leq m$. Þetta merkir að leiðin frá gatnamótunum $(x_1, y_1)$
að gatnamótunum $(x_2, y_2)$ er lokað. Þessi gatnamót eru ávallt aðlæg hvorum öðrum. 

\section*{Úttak}
Ef fyrirspurn númer $i$ lokar á allar leiðir frá $(0, 0)$ til $(n, m)$, prentið $i$. Ef ennþá er hægt að fara frá $(0, 0)$ til $(n, m)$ eftir
allar fyrirspurnir, prentið $-1$ í staðinn.

\section*{Stigagjöf}
\begin{tabular}{|l|l|l|}
\hline
Hópur & Stig & Takmarkanir \\ \hline
1     & 20   & $n = 1$, $m, q \leq 500$ \\ \hline
2     & 20   & $n, m, q \leq 500$ \\ \hline
3     & 30   & $n, m, q \leq 5\,000$ \\ \hline
4     & 30   & $n, m \leq 10^9$, $q \leq 5 \cdot 10^5$ \\ \hline
\end{tabular}

