\problemname{Sara}
\illustration{0.3}{privacy}{Mynd fengin af \href{https://flickr.com/photos/136770128@N07/41587398905/}{flickr.com}}

Eins og hefur komið fram nokkrum sinnum áður að þá er Sara frekar skrýtin.
Sara og besta vinkona hennar voru báðar í Menntaskólanum í Kópavogi.
Þær töluðu mikið saman, hvort sem það var um námið, skemmtun eða samnemendur.

Þar sem sum málefni voru viðkvæm, vildi Sara ekki að fólkið í kringum þær gætu skilið hvað þær sögðu.
Sara var á tungumálalínu og því frekar góð í því að læra mál sem fáir myndu skilja.
Hún gat samt ekki notað bara hvaða mál sem er.
Það þurfti að vera nógu lítið notað til að ólíklegt væri að önnur manneskja í skólanum gæti skilið umræður á því máli.
Þess vegna ákvað Sara að betra væri að forðast almenn mannamál, til dæmis íslensku, ensku, frönsku eða önnur mál sem gætu verið móðurmál fólks.

Sara skoðaði því frekar dulritunaraðferðir.
Dulritunaraðferðir á þessum tíma byggðu mikið á talnafræði.
Þar sem hún var ekki stærðfræðiséní leitaði hún frekar að öðrum aðferðum.

Sara fann tungumálaleikinn \emph{Ubbi dubbi} og ákvað að nota hann til þess að dulkóða samtölin sín.
Leikurinn felst í því að bæta við \texttt{ub} fyrir framan hvert sérhljóð.
Sara lagði hart að sér að læra að tala íslensku á dulkóðaðan máta og tók sér svo tíma að kenna vinkonu sinni.
Sara og vinkona hennar tala og skilja dulkóðaða málið reiprennandi.

Hannes er líka vinur hennar Söru og talar hún stundum dulkóðaða málið við hann.
Hann er því miður í aðeins meiri vandræðum með málið.
Hann skilur reglurnar á bakvið það en þarf aðeins meiri tíma í að þýða fram og til baka.
Hannes ákveður því að hann þurfi forrit til hjálpa sér að dulkóða og afkóða samskiptin sín við Söru.

Heppilegt er að hann er útskrifaður tölvunarfræðingur sem kann að forrita.
Það vill hins vegar til að Hannes er upptekinn við að forrita hugbúnað sem kemur í veg fyrir flugslys.
Því biður hann þig um að vinna þetta verkefni fyrir sig.
Getur þú skrifað forritið?

\section*{Inntak}
Fyrsta línan í inntakinu lýsir því hvort skal dulkóða eða afkóða.
Næst kemur lína með jákvæðri heiltölu $n$.
Svo koma $n$ línur, þar sem hver lína samanstendur af orðum, sem eru aðskilin með bilum.
Orð samanstanda af enskum stöfum, punktum og kommum.

Ef fyrsta línan er \texttt{D} þá skal dulkóða afganginn af inntakinu, en ef hún er \texttt{A} þá skal afkóða hann.
Gera má ráð að ef sérhljóði fylgir sérhljóða í textanum, þá er það ekki samsettur sérhljóði, heldur er hvor sérhljóði fyrir sig borinn fram sem sérhljóð.
Öll tákn í inntakinu eru annaðhvort bil eða enskir stafir.

\section*{Úttak}
Ef fyrsta línan í inntakinu er \texttt{D} skaltu skrifa út textann í inntakinu eftir að hann er dulkóðaður.
Annars skaltu skrifa út textann í inntakinu eftir að hann er afkóðaður.
Ekki þarf að viðhalda formi textans.

\section*{Stigagjöf}
\begin{tabular}{|l|l|l|}
\hline
Hópur & Stig & Takmarkanir \\ \hline
1     & 30   & Dulkóðun, inntak er í mesta lagi $100$ tákn \\ \hline
2     & 30   & Dulkóðun, inntak er í mesta lagi $2 \cdot 10^5$ tákn \\ \hline
3     & 30   & Afkóðun, inntak er í mesta lagi $100$ tákn \\ \hline
4     & 30   & Afkóðun, inntak er í mesta lagi $2 \cdot 10^5$ tákn \\ \hline
\end{tabular}
