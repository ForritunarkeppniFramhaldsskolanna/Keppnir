\problemname{Umferð}
\illustration{0.3}{umferd.jpg}{Mynd fengin af \href{https://www.flickr.com/photos/seattlecamera/51511583493/}{flickr.com}}

Umferðarteppa er mikið vandamál í stórborgum, eins og til dæmis Njú Jork og Síattúl, sérstaklega á álagstímum.

Í umferðarteppu eru margir bílar með litlu bili milli hvors annars, og ekki mikil hreyfing er á þeim.
Til að einfalda hlutina, þá skulum við gefa okkur það að umferð myndast á \emph{akrein}, og hver akrein samanstendur
af $m$ \emph{reitum}, í hverjum reit getur verið að hámarki $1$ bíll, en reitur getur líka verið auður.

Gefin er lengd hraðbrautarinnar $m$, fjölda akreina $n$, og textræn lýsing á hverjum reit, þar sem reitur með bíl
á er táknaður með \texttt{\#}, og \texttt{.} táknar auðann reit.

Verkefnið er að reikna út hlutfall auðra reita á hraðbrautinni, sem tölu á milli $0$ og $1$.

\section*{Inntak}
Fyrsta línan inniheldur eina heiltölu, $m$, lengd hraðbrautarinnar.
Næsta lína inniheldur eina heiltölu, $n$, fjölda akreina á hraðbrautinni.
Næst fylgja $n$ línur.
Hver lína inniheldur $m$ stafi, og er sérhver þeirra annaðhvort \texttt{.} eða \texttt{\#}.

\section*{Úttak}
Skrifið út hlutfall auðra reita á hraðbrautinni.

Úttakið er talið rétt ef annaðhvort hlutfallsleg eða bein skekkja þess er innan við 
$10^{-5}$. Þetta þýðir að það skiptir ekki máli með hversu margra aukastafa nákvæmni tölurnar eru skrifaðar út, svo lengi sem þær er nógu nákvæmar.

\section*{Stigagjöf}
\begin{tabular}{|l|l|l|}
\hline
Hópur & Stig & Takmarkanir \\ \hline
1     & 40   & $n = 1$ og $1 \leq m \leq 500$ \\ \hline
2     & 60   & $1 \leq n, m \leq 500$ \\ \hline
\end{tabular}

