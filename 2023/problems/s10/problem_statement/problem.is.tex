\problemname{S10}
\illustration{0.3}{garage}{Mynd fengin af \href{https://commons.wikimedia.org/wiki/File:Empty_parking_garage_\%2842791016642\%29.jpg}{commons.wikimedia.org}}

Á Suðurlandsbraut getur oft reynst erfitt að finna bílastæði, þar sem margir eiga erindi til fyrirtækjanna sem eiga heima þar.
Mörg bílastæðin eru sérmerkt fyrir viðskiptavini fyrirtækjanna og eru því skammtímastæði.
Því má aðeins leggja í 15--30 mínútur í þessi bílastæði.
Starfsmenn fyrirtækjanna geta ekki notað þessi bílastæði og því þurfa fyrirtækin að tryggja að einhver bílastæði séu úthlutuð fyrir starfsmenn sína.

Hjá Suðurlandsbraut 10 er bílastæðahús á þremur hæðum með mörg bílastæði.
Þó svo bílastæðin séu mörg þá getur verið að þau séu öll í notkun og að bíl sé lagt ólöglega í pláss sem er ekki merkt sem bílastæði, til dæmis fyrir framan önnur bílastæði eða í inngangnum.
Þar sem fyrirtækin deila bílastæðahúsinu að þá hefur stæðunum verið skipt upp á milli fyrirtækjanna.
Bílastæðin eru ekki sérmerkt eftir fyrirtækjum, heldur leggja starfsmenn í bílastæði og má hvert fyrirtæki ekki nota fleiri bílastæði samtímis en því var úthlutað.
Við inngang bílastæðahússins er myndavél sem tekur mynd af bílnúmeraplötu bílsins þegar hann keyrir inn og út úr bílastæðahúsinu.
Það má einnig ekki skilja bíla eftir yfir nóttina og er því bílastæðahúsið alltaf tómt í byrjun hvers dags.

Sérhver bíll sem fer inní bílastæðahúsið leggur annaðhvort löglega eða ólöglega sem er ákvarðað með eftirfarandi reglum.
\begin{itemize}
    \item Ef það er eitthvað laust bílastæði þá leggur bíllinn í það, annars leggur bíllinn ólöglega utan bílastæðis.
    \item Ef bílnúmer bílsins er ekki skráð á neitt fyrirtæki, þá er bílnum lagt ólöglega, hvort sem honum sé lagt í bílastæði eða ekki.
    \item Ef bílnúmerið er skráð á fyrirtæki og fyrirtækið er þegar að fullnýta úthlutaða kvótann sinn, þá er bílnum lagt ólöglega.
\end{itemize}
Aðeins þeir bílar sem eru lagðir í bílastæði telja upp í kvótann hjá fyrirtækinu sem bílnúmerið er skráð.

Eyþór er forritari hjá hugbúnaðarfyrirtækinu Karlar \& Kanínur ehf. og finnst honum mjög einkennilegt að það sé aldrei laust bílastæði fyrir sig.
Hann grunar að eitthvað gruggugt sé í gangi því það eru ekki svo margir á bíl hjá Körlum og Kanínum.
Einhver hlýtur að vera leggja í leyfisleysi.
Honum dettur í hug að nota gögnin frá myndavélinni til þess að athuga hverjir eru að brjóta reglurnar.

Eyþór biður húseigandann um myndirnar úr myndavélini fyrir gærdaginn.
Hann biður einnig um lista yfir bílnúmer frá hverju fyrirtæki.
Þegar upplýsingarnar eru komnar í hendurnar hans skrifar hann forrit sem les myndirnar
og breytir þeim í textaskrá, til að auðvelda vinnslu á gögnunum.

Ansans! Eyþór fattar að hann er seinn á daglega scrum fundinn, fyrsta af sjö fundum hans þennan dag.
Svo virðist sem Eyþór mun ekki hafa færi á að finna sökudólgana í dag einsamall og því biður hann þig um að klára verkefnið sitt.
Getur þú klárað að greina gögnin fyrir Eyþór og fundið bílnúmer þeirra sem lögðu ólöglega?

\section*{Inntak}
Fyrsta línan í inntakinu samanstendur af tveimur heiltölum $n$, fjölda bílastæða í bílastæðahúsinu, og $k$, fjölda fyrirtækja sem nýta bílastæðahúsið.

Næst koma $k$ lýsingar, ein fyrir hvert fyrirtæki.
Lýsingin á fyrirtæki $i$ byrjar fyrst á línu með tveimur heiltölum $n_i$, fjölda bílastæða sem fyrirtæki $i$ má nota í bílastæðahúsinu, og $m_i$, fjölda bílnúmera á skrá hjá fyrirtæki $i$.
Næst fylgja $m_i$ línur, hver þeirra með einu bílnúmeri.

Næst kemur lína með einni heiltölu $m$, fjöldi mynda sem myndavélin tók.
Að lokum koma $m$ línur, ein fyrir hverja mynd sem myndavélin tók, í sömu röð og myndirnar voru teknar.
Hver lína inniheldur eitt bílnúmer.

Sérhvert bílnúmer í inntaki er tvö til sex tákn að lengd og inniheldur einungis enska hástafi og tölustafi.
Það gildir fyrir öll $i$ að $0 \leq n_i \leq n$ og gera má ráð fyrir að samanlagður fjöldi bílastæða sem hefur veruð úthlutaður til fyrirtækja er ekki meiri en fjöldi bílastæða í bílastæðahúsinu.
Fjöldi bílnúmera sem eru skráð á fyrirtæki eru í mesta lagi tvöfaldur fjöldi bílastæða.
Einnig má gera ráð fyrir að bílnúmer séu hvorki skráð á tvö mismunandi fyrirtæki né skráð tvisvar á sama fyrirtækið.
Samtals fjöldi bílnúmera í inntakinu er í mesta lagi $4 \cdot 10^5$.

\section*{Úttak}
Fyrst skal skrifa út eina heiltölu, fjölda bílnúmera sem lausnin mun skrifa út.
Skrifaðu út öll bílnúmer þeirra bíla sem lögðu að minnsta kosti einu sinni í leyfisleysi.
Bílnúmerin skal skrifa út í stafrófsröð.

\section*{Stigagjöf}
\begin{tabular}{|l|l|l|}
\hline
Hópur & Stig & Takmarkanir \\ \hline
1     & 10   & $1 \leq n \leq 100$, $1 \leq k \leq 10$, $1 \leq m \leq 200$, öllum bílum er lagt í bílastæði\\
      &      & og öll fyrirtæki fylgja úthlutaða kvótanum sínum \\ \hline
2     & 15   & $1 \leq n \leq 100$, $1 \leq k \leq 10$, $1 \leq m \leq 200$, öllum bílum er lagt í bílastæði \\ \hline 
3     & 15   & $1 \leq n \leq 100$, $1 \leq k \leq 10$, $1 \leq m \leq 200$, öll fyrirtæki fylgja úthlutaða kvótanum sínum \\ \hline
4     & 15   & $1 \leq n \leq 100$, $1 \leq k \leq 10$,  $1 \leq m \leq 200$ \\ \hline
5     & 15   & $1 \leq n \leq 10^5$, $1 \leq k \leq 10^4$, $1 \leq m \leq 2 \cdot 10^5$, öllum bílum er lagt í bílastæði  \\ \hline % O(n + m log m)
6     & 15   & $1 \leq n \leq 10^5$, $1 \leq k \leq 10^4$, $1 \leq m \leq 2 \cdot 10^5$, öll fyrirtæki fylgja úthlutaða kvótanum sínum  \\ \hline % O(n + m log m)
7     & 15   & $1 \leq n \leq 10^5$, $1 \leq k \leq 10^4$, $1 \leq m \leq 2 \cdot 10^5$ \\ \hline % O(n + m log m)
\end{tabular}
