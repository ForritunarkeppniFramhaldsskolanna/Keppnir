\problemname{Þarasöfnun}
\illustration{0.3}{kelp}{Mynd eftir Peter Southwood, fengin af \href{https://commons.wikimedia.org/wiki/File:Kelp_forest_at_Taranga_pinnacles_Hen_and_Chicken_Islands_PA232359.JPG}{Wikimedia Commons}}

Enn eina ferðina eru tilteknir ónefndir einstaklingar í KFFÍ að reyna að komast
í auðveldan pening og höfðu þeir heyrt frá verkfræðingavinum sínum að allt
umhverfisvænt og framtíðarlegt fær næga styrki þessa dagana. Með þetta í
huga fékk einn þeirra þá hugmynd að smíða þararæktunarvélmenni. Síðar kom
auðvitað í ljós að það var ekki alveg jafn einfalt og þeir héldu, sérstaklega
þegar þurfti að smala saman verkfræðingum til að smíða sjálft vélmennið. Svo
illa gekk þetta að það steingleymdist að forrita vélmennið til að gera það
sem það á að gera, allur tíminn fór í annað. 

Búið er að koma fyrir mælitækjum í sjónum þar sem þarinn vex, svo eina sem
þarf að gera er að koma vélmennið á réttan stað til að sækja þarann þegar
boð berast frá mælitækjunum. Ræktarreitinn má líta á sem $C \times R$ reiti
með þaraplöntu í hverjum reit. Vélmennið er staðsett í $(0, 0)$ í byrjun og
lítum svo á að við byrjum á tíma $0$. Vélmennið getur ferðast um einn reit á 
sekúndu lárétt, lóðrétt eða skáhallt. Þegar þari er tilbúinn til að sækja skal 
vélmennið ferðast á þann reit með sem stysta hætti og á ávallt að ferðast 
skáhallt meðan það borgar sig. Hunsa má tímann sem það tekur vélmennið að 
sækja þarann þegar hann er kominn á réttan stað. Ávallt skal sækja þarann 
sem kom fyrst fyrir í inntaki ef fleiri en ein fyrirspurn bíður eftir afgreiðslu. 
Hins vegar, ef farið er á reit með þara sem bíður við það að fara eitthvert 
annað skal sækja hann í leiðinni. Ef vélmennið hefur ekkert að gera skal það
einfaldlega halda sér kyrrt.

\section*{Inntak}
Fyrsta línan í inntakinu inniheldur tvær jákvæðar heiltölur $C, R$,
fjöldi dálka og raða af reitum á ræktunarsvæðinu. Næsta lína inniheldur eina
jákvæða heiltölu $q$, fjölda fyrirspurna. Loks koma $q$ línur, hver með
þremur heiltölum $t$, þar sem $0 \leq t \leq 10^{18}$, $x$, þar sem $0 \leq x < C$,
 og $y$, þar sem $0 \leq y < R$.
$t$ er tíminn sem mælingin berst um að þarinn sé tilbúinn og $(x, y)$ er staðsetning
þarans. Hver staðsetning kemur mest fyrir einu sinni, $(x, y) \neq (0, 0)$ og 
tími er í veikt vaxandi röð.

\section*{Úttak}
Skrifið út tíma og staðsetningu á sama formi og í inntaki hvert sinn sem þara er 
safnað, í þeirri röð sem þeim er safnað. 

\section*{Stigagjöf}
\begin{tabular}{|l|l|l|}
\hline
Hópur & Stig & Takmarkanir \\ \hline
1     & 20   & $q = 1$, $1 \leq R, C \leq 1\,000$ \\ \hline
2     & 20   & $1 \leq q, R, C \leq 1\,000$, þara er safnað í sömu röð og í inntaki \\ \hline
3     & 20   & $1 \leq q, R, C, \leq 1\,000$ \\ \hline
4     & 40   & $1 \leq q \leq 10^5$, $1 \leq r, c \leq 10^{12}$ \\ \hline
\end{tabular}

