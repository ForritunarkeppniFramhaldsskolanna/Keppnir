\problemname{Flatbökuveisla}
\illustration{0.3}{pizza}{Mynd fengin af \href{https://www.flickr.com/photos/thepizzareview/2329552391}{flickr.com}}

Ómar er nýbúinn að panta flatbökur.
Eftir örstutta bið heyrist í dyrabjöllunni.
Hann opnar útidyrahurðina og blasa við þar kassar merktir Flýtiböku.
Hann grípur kassana og kemur þeim fyrir á matarborðinu, svo kallar hann hátt og snjallt að maturinn sé kominn.
Þá koma allir íbúar heimilisins á augabragði að matarborðinu.

Þegar flatbökukassarnir eru opnir má telja fjölda flatbökusneiða, og eru þær $n$ samtals.
Á heimili Ómars búa $m$ manneskjur.
Íbúarnir eru allir mjög svangir og borða eins mikið og þeir geta, en reglur heimilisins segja að allir íbúar fái jafn margar sneiðar.
Gallinn við þessa reglu er að stundum verður afgangur af matnum sem enginn borðar.
Hversu margar sneiðar verða í afgang?

\section*{Inntak}
Inntak er tvær línur.
Fyrri línan inniheldur eina heiltölu $n$, fjölda sneiða.
Seinni línan inniheldur eina heiltölu $m$, fjölda íbúa á heimili Ómars.

\section*{Úttak}
Skrifaðu út eina heiltölu, fjölda sneiða sem verða eftir í afgang.

\section*{Stigagjöf}
\begin{tabular}{|l|l|l|}
\hline
Hópur & Stig & Takmarkanir \\ \hline
1     & 100   & $1 \leq n, m \leq 10^6$ \\ \hline
\end{tabular}

