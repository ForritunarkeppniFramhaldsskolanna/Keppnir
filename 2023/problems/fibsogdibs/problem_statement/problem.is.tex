\problemname{Fibs og Dibs}
\illustration{0.3}{numbers}{Mynd fengin af \href{https://flickr.com/photos/morebyless/9423385629}{flickr.com}}
Eins og allir vita eru Dagur og Elvar miklir spilarar og elska að spila langa leiki.
Þeir vilja núna spila næstuppáhaldsleikinn sinn Fibs og Dibs.
Hann virkar þannig að fyrst velur Dagur sér eina heiltölu $a$, síðan velur Elvar sér stærri eða jafnstóra heiltölu $b$.
Að lokum velja þeir $n$, fjölda umferða til að spila.

Í hverri umferð byrjar Dagur á því að segja ,,Fibs``,
þá leggja þeir tölurnar sínar saman og Dagur skiptir út gömlu tölunni sinni fyrir þessa nýju, þannig $a$ tekur gildi $a + b$.
Síðan segir Elvar ,,Dibs`` og gera þeir þá sömu skref aftur nema Elvar skiptir út sinni tölu fyrir nýju, þannig $b$ tekur gildi $a + b$.

Þar sem þeir hafa ekki þolinmæðina til að spila $n$ umferðir, því þeir vilja drífa sig í uppáhaldsleikinn sinn Dibs og Fibs,
þá biðja þeir þig um að hjálpa sér að finna út hvaða tölur þeir enda með eftir $n$ umferðir.

\section*{Inntak}
Fyrsta línan í inntakinu inniheldur tvær heiltölur, $a$, talan sem Dagur velur, og $b$,
talan sem Elvar velur, þar sem $1 \leq a \leq b \leq 10^5$.
Síðan kemur ein heiltala $n$, fjöldi umferða sem þeir munu spila.

\section*{Úttak}
Skrifið út hvaða tvær tölur þeir enda með eftir $n$ umferðir í leiknum.
Skrifið út tölurnar tvær á einni línu, fyrst töluna hans Dags og svo töluna hans Elvars, aðskilnar með bili.
Þar sem tölurnar geta orðið stórar, þá skuluð þið skrifa þær módúlus $10^9 + 7$.

\section*{Stigagjöf}
\begin{tabular}{|l|l|l|}
\hline
Hópur & Stig & Takmarkanir \\ \hline
1     & 20   & $0 \leq n \leq 10$ \\ \hline
2     & 40   & $0 \leq n \leq 10^5$ \\ \hline
3     & 40   & $0 \leq n \leq 10^{12}$ \\ \hline
\end{tabular}

