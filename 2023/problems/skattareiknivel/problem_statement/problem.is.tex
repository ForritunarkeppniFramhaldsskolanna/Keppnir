\problemname{Skattareiknivel}

Sigurjón var valinn starfsmaður mánaðarins hjá Krabbaborgurum tólf mánuði í röð
og fékk því mikla launahækkun. Sigurjón veit þó að hann fær ekki öll launin sín
beint í vasann, heldur þarf hann að fyrst að greiða í lífeyrissjóð og
séreignarsparnað og svo þarf hann að greiða tekjuskatt og útsvar af restinni.
Sigurjón veit líka að í hverjum mánuði fær hann ákveðinn persónuafslátt sem
dreginn er frá útreiknuðum tekjuskatti og útsvari og ef hann notar ekki allan
persónuafsláttinn sinn einn mánuðinn þá má hann nota afganginn í næsta mánuði!
Sigurjón vill vita hversu há laun hann fær útborguð í hverjum mánuði eftir alla
skatta og gjöld en útreikningarnir eru of flóknir fyrir hann. Getur þú hjálpað
Sigurjóni?

Skattur af launum einstaklinga er reiknaður á eftirfarandi hátt:

Fyrst þarf að greiða iðgjald í lífeyrissjóð og séreignarsparnað. Gjöldin eru
reiknuð sem prósenta af heildarlaunum fyrir skatt. Heildarlaunin að frádregnum
iðgjöldum eru kölluð skattstofn og er hann námundaður niður að næstu heiltölu.
Við táknum þetta með $\lfloor x \rfloor$, því er
$\lfloor 11.3 \rfloor = 11$ og $\lfloor 25.99 \rfloor = 25$.
Til dæmis, ef einstaklingur er með $500\,000$ krónur í mánaðarlaun og greiðir
$4\%$ iðgjald í lífeyrissjóð og $1\%$ iðgjald í séreignarsparnað, þá greiðir
einstaklingurinn $\lfloor 500\,000 \cdot 0.04 \rfloor= 20\,000$ krónur í
lífeyrissjóð og $\lfloor 500\,000 \cdot 0.01 \rfloor = 5\,000$ krónur í
séreignarsparnað. Skattstofninn er þá
$500\,000 - 20\,000 - 5\,000 = 475\,000$ krónur.

Af skattstofni þarf að greiða tekjuskatt til ríkisins og útsvar til
sveitarfélagsins sem einstaklingurinn býr í. Tekjuskattinum er skipt upp í
mismunandi skattþrep en í hverju þrepi þarf að borga ákveðna prósentu af
skattstofninum sem fellur inn í það skattþrep en útsvarinu er ekki skipt upp í
þrep. Samanlagður tekjuskattur og útsvar nefnist staðgreiðsluskattur og er hann
einnig námundaður niður að næstu heiltölu.

Árið 2023 eru eftirfarandi þrjú skattþrep á Íslandi og er gert ráð fyrir $14.67\%$
meðalútsvari:

\begin{table}[h]
\begin{tabular}{lll}
    Þrep & Laun & Skattshlutfall \\ \hline
    $1$ & $0$ kr.\ -- $409\,986$ kr. & $31.45\%$, þar af $16.78\%$ tekjuskattur \\
    $2$ & $409\,987$ kr.\ -- $1\,151\,012$ kr. & $37.95\%$, þar af $23.28\%$ tekjuskattur \\
    $3$ & $1\,151\,013$ kr.\ og meira & $46.25\%$, þar af $31.58\%$ tekjuskattur \\
\end{tabular}
\end{table}

Sé skattstofninn $475\,000$ krónur fellur hann alveg yfir fyrsta skattþrepið, 
$409\,986$ krónur í því þrepi, og að hluta yfir annað þrepið, $65\,014$ krónur
í því þrepi. Einstaklingurinn borgar því $31.45\%$ af fyrstu $409\,986$
krónunum og $37.95\%$ af síðustu $65\,014$ krónunum. Reiknaður
staðgreiðsluskattur er þá
$\lfloor 0.3145 \cdot 409\,986 + 0.3795 \cdot 65\,014 \rfloor = 153\,613$
krónur.

Að lokum er persónuafslátturinn dreginn frá reiknuðum staðgreiðsluskatti og er
útkoman sú upphæð sem einstaklingurinn þarf að greiða af skattstofninum. Ef
reiknaður staðgreiðsluskattur er lægri en persónuafslátturinn þá safnanst
persónuafslátturinn upp milli mánaða, en ekki aftur í tímann. Árið 2023 er
persónuafslátturinn $59\,665$ krónur á mánuði. Til dæmis, ef reiknaður
staðgreiðsluskattur er $153\,613$ krónur, þá þarf einstaklingurinn
aðeins að greiða $153\,613 - 59\,665 = 93\,948$ krónur af skattstofninum
og fær því $475\,000 - 93\,948 = 381\,052$ krónur í laun eftir að hafa
greitt öll gjöld og skatta. Í þessu tilfelli var persónuafslátturinn fullnýttur
og ekkert safnast upp milli mánaða.

Ef reiknaður staðgreiðsluskattur hefði verið $30\,000$ krónur og
persónuafslátturinn ennþá $59\,665$ krónur þá hefði einstaklingurinn ekki þurft
að greiða neitt af skattstofninum og $59\,665 - 30\,000 = 29\,665$ krónur af
persónuafslættinum hefðu verið ónýttar þennan mánuð. Næsta mánuð hefði
einstaklingurinn þá átt verið með $59\,665 + 29\,665 = 89\,330$ krónur í
persónuafslátt.

\section*{Inntak}
Fyrsta línan inniheldur eina rauntölu $l$, þar sem $0 \leq l \leq 4$, sem táknar iðgjaldið í
lífeyrissjóð sem prósentu.

Næsta lína inniheldur eina rauntölu $u$, þar sem $0 \leq l \leq 4$, sem táknar iðgjaldið í
séreignarsjóð sem prósentu.

Rauntölurnar tvær eru gefnar með nákvæmlega tveimur aukastöfum.

Næst koma tólf línur. Hver lína inniheldur eina heiltölu $m_{i}$, þar sem $0 \leq m_i \leq 10^8$ fyrir öll sérhvert $i$, sem táknar launin sem Sigurjón fær í mánuði $i$.

\section*{Úttak}
Skrifa skal út eina heiltölu $h$, samanlögð útborguð mánaðarlaun Sigurjóns yfir árið
eftir alla skatta og gjöld.

\section*{Stigagjöf}
\begin{tabular}{|l|l|l|}
\hline
Hópur & Stig & Takmarkanir \\ \hline
1     & 100  & Engar frekari takmarkanir\\ \hline
\end{tabular}
