\problemname{Skattareiknivel}

Sigurjón was chosen employee of the month at Krabbaborgarar for twelve months in
a row, and therefore got a big raise. However, Sigurjón knows that he will not
directly receive all of his salary, first he must pay into a pension fund and a
supplementary pension fund, and then he has to pay income tax and communal tax of the
rest. Sigurjón also knows that he gets a personal tax exemption every month that
can be deducted from the total tax, and if he does not use all of his personal
tax exemption one month, he can use the rest of it the following month! Sigurjón
wants to know how much of his salary he will receive after paying all fees and
taxes, but the calculations are too complicated for him. Can you help Sigurjón?

Taxes of individual's salary are calculated as follows:

First you have to pay into a pension fund and a supplementary pension fund. The fees
are calculated as a percentage of the total salary before tax. The total salary
minus the fund fees are called tax base, which is rounded down to the nearest
integer. We represent this using $\lfloor x \rfloor$, therefore $\lfloor 11.3 \rfloor = 11$
and $\lfloor 25.99 \rfloor = 25$.
For example, if an individual's monthly salary is $500\,000$ ISK, and
he pays $4\%$ into a pension fund, and $1\%$ into a supplementary pension fund, then
the individual pays $\lfloor 500\,000 \cdot 0.04 \rfloor = 20\,000$ ISK into a pension fund and
$\lfloor 500\,000 \cdot 0.01 \rfloor = 5\,000$ ISK into a supplementary pension fund. The tax base is
then $500\,000 - 20\,000 - 5\,000 = 475\,000$ ISK.

It is necessary to pay income tax, and communal tax of the tax base to the
state, and to the municipality that the individual lives in, respectively.  The
income tax is divided into different tax brackets, and in each bracket it is
necessary to pay a certain percentage of the tax base that is covered by the
bracket. The communal tax is not divided into brackets. The sum of the income
tax and the communal tax is called withholding tax, which is also rounded down
to the nearest integer.

In 2023, we have the following three tax brackets in Iceland and we assume mean
communal tax of $14.67\%$:

\begin{table}[h]
\begin{tabular}{lll}
    Bracket & Salary & Tax rate \\ \hline
    $1$ & $0$ ISK.\ -- $409\,986$ ISK. & $31.45\%$, which includes $16.78\%$ income tax \\
    $2$ & $409\,987$ ISK.\ -- $1\,151\,012$ ISK. & $37.95\%$, which includes $23.28\%$ income tax \\
    $3$ & $1\,151\,013$ ISK.\ and over & $46.25\%$, which includes $31.58\%$ income tax \\
\end{tabular}
\end{table}

If the tax base is $475\,000$ ISK, then it completely covers the first bracket,
$409\,986$ ISK in that bracket, and partially covers the second bracket,
$65\,014$ ISK in that bracket. The individual therefore pays $31.45\%$ of the
first $409\,986$ ISK, and $37.95\%$ of the remaining $65\,014$ ISK. The total
withholding tax is then
$\lfloor 0.3145 \cdot 409\,986 + 0.3795 \cdot 65\,014 \rfloor = 153\,613$ ISK.

Finally, the personal tax exemption is deducted from the withholding tax, and
the result is the amount that the individual has to pay in taxes of the tax
base. If the withholding tax is lower than the personal tax exemption, then the
personal tax exemption is accrued (but not retroactively). In 2023, the personal
tax exemption is $59\,665$ ISK per month. For example, if the withholding tax is
$153\,613$ ISK, then the individual has to pay
$153\,613 - 59\,665 = 93\,948$ ISK of the tax base, and therefore receives
a salary of $475\,000 - 93\,948 = 381\,052$ ISK after paying all fees and
taxes. In this case, the personal tax exemption was fully used, and nothing is
carried over to the next month.

Had the withholding tax been $30\,000$ ISK, and the personal tax exemption still
$59\,665$ ISK, then the individual would not have needed to pay anything of the
tax base, and $59\,665 - 30\,000 = 29\,665$ ISK of the personal tax exemption
would have carried over to the next month. The invididual would therefore have a
personal tax exemption of $59\,665 + 29\,665 = 89\,330$ ISK the following month.

\section*{Input}
The first line contains a real number $l$, where $0 \leq l \leq 4$, the
contribution percentage to a pension fund.

The second line contains a real number $s$, where $0 \leq s \leq 4$, the
contribution percentage to a private property fund.

The real numbers are given with exactly two digits after the decimal
point.

Following that are twelve lines. Each line contains one integer $m_{i}$,
where $0 \leq m_i \leq 10^8$ for each $i$, denoting Sigurjón's salary for month $i$.

\section*{Output}
The output should contain one integer $h$, the sum of Sigurjón's monthly
salaries over the year after he has paid all fees and taxes.

\section*{Scoring}
\begin{tabular}{|l|l|l|}
\hline
Group & Points & Constraints \\ \hline
1     & 100  & No further constraints\\ \hline
\end{tabular}
