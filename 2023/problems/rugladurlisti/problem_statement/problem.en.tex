\problemname{Ruglaður listi}
\illustration{0.3}{arcade}{Picture taken from \href{https://www.flickr.com/photos/kimli/3942113170}{flickr.com}}

Dagur likes taking walks downtown and one day when Dagur was on his daily walk, he bumped into an
arcade machine in the middle of the street. Since Dagur is very curious, he just had to check this
machine out. When he finally saw what was on the screen, he could not believe his eyes. It has a scoreboard!
Since Dagur loves playing games, he really wanted to get his name to the top of the scoreboard.

Since Dagur did not know how this game worked, he decided to start fiddling with the machine.
Until he finally made it to a screen which displayed the instructions for the game.
The page read:

\begin{quote}
``Welcome to the newest version of scrambled list. Before you start playing you'll have
to get familiar with the rules, they are as follows. I (the arcade machine) will create a list of size $n$
with the values $[1, 2, \dots, n]$, and then I will shuffle it. Next I will write an integer $n$ to the screen.
Afterwards you can ask me about up to $n$ indices and I will write the values in my scrambled list, however
you will get them in some shuffled order. Your goal is to find my list asking the fewest amount of questions.''
\end{quote}

Dagur really wants to be at the top of the scoreboard, but does not know how to do it.
So he asks you to help him find the shuffled list in as few guesses as possible.

\section*{Interaction}
This is an interactive task. Your solution will be tested against a dynamic judge, which reads
the output from your solution and writes to the input of your solution.
This interaction follows specific rules, which are the following:

The judge will first write out one line containing an integer $n$, the length of the shuffled list.
In the test cases, $n$ is always $10\,000$, but in the examples it is less.

Afterwards your solution can perform two types of actions.
The former is a query, which can be repeated many times.
A query starts with the symbol \texttt{?}, followed by a space and then an integer $k$, representing the 
number of indices which will be queried. There after, in the same line, you shall write $k$ indices,
in the range $1$ to $n$, separated by a space, of which you want the judge to give the values.
The judge then answers, in one line, containing the $k$ integeres, seperated by a space,
which represent the values at the indices, in shuffled order. Each number the judge gives
will be in the range from $1$ to $n$ and no number will appear more than once.

If any index, in any query, is not in the range $1$ to $n$, then your solution will be considered incorrect.
If $k < 1$ or $n < k$ in any of the queries, then your solution will be considered incorrect.
If the same index appears more than once in the same query, then your solution will be considered incorrect.

When you think you know what the shuffled list looks like, then you should write \texttt{!}
with a space followed by all $n$ values of the shuffled list, in the correct order.
Then your program should terminate.
If you guess the correct order, then your solution will be accepted and you will score points based on
the number of queries made.

\section*{Scoring}
Your solution will be tested on multiple testcases and the average amount of points
over all testcases will determine the total score.
In the following table you can see some guarantees regarding scores for individual test cases based on the number of queries made.
If the number of queries lies between two rows of the table then the score will be a linear combination of the score in the same two rows.

\begin{tabular}{|l|l|}
\hline
Queries & Points   \\ \hline
$> 10\,000$  & $0$     \\ \hline
$10\,000$    & $1$     \\ \hline
$9\,999$     & $5$     \\ \hline
$6\,668$     & $25$    \\ \hline
$3\,334$     & $50$    \\ \hline
$100$      & $80$    \\ \hline
$\leq 14$  & $100$ \\ \hline
\end{tabular}
