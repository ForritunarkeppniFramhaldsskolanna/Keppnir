\problemname{Ruglaður listi}
\illustration{0.3}{arcade}{Mynd fengin af \href{https://www.flickr.com/photos/kimli/3942113170}{flickr.com}}

Dag einn þegar Dagur var á rölti í miðbænum rakst hann á spilakassa á miðri götu.
Þar sem Dagur er mjög forvitinn þá þurfti hann að skoða þennan spilakassa.
Þegar hann gat loksins séð hvað stóð á spilakassanum, þá trúði hann ekki sínum eigin augum.
Það var stigatafla á honum! Auðvitað þar sem Dagur er mikill spilakall, þá varð hann að koma sér á töfluna.

Þar sem Dagur var ekki enn viss hvernig leikurinn virkaði, þá byrjaði hann að fikta í spilakassanum
þangað til að hann komst á skjámynd þar sem leikreglurnar voru sýndar. Á skjámyndinni stóð:

\begin{quote}
,,Velkominn í nýjustu útgáfu að rugluðum lista. Áður en byrjað er að spila leikinn skal fyrst fara yfir leikreglurnar, þær eru eftirfarandi.
Ég mun búa til lista af stærð $n$, með stökunum $[1, 2, \dots, n]$ og síðan stokka ég hann.
Eftir það mun ég skrifa út heiltöluna $n$ á skjáinn. Síðan mátt þú spyrja mig endurtekið um allt að $n$ vísa, eða staðsetningar, í hvert skipti
og ég mun skrifa út gildin í ruglaða listanum mínum sem eru þar, en þú færð þau hinsvegar líka stokkuð.
Markmiðið þitt er að finna listann í eins fáum spurningum og þú getur.``
\end{quote}

Dagur vill núna ná eins háu sæti og hann getur, en veit ekki alveg hvernig hann á að fara að því.
Hann biður þig því um hjálp við að giska á ruglaða listann í eins fáum giskum og hægt er.

\section*{Gagnvirkni}
Þetta er gagnvirkt verkefni. Lausnin þín verður keyrð á móti gagnvirkum dómara
sem les úttakið frá lausninni þinni og skrifar í inntakið á lausninni þinni.
Þessi gagnvirkni fylgir ákveðnum reglum:

Dómarinn skrifar fyrst út eina línu með heiltölunni $n$, sem er lengdin á ruglaða listanum.
Í prufutilvikunum er $n$ alltaf $10\,000$, en í sýnidæmum er það minna.

Eftir það mun lausnin þín framkvæma tvær týpur af aðgerðum.
Fyrri aðgerðin er spurning og má endurtaka hana oft.
Spurning hefst á tákninu \texttt{?}, næst kemur bil og svo heiltala $k$, sem táknar
fjölda vísa sem spurt er um. Þar eftir, í sömu línu, skalt þú skrifa út $k$ vísa, á bilinu $1$ upp í $n$,
aðskilna með bilum, sem þú vilt að dómarinn gefi þér gildin á.
Dómarinn svarar þá, með einni línu sem samanstendur af $k$ heiltölum aðskilnum
með bili, sem tákna gildin sem eru á þessum vísum í stokkaðri röð. Hver einasta tala í svari dómarans
mun vera á bilinu $1$ upp í $n$ og engin tala mun koma fyrir oftar en einu sinni í sama svari.

Ef vísir er ekki á bilinu $1$ upp í $n$ í einhverri spurningu mun lausnin vera dæmd röng.
Ef $k < 1$ eða $n < k$ í einhverri spurningu mun lausnin vera dæmd röng.
Ef sami vísir kemur fyrir oftar en einu sinni í sömu spurningu mun lausnin vera dæmd röng.

Þegar þú telur þig vita röðina á ruglaða listanum þá skalt þú
skrifa \texttt{!} með bili á eftir og svö öll $n$ stökin í ruglaða listanum,
aðskilin með bilum, í réttri röð. Síðan skal forritið hætta keyrslu.
Ef þú skrifar rétta röð, þá mun lausnin þín vera samþykkt og þú munt fá stig
sem eru ákvörðuð út frá fjölda spurninga sem voru spurðar.

\section*{Stigagjöf}
Lausnin þín verður keyrð á mörgum prufutilvikum og meðaltal stiga yfir
öll prufutilvik mun gilda til stigagjafar.
Eftirfarandi eru tryggð stig miðað við fjölda spurninga sem spurt var um.
Ef fjöldi fyrirspurna lendir á milli tveggja raða í töflunni að þá er
stigafjöldi ákvarðaður sem línuleg samsetning af stigunum í sömu tveimur röðum.

\begin{tabular}{|l|l|}
    \hline
Spurningar & Stig    \\ \hline
$> 10\,000$  & $0$     \\ \hline
$10\,000$    & $1$     \\ \hline
$9\,999$     & $5$     \\ \hline
$6\,668$     & $25$    \\ \hline
$3\,334$     & $50$    \\ \hline
$100$      & $80$    \\ \hline
$\leq 14$  & $100$ \\ \hline
\end{tabular}
