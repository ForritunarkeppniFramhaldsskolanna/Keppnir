\problemname{Heimavinna}
\illustration{0.3}{books}{Mynd fengin af \href{https://www.flickr.com/photos/breatheindigital/4692258762/}{flickr.com}}

\noindent
Hneitir er menntaskólanemandi, en það þýðir að hann þarf að sinna 
heimavinnu fyrir hinar ýmsu námsgreinar. Hneitir er ekki latur,
en honum leiðist að læra heima, sérstaklega fyrir efnafræði. 
Á morgun verður hann tekinn upp á töflu, þar sem hann á að leysa 
dæmin sem honum er sett fyrir.
Gallinn er að hann veit ekki hvaða dæmi hann á að gera upp á töflu 
því efnafræðikennarinn hans, Ormhildur, á það til að sleppa því að
fara yfir stöku dæmi, til þess að flýta yfirferðinni. 
Það sem verra er, Hneitir hefur verið að fresta því að leysa dæmin
og því eru fjölmörg dæmi sem hann þarf að leysa fyrir morgundaginn.

Hvað þarf Hneitir að leysa mörg dæmi?

\section*{Inntak}
Inntakið er ein lína og tiltekur þau dæmi sem Hneitir þarf að leysa. Hneitir þarf alltaf að leysa í minnsta lagi eitt dæmi.
Dæmin eru númerið frá $1$ uppí $1\,000$ og eru talin upp í vaxandi röð, afskilin með semíkommu (\texttt{;}).
Ef Hneitir á að leysa tvö eða fleiri dæmi í röð er það tiltekið með \texttt{-}.
Dæmi um inntak er \texttt{1-3;5;7;10-13} og þýðir það að Hneitir þarf að leysa dæmi 1,2,3,5,7,10,11,12,13.

\section*{Úttak}
Skrifaðu út eina línu með einni heiltölu $n$, fjöldi dæma sem Hneitir þarf að leysa.

\section*{Stigagjöf}
\begin{tabular}{|l|l|l|l|}
\hline
Hópur & Stig & Takmarkanir \\ \hline
1     & 25  & Dæmin eru öll á einu bili, t.d. \texttt{3-10} \\ \hline
2     & 25  & Engin tvö dæmi eru samliggjandi, t.d. \texttt{1;3;5;8;13} \\ \hline
3     & 50  & Engar frekari takmarkanir \\ \hline
\end{tabular}
