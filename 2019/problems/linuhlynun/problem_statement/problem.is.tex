\problemname{Línuhlýnun}
\illustration{0.3}{traffic}{Mynd fengin af \href{https://www.flickr.com/photos/shiyang_huang/11319287966/}{flickr.com}}

Íbúar Línulands fengu þær fréttir nýlega að byggja ætti háskóla í landinu.
Þetta eru þau himinlifandi með, enda á að kenna Línulega algebru, Línulega
bestun og Línulega fallagreiningu í skólanum, svo eitthvað sé nefnt.

Gróðurhúsaáhrif hafa farið illa með Línuland eins og önnur lönd, og hefur þetta
valdið mikilli Línuhlýnun. Yfirvöld hafa áhyggjur af því að þetta gæti versnað
þegar nemendur nýja háskólans fara að keyra í skólann, enda gæti það aukið
koltvíoxíðmengun. Þau hafa því fengið þig til að aðstoða við að velja
staðsetningu skólans til að lágmarka mengun.

Eins og nafnið gefur til að kynna þá býr fólkið í Línulandi á línu. Húsnúmer í
landinu eru jákvæðar heiltölur, en fjarlægð á milli húsa númer $a$ og $b$ er
$|a-b|$ kílómetrar.

Gefinn listi af nemendum háskólans, hvar þeir eiga heima, og hversu mikið af
koltvíoxíð bíllinn þeirra myndar á hverjum kílómetra, finndu í hvaða húsnúmeri
væri best að byggja háskólann til lágmarka heildarkoltvíoxíðmengun á hverjum
morgni þegar nemendur keyra í skólann.

Athugið að margir nemendur gætu búið í sama húsnúmeri. Það er í lagi að byggja
háskólann í húsi þar sem nemendur eiga þegar heima, en þá fá þeir einfaldlega
að búa í háskólanum. (Vá, heppnir þeir!)

\section*{Inntak}
Fyrsta línan í inntakinu inniheldur eina heiltölu $n$ ($1 \leq n \leq
2\cdot10^5$), fjöldi nemenda í nýja háskólanum.

Síðan koma $n$ línur, ein fyrir hvern nemenda, þar sem lína $i$ inniheldur tvær
heiltölur $x_i$ ($1 \leq x_i \leq 10^9$), númer hússins sem nemandi $i$ býr í,
og $c_i$ ($1 \leq c_i \leq 10^3$), magn koltvíoxíðs sem bíll nemanda $i$ mengar
á hverjum kílómetra.

\section*{Úttak}
Skrifið út í hvaða húsnúmeri væri best að byggja háskólann til að lágmarka
heildarkoltvíoxíðmengun á hverjum morgni þegar nemendur keyra í skólann. Ef mörg
húsnúmer koma til greina, skrifið út minnsta húsnúmerið sem kemur til greina.

\section*{Stigagjöf}
\begin{tabular}{|l|l|l|l|}
\hline
Hópur & Stig & Takmarkanir \\ \hline
1     & 25   & $n \leq 10^3$, $x_i \leq 10^3$ \\ \hline
2     & 25   & $x_i \leq 10^3$ \\ \hline
3     & 20   & $c_i = 1$ \\ \hline
4     & 30   & Engar frekari takmarkanir\\ \hline
\end{tabular}

