\problemname{Dagatal}
\illustration{0.3}{calendar}{Mynd fengin af \href{https://www.flickr.com/photos/dafnecholet/5374200948/in/}{flickr.com}}
Líklega það einkennilegasta við Gregoríska dagatalið er uppskipting daga milli mánuða.
Af einhverri ástæðu virðist dreifing daganna á milli mánuða nánast handahófskennd þó það sé í raun ástæða fyrir henni.
Þetta ruglar marga í ríminu og eiga þeir oft erfitt með að muna fjölda daga í hverjum mánuði.
Skrifaðu forrit til að hjálpa þessum vesalings sálum. Forritið skal gera ráð fyrir að árið sé 2019.

\section*{Inntak}
Ein lína með einni heiltölu $m$, númer mánaðarins. Það gildir ávallt að $1 \leq m \leq 12$.

\section*{Úttak}
Ein lína með einni heiltölu, fjöldi daga í mánuði númer $m$.

\section*{Stigagjöf}
\begin{tabular}{|l|l|l|l|}
\hline
Hópur & Stig & Takmarkanir \\ \hline
1     & 100  & Engar frekari takmarkanir \\ \hline
\end{tabular}
