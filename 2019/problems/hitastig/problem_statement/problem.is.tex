\problemname{Hitastig}
\illustration{0.3}{thermometer}{Mynd fengin af \href{https://www.flickr.com/photos/renaissancechambara/4938639714/}{flickr.com}}
Prófessor Hannes er dugnaðarmaður.
Áhugasvið Hannesar er hitastig, en honum finnst það mjög áhugavert og fylgist mikið með því.
Honum finnst mjög skemmtilegt þegar hitastig er mjög hátt eða mjög lágt.
Honum finnst í raun eiginlega allt sem tengist hitastigi vera mjög skemmtilegt, alveg ótrúlega skemmtilegt!

Síðustu $n$ daga hefur Hannes verið að skrifa niður hitastigin.
Núna er kominn tími til að fara yfir öll gögnin og finna skemmtilega hluti.
Hannesi finnst skemmtilegast að vita hvað er hæsta og lægsta hitastigið.
Hann byrjar að skoða gögninn og reynir að finna út hvað er lægsta og hvað er hæsta hitastigið en váááá þetta eru svo mikið af gögnum.
Hann getur bara ómögulega gert þetta með höndunum.

Getur þú hjálpað Hannesi að finna hæsta og lægsta histastigið sem hefur verið síðustu $n$ daga?

\section*{Inntak}
Inntakið er tvær línur. Fyrri línan inniheldur eina heiltölu $1 \le n \le 1\,000$.
Seinni línan inniheldur $n$ heiltölur $-10^{18} \le a_i \le 10^{18}$, þar sem  $a_i$ táknar hitastig $i$-ta dagsins.

\section*{Úttak}
Skrifaðu út tvær heiltölur, fyrst hæsta hitastigið og svo lægsta hitastigið sem hefur verið síðustu $n$ daga.

\section*{Stigagjöf}
\begin{tabular}{|l|l|l|l|}
\hline
Hópur & Stig & Takmarkanir \\ \hline
1     & 80    & $-10^9 \le a_i \le 10^9 $\\ \hline
2     & 20    & Engar frekari takmarkanir \\ \hline
\end{tabular}
