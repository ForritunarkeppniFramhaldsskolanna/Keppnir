\problemname{Stalínröðun}
\illustration{0.3}{stalinrodun}{Sýnidæmi 2}
Eitt af þeim fjölmörgu skiptum sem Unnar var að skruna og skoða nýjar færslur á
samfélagsmiðlinum LasÞað sá hann færslu á /l/forritunarhúmor um Stalínröðun.
Þar var lýst línulegu reikniriti til þess að raða lista en það virkaði með því að
fjarlæga öll stök í listanum sem voru ekki í vaxandi röð.

Nú fór Unnar að pæla ,,Hvað með að í staðinn fyrir að eyða út öllum stökum sem eru ekki í vaxandi
röð að þá eyðum við út öllum þeim stökum sem eru í vaxandi röð?''. Mjög eðlileg spurning til að spyrja er
þá hvað tæki það margar ítranir þangað til við endum með tóman lista?

Stak $a_i$ er í vaxandi röð ef að fyrir öll $j < i$ gildir að $a_i \geq a_j$.

\section*{Inntak}
Fyrsta línan inniheldur eina heiltölu $1 \leq n \leq 5 \cdot 10^5$.
Næsta lína inniheldur $n$ heiltölur $1 \leq a_i \leq 10^6$. 

\section*{Úttak}
Skrifið út fjölda ítrana til þess að enda með tóman lista.

\section*{Sýnidæmi}

[ 1 7 5 8 6 3 2 4 ]\newline
Eftir fyrstu ítrun:\newline
[ 5 6 3 2 4 ]\newline
Eftir aðra ítrun:\newline
[ 3 2 4 ]\newline
Eftir þriðju ítrun:\newline
[ 2 ]\newline
Eftir fjórðu ítrun:\newline
[ ]\newline
Svarið er því fjórir.

\section*{Stigagjöf}
\begin{tabular}{|l|l|l|l|}
\hline
Hópur & Stig & Takmarkanir \\ \hline
1     & 20   & $n \leq 50$, öll $a_i$ mismunandi \\ \hline
2     & 30   & $n \leq 2\,500$, öll $a_i$ mismunandi \\ \hline
3     & 50   & Engar frekari takmarkanir\\ \hline
\end{tabular}
