\problemname{Blaðra}
\illustration{0.3}{ballon}{Blöðrur}

\noindent
Í Forritunarkeppni Framhaldsskólanna fá öll lið blöðru fyrir hvert dæmi sem þau
leysa, en blöðrurnar eru í mismunandi litum eftir því hvaða dæmi var leyst.

Í ár verða $k$ dæmi. Því þarf að ráða $k$ starfsmenn til að blása blöðrur.
Hver starfsmaður mun blása eina blöðru fyrir hvert lið sem leysir dæmið hans.

Hannes var einn af þeim sem var ráðinn til að blása blöðrur í ár og mikið þurfti hann að blása
því svo margir leystu dæmið hans. Þetta finnst honum ósanngjarnt þar sem hann var á sömu launum
og allir hinir starfsmennirnir.

Eftir keppnina pælir Hannes:
\begin{quote}
    Hvaða dæmi hefði ég getað fengið þannig ég hefði þurft að blása sem minnst?
\end{quote}

Getur þú svarað þessari spurningu fyrir Hannes?

\section*{Inntak}
Fyrsta lína inniheldur tvær heiltölur $1 \le k,q \le 10^5$, þar sem
$k$ táknar fjölda dæma og $q$ táknar hversu mörg dæmi voru leyst í heildina.
Næst fylgja $q$ línur, hver með tvær heiltölur $1 \le a_i \le 10^5, 1 \le b_i \le k$ sem táknar að lið númer $a_i$ leysti dæmi $b_i$.
Ekkert lið leysir sama dæmi oftar en einu sinni.

\section*{Úttak}
Skrifaðu út eina heiltölu, minnsta fjölda blaðra sem Hannes hefði þurft að blása.

\section*{Stigagjöf}
\begin{tabular}{|l|l|l|l|}
\hline
Hópur & Stig & Takmarkanir \\ \hline
1     & 50  & $1 \le k,q,a_i \le 100$ \\ \hline
2     & 50  & Engar frekari takmarkanir \\ \hline
\end{tabular}
