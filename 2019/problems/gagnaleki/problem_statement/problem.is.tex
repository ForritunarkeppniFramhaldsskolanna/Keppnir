\problemname{Gagnaleki}
\illustration{0.3}{fall}{}


Hakkari braust inn á Forritun.is og lak út mörgum gígabætum af gögnum á netið,
þar með talið skráningarupplýsingum keppenda Forritunarkeppni Framhaldsskólanna
frá upphafi. Í þessum gögnum eru því meðal annars lykilorð frá þér og öðrum
keppendum. Þetta finnst þér mjög spennandi. Þú finnur gögnin á svokölluðu
Hulduneti og ákveður að skoða lykilorð andstæðinga þinna í keppninni í ár. En
þér til mikillar mæðu eru lykilorðin sjálf ekki geymd í gögnunum, heldur eru
lykilorðin geymd á tættu formi. Það er, þegar keppandi skráir sig fyrst í
keppnina og slær inn lykilorð, þá er lykilorðið brenglað með svokölluðu
tætifalli áður en það er sett inn í gagnagrunninn. Svo þegar keppandi ætlar að
skrá sig inn með lykilorðinu sínu, þá er lykilorðið brenglað með sama
tætifalli, og niðurstaðan borin saman við tætta lykilorðið sem var geymt í
gagnagrunninum. Ef tættu lykilorðin eru eins, þá er keppandinn skráður inn;
annars ekki.

Það er því ljóst að þeir sem forrituðu Forritun.is höfðu öryggið í fyrirrúmi ef
ske kynni að vefsíðan yrði hökkuð með því að geyma ekki sjálf lykilorðin. Þetta
gerir þér erfiðara fyrir, en þú gefst ekki svo einfaldlega upp! Á GitHub finnur
þú eftirfarandi kóða fyrir tætifallið í forritunarmálinu Python:

\begin{verbatim}
def heiltala(c):
    if '0' <= c <= '9':
        return int(c)
    return ord(c) - ord('a') + 10

def stafur(c):
    if 0 <= c <= 9:
        return str(c)
    return chr(ord('a') + c - 10)

def leggjaSaman(a, b):
    carry = 0
    s = ''
    for at in range(31, -1, -1):
        carry += heiltala(a[at]) + heiltala(b[at])
        s = stafur(carry % 16) + s
        carry = carry // 16
    return s

def brengla(s, at, u):
    magic = 'b058592efd277ae75f27bd99d1628fbd'
    if at >= len(s):
        return magic

    res = leggjaSaman(brengla(s, at+1, True), brengla(s, at+1, False))
    for i in range(6):
        res = leggjaSaman(res, res)

    cnt = ord(s[at])
    for i in range(cnt):
        res = leggjaSaman(res, magic)

    return res

def taetaLykilord(s):
    return brengla(s, 0, True)


print(taetaLykilord('forrit123'))
\end{verbatim}

Þar að auki fannstu líka kóðann í forritunarmálunum
\href{/problems/iceland.gagnaleki/file/statement/attachments/taeta.cpp}{C++},
\href{/problems/iceland.gagnaleki/file/statement/attachments/Taeta.cs}{C\#},
\href{/problems/iceland.gagnaleki/file/statement/attachments/Taeta.java}{Java},
\href{/problems/iceland.gagnaleki/file/statement/attachments/taeta.js}{JavaScript} og
\href{/problems/iceland.gagnaleki/file/statement/attachments/taeta.rb}{Ruby}.

Þú varst líka búinn að safna saman tættu lykilorðunum -- sem eru hvorki færri
né fleiri en $2\,500$ -- í skrá sem má nálgast
\href{/problems/iceland.gagnaleki/file/statement/attachments/gagnaleki.txt}{hér},
eitt tætt lykilorð á hverri línu.

Með þessa vitneskju um hvernig tætifallið virkar, þá viltu reyna að brjóta eins
mörg af tættu lykilorðunum og þú getur. Tætt lykilorð telst brotið þegar þú
finnur einhvern streng sem gefur sömu útkomu ef þú setur hann í gegnum
tætifallið. Þú mátt nota hvaða aðferðir sem þú vilt til að brjóta tættu
lykilorðin, þar með talið finna enska orðalista á netinu eða breyta tætifallinu
til að gera það hraðara (svo lengi sem það gefur sömu niðurstöður).

\section*{Inntak}
Forritið verður bara keyrt einu sinni, og mun inntakið vera nákvæmlega það sama
og er í skránni sem bent er á að ofan. Á fyrstu línu er fjöldi tættra
lykilorða, sem er alltaf $2\,500$. Þar eftir fylgja $2\,500$ tætt lykilorð,
eitt á hverri línu.

\section*{Úttak}
Fyrir hvert tætt lykilorð úr inntakinu sem þú nærð að brjóta, skrifaðu út eina
línu á forminu ``\textit{tætt lykilorð}\texttt{:}\textit{lykilorð}'', þar sem
\textit{lykilorð} er strengur sem gefur niðurstöðuna \textit{tætt lykilorð} úr
tætifallinu.

\section*{Stigagjöf}
Þú færð $0.04$ stig fyrir hvert tætt lykilorð sem þú nærð að brjóta. Ef
heildarfjöldi stiga er kommutala, þá er hún námunduð upp að næstu heiltölu. Þú
færð líka að vita eftirfarandi um upprunalegu lykilorðin, áður en þau voru
tætt:

\begin{itemize}
    \item 20\% af lykilorðunum innihalda bara tölustafi og eru að lengd í mesta lagi 4.
    \item 20\% af lykilorðunum eru ensk orð.
    \item 20\% af lykilorðunum innihalda bara enska lágstafi og eru að lengd í mesta lagi 5.
    \item 20\% af lykilorðunum eru ensk orð með tölustafi fyrir aftan.
    \item 20\% af lykilorðunum samanstanda af allt að 15 enskum bókstöfum, tölustöfum og táknum.
\end{itemize}

\section*{Sýnidæmi}
Ef lausnin þín skrifar út eftirfarandi, þá fær hún $0.2$ stig, sem er svo
námundað upp í $1$ stig.
\begin{verbatim}
73f5bcbc047219f8ce6fa72e7910b82d:1234
109de74e2dc5851b9e4dc32f40a34eda:benni
c5e0062df0ca99faa2d710c058b3f130:password
3c50164525d099e1b909b3e4b3675a67:sexy1000
053e0ff2244d8f9f1633c0b4f7223e79:mN9`C6ZjGa@!$/L
\end{verbatim}

